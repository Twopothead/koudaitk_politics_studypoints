% This tex file is generated by get_politics_chapters/index.sh automatically.
\section{[马原]马克思主义基本原理概论}

\subsection{01451-哲学基本问题及其内容(对哲学的划分)}
\textbf{{恩格斯}}第一次明确提出``\textbf{{{全部哲学,特别是近代哲学的重大的基本问题,是思维和存在的关系问题}}}''哲学基本问题包括\textbf{{{两个方面}}}的内容:其一,意识和物质、精神和自然界,究竟谁是\textbf{{{世界的本原}}},即物质和精神何者是\textbf{{{第一性}}}的问题;其二,思维能否认识或正确认识存在,即思维和存在有无\textbf{{{同一性}}}的问题。

根据对哲学基本问题\textbf{{{第一方面}}}的不同回答,哲学可划分为\textbf{{{唯物主义和唯心主义}}}两个对立的基本派别{。}\textbf{{{唯物主义}}}把世界的本原归结为物质,\textbf{{{主张物质第一性,意识第二性}}},意识是物质的产物,\textbf{{{唯心主义}}}把世界的本原归结为精神,\textbf{{{主张意识第一性,物质第二性}}},物质是意识的产物。

{根据对哲学基本问题}\textbf{{{第二方面}}}{的不同回答,哲学又可以划分为}\textbf{{可知论和不可知论}}{。}\textbf{{可知论}}{认为世界是可以被认识的,思维和存在具有}\textbf{{同一性}}{;}\textbf{{不可知论}}{认为世界是不能被人所认识或不能被完全认识的,}\textbf{{否认思维和存在的同一性}}{。}

\subsection{01452-哲学的其他划分}
\textbf{{社会存在与社会意识的关系问题}}是社会\textbf{{历史观的基本问题}}。\textbf{{历史唯物主义认为社会存在决定社会意识,历史唯心主义认为社会意识决定社会存在}}。

{回答世界是}\textbf{{怎样存在}}{的问题,形成了}\textbf{{辩证法和形而上学}}{两种不同的观点。}\textbf{{辩证法}}{坚持用}\textbf{{联系的、发展的}}{观点看世界,认为发展的根本原因在于事物的内部}\textbf{{矛盾}}{;}\textbf{{形而上学}}{则坚持用}\textbf{{孤立的、静止的}}{观点看问题,否认事物内部矛盾的存在和作用。}

\subsection{01454-物质与运动}
\textbf{{运动}}作为物质的\textbf{{存在方式、固有属性或根本属性}},是指宇宙间所发生的一切变化和过程,简要的说运动就是指\textbf{{变化}}。

\textbf{{物质和运动密不可分,物质是运动的,运动是物质的运动}}{。}

{坚持物质和运动相统一的观点,要反对否认物质是运动的}\textbf{{形而上学的观点}}{和反对否认运动是物质的运动的}\textbf{{唯心主义的观点}}{。}

\subsection{01456-时间与空间}
\textbf{{时间、空间是运动着的物质存在形式}},时间指物质运动的持续性,是一维的、不可逆的,空间是指运动物质的广延性,是三维的。

\textbf{{物质运动与时间空间是不可分的}}。物质运动在时间空间中存在,\textbf{{{时间空间是物质运动的存在形式。}}{时间和空间具有相对性。}}

\subsection{01457-意识的起源与本质}
\textbf{{意识的起源}}:是自然界长期发展的产物;是社会的产物。意识起源经历了三个阶段:一切物质都具有的反应特性到低等生物的刺激感应性,再到高等动物的感觉和心理,最终发展为人类的意识。

\textbf{{意识的本质}}{{:}是}\textbf{{人脑的机能与属性}}{,}\textbf{{是人脑对客观世界的主观映象}}{。}

{马克思指出,``观念的东西不外是移入人的头脑并在人的头脑中改造过的物质的东西而已。''}

\textbf{{意识具有的能动作用。}}

\subsection{01458-意识的特征}
\textbf{{意识的内容是客观的,形式是主观的}}{。}

\textbf{{物质和意识的关系}}{:}\textbf{{物质是根源,意识是派生}}{;物质不能代替意识,意识不能代替物质。物质决定意识,意识反作用于物质;物质可以变成意识,意识可以变成物质;意识和物质具有同一性。}

\subsection{01461-人类社会与自然界}
自然界和人类社会的关系:\textbf{{分化和统一}}。

{\textbf{实践}}是使物质世界分化为自然界与人类社会的\textbf{{历史前提}}{,}又是使自然界与人类社会统一起来的{\textbf{现实基础}}。

\subsection{01465-发展的实质}
\textbf{{运动变化的基本趋势是发展}}{。}

发展的\textbf{{含义}}:是指事物由低级到高级、由简单到复杂的前进性变化。

{发展的}\textbf{{实质}}{:是}\textbf{{新事物的产生,旧事物的灭亡}}{。}

{发展的过程性:是指一切事物都处在生存和灭亡的不断变化中,}事物的发展是一个过程,一切现象都是作为过程而存在、作为过程而发展的。

{\textbf{{新事物能够战胜旧事物}}}

(1)新事物符合事物发展的必然趋势;

(2)新事物脱胎于旧事物,既克服了旧事物的消极因素,又吸收了旧事物的积极因素;

(3)新事物代表了最广大人民群众的根本利益,受到群众的支持。

\subsection{01466-量变与质变}
{\textbf{量}}:事物存在和发展的规模、程度、速度以及它的构成成分在空间上的排列组合等可以用数量表示的规定性。

质:一事物成为自身并区别于它事物的规定性。事物质的规定性是由事物内部矛盾的特殊性所决定的。

\textbf{{度}}:\textbf{{事物保持自己质的量的界限}},即事物的范围、幅度和限度。它的极限叫\textbf{{关节点}}。认识度才能为实践活动提供正确的准则即适度原则,防止``过''或``不及''。

\textbf{{区分量变和质变的根本标志是事物的变化是否超出度}},在度的范围内的变化是量变,超出度的变化是质变。

{量变与质变是事物变化和发展的状态与形式;在实践上,即要重视量变,又要重视质变。}

\subsection{01467-对立统一规律}
矛盾的同一性和斗争性:

\textbf{{同一性}}是指矛盾双方相互依存、相互贯通的性质和趋势。

\textbf{{斗争性}}是矛盾着的对立面之间相互排斥、相互分离的性质和趋势。

矛盾的同一性和斗争性相互联结、相互制约的原理,要求我们{在分析和解决矛盾时,必须从对立中把握同一,从同一中把握对立。}矛盾的同一性和斗争性不仅揭示了事物联系的实在内容,而且\textbf{{揭示了事物发展的内在动力}}。\textbf{{矛盾是事物发展的动力和源泉}}。矛盾推动事物的发展,事物发展的根本原因不在事物的外部,而在于事物内部的矛盾。

\textbf{{和谐}}作为矛盾的一种特殊表现形式,体现着矛盾双方的相互依存、相互促进、共同发展。

矛盾的\textbf{{普遍性和特殊性}}的关系:矛盾普遍性和特殊性相互区别、相互联系,并在一定条件下相互转化。任何事物都是共性和个性的统一,共性寓于个性之中,个性与共性相联系而存在。

{事物的}\textbf{{主要矛盾和非主要矛盾}}{相互区别,相互作用,并在一定条件下相互转化。}\textbf{{矛盾的主要方面与非主要方面}}{相互区别、相互作用,并在一定条件下相互转化。}

\subsection{01468-肯定与否定}
事物内部都存在着\textbf{{肯定因素和否定因素}}。肯定因素是维持现存事物存在的因素,否定因素是促使现存事物灭亡的因素。

否定是事物的\textbf{{自我否定}},是事物内部矛盾运动的结果。\textbf{{否定是事物发展的环节}}。否定是新旧事物联系的环节,新事物孕育产生于旧事物,新旧事物是通过否定环节联系起来的。辩证否定的实质是{``}\textbf{{扬弃}{''}},即\textbf{{新事物对旧事物既批判又继承}},既克服其消极因素又保留其积极因素。

{事物的辩证发展经过第一次否定,使矛盾得到初步解决。而处于否定阶段的事物仍然具有片面性,还要经过再次否定,即}\textbf{{否定之否定}。}{从}\textbf{{内容上看}}{,是自己发展自己、自己完善自己的过程;从}\textbf{{形式上看}}{,是}{螺旋式上升或波浪式前进}{,方向是前进上升的,道路是迂回曲折的,是前进性和曲折性的统一。}

\subsection{01469-必然与偶然}
必然性和偶然性是揭示客观事物发生、发展和灭亡的不同趋势的范畴。

{{\textbf{必然性}}:指事物联系和发展过程中一定要发生、确定不移的趋势。}

\textbf{{偶然性}}:指事物联系和发展过程中并非确定发生的,可以出现,也可以不出现,可以这样出现,也可以那样出现的不确定的趋势。

{{关系}:必然性产生于事物内部的根本矛盾,偶然性产生于非根本矛盾和外部条件}
;必然性存在于偶然性之中,通过大量的偶然性表现出来,并为自己开辟道路;偶然性背后隐藏着必然性,受必然性的支配,偶然\textbf{}性是必然性的表现形式和补充。\textbf{{{必然是偶然的,偶然是必然的}{。}}}

\subsection{01471-现象与本质}
现象和本质是揭示客观事物的外部表现和内部联系相互关系的范畴。

{\textbf{现象}}:事物的外部联系和表面特征,人们可通过感官感知。

\textbf{{本质}}:是事物的内在联系和根本性质,只有靠人的理性思维才能把握。

{\textbf{区别}}:现象是个别的、具体的,本质是一般的、共同的;现象是多变的,本质是相对稳定的;现象是生动、丰富的,本质是比较深刻、单纯的。

\textbf{{联系}}:任何本质都是通过现象表现出来,任何现象都从一定的方面表现着本质。

\textbf{现象和本质的区别决定了认识的必要性,现象和本质的联系决定了认识的可能性。}

\subsection{01472-可能与现实}
可能性和现实性是揭示事物的过去、现在和将来的相互关系的范畴。

{可能性:指事物发展过程中潜在的东西,是\textbf{包含在事物中并预示事物发展前途的种种趋势}。}

{现实性:指已经\textbf{产生出来的有内在根据、合乎必然性的存在}。}

{把握事物的可能性,要\textbf{{注意区分}}可能性和不可能性、现实的可能性和抽象的可能性、好的可能性和坏的可能性。}

\subsection{01474-规律及其客观性}
{事物的联系和运动构成}\textbf{{规律}}{,}规律是事物的内部联系和发展的必然趋势{。规律这一范畴揭示的是事物运动发展中的}\textbf{本质的、必然的、稳定的}{联系。}\textbf{客观性是规律的根本特点}{,它的存在不依赖于人的意识。相反,人的意识及其指导下的实践却要受规律的支配。}

\subsection{01475-自然规律和社会规律}
\textbf{自然规律和社会规律都是规律,都具有客观性}{。两者之间的区别主要表现在:}\textbf{自然规律是作为一种盲目的无意识的力量起作用,社会规律则是通过抱有一定目的和意图的人的有意识的活动实现的}{。}

{}

\subsection{01476-客观规律性和主观能动性}
尊重客观规律是发挥主观能动性的前提。在尊重客观规律的基础上充分发挥主观能动性。

\textbf{{举例}}{{:}登高而招,臂非加长也,而见者远;顺风而呼,声非加疾也,而闻者彰。假舆马者,非利足也,而致千里;假舟楫者,非能水也,而绝江河。------荀子}

\subsection{01478-可知论的两个对立}
{\textbf{唯物主义认识论和唯心主义认识论}}的对立:\\
1. 前提或基础即世界观不同,前者是唯物主义,后者是唯心主义。\\
2. 核心或本质不同,前者是反映论。\\
3.
认识路线(顺序)不同,前者是从物到感觉和思想,后是从思想和感觉到物。\\[2\baselineskip]{\textbf{能动反映论}}(辩证的)较\textbf{{直观反映论}}(形而上学性的)的优势:\\
1.
彻底的科学的反映论,克服了形而上学唯物主义反映论的狭隘性、机械性和被动直观性。\\
2. 把科学的实践观引入认识论,把实践作为全部认识论的基础。\\
3.
把辩证法贯彻于反映论,科学地说明了人认识发展的辩证过程,揭示了认识运动的基本规律,克服了形而上学唯物主义把认识直观化、凝固化、片面化的形而上学的缺陷。\\[2\baselineskip]

\subsection{01479-实践及其和认识的关系}
\textbf{{认识的本质}}:以实践为基础的主体对客体的能动反映

实践决定认识或实践是认识的\textbf{{基础}}:实践是认识的\textbf{{来源}},实践是认识发展的\textbf{{动力}},实践是检验认识真理性的唯一\textbf{{标准}},实践是认识的最终\textbf{{目的}}。\textbf{{认识反作用于实践}}:对实践有指导作用。

\textbf{{提示}}{:实践和认识的关系注意类比摸着石头过河和顶层设计的关系。}

\subsection{01481-认识发展过程的两次飞跃}
{\textbf{感性认识是理性认识的基础,感性认识有待发展到理性认识}},反对割裂二者关系的教条主义和经验主义。\\[2\baselineskip]\textbf{从感性认识到理性认识的第一次飞跃}。\\
理由:认识的真正任务是要达到理性认识才能有效的指导实践。\\
条件:要有丰富而真实的感性材料,并对感性材料进行辩证思维的加工。\\[2\baselineskip]\textbf{从理性认识到实践的第二次飞跃}。\\
理由:认识的需要和要求,实践的需要和要求。\\
条件:主要是要坚持从实际出发,坚持一般理论和具体实践相结合的原则。\\[2\baselineskip]认识过程的\textbf{{反复性和无限性}}:实践、认识、再实践、再认识,无限循环,由低级阶段向高级阶段不断推移的永无止境的前进运动。\\[2\baselineskip]{思想路线是马克思主义认识论在实际工作中的具体运用},是体现在实际工作中的认识路线。党的思想路线,亦称党的认识路线:一切从实际出发,理论联系实际,实事求是,
在实践中检验真理和发展真理。\\[2\baselineskip]

\subsection{01482-认识过程的理性因素和非理性因素}
\textbf{{}}

认识过程中的理性因素(指导,解释,借鉴)和非理性因素(激活,驱动,控制)

\textbf{{理性因素}}是指人的理性直观、理性思维等能力,它在认识活动中的作用主要有:指导作用、解释作用、预见作用。\\[2\baselineskip]{\textbf{非理性因素}}是指人的情感、意志,包括动机、欲望、信念、信仰、习惯、本能等,以非逻辑形式出现的幻想、想象、直觉、灵感等也属非理性因素。非理性因素具有动力作用,诱导作用和激发作用。\\

\subsection{01483-真理及其客观性(一元性)}
{真理是指人类对客观事物及其规律的正确认识}。真理的根本属性是客观性,是指真理的内容是客观的,真理具有真实的客观内容;检验认识真理性的标准(实践)也是客观的。

\textbf{{实践是检验真理的唯一标准。}}

{实践标准的确定性与不确实性:}{确定性在于实践是检验真理的唯一标准。不确定性在于一定历史阶段上的具体实践具有局限性}{,它往往不能充分证明或驳倒某一认识的真理性;实践检验真理是一个过程,不是一次完成的;已被实践检验过的真理还要继续经受实践的检验。}

{}

\textbf{真理的客观性(一元性)}:一是指真理的内容是客观的;二是指真理的标准是客观的。

~真理的绝对性:真理的绝对性是指真理的内容表明了主客观统一的确定性和发展的无限性,它有两个方面的含义,一是任何真理都必然包含同客观对象相符合的客观内容,都同谬误有原则的界限,这一点是无条件的,绝对的;二是人类认识按其本性来说能够正确认识无限发展着的物质世界,认识每前进一步,都是对无限发展着的世界的接近,这一点也是无条件的,绝对的。因此承认世界的可知性,承认人能获得关于无限发展着的世界的正确认识,也就是承认了真理的绝对性。



\subsection{01484-真理的绝对性与相对性}
{同一真理都是真理的绝对性和相对性的统一}。

\textbf{{真理的绝对性(绝对真理)}}:一是就真理的客观性而言,任何真理都是对客观事物及其规律的正确认识,这是无条件的、绝对的。二是就人类认识的本性来说,完全可以正确认识无限发展的物质世界,这也是无条件的、绝对的。三是从真理的发展来说,无数相对真理的总和构成绝对真理。

\textbf{{真理的相对性(相对真理)}}{:真理的相对性或有三层含义:一是从}\textbf{{广度}}{上说有待于扩展;二是从}\textbf{{深度}}{上说有待于深化;三是从}\textbf{{进程}}{上说有待发展。}

{}

真理的相对性:指人们在一定条件下对事物及其发展规律的正确认识总是有限度的。相对性有两方面含义,一是真理所反映的对象是有条件的,有限的;二是真理反映客观对象的正确程度也是有条件的,有限的。任何真理都只能是主观对客观事物近似正确即相对正确的反映。

真理是由相对不断走向绝对的永无止境的发展过程。这是真理发展的一个规律。\\

\textbf{{\\
}}

\textbf{{真理的绝对性和相对性是辩证统一的,其一,二者相互依存。}}所谓相互依存是说人们对于客观事物及其本质和规律的每一个正确认识,都是在一定范围内一定程度上一定条件下的认识,因而必然是相对的和有局限性的,但是在这一范围内,一定程度上一定条件下,它又是对客观对象的正确反应,因而它又是无条件的绝对的。其二,二者相互包含,所谓相互包含,一是说真理的绝对性寓于相对性之中,二是说真理的相对性必然包含并表现着真理的绝对性,所以绝对真理和相对真理是不可分的,没有离开绝对真理的相对真理,也没有离开相对真理的绝对真理。

~

在二者的辩证关系中还要明确,{\textbf{真理永远处在由相对向绝对的转化和发展中,是从真理的相对性走向绝对性}},接近绝对性的过程,{\textbf{任何真理性的认识都是由真理的相对性向绝对性转化过程中的一个环节}},这是真理发展的规律。

既要反对绝对主义,又要反对相对主义。

\subsection{01485-真理与谬误}
{真理的}\textbf{{具体性}}{是指任何真理都是在特定条件、范围的限定下主体的认识同特定对象的一致或者符合。}如果超出这些限定``只要再多走一小步,真理就会变成错误''。

{真理是有}\textbf{{条件的}}{,真理是}\textbf{{历史的}}{,}\textbf{{具体的}}{,而不是抽象的。同一个真理不能因人而异。}

\textbf{真理和谬误}{是性质不同的两种认识,它们是对立的。但是,真理和谬误又是统一的,它们相互依存、相互转化。}\textbf{{真理和谬误在同一条件和范围下对立,超出这个范围就可以相互转化。}}

{在人们的认识和实践活动中,正确的认识往往会导致成功的实践,而由于主客观条件的限制,人们的实践活动也会达不到自身所期待的结果,导致失败。}\textbf{{错误往往是正确的先导}}{,失败常常是成功之母。要人们分析失败的原因,化不利条件为有利因素,就能从失败中吸取教训,变失败为成功。}

{}

实践是检验真理的唯一标准。


\subsection{01487-唯物史观和唯心史观的对立}
\textbf{{历史观}}是人们在认识社会历史现象、解决社会问题时所采取的根本观点。

\textbf{{社会存在与社会意识的关系问题,是社会历史观的基本问题}}。

在对待社会历史发展及其规律问题上,存在着两种根本对立的历史观:一种是\textbf{{唯物史观}},另一种是\textbf{{唯心史观}}。

在马克思主义产生之前,唯心史观一直占据统治地位,它的主要缺陷是:一是至多考察人们活动的思想动机,而\textbf{{没有进一步追究思想动机背后的物质动因}};二是只看到个人在历史上的作用,而\textbf{{忽视人民群众创造历史的决定作用}}。

马克思正确地解决了社会存在与社会意识的关系问题,发现了人类社会发展的客观规律,创立了唯物史观。

\textbf{{唯物史观认为}{:社会历史发展是有规律的而不是无序的,社会存在决定社会意识,人民群众是历史的创造者。}}

\subsection{01488-社会存在和社会意识及其关系}
\textbf{{社会存在}:}社会存在属于社会生活的物质方面,是社会实践和物质生活条件的总和,包括\textbf{{物质生活资料的生产}}以及\textbf{{生产方式}}、\textbf{{地理环境}}和\textbf{{人口因素}}。

{\textbf{社会意识}}\textbf{:}社会生活的精神方面,它既包括社会意识的各种形式,也包括社会心理与自发形成的风俗、习惯。\textbf{{属于上层建筑的社会意识形式称为社会意识形态}},主要包括政治法律思想、道德、艺术、宗教、哲学等。

社会意识具有\textbf{{相对独立性}}主要表现在:社会意识和社会存在发展的\textbf{{不平衡性}};社会意识的\textbf{{历史继承性}};社会意识内部各种形式的\textbf{{相互作用性}};\textbf{{社会意识对社会存在的反作用性}}。

{社会存在决定社会意识,社会意识是社会存在的反映,并反作用于社会存在。}

\subsection{01489-生产方式}
\textbf{{生产方式}}{就是劳动者和劳动资料结合的特殊方式,}\textbf{{是生产力和生产关系}}{的统一,它集中地}\textbf{{体现了人类社会的物质性}}{。}\textbf{{生产方式是社会历史发展的决定力量}}{。}

\subsection{01491-生产力和生产关系}
\textbf{{生产力}:}人们解决社会同自然矛盾的实际能力,是人类改造自然使其适应社会需要的物质力量。在哲学上,生产力是标志人类改造自然的实际程度和实际能力的范畴,它表示\textbf{{人和自然的关系}}。\textbf{{一是劳动资料即劳动手段}}。\textbf{{二是劳动对象}}。\textbf{{三是劳动者}}。生产力中还包含着科学技术。\textbf{{科学技术是先进生产力的集中体现和主要标志,是第一生产力}}。{生产力包括生产力水平,生产力的性质和生产力状况。}\textbf{{生产力的水平}}是生产力的\textbf{{量的规定性}}。\textbf{{生产力的性质}}是生产力的\textbf{{质的规定性}}。\textbf{{生产力状况是生产力的水平和生产力的性质的统一}}。

\textbf{{生产关系}}:人们在物质生产过程中形成的不以人的意志为转移的经济关系。\textbf{{生产关系是社会关系中最基本的关系}}。狭义的生产关系包括生产资料所有制的关系、生产中人与人的关系和产品分配关系。广义的生产关系包括\textbf{{生产}}、\textbf{{分配}}、\textbf{{交换}}和消费中的关系。在生产关系中,\textbf{{生产资料所有制的关系是最基本的}}。

\textbf{{生产力决定生产关系,生产关系反作用于生产力}}{。生产力和生产关系、经济基础和上层建筑的矛盾是}\textbf{{社会基本矛盾}}{。生产力和生产关系的矛盾是}\textbf{{更为根本的矛盾}}{。社会基本矛盾作为社会发展的}\textbf{{根本动力}}{,生产力是社会基本矛盾运动中}\textbf{{{最基本的动力因素}}}{{},是人类社会发展和进步的}\textbf{{{最终决定力量}}}{。生产力是社会发展的}\textbf{{根本内容}}{,是实现社会发展多重目标的}\textbf{{{根本条件}}}{,是社会发展的}\textbf{{{集中体现和客观标志}}}{,}\textbf{{{是衡量社会进步的根本尺度}}}{。}

\subsection{01492-经济基础与上层建筑}
\textbf{{经济基础}:}指由社会一定发展阶段的生产力所决定的生产关系的总和。

\textbf{{上层建筑}:}指建立在一定经济基础之上的意识形态以及相应的制度、组织和设施。

上层建筑由意识形态和政治法律制度及设施、政治组织等两部分构成。\textbf{{意识形态又称观念上层建筑}},包括政治{法律思想、道德、艺术、宗教、哲学}等思想观点,政治法律制度及设施和政治组织又称政治上层建筑,包括:{国家政治制度、立法司法制度和行政制度;国家政权机构、政党、军队、警察、法庭、监狱等政治组织形态和设施}。

政治上层建筑是在一定意识形态指导下建立起来的,是统治阶级意志的体现;政治上层建筑一旦形成,就成为一种现实的力量,影响并制约着人们的思想理论观点。\textbf{{在整个上层建筑中,政治上层建筑居主导地位,国家政权是它的核心。}}

\textbf{{经济基础决定上层建筑,上层建筑反作用于经济基础}}{。}

\subsection{01493-其他动力}
{\textbf{阶级斗争}}是阶级社会发展的\textbf{{直接动力}}(重要动力)

{\textbf{革命}}的根本问题是国家政权问题,\textbf{{重要动力}}。

{\textbf{改革}}是推动社会发展的又一\textbf{{重要动力}}。

{\textbf{科学技术革命}}是社会动力体系中的一种重要动力:

\textbf{首先},对生产方式产生了深刻影响。一是\textbf{{改变了社会生产力的构成要素}},二是\textbf{改变了人们的劳动形式},三是{\textbf{改变了社会经济结构}},特别是导致了产业结构发生变革。

\textbf{其次},{\textbf{对生活方式产生了巨大影响}}。现代科学技术革命直接或间接地作用于人们生活方式的四个基本要素,即生活主体、生活资料、生活时间和生活空间,从而引起生活方式发生新的变革。

\textbf{最后},{\textbf{促进了思维方式的变革}}。

\subsection{01494-社会形态及其发展}
\textbf{{}}

\textbf{{社会形态}}{:关于社会运动的具体形式、发展阶段和不同质态的范畴,}{是同生产力发展一定阶段相适应的}\textbf{{经济基础与上层建筑的统一体}}{。社会形态包括社会的}{经济形态、政治形态和意识形态}{。}

\subsection{01495-社会形态更替的特点}
{社会形态更替归根结底是社会基本矛盾运动的结果};生产力的发展具有\textbf{{最终的决定意义}}。

第一,社会发展的客观必然性造成了一定阶级阶段社会发展的基本趋势,为人们的历史选择提供了{基础、范围和可能性空间}。第二,社会形态更替的过程也是一个{和目的性与合规律性相统一}的过程。第三,人们的历史选择性,归根到底是人民群众的选择性

社会形态更替具有\textbf{必然性}与人们的历史\textbf{选择性}:一个民族之所以做出这种或那种选择,有其特定的原因:{一是取决于民族利益。二是取决于交往。三是取决于对历史必然性以及本民族特点的把握程度。}人们的历史选择性,归根结底是人民群众的选择性。

社会形态更替具有{统一性和多样性};

{社会形态更替的}{前进性与曲折性、}{顺序性与跨越性的统一}{。}

\subsection{01496-国家的起源和实质}
\textbf{{政治上层建筑居上层建筑的主导地位}},国家政权是其\textbf{{核心}}。

{国家是阶级矛盾不可调和的产物。国家的实质是一个阶级统治另一个阶级的工具。国家是为自己的经济基础服务的。}

\subsection{01498-人民群众与英雄人物的关系}
\textbf{{人民群众}}:人民群众是一个历史范畴,是指一切对社会历史起推动作用的人。在不同的历史时期,人民群众有着不同的作用,{主体部分始终是劳动群众及其知识分子}。

{人民群众的作用:在社会历史发展过程中,人民群众起着}\textbf{{决定性作用}}{,人民群众是}\textbf{{历史的主体}}{,是历史的}\textbf{{创造者}}{。主要表现:人民群众是社会}\textbf{{物质财富的创造者}}{,人民群众是}\textbf{{精神财富的创造者}}{,}\textbf{{人民群众是人类社会变革的决定力量}}{。}

{我党的}\textbf{{群众路线}}{:坚持一切为了群众,一切依靠群众,从群众中来,到群众中去的路线。}

{历史人物,特别是杰出人物在社会发展过程中起着特殊的作用,主要表现在历史人物是历史事件的发起者、当事者,历史人物是实现一定历史任务的组织者、领导者;历史人物是历史进程的影响者,它可以加速或延缓历史任务的解决。
历史人物对历史发展的具体过程始终起着一定的作用,有时甚至对历史事件的进程和结局发生决定性的影响,但}\textbf{{不能决定历史发展的基本趋势}}{。}

{所以,社会历史发展是}\textbf{{无数个人合力作用的结果}}{。历史人物在历史发展中起着特殊的作用。历史人物的产生是}\textbf{{必然性和偶然性的统一}}{。坚持时势造英雄的观点,既要承认杰出人物的历史作用,又要反对个人崇拜。}

\subsection{01501-劳动二重性}
\textbf{{具体劳动}}是指\textbf{{生产一定使用价值的具体形式的劳动}},即\textbf{{有用劳动}}。具体劳动形成\textbf{{商品的使用价值}}。

\textbf{{抽象劳动}}是指撇开一切具体形式的、\textbf{{无差别的一般人类劳动}},即\textbf{{人的体力和脑力的消耗}}。抽象劳动形成商品的\textbf{{价值}}实体。

具体劳动和抽象劳动的关系,是\textbf{{同一劳动过程的两个方面}},在时间和空间上是统一的。

\textbf{{劳动二重性决定商品二因素}}{。即}\textbf{{具体劳动创造使用价值,抽象劳动形成价值}}{。}

{劳动二重性学说是马克思的重大贡献,是理解马克思政治经济学的}\textbf{{枢纽}}{。}

\subsection{01502-商品二因素}
\textbf{{商品是用来交换的能满足人们某种需要的劳动产品}},具有\textbf{{使用价值和价值}}两个因素,是\textbf{{使用价值和价值的矛盾统一体}}。

\textbf{{使用价值}}{是指}\textbf{{商品能满足人们某种需要的属性}}{,即}\textbf{{商品的有用性}}{,反映}\textbf{{人与自然之间的物质关系}}{,是商品的}\textbf{{自然属性}}{,}\textbf{{是一切劳动产品共有的属性}}{。}\textbf{{使用价值构成社会财富的物质内容}}{。是商品价值和交换价值的}\textbf{{物质承担者}}{。}

\textbf{{价值}}是凝结在商品中的\textbf{{无差别的一般人类劳动}},即\textbf{{人类脑力和体力的耗费}}。价值是商品特有的\textbf{{社会属性}}。商品的价值实体是凝结在商品中的无差别的人类劳动,它\textbf{{本质上体现生产者之间的一定社会关系}}。\textbf{{价值是交换价值的基础和内容。}}

\textbf{{交换价值}}{:是一种使用价值和另一种使用价值相交换的量的关系或比例。}\textbf{{交换价值是价值的表现形式}}{。}

\textbf{{任何商品都是使用价值和价值的统一体}}{。而同时,}\textbf{{对买卖双方当事人来说,使用价值和价值不能兼得}}{。}{使用价值和价值的这种矛盾对立,只有通过交换才能得到解决}{。}

\subsection{01503-货币理论}
在商品经济发展过程中,价值形式发展经历了\textbf{四个阶段}:\textbf{简单价值形式、扩大价值形式、一般价值形式和货币形式}。货币的产生使整个\textbf{商品世界分化为两极}:一极是各种各样的\textbf{具体商品},它们分别代表不同的\textbf{使用价值}:一极是\textbf{货币},它们只代表\textbf{商品的价值}。\textbf{商品内在的使用价值和价值的矛盾就发展成为外在的商品和货币的矛盾}。

{货币的职能:货币的职能是它的本质的具体体现,在商品经济中货币具有五个职能:}\textbf{价值尺度、流通手段、贮藏手段、支付手段和世界货币}{。其中,价值尺度和流通手段是最基本的职能。}

\subsection{01504-价值与价格关系}
\textbf{{价值规律}}的表现形式是\textbf{商品的价格围绕价值自发波动}。\textbf{价格是商品价值的货币表现}。由于\textbf{供求关系}变动的影响,商品的\textbf{价格不停地围绕价值这个中心上下波动}。从较长时间来看,\textbf{商品的平均价格和价值是一致的}。

价值规律是商品经济的基本规律,其基本内容和要求是:商品的价值量由社会必要劳动时间决定,商品交换按照等价交换原则来进行。

价值规律在垄断条件下依然成立。

\subsection{01507-劳动力转化为商品和货币转化为资本}
劳动力是指人的劳动能力,是人的体力和脑力的总和。劳动力的使用即劳动。在资本主义条件下,\textbf{资本家购买的是雇佣工人的劳动力而不是劳动}。

{劳动力成为商品需具备}\textbf{{两个基本条件}}{:第一,劳动者有人身自由;第二,劳动者没有生产资料,必须出卖劳动力。}

{劳动力商品的价值由}\textbf{维持劳动者生存和延续者后代所需要的生活资料的价值以及劳动者接受教育和训练费用}等三部分{构成。劳动力商品的价值决定还有一个特别,即包括历史和道德因素。}劳动力商品的使用价值是劳动,其特点是\textbf{劳动是价值是剩余价值的源泉}{。}

\textbf{资本是可以带来剩余价值的价值}{。但}资本的本质不是物,而是\textbf{一定的历史社会形态下}的生产关系{。}

\textbf{{劳动力转化为商品}是货币转化为资本的前提条件}{。}

\subsection{01508-资本主义所有制和工资的本质}
资本家凭借对生产资料的占有,在等价交换原则的掩盖下雇佣工人从事劳动,占有雇佣工人的剩余价值,这就是\textbf{{资本主义所有制的实质}}。

在资本主义社会,工人在市场上卖给资本家的是劳动力,而不是劳动。

在资本主义制度下,\textbf{工资的本质是劳动力的价值或价格}。但是,在资本主义经济的现象中,工资却表现为劳动的价格,因此,资本主义工资掩盖了剥削。

资本主义工资包括计时工资和计件工资两种主要形式。

只要资本和雇佣劳动的基本经济关系不变,资本主义工资的本质就不会发生根本变化{。}

\subsection{01509-资本理论}
预付资本在生产中以生产资料和劳动力这两部分存在。根据这两部分资本在生产剩余价值中的作用不同,把它们区别为\textbf{不变资本和可变资本}。

\textbf{不变资本}:以生产资料形式存在的资本,在生产过程中只是把价值转移到新产品中去,并不改变自己的价值量。

\textbf{可变资本}:以劳动力形式存在的资本,它的价值在生产过程中不会转移到新产品中,而要由工人劳动再创造出来,而且还要\textbf{生产出剩余价值},改变了原来的价值量。

{区分不变资本和可变资本的意义:第一,提示了}\textbf{剩余价值的源泉}{和}\textbf{资本主义剥削的实质}{。第二,}\textbf{为确定资本家对工人的剥削程度}{,提供了科学依据。体现资本家对工人剥削程度的概念是}\textbf{剩余价值率}{。}

\subsection{01511-剩余价值}
\textbf{{剩余价值}}是\textbf{雇佣工人所创造的并被资本家无偿占有的超过劳动力价值的那部分价值},它是\textbf{雇佣工人剩余劳动的凝结},体现了资本家与雇佣工人之间\textbf{剥削与被剥削}的关系。

资本主义生产的\textbf{直接目的和决定性动机},就是无休止地采取各种方法获取\textbf{尽可能多的剩余价值}。这样种不以人的意志为转移的客观必然性,就是剩余价值规律,雇佣工人的劳动分为两部分:一部分是必要劳动,用于再生产劳动力的价值;另一部分是剩余劳动,用于无偿地为资本家生产剩余价值。

\textbf{{绝对剩余价值}:}绝对剩余价值生产的前提是必要劳动时间不变;增加劳动时间也就是\textbf{增加剩余劳动时间};由提高劳动强度而生产的剩余价值,也是绝对剩余价值。

\textbf{{超额剩余价值{:}}}{是}\textbf{商品的个别价值低于社会价值的差额}{;超额剩余价值对个别资本家来说,是一种暂时现象,但从整个社会来看,却是经常存在的。}\textbf{个别资本家提高劳动生产率的直接目的是追求超额剩余价值}{。}

\textbf{\textbf{{相对剩余价值}}}{\textbf{:}相对剩余价值生产的前提是工作日长度不增加;相对剩余价值生产是}\textbf{以社会劳动生产率提高为条件}{的;社会劳动生产率提高,进而资本家普遍}\textbf{获得相对剩余价值是通过各个资本家率先提高劳动生产率追求超额剩余价值实现的}{。}

\subsection{01513-扩大再生产与资本积累}
\textbf{资本主义再生产的实质是物质资料再生产和资本主义生产关系再生产的统一},资本主义再生产包括\textbf{简单再生产和扩大再生产}两种类型。其中,\textbf{扩大再生产是资本主义再生产的特征}。

社会再生产的核心问题是社会总产品的实现问题。社会总产品在实物形态上是生产资料和消费资料;在价值形态上是社会总价值。社会再生产顺利的基本条件是:总量平衡,结构合理,顺利完成价值补偿和实物补偿。

\textbf{把剩余价值转化为资本,或者说剩余价值的资本化,就是资本积累}{。资本积累的}\textbf{本质}{,}就是资本家不断利用无偿占有的工人创造的剩余价值,来扩大自己的资本规模,进一步扩大和加强对工人的剥削和统治。随着资本积累,必然加剧社会的两极分化。{资本积累不但是社会财富占有两极分化的重要原因,而且是}\textbf{资本主义社会失业现象产生的根源}{。}

\subsection{01514-资本的有机构成}
\textbf{{技术构成}:}由生产的技术水平所决定的生产资料和劳动力之间的比例。

\textbf{{价值构成}:}不变资本和可变资本之间的比例。

\textbf{{有机构成}:}由资本技术构成决定并反映技术构成变化的资本价值构成。

{在资本积累过程中,伴随技术不断进步,资本有机构成有不断提高的趋势。}

\subsection{01516-资本主义社会的基本矛盾与经济危机}
\textbf{{生产资料资本主义私人占有和生产社会化之间的矛盾,是资本主义的基本矛盾。}}

资本主义经济危机是生产过剩的经济危机。生产过剩是资本主义经济危机的主要表现和本质特征。

{资本主义基本矛盾是经济危机的}\textbf{根源}{。经济危机是资本主义基本矛盾激化的产物。}

{资本主义经济危机具有}\textbf{{周期性}}{:}危机、萧条、复苏、高涨{。}

\subsection{01517-资本的循环与周转}
资本循环的三个阶段和三种职能形式:三个阶段是购买阶段、生产阶段和销售阶段。三种职能形式是\textbf{货币资本、生产资本和商品资本}。资本的三种职能形式在空间上并存,在时间上继起。

资本是在运动中增殖的,资本周而复始、不断反复的循环,就叫\textbf{资本的周转}。资本周转越快,在定时期内带来的剩余价值就越多。影响资本周转快慢的因素有很多,关键的因素有两个:一是资本周转时间,二是生产资本中固定资本和流动资本的构成。

\textbf{{注意}}{:资本周转速度影响利润但不影响利润率。}

\subsection{01518-平均利润与生产价格}
{资本主义生产是为了获得利润,因此,不同部门之间如果利润率不同,资本家之间就会展开激烈的竞争,使资本从利润率低的部门转向利润率高的部门,从而导致}利润率趋于平均化{,按照平均利润率来计算和获得的利润,叫做}\textbf{平均利润}{。}\textbf{随着利润转化为平均利润,商品价值就转化为生产价格}{,即}\textbf{商品的成本价格加平均利润}{。}

\subsection{01519-资本主义产生的产生与原始积累}
资本主义生产关系的产生和萌芽的途径:小商品经济的分化和由商人和高利贷转化。

\textbf{资本原始积累}{,就是生产者与生产资料相分离,货币资本迅速集中于少数人手中的历史过程。资本原始积累的途径有:}\textbf{以暴力手段剥夺农民的土地(基础);以暴力手段掠夺货币财富}{。}

\subsection{01520-资本主义的政治制度和意识形态}
资本主义国家的对内职能主要是政治统治职能,还具有社会公共管理职能。资本主义国家对外职能是对外进行国家交往与维护国家安全和利益。资本主义国家职能的\textbf{{本质}},从根本上说,\textbf{{是资产阶级进行政治统治的工具}}。

{资本主义政治制度:资本主义国家的政治经济是通过具体的政治制度实现的。资本主义政治制度包括资本主义的民主与法制、政权组织形式、选举制度和政党制度。}{资本主义政治制度的本质上是为资产阶级利益服务的,是服从于资产阶级统治和压迫需要的政治工具,不可避免地有其历史和阶级局限性}{。}

{资本主义意识形态是资产阶级在长期的反对封建专制主义和宗教神学的斗争逐步形成和发展起来的。}

{资本主义意识形态作为资本主义的观念上层建筑,是为资本主义经济基础服务的;资本主义意识形态是资产阶级的阶级意识的集中体现}。

\textbf{{注意}}{:资本主义的政治和意识形态都是虚伪的,为资产阶级统治服务的。}

\subsection{01521-垄断与竞争}
资本主义的发展经历{\textbf{自由竞争资本主义和垄断资本主义}}两个阶段。

垄断取代自由竞争在资本主义经济中占据统治地位,垄断资本主义的发展包括\textbf{{私人垄断资本主义和国家垄断资本主义}}两种形式。

{垄断形成的一般过程:自由竞争引起生产集中,生产集中发展到一定阶段就自然形成垄断。}{\textbf{{生产集中}}是指生产资料、劳动力和商品的生产日益集中于少数大企业的过程}{,其结果是大企业在社会生产中所占约比重不断增加。资本集中是指大资本吞并小资本,或由许多小资本合并而成大资本的过程,其结果是越来越多的资本为少数大资本所支配。}

垄断形成的原因:\\
1. 少数大企业容易达成联合起来,进行垄断的协议;\\
2. 大企业形成对竞争的限制,也会产生垄断;\\

3. 大企业为了避免竞争造成两败俱伤也会联合起来,进行垄断。

{\textbf{垄断没有消灭竞争}}:\\
1. 垄断没有消灭商品经济,因而就不可能消除竞争;\\
2. 垄断形成后,垄断组织之间和内部仍存在竞争;\\
3. 任何垄断组织不可能垄断一切。\\

{同自由竞争相比,垄断条件下的竞争,规模大、时间长、手段残酷、程度更加激烈,具有更大的破坏性。}

\subsection{01522-金融资本和垄断利润}
\textbf{金融资本}:工业垄断资本和银行垄断资本融合在一起的垄断资本,金融资本是垄断资本主义阶段占统治地位的资本形式。

金融寡头是指操纵国民经济命脉,并在实际上控制国家政权的少数垄断资本家或垄断资本家集团。{金融寡头在经济中的统治,主要是通过}{"}{参与制}{"}{来实现的}{;}{在政治上对国家机器的控制主要是通过同政府的}{"}{个人联合}{"}{来实现的。金融寡头还通过建立政策咨询机构,掌握新闻科教文化等来左右和影响内政外交与社会生活}。

\textbf{{垄断利润}}{:垄断组织进行垄断的目的或社会化大资本的实质是取得垄断利润。}

\textbf{{垄断价格}}{:保证垄断利润的主要手段是规定垄断价格,}{垄断价格}{=}{成本价格}{+平均利益+}{垄断利润}{。它包括垄断高价和垄断低价两种形式。}

\textbf{{垄断价格的产生并没有否定价值规律}}{,它是价值规律在垄断资本主义阶段作用的具体表现。}

\subsection{01523-垄断资本从国家走向全球}
国家垄断资本主义是国家政权和私人垄断资本融合在一起的垄断资本主义。国家垄断资本主义产生和发展是{\textbf{资本主义基本矛盾}}发展的结果。

国家垄断资本主义并\textbf{{{没有}根本改变垄断主义的性质}};没有从根本上{\textbf{消除资本主义的基本矛盾}}{\textbf{;{其实质}是私人垄断资本主义利用国家机器为其自身服务。}}{\textbf{国家垄断资本主义具有根本的局限性}}{\textbf{。}}

{垄断资本在世界范围的扩张的基本形式:}{{\textbf{借贷资本输出;生产资本输出;商品资本输出}}}{。}

{垄}{断资本主义的}\textbf{{基本经济特征}}{:垄断组织在经济生活中起决定作用;在金融资本的基础上形成金融寡头的统治;资本输出有了特别重要的意义;瓜分世界的资本家国际垄断同盟已经形成;最大资本主义列强已把世界上的领土分割完毕。}

\subsection{01526-当代资本主义变化的表现}
{当代资本主义经济政治的新变化:}\textbf{生产资料所有制的变化。劳资关系和分配关系的变化。社会阶层、阶级结构的变化。经济调节机制和经济危机形态的变化。政治制度的变化}。

\subsection{01527-当代资本主义变化的原因和实质}
当代资本主义变化的原因\\
1. 科学技术革命和生产力的发展;\\
2. 工人阶级的斗争;\\
3. 社会主义优越性的影响;\\
4.
改良主义的政党对资本主义制度的改革。\\[2\baselineskip]当代资本主义变化的实质\\
1. 从根本上说是人类社会发展一般规律和资本主义经济规律作用的结果;\\
{2. {\textbf{资本主义生产关系的根本性质没有变化}}。 }\\[2\baselineskip]

\subsection{01528-资本主义的历史地位}
{与封建社会相比,资本主义显示了巨大的历史进步性。然而,资本主义的历史进步性并不能掩盖其自身的局限性。其表现是:第一,资本主义基本矛盾阻碍社会生产力的发展。第二,资本主义制度下财富占有两极分化,引起经济危机。第三,资本家阶级支配和控制资本主义经济和政治的发展和运行,不断激化社会矛盾和冲突。}\textbf{资本主义必将被社会主义取代}{。}

\subsection{01529-空想社会主义}
{空想社会主义是科学社会主义的}\textbf{{{理论先驱}}}{。19世纪初,以{{\textbf{圣西门、傅立叶、欧文}}}为代表的空想社会主义是科学社会主义的}\textbf{{{直接思想来源}}}{。}

空想社会主义\textbf{{对资本主义制度进行批判}};对{\textbf{社会主义制度进行了天才描绘}}。

{空想社会主义的局限性:认为资本主义必然灭亡,却}\textbf{{不能揭示其必然灭亡的经济根源}}{;要求埋葬资本主义,却}\textbf{{找不到埋葬资本主义的社会力量}}{;主张建立社会主义理想社会,却}\textbf{{找不到通往理想社会的现实道路}}{。}

\subsection{01530-科学社会主义}
社会主义从空想到科学的\textbf{{两}{大发现}}:唯物史观和剩余价值学说。

{科学社会主义问世的}\textbf{{标示}}{:{\textbf{《共产党宣言》发表}}。}

{科学社会主义是全部马克思主义的}\textbf{{理论结论}}{。}

{科学社会主义不是靠想象而是通过合理的科学的立场和观点做的预见,其具体表现有:}\textbf{{在揭示人类发展一般规律的基础上指明社会发展的方向}}{;}\textbf{{在剖析资本主义社会中阐发新世界的特点}}{;}\textbf{{立足于社会的一般特征而不做细节的描述。}}

\subsection{01531-十月革命与列宁时期的探索}
社会主义从理论到实践的飞跃是通过无产阶级革命实现的,无产阶级革命有暴力与和平两种形式。其中,{\textbf{暴力革命}}\textbf{{是主要的基本形式}}。

社会主义革命{\textbf{首先在一个或几个国家取得胜利}}{:进入垄断资本主义阶段,}{\textbf{经济和政治发展不平衡是资本主义的绝对规律}}{,社会主义可能在少数甚至单独一个资本主义国家获得胜利。}

列宁领导的苏维埃俄国对社会主义道路的探索经历了{\textbf{三个时期}}:{{\textbf{第一个时期}}}{}进步{\textbf{巩固苏维埃政权时期}}。{{\textbf{第二个时期}}}{},外国武装干涉和国内战争时期,即{\textbf{战时共产主义时期}}。{{\textbf{第三个时期}}}{}由战时共产主义转变为\textbf{{新经济政策时期}}。

{列宁关于社会主义建设提出了许多{\textbf{精辟的论述}}:}首先,把{\textbf{建设社会主义作为一个长期探索、不断实践的{\textbf{过程}}}}。其次,把{\textbf{大力发展生产力、提高劳动生产率放在首要地位}}\textbf{。}再次,在多种经济成分并存的条件下,\textbf{{利用商品、货币和市场发展经济}}。最后,{\textbf{利用资本主义,建设社会主义}}{。}

\subsection{01534-社会主义发展的性质}
社会主义建设的\textbf{{艰巨性和长期性}}:生产力发展状况的制约;经济基础和上层建筑发展状况的制约;国际环境的严峻挑战;对社会主义发展道路的探索和对社会主义建设规律的认识,需要一个长期过程。

{社会主义发展道路的}\textbf{{多样性}}{:各国的生产力状况、历史文化传统不同以及时代和实践的不断发展决定了社会主义发展道路的多样性。}

{社会主义发展的}\textbf{{前进性和曲折性}}{:同任何事物的发展一样,社会主义的发展也会发生曲折,是前进性与曲折性的统一。}

\subsection{01535-无产阶级政党及其作用}
无产阶级专政和社会主义民主是科学社会主义的\textbf{{核心内容}}。

马克思主义政党是工人阶级的先锋队。

马克思主义政党代表着最广大人民的根本利益,以服务人民群众、为人民群众谋利益作为自己的根本宗旨。

马克思主义政党的组织原则是民主集中制。

马克思主义政党是社会主义革命的领导核心。

\subsection{01561-法律的运行}
法律的\textbf{运行:}法律的运行是一个从\textbf{创制、实施到实现}的过程。这个过程主要包括\textbf{法律制定(立法)、法律遵守(守法)、法律执行(执法)、法律适用(司法)}等环节。{}

法律\textbf{制定}:{\textbf{全国人民代表大会及其常务委员会}}行使国家立法权。大体包括以下四个环节:法律案的\textbf{提出},法律案的\textbf{审议},法律案的\textbf{表决},法律的\textbf{公布(国家主席)}。{}

法律\textbf{遵守}:依法办事包括两层含义:一是\textbf{依法享有并行使}权利,二是\textbf{依法承担并履行}义务。一切组织和个人都是守法的\textbf{主体}。{}

法律\textbf{执行}:在法律运行中,{\textbf{行政执法}}\textbf{是最大量、最经常的工作,是实现国家职能和法律价值的重要环节}。

法律\textbf{适用}{:在我国,}司法机关是指\textbf{国家检察机关(人民检察院)和审判机关(人民法院)}{。
}\textbf{{司法的基本}{要求}}{}{}{}{}是\textbf{正确、合法、合理、及时}{。}\textbf{{司法原则}}{主要有:}司法公正;公民在法律面前一律平等;以事实为依据,以法律为准绳;司法机关依法独立行使职权{。}

\subsection{01563-我国的国家制度}
\textbf{人民民主专政制度}:人民民主专政是我国的{\textbf{国体}}。国体即国家性质,是国家的阶级本质,是指社会各阶级在国家生活中的地位和作用。{}

\textbf{人民代表大会制度}:人民代表大会制度是中国社会主义民主政治最鲜明的特点,是人民当家作主的\textbf{重要途径和最高实现形式},是社会主义政治文明的重要制度载体,是我国的{\textbf{根本政治制度}}。人民代表大会制度是我国的{\textbf{政权组织形式}}。{}

\textbf{政党制度}:\textbf{中国共产党领导的多党合作和政治协商制度是}我国的一项基本政治制度,是中国特色社会主义政党制度。中国社会主义政党制度的{\textbf{特点}}是\textbf{共产党领导、多党派合作,共产党执政、多党派参政}。{}

\textbf{民族区域自治制度}:民族区域自治制度是我国为解决民族问题,处理民族关系,实现民族平等、民族团结、各民族共同繁荣发展而建立的基本政治制度。{}

\textbf{基层群众自治制度}:基层群众自治制度是城乡基层群众在党的领导下,依法直接行使民主权利,管理基层公共事务和公益事业,实行自我管理、自我服务、自我教育、自我监督的一项重要政治制度。基层群众自治是基层民主的\textbf{主要实现形式},是人民当家作主\textbf{最有效、最广泛}的途径。{}

\textbf{基本经济制度}:基本经济制度是指一国通过宪法和法律调整以生产资料所有制为核心的各种基本经济关系的规则、原则和政策的总和。{\textbf{社会主义公有制}}是我国经济制度的基础。\textbf{全民所有制和劳动群众集体所有制}是我国社会主义公有制{的两种基本形式。}

\subsection{01564-公民的基本权利和义务}
\textbf{我国公民基本权利有}:\textbf{平等权}、
\textbf{政治权利和自由}(包括两个方面:\textbf{选举权和被选举权;政治自由}),\textbf{宗教信仰自由}。\textbf{人身自由权}。\textbf{监督权和取得国家赔偿权}。\textbf{社会经济权}(包括:财产权,劳动权,休息权,物质帮助权)
\textbf{文化教育权}。 \textbf{特定主体权利}。

\textbf{我国公民的基本义务}{
:也称宪法义务,包括:维护国家统一和全国务民族团结。 遵守宪法和法律。
维护祖国的安全、荣誉和利益。 保卫祖国、依法服兵役和参加民兵组织。
依法纳税。其他义务。}

\subsection{01569-社会主义法治观念的基本内容}
\textbf{\textbf{}{坚持走中国特色社会主义}}

{党的领导是中国特色社会主义最本质的特征,是社会主义法治最根本的保证。}

{中国特色社会主义制度是中国特色社会主义法治体系的根本制度基础,是全面依法治国的根本制度保障。}

{中国特色社会主义法治理论是中国特色社会主义法治体系的理论指导和学理支撑,是全面依法治国的行动指南。}

\textbf{{坚持党的领导、人民当家作主与依法治国相统一}}

{党的领导是人民当家作主和依法治国}{的根本保证;}

{人民当家作主是党的领导和依法治国的本质要求;}

{依法治国是党领导人民当家作主的治国方略。}

\textbf{{坚持依法治国和以德治国相结合}}

{正确认识法治和德治的地位;}

{正确认识法治和德治的作用;}

{正确认识法治和德治的实现途径。}

\textbf{{加强宪法实施,落实依宪治国}}

{深刻认识宪法实施和依宪治国的重大意义;}

{全面实施宪法的基本要求:要在全社会树立宪法意识,弘扬宪法精神;要加强宪法实施;要加强你宪法实施,要坚持党的依宪执政,自觉在宪法法律范围内活动;}

{准确把握宪法实施的正确方向。}

\subsection{01570-法治思维的含义、特征}
\textbf{{\textbf{含义:}法治思维方式是指人们按照法治的理念、原则和标准判断、分析和处理问题的理性思维方式{。}\\
}}

{\textbf{{1、法律至上。}}}{}{法律至上尤其指宪法至上,因为宪法具有最高的法律效力。法律至上具体表现为法律的}{\textbf{{普遍适用性、优先适用性和不可违抗}}{性}}{。}{法律的普遍适用性}{,是}{指法律在本国主权范围内对所有人具有普遍的约束力。}{法律的优先适用}{性}{,}{是指同一项社会关系同时受到多种社会规范的调整时,要优先考虑法律规范的适用}{。}{法律的不可违抗性}{,}{是指法律必须遵守,违反法律要受到惩罚}{。}

{\textbf{{2、权力制约。}}}{}{是指国家机关的权力必须受到法律的规制和约束,也就是要把权力关进制度的笼子里。}

\textbf{{3、公平正义。}}{公平正义是指社会的政治利益、经济利益和其他利益在全体社会成员之间合理、平等分配和占有。公平正义包括:权利公平、机会公平、规则公平和救济公平。}

{\textbf{4、人权保障。}}{人权的法律保障包括:宪法保障、立法保障、行政保护和司法保障。}

\textbf{{5、正当程序。}}{程序的正当,表现在程序的合法性、中立性、参与性、公开性、时限性等方面。}

\subsection{01574-尊重法律权威的基本要求}
{1信仰法律}

{2遵守法律}

{3服从法律}

{4维护法律}

\subsection{01664-新文化运动的内容和意义}
五四以前的新文化运动是资产阶级民主主义的新文化反对封建主义的旧文化的斗争。\textbf{《新青年》杂志和北京大学成了新文化运动的主要阵地}。\textbf{新文化运动的{基本口号}是民主和科学}。新文化运动的{基本内容}:提倡民主和科学,反对专制和迷信盲从;提倡个性解放,反对封建礼教;提倡新文学,反对旧文学,实行文学革命。

\textbf{新文化运动是中国历史上一次前所未有的启蒙运动和空前深刻的思想解放运动}{。}

\subsection{01665-五四运动及其意义}
\textbf{五四运动是{新民主主义革命}的开端}:{1919}年{5}月爆发的五四运动,是中国近代史上的一个划时代的事件。五四运动的直接导火线,是巴黎和会上中国外交的失败。

{五四运动是在新的社会历史条件下发生的,它具有以辛亥革命为代表的旧民主主义革命所不具备的一些特点。主要是:1.
}\textbf{它表现了反帝反封建的彻底性}{。2.
}\textbf{是一次真正的群众运动}{。3.
}\textbf{五四运动促进了马克思主义在中国的传播及其与中国工人运动的结合}{。4.
}\textbf{五四运动发生在俄国十月革命之后,发生在无产阶级社会主义革命的新时代}{。}

\subsection{01666-十月革命的影响与新民主主义革命的开端}
毛泽东指出:{``}十月革命一声炮响,给中国送来了马克思列宁主义{''}。{}

在中国传播马克思主义的先驱者是李大钊。十月革命以后,他于{1918}年发表《法俄革命之比较观》、《庶民的胜利》、《{Bolshevism}的胜利》。

{五四运动具备了上述新的历史特点,成为了中国革命的新阶段即新民主主义革命阶段的开端。中国的民族民主革命,在十月革命后就属于世界无产阶级社会主义革命的一部分了。}

\subsection{01667-中国共产党的成立}
\textbf{中共一大确定党的名称为中国共产党}。党的纲领是:以无产阶级革命军队推翻资产阶级,采用无产阶级专政以达到阶级斗争的目的{------}消灭阶级,废除资本私有制,以及联合第三国际等。大会选举产生了由陈独秀、张国焘、李达组成的党的领导机构{------}中央局,以陈独秀为书记。{}

\textbf{特点:}源于马克思主义理论,源于工人运动。

\textbf{{意义:自从有了中国共产党,中国人民有了可以信赖的组织者和领导者,中国革命有了坚强的领导力量}}{。}

\subsection{01668-中国共产党的早期运动}
{研究和宣传马克思主义;到工人中去进行宣传和组织工作;建立社会主义青年团;进行有关建党问题的研究和讨论。}

\subsection{01669-国共合作的形成}
原因:建立革命统一战线是中国革命发展的客观需要。1.
中国处于帝国主义与北洋军阀的残酷统治之下,社会矛盾日益加深,人民生活更趋恶化。2.
京汉铁路罢工遭血腥镇压,中国共产党由此认识到要胜利,必须组织革命的统一战线。{\textbf{3.}
}\textbf{中共三大正式决定全体共产党员以个人名义加入国民党,同孙中山领导的国民党建立统一战线}{。}{4.
孙中山的转变。}\textbf{国民党一大的成功召开标志着第一次国共合作的正式形成}{。}

国民党一大通过的宣言对三民主义作出了新的解释{。这个新三民主义的政纲同中共在民主革命阶段的纲领基本一致,因而成为}\textbf{{国共合作的政治基础}}{。在}民族主义中突出了反帝的内容,强调对外实现中华民族的独立,同时主张国内各民族一律平等{;}在民权主义中强调了民主权利应``为一般平民所共有'',不应为``少数人所得而私''{;}把民生主义概括为``平均地权''和``节制资本''两大原则{(后来又提出了``}耕者有其田{''的主张)。大会实际上确定了}\textbf{{联俄、联共、扶助农工}}{三大革命政策。}

{国共合作的形成,极大地推动了国民革命的迅猛发展。主要表现在:黄埔军校的创建;广东革命根据地的统一;工农运动的发展;北伐战争的胜利进展。}

\subsection{01671-大革命的意义,失败原因和教训}
蒋介石在上海发动``四一二''政变,汪精卫在武汉发动``七一五''政变。这标志着国共合作全面破裂,大革命最终失败。

\textbf{{大革命的意义}}:1.{
}沉重打击了帝国主义在华势力,基本推翻了北洋军阀统治。中国共产党提出的反帝反封建的口号成为广大人民的共同呼声。2.{
}教育和锻炼了各革命阶级,党领导的工农大众经受了革命的洗礼,提高了政治觉悟,为后来共产党领导的土地革命的开展奠定了群众基础。3.{
}提高了中国共产党在全国人民中的政治威望,党在马克思列宁主义指导下,制定民主革命纲领,发挥了党的政治优势和组织优势。

\textbf{{失败的原因}}{:1.
从客观方面来讲,是由于}\textbf{反革命力量的强大}{,大大超过了革命的力量;资产阶级发生严重的动摇、统一战线出现剧烈的分化;蒋介石集团、汪精卫集团先后被帝国主义势力和地主阶级、买办资产阶级拉进反革命营垒里去了。2.
从主观方面来说,是由于}\textbf{陈独秀为代表的右倾机会主义}{的错误;当时的中国共产党还处于幼年时期;共产国际的错误干预。}

\textbf{{经验教训}}{:1. 必须建立广泛的革命统一战线。2.
无产阶级领导权的中心问题是农民问题。3.
中国革命的主要斗争形式是武装斗争。4.
必须不断加强共产党思想上、政治上和组织上的建设。}

\subsection{01672-国民党的统治和反抗国民党的斗争}
\textbf{国民党在全国统治的特点:}首先,国民党建立了庞大的军队。其次,国民党还建立了庞大的全国性特务系统。再次,为了控制人民,禁止革命活动,国民党还大力推行保甲制度。最后,为了控制舆论,剥夺人民的言论和出版自由,国民党还厉行文化专制主义。{}

\textbf{国民党在全国统治的性质:}维护帝国主义、封建主义、官僚资本主义的利益,巩固自身统治。{}

{1927}年{8}月{7}日,\textbf{中共中央在汉口秘密召开紧急会议}(即\textbf{{八七会议}}),\textbf{彻底清算了大革命后期陈独秀的右倾机会主义错误,确定了土地革命和武装反抗国民党的总方针,并选出了以瞿秋白为书记的中央临时政治局}。\textbf{毛泽东}在会上着重阐述了党必须依靠农民和掌握枪杆子的思想,强调党{``}\textbf{以后要非常注意军事,须知政权是由枪杆子中取得的}。{''}\textbf{八七会议开始了从大革命失败到土地革命战争兴起的{历史性转折}}。

\textbf{三大起义}{:南昌起义(8.1),广州起义,秋收起义中国革命由此发展到了一个新的阶段,即}土地革命战争{时期,或称}十年内战时期{。}

\subsection{01673-革命新道路的探索}
1.
背景:①大革命失败后,中共中央继续留在上海,党的工作重心仍然放在中心城市。②农村包围城市、武装夺取政权的理论,是对1927年革命失败后中国共产党领导的红军和根据地斗争经验的科学概括。它是在以毛泽东为主要代表的中国共产党人同当时党内盛行的把马克思主义教条化、把共产国际决议和苏联经验神圣化的错误倾向作坚决斗争的基础上逐步形成的。

{2.
毛泽东在开辟中国革命新道路过程中的杰出贡献。在实践上,他领导秋收起义,开始了创建井冈山农村革命根据地的斗争。在理论上,1928年10月和11月,毛泽东写了}《中国的红色政权为什么能够存在?》和《井冈山的斗争》{两篇文章,科学地阐述了共产党领导的}土地革命、武装斗争与根据地建设{这三者之间的辩证统一关系。1930年1月,毛泽东在}《星星之火,可以燎原》{一文中进一步}指出以乡村为中心的思想{。1930年5月,毛泽东在}《反对本本主义》{一文中,}阐明了坚持辩证唯物主义的思想路线即坚持理论与实际相结合的原则的极端重要性{。}

{3.
}\textbf{农村包围城市、武装夺取政权理论的提出,标志着中国化的马克思主义即毛泽东思想的初步形成}{。}

\subsection{01674-土地革命}
{毛泽东在}井冈山主持制定了中国共产党历史上第一个土地法{(}《井冈山土地法》{),以立法的形式,首次肯定了广大农民以革命的手段获得土地的权利。1929年4月,毛泽东在赣南发布}第二个土地法《兴国土地法》){,}\textbf{将``没收一切土地''改为``没收一切公共土地及地主阶级的土地''。这是一个原则性的改正,保护了中农的利益使之不受侵犯}{。毛泽东还和邓子恢等一起制定了土地革命中的阶级路线和土地分配方法:}\textbf{坚定地依靠贫农、雇农,联合中农,限制富农,保护中小工商业者,消灭地主阶级;以乡为单位,按人口平分土地,在原耕地的基础上,实行抽多补少、抽肥补瘦}{。}

\subsection{01675-“左”的错误的危害}
\textbf{主要错误}:1.
在革命性质和统一战线问题上,混淆民主革命与社会主义革命的界限。2.
在革命道路问题上,继续坚持以城市为中心。3.
在土地革命问题上,提出坚决打击富农和``地主不分田,富农分坏田''的主张。4.
在军事斗争问题上,实行进攻中的冒险主义、防御中的保守主义、退却中的逃跑主义。5.
在党内斗争和组织问题上,推行宗派主义和``残酷斗争,无情打击''的方针。

\textbf{形成原因}{:1.
八七会议以后党内一直存在着的浓厚的``左''倾情绪始终没有得到认真的清理。2.
共产国际对中国共产党内部事务的错误干预和瞎指挥。3.
全党不善于把马克思列宁主义与中国实际全面地、正确地结合起来。}

\textbf{危害}{:1.
其最大的恶果就是,使红军在第五次反``围剿''作战中遭到失败,不得不退出南方根据地实行长征。2.
这次错误使红军和根据地损失了90%。3.
国民党统治区党的力量几乎损失了100%。}

\subsection{01676-遵义会议与长征胜利}
{1935}年{1}月{15}日至{17}日,中共中央政治局在这里召开了扩大会议(史称{``}\textbf{遵义会议}{''})。集中全力解决了当时具有决定意义的\textbf{军事问题和组织问题}。{}

\textbf{遵义会议开始确立了以毛泽东为代表的马克思主义的正确路线在中共中央的领导地位,从而在极其危急的情况下挽救了中国共产党、挽救了中国工农红军、挽救了中国革命,成为中国共产党历史上一个生死攸关的转折点}。
\textbf{标志着中国共产党在政治上从幼年达到了成熟}。

{1936年10月,红二、四方面军先后同红一方面军在甘肃会宁、静宁将台堡(今属宁夏回族自治区)会师,}标志着三大主力红军的长征胜利结束{。}

\subsection{01679-西安事变与抗日民族统一战线的形成}
{1936}年{12}月{12}日,张学良、杨虎城发动了西安事变。西安事变的和平解决的意义:1.
它成为了时局转换的枢纽,十年内战的局面结束了,国内和平基本实现。2.
这是中国共产党抗日民族统一战线策略和逼蒋抗日方针的重大胜利,它为国共两党的合作抗日创造了有利条件。{}

中共关于抗日民族统一战线政策的制定:毛泽东在瓦窑堡会议上系统地解决了党的政治路线上的问题。(1)阐明建立抗日民族统一战线的可能性。(2)批判了``左''倾关门主义错误。(3)规定了建立广泛的抗日民族统一战线的具体政策。

{1}{936}年{5}月,中共中央发布《停战议和一致抗日》通电,放弃了``反蒋抗日''的口号。后正式确定了``逼蒋抗日''的策略方针。国民党五届三中全会在会议文件上第一次写上了``抗日''的字样。\textbf{国民党中央通讯社发表《中共中央为公布国共合作宣言》;23日,蒋介石发表实际承认共产党合法地位的谈话。标志着以国共两党第二次合作为基础的抗日民族统一战线正式形成。}

\textbf{抗日民族统一战线的特点}{:}1. 广泛的民族性和复杂的阶级矛盾。2.
国共双方有政权有军队的合作。国民党领导全国政权和军队,共产党领导局部政权和军队。3.
没有正式的固定的组织形式和协商一致的具体的共同纲领{。}

抗日民族统一战线的巩固、发展和壮大,是夺取抗日战争最后胜利的根本保证{。}

\subsection{01681-正面战场与敌后战场}
蒋介石集团实行的是片面抗战路线,不敢放手发动和武装民众,将希望单纯寄托在政府和正规军的抵抗上。尽管国民党军队在抗战初期对日作战比较积极比较努力,但是国民党正面战场除了台儿庄战役取得大捷外,其他战役几乎都是以退却、失败而结束。造成这种状况的客观原因,是由于在敌我力量对比上,日军占很大的优势;主观原因,则是国民党战略指导方针上的失误。主要是实行片面抗战路线及消极防御的作战方针。

{抗日战争进入相持阶段,}日本对国民党政府采取以政治诱降为主、军事打击为辅的方针{。国民党在重申坚持持久抗战的同时,其对内对外政策发生重大变化。}采取消极抗日、积极反共的政策{。}

中国共产党主张实行全面抗战的路线,即\textbf{人民战争路线}。中共在洛川召开政治局扩大会议(\textbf{洛川会议}),制定了抗日救国\textbf{十大纲领},\textbf{强调要打倒日本帝国主义,关键在于使已经发动的抗战成为全面的全民族的抗战}。

\textbf{平型关战役}歼敌{1000}余人,取得全\textbf{民族抗战以来中国军队的第一次重大胜利。}

{在抗日战争的}\textbf{初期和中期}{,}\textbf{游击战被提到了战略的地位,具有全局性的意义}{。在战略防御阶段,对阻止日军的进攻、减轻正面战场压力、使战争转入相持阶段起了关键性的作用。}\textbf{在战略相持阶段,敌后游击战争成为主要的抗日作战方式}{。游击战还为人民军队进行战略反攻准备了条件。}

\subsection{01682-共产党维护统一战线的策略}
\textbf{统一战线中的独立自主原则}:中国共产党在统一战线中坚持独立自主原则,既统一,又独立。这样做的目的,是为了保持共产党领导的革命力量已经取得的阵地,尤其是为了发展这些阵地,以动员千百万群众进入抗日民族统一战线,实质上就是力争中国共产党对抗日战争的领导权,使自己成为团结全民族抗战的中坚力量。这是把抗日战争引向胜利的中心一环。

\textbf{坚持抗战、团结、进步,反对妥协、分裂、倒退}{:抗日战争相持阶段到来以后,由于以蒋介石为代表的国民党亲英美派开始推行消极抗日、积极反共的政策,团结抗战的局面逐步发生严重危机,出现了中途妥协和内部分裂两大危险。针对这种情况,中国共产党明确提出}\textbf{``坚持抗战到底,反对中途妥协''、``巩固国内团结,反对内部分裂''、``力求全国进步,反对向后倒退''}{三大口号,在抗战情况下,民族斗争高于阶级斗争,民族斗争与阶级斗争表现出一致性。}

{\textbf{巩固抗日民族统一战线的策略总方针:}}

为了抗日民族统一战线的坚持、扩大和巩固,中国共产党制定了``\textbf{发展进步势力,争取中间势力,孤立顽固势力}''的策略总方针。

进步势力主要的是指工人、农民和城市小资产阶级。他们是统一战线的基础,抗日战争的主要依靠力量。{}

中间势力主要是指民族资产阶级、开明绅士和地方实力派。争取中间势力,是中国共产党在抗日民族统一战线中的一项极严重的任务。争取中间势力需要一定的条件:一是共产党要有充足的力量;二是尊重他们的利益;三是要同顽固派作坚决的斗争,并能一步一步地取得胜利。

顽固势力是指大地主大资产阶级的抗日派,即以蒋介石集团为代表的国民党亲英美派{。他们采取两面政策,既主张团结抗日,又限共、溶共、反共并摧残进步势力。为此,共产党必须以革命的两面政策来对付他们,}即贯彻又联合又斗争的政策{,斗争不忘统一,统一不忘斗争,二者不可偏废,而以统一为主。}同顽固派斗争的策略原则是``有理''、``有利''、``有节''{。}{}

\subsection{01683-根据地建设}
\textbf{三三制政权}根据地是共产党领导的抗日民族统一战线性质的政权。三三制是指抗日民主政府在工作人员分配上实行``三三制''原则。即\textbf{共产党员、非党的左派进步分子和中间派各占1/3}。
{}

\textbf{减租减息,发展生产}:减租减息是中国共产党在抗日根据地为适当调节各抗日阶层的利益实行的土地政策。

{为了克服根据地的严重困难,毛泽东提出了``}\textbf{发展生产,保障供给}{''的经济工作和财政工作的总方针,发出了``}\textbf{自己动手,丰衣足食}{''的号召。抗日民主政府还厉行}\textbf{精兵简政}{,以减轻人民负担,为坚持抗战、争取胜利奠定了物质基础。}

\subsection{01684-党的建设与延安整风}
在中国共产党召开的六届六中全会上,毛泽东明确提出了``\textbf{马克思主义的中国化}''这个命题。{}

毛泽东先后作了《改造我们的学习》、《整顿党的作风》和《反对党八股》的讲演,成为了延安整风的指导性文献,\textbf{延安整风}运动在全党范围普遍展开。

\textbf{反对主观主义以整顿学风、反对宗派主义以整顿党风、反对党八股以整顿文风}{。其中,}\textbf{反对主观主义是整风运动最主要的任务}{。}

整风运动是一场伟大的思想解放运动。一切从实际出发、理论联系实际、实事求是的马克思主义思想路线,在全党范围确立了起来{。}

\textbf{中共七大}{将以毛泽东为代表的中国共产党人把马克思主义基本原理同中国具体实际相结合所创造的理论成果,正式命名为}\textbf{毛泽东思想}{,并将毛泽东思想规定为党的一切工作的指针。}

\subsection{01685-抗战的胜利与中国在二战中的地位}
{中国的抗日战争是世界反法西斯战争的重要组成部分,是世界反法西斯战争的东方主战场。中国的抗日战争为世界反法西斯战争的胜利作出了重大贡献。}

\subsection{01686-抗日战争胜利的意义、原因和基本经验}
\textbf{{历史意义}}:\textbf{{①}中国人民抗日战争,是近代以来中华民族反抗外敌入侵第一次取得完全胜利的民族解放战争。{②}捍卫了中国的国家主权和领土完整,使中华民族避免了遭受殖民奴役的厄运。{③}促进了中华民族的觉醒,使中国人民在精神上、组织上的进步达到了前所未有的高度。{④}促进了中华民族的大团结,弘扬了中华民族的伟大精神。{⑤}中国人民抗日战争的胜利,对世界各国夺取反法西斯战争的胜利、维护世界和平的伟大事业产生了巨大影响}。

\textbf{{胜利原因}}{:}\textbf{①中国共产党在全民族抗战中起到了中流砥柱的作用。②中国人民巨大的民族觉醒、空前的民族团结和英勇的民族抗争,是中国人民抗日战争胜利的决定性因素。③中国人民抗日战争的胜利,同世界所有爱好和平和正义的国家和人民、国际组织以及各种反法西斯力量的同情和支持也是分不开的}{。}

\textbf{{历史经验}}{:}\textbf{①全国各族人民的大团结是中国人民战胜一切艰难困苦、实现奋斗目标的力量源泉。②以爱国主义为核心的伟大民族精神是中国人民团结奋进的精神动力。③提高综合国力是中华民族自立于世界民族之林的基本保证。④中国人民热爱和平,反对侵略战争,同时又决不惧怕战争。⑤只有坚持中国共产党的领导,中华民族才能捍卫自己的生存和发展的权利,才能创造美好的未来}{。}

\subsection{01693-新民主主义社会的性质}
\textbf{我国社会的性质是新民主主义社会}。新民主主义\textbf{五种经济成分}中,\textbf{主要的是三种},即\textbf{社会主义经济、个体经济和私人资本主义经济}。通过没收官僚资本而形成的社会主义国营经济,掌握了主要经济命脉,居于\textbf{领导地位}。而以农业和手工业为主体的个体经济,则在国民经济中占\textbf{绝对优势}。工人阶级、农民阶级和其他小资产阶级、民族资产阶级等是新民主主义社会基本的阶级力量。三种基本的经济成分及与之相应的三种基本的阶级力量(工人阶级、农民及其他小资产阶级、资产阶级)之间的矛盾,就\textbf{{集中地表现为无产阶级与资产阶级的矛盾、社会主义与资本主义的矛盾}}{。}{}

\textbf{中共七届二中全会}决议分析了新民主主义社会的经济状况和基本矛盾,提出中国从农业国转变为工业国并解决了土地问题以后,中国还存在着\textbf{两种基本的矛盾}:\textbf{国际上是新中国同帝国主义的矛盾,国内是工人阶级和资产阶级的矛盾}。

\textbf{{阶级构成}:工人阶级、农民阶级和其他小资产阶级、民族资产阶级等是新民主主义社会基本的阶级力量}{。}由于农民和手工业者的个体经济既可以自发地走向资本主义,也可以被引导走向社会主义,其本身并不代表一种独立的发展方向{。随着土地改革的基本完成,}工人阶级和资产阶级的矛盾逐步成为国内的主要矛盾。而解决这一矛盾,必然使中国社会实现向社会主义的转变{。}这一时期的民族资产阶级仍然是一个具有两面性的阶级:既有剥削工人的一面,又有接受工人阶级及其政党领导的一面{。{因此},}民族资产阶级与工人阶级的矛盾也具有两重性,既有剥削者与被剥削者的阶级利益相互对立的对抗性的一面,又有相互合作、具有相同利益的非对抗性的一面{。对于工人阶级和社会主义革命来说,民族资产阶级作为一个剥削阶级是被消灭的对象,作为可以接受共产党和工人阶级领导的社会力量,又是团结和改造的对象。}

\subsection{01696-对农业手工业的改造}
\textbf{农业:互助组,初级社,高级社}。\textbf{先合作化、后机械化}。实行\textbf{积极发展、稳步前进、逐步过渡的方针};农业互助合作的发展,要坚持\textbf{自愿和互利的原则},采取\textbf{典型示范、逐步推广的方法},发展一批,巩固一批;正确分析农村的阶级和阶层状况,制定正确的阶级政策。要始终\textbf{把是否增产作为衡量合作社是否办好的标准};要把社会改造同技术改造相结合。在实现农业合作化\textbf{{以后}},国家应努力用先进的技术和装备发展农业经济。

\textbf{手工业:}{中国共产党采取的是}\textbf{积极领导、稳步前进的方针}{。手工业合作化的组织形式,是由}\textbf{手工业生产合作小组(手工业供销小组)}{、}\textbf{手工业供销合作社到手工业生产合作社}{;步骤是从供销人手,由小到大,由低到高,逐步实行社会主义改造和生产改造。}

\subsection{01698-社会主义制度的确立及其意义}
\textbf{经验:}第一,坚持社会主义工业化建设与社会主义改造同时并举。第二,采取积极引导、逐步过渡的方式。第三,用和平方法进行改造。

\textbf{{意义}:新民主主义革命的胜利,社会主义基本制度的建立,我国进入社会主义初级阶段。标志着中国历史上长达数千年的阶级剥削制度的结束。为当代中国一切发展进步奠定了根本政治前提和制度基础}{。}

\subsection{02511-马克思主义中国化的提出与内涵}
\textbf{{马克思主义中国化}}的提出:1938年,毛泽东在党的\textbf{{六届六中全会}}上作了题为\textbf{{《论新阶段》}}的政治报告。这是在全党范围内,\textbf{{最早明确提出}}``马克思主义中国化''的命题。经过\textbf{{延安整风}},马克思主义中国化的思想成为\textbf{{全党的共识}}。1945年,\textbf{{刘少奇}}代表党中央在党的\textbf{{七大}}上作的关于修改党章的报告。中共七大通过的党章指出,{毛泽东思想是马克思主义中国化的第一个重大理论成果,是}{``}{中国化的马克思主义''}。

\textbf{{马克思主义中国化的科学内涵}}:{马克思主义中国化,就是将马克思主义基本原理与中国具体实际相结合}。具体地说,就是把马克思主义的基本原理同中国革命、建设和改革的实践结合起来,同中国的优秀历史传统和优秀文化结合起来,既坚持马克思主义,又发展马克思主义。

\textbf{{马克思主义中国化就是}}{:}{运用马克思主义指导中国革命、建设和改革的实践}{;}{就是把中国革命、建设和改革的实践经验和历史经验上升为马克思主义理论}{;}{就是把马克思主义植根于中国的优秀文化之中}{。}


\section{[毛中特]毛泽东思想和中国特色社会主义理论体系}

\subsection{02513-毛泽东思想的形成过程}
毛泽东思想是马克思列宁主义在中国的运用和发展,是被实践证明了的关于中国革命和建设的正确的理论原则和经验总结,是中国共产党\textbf{{集体智慧的结晶}}。

帝国主义战争与无产阶级革命的时代主题,是毛泽东思想形成的\textbf{{时代背景}}。

中国共产党领导的革命和建设的实践,是毛泽东思想形成的\textbf{{实践基础}}。

\textbf{{开始形成}}:第一次国内革命战争时期,通过调查研究论证阶级关系;土地革命战争时期开辟了农村包围城市、武装夺取政权的道路,与教条主义作斗争,论证中国革命新道路。

\textbf{{走向成熟}}:遵义会议后到抗日战争时期,完成新民主主义革命理论和政策是\textbf{{成熟的标志}}。\textbf{{七大毛泽东思想被确立为党的指导思想}}。

\textbf{{进一步补充}}{:解放战争时期和新中国成立后,提出人民民主专政理论、社会主义改造理论以及``第二次结合''等。}

\subsection{02514-毛泽东思想的内容与评价}
{毛泽东思想的主要内容}:新民主主义革命理论;社会主义革命和社会主义建设理论;革命军队建设和军事战略的理论;政策和策略的理论;思想政治工作和文化工作的理论;党的建设理论。

\textbf{{毛泽东思想的}{活的灵魂}},是贯穿于上述各个理论的立场、观点和方法。它们有三个基本方面,即\textbf{{实事求是,群众路线,独立自主}}。其\textbf{{核心和精髓就是实事求是}}。

{毛泽东思想的重大意义}{:马克思主义中国化第一次历史性飞跃的理论成果;中国革命和建设的科学指南;中国共产党和中国人民宝贵的精神财富。}

\subsection{02519-近代中国的基本国情与产生原因}
{认清国情,是认清和解决革命问题的基本依据}。近代中国,已经沦为一个\textbf{{半殖民地半封建性质}}的社会,这是\textbf{{最基本的国情}}。

鸦片战争以后,随着外国资本---帝国主义的入侵,中国社会发生了\textbf{{两个根本性的变化}}:

1.
中国由一个领土完整、主权独立的国家沦为表面上独立、实际上受帝国主义列强共同支配的\textbf{{半殖民地国家}}。

2.
中国由一个完全的封建社会变为有了一定程度资本主义成分的\textbf{{半封建社会}}。

\textbf{{中国逐步变成半殖民地的原因}}:

1.
鸦片战争以后,西方列强通过发动侵略战争,强迫中国签订一系列不平等条约,一步一步地控制中国的政治、经济、外交和军事。中国已经丧失了完全独立的地位,在\textbf{{相当程度上被殖民地化}}了。

2.
西方列强侵略中国的目的,是要把它变成自己的殖民地。但是由于中国长期以来一直是一个统一的大国,特别是\textbf{{中国人民顽强、持久的反抗}},同时也由于\textbf{{帝国主义列强间争夺中国的矛盾无法协调}},使得它们中的任何一个国家都无法单独征服中国,也使得它们不可能共同瓜分中国。因此,近代中国尽管在实际上已经丧失拥有完整主权的独立国的地位,但是仍然维持着独立国家和政府的名义,还有一定的主权,\textbf{{因此被称作半殖民地}}。

\textbf{{中国逐步变成半封建社会的原因}}{。}

{1.
外国资本主义列强用武力打开中国的门户,把中国卷入世界资本主义经济体系和世界市场之中。随着外国资本主义的入侵,一方面,}{破坏了中}{国自给自足的自然经济的基础,破坏了城市的手工业和农民的家庭手工业;另一方面,促进了中国城乡商品经济的发展,给中国资本主义的产生提供了某些客观条件。}{破产的农民和手工业者成为产业工人的后备军。一批官僚、买办、地主、商人投资兴办新式工业。}\textbf{{中国出现了资本主义生产关系}}{。}

{2.
}\textbf{{西方列强并不愿意中国成为独立的资本主义国家}}{。中国的民族资本主义经济虽然有了某些发展,但是并没有也不可能成为中国社会经济的主要形式。在中国农村中,地主剥削农民的封建生产关系,在社会经济生活中依然占着明显的优势。这样,中国的经济就成为半殖民地半封建的经济了。}

\subsection{02520-近代以来的主要矛盾和革命性质}
{近代中国社会是半殖民地半封建社会},这是一个特殊的过渡性社会。近代中国社会的性质,即是中国特殊的国情。\textbf{{近代中国半殖民地半封建的社会性质,决定了社会主要矛盾是帝国主义和中华民族的矛盾、封建主义和人民大众的矛盾}}。\textbf{{而帝国主义和中华民族的矛盾,又是各种矛盾中最主要的矛盾}}。

近代中国社会的性质和\textbf{{主要矛盾}},\textbf{{决定}}了近代中国革命的\textbf{{根本任务是推翻帝国主义、封建主义和官僚资本主义的统治}},从根本上推翻反动腐朽的政治上层建筑,变革阻碍生产力发展的生产关系,为建设富强民主的国家、改善人民的生活、确立人民当家作主的政治地位扫清障碍,创造必要的前提。

\textbf{{俄国革命的胜利}},改变了整个世界历史的方向,划分了整个世界历史的时代,\textbf{{开辟了世界无产阶级社会主义革命的新纪元,标志着人类历史开始了由资本主义向社会主义转变的进程}}。这使\textbf{{中国的资产阶级民主主义革命}},从\textbf{{原来属于旧的世界资产阶级民主主义革命}}的范畴,属于旧的世界资产阶级民主主义革命的一部分,\textbf{{转变为}}属于新的资产阶级民主主义革命的范畴,\textbf{{属于世界无产阶级社会主义革命的一部分}}。

近代中国革命\textbf{{以五四运动为开端}},进入新民主主义革命阶段。

\textbf{{革命的性质是由社会性质和革命任务等因素决定的}}。\textbf{{中国革命的性质,就不是无产阶级社会主义革命,而是资产阶级民主主义革命}}。

{新民主主义革命与旧民主主义革命相比有其}\textbf{{新的内容和特点}}{,集中表现在:1.
中国新民主主义革命处于世界无产阶级社会主义革命的}\textbf{{时代条件}}{,是世界无产阶级社会主义革命的一部分;2.
革命的}\textbf{{领导力量}}{是中国无产阶级及其先锋队------中国共产党;(}\textbf{{根本标志}}{)③革命的}\textbf{{指导思想}}{是马克思列宁主义;3.
革命的}\textbf{{前途}}{不再是资本主义,而是经过新民主主义实现社会主义。}

\subsection{02522-新民主主义革命的动力与领导}
认清这个革命的动力问题,才能正确地解决中国革命的基本策略问题。\textbf{{新民主主义革命的动力}{是工人阶级、农民阶级、城市小资产阶级和民族资产阶级}},而\textbf{{根本的动力是工人和农民}}。

\textbf{{中国无产阶级是中国革命最基本的动力}}。{无产阶级的领导权是中国革命的中心问题,也是新民主主义革命理论的核心问题}。{革命的领导权是区别新旧两种不同范畴的民主主义革命的}\textbf{{根本标志}}。中国工人阶级是新的社会生产力的代表,是近代中国最进步的阶级。中国工人阶级具有{自身的特点和优点}。1.
它从诞生之日起,就身受外国资本主义、本国封建势力和资产阶级的三重压迫,而这些压迫的严重性和残酷性,是世界各民族中少见的,这就形成了中国无产阶级\textbf{{坚强的斗争性和彻底的革命性}}。2.
它\textbf{{分布集中}},有利于无产阶级队伍的组织和团结,有利于革命思想的传播和形成强大的革命力量。3.
它大部分出身于破产的农民,\textbf{{和农民有着天然的联系}},使无产阶级便于和农民结成亲密的联盟,共同团结战斗。无产阶级及其政党的领导,是中国革命取得胜利的根本保证。

\textbf{{农民是中国革命的主力军}},其中的贫{农是无产阶级最可靠的同盟军,而中农是无产阶级可靠的同盟军}。{农民问题是中国革命的基本问题},{新民主主义革命实质上就是党领导下的农民革命,中国革命战争实质上就是党领导下的农民战争}。

{城市小资产阶级,包括广大的知识分子、小商人、手工业者和自由职业者,城市小资产阶级同样是中国革命的动力之一},{是中国革命可靠的同盟军}。

\textbf{{民族资产阶级是一个带有两面性的阶级}}。中国民族资产阶级由一部分买办、地主官僚、商人、手工工场主转变而来。中国共产党对民族资产阶级在政治上争取它,对其动摇性和妥协性进行批评和斗争,在经济上实行保护民族工商业的政策,中国共产党对民族资产阶级采取{既联合又斗争}的方针。

{无产阶级及其政党对中国革命的领导权不是自然而然得来的,而是在与资产阶级争夺领导权的斗争中实现的}{。毛泽东指出:``领导的阶级和政党,要实现自己对于被领导的阶级、阶层、政党和人民团体的领导,必须具备}\textbf{{两个条件}}{:(甲)率领被领导者(同盟者)向着共同敌人作坚决的斗争,并取得胜利;(乙)对被领导者给以物质福利,至少不损害其利益,同时对被领导者给以政治教育。''}

\subsection{02523-新民主主义革命的前途与两种错误倾向}
毛泽东根据中国社会的性质和新民主主义革命的特点,认为\textbf{{中国革命必须分两步走}}:\textbf{{第一步是新民主主义革命}},反对帝国主义、封建主义和官僚资本主义,改变中国半殖民地半封建社会形态,使中国成为一个独立的新民主主义的国家;\textbf{{第二步是社会主义革命}}。即在新民主主义国家的基础上,使革命继续向前发展,建立社会主义制度,使中国成为一个社会主义国家。\textbf{{新民主主义革命与社会主义革命又是互相联系、紧密衔接的,中间不容横插一个资产阶级专政}}。民主主义革命是社会主义革命的必要准备,社会主义革命是民主主义革命的必然趋势。

{中共党内在革命前途问题上曾经有过}\textbf{{两种错误倾向}}{:一是陈独秀的``}\textbf{{二次革命论}}{'',把中国革命过程中两个紧密联系的阶段割裂开来,只看到两者之间的区别,没有看到两者之间的联系,要在两个阶段之间硬插一个资产阶级专政的和发展资本主义的阶段;二是以王明为代表的}\textbf{{``左''倾教条主义}}{,主张把民主革命和社会主义革命``毕其功于一役''的``一次革命论'',混淆了民主革命和社会主义革命的界限,企图把两种不同性质的革命阶段并作一步走,一举取得社会主义革命的胜利。这种观点只看到两者之间的联系,而忽视了两者之间的区别。}

\subsection{02524-政治纲领}
1940年,毛泽东在{《新民主主义论》}中阐述了新民主主义的基本纲领,即政治、经济和文化纲领。1945年,他在党的七大所作的{《论联合政府》}的政治报告中,进一步把新民主主义的政治、经济和文化与党的基本纲领联系起来,进行了具体阐述。

新民主主义革命的政治纲领是:\textbf{{推翻帝国主义和封建主义的统治,建立一个无产阶级领导的、以工农联盟为基础的、各革命阶级联合专政的新民主主义的共和国}}。

新民主主义国家的\textbf{{国体}}是无产阶级领导的以工农联盟为基础,包括小资产阶级、民族资产阶级和其他反帝反封建的人们在内的各革命阶级的联合专政。

{国体------各革命阶级联合专政,政体------民主集中制的人民代表大会制度,这就是新民主主义政治}{。}

\subsection{02526-文化纲领}
\textbf{{新民主主义文化就是无产阶级领导的人民大众的反帝反封建的文化,即民族的科学的大众的文化}}。

第一,``无产阶级领导'',是指新民主主义文化由无产阶级思想即共产主义思想领导的,也就是说,\textbf{{新民主主义文化中居于指导地位的是共产主义思想}}。\textbf{{是否以共产主义思想为指导,这是新民主主义文化同旧民主主义文化相区别的标志}}。

第二,新民主主义文化是\textbf{{民族的}},强调的是文化的民族内涵和形式。新民主主义文化是民族的,就\textbf{{其内容说是反对帝国主义压迫,主张中华民族的尊严和独立}};就其\textbf{{形式说是具有鲜明的民族风格、民族形式和民族特色,要有中国作风和中国气派}}。

第三,新民主主义文化是\textbf{{科学的}},强调的是这种文化内容的科学性。它\textbf{{反对一切封建思想和迷信思想,主张实事求是,主张客观真理,主张理论和实践的统一}}。同时,\textbf{{对于中国古代文化,要剔除其封建性的糟粕,吸取其民主性的精华,决不能无批判的兼收并蓄}}。

{第四,新民主主义文化是}\textbf{{大众的}}{,强调的是这种文化的}\textbf{{民主性}}{。这种文化}\textbf{{应该为全民族90\%以上的工农大众服务}}{,并逐渐成为他们的文化,因此是最民主的文化。}

\subsection{02529-新民主主义社会的经济、政治和文化}
我国从中华人民共和国成立到社会主义改造基本完成时期,是一个\textbf{{过渡时期}}。这一时期,\textbf{{我国社会的性质是新民主主义社会}}。新民主主义社会不是一个独立的社会形态,而是由新民主主义到社会主义转变的过渡性的社会形态,它\textbf{{属于社会主义体系}}。

在\textbf{{经济上}},\textbf{{五种经济成分并存}}。即:社会主义性质的国营经济、半社会主义性质的合作社经济、农民和手工业者的个体经济、私人资本主义经济和国家资本主义经济。\textbf{{主要的经济成分是三种}}:\textbf{{社会主义经济、个体经济和资本主义经济}}。在这些经济成分中,\textbf{{通过没收官僚资本而形成的社会主义的国营经济}},\textbf{{掌握了主要经济命脉}},居于\textbf{{领导地位}}。

在\textbf{{政治上}},\textbf{{新民主主义国家实行工人阶级领导的,以工农联盟为基础的,包括工人阶级、农民阶级、城市小资产阶级和民族资产阶级在内的人民民主专政}}。新中国建立之初,我国人民民主专政,是属于新民主主义政权性质。

{在}\textbf{{文化上}}{,}\textbf{{发展以马克思主义指导下的新民主主义的文化,即民族的、科学的、大众的文化}}{。}

\subsection{02531-过渡时期总路线的提出过程}
\textbf{{向社会主义过渡原因}:}第一,社会主义性质的国营经济力量相对来说比较强大,它是实现国家工业化的主要基础。第二,资本主义经济力量弱小,发展困难,不可能成为中国工业起飞的基础。第三,对个体农业进行社会主义改造,是保证工业发展、实现国家工业化的一个必要条件。第四,当时的国际环境也促使中国选择社会主义。

\textbf{{必然性}:}第一,实现社会主义工业化,是国家独立和富强的必然要求和必要条件。第二,对个体经济和私营资本主义工商业进行社会主义改造,是实现社会主义工业化的客观需要。

\textbf{{可能性}}:第一,我国已经有了相对强大和迅速发展的社会主义国营经济。第二,土地改革完成后,为发展生产、抵御自然灾害,广大农民具有走互助合作道路的要求。第三,新中国成立初期,党和国家在合理调整工商业的过程中,出现了加工订货、经销代销、统购包销、公私合营等一系列从低级到高级的国家资本主义形式。第四,当时的国际形势也有利于中国向社会主义过渡。

\textbf{{计划}:}{党和毛泽东对于何时向社会主义过渡、怎样过渡的问题,}\textbf{{经历了一个从先搞工业化建设再一举过渡,到建设和改造同时并举}}{、从中华人民共和国成立起即逐步过渡的发展变化过程。}

\subsection{02536-社会主义制度的确立及其意义}
\textbf{{经验}:}{第一,坚持社会主义工业化建设与社会主义改造同时并举。第二,采取积极引导、逐步过渡的方式。第三,用和平方法进行改造。}

\textbf{{意义}:{新民主主义革命的胜利,社会主义基本制度的建立,我国进入社会主义初级阶段。标志着中国历史上长达数千年的阶级剥削制度的结束。为当代中国一切发展进步奠定了根本政治前提和制度基础}}{。}

\subsection{02550-改革的性质}
\textbf{{改革开放是决定当代中国命运的关键抉择。改革是唯一的出路。改革开放是社会主义制度的自我完善和发展。改革是社会主义社会发展的直接动力{。}}{社会主义社会仍然存在着基本矛盾,可以通过社会主义自身的力量解决。改革是社会主义制度下解放和发展生产力的必由之路。}}

{革命是解放生产力,改革也是解放生产力,改革的目的同过去的革命一样,是为了扫除社会生产力发展的障碍,使中国摆脱贫穷落后的状态}。\textbf{{改革是党在新的时代条件下带领人民进行的新的伟大革命}}{。}\textbf{{改革不是对原有体制的细枝末节的修补,是一场深刻而全面的社会改革}}。

{改革是一场伟大的革命。但它}{不是个阶级推翻另一个阶级意义上的革命,不是也不允许否定和抛弃我们已经建立起来的社会主义制度}{,它是社会主义制度的自我完善和发展。}

\subsection{02584-和平统一、一国两制思想的提出、内容与意义}
{台湾问题的}\textbf{{由来和实质}}{:台湾自古以来是中国领土不可分割的重要组成部分,台湾人民同大陆人民同根、同宗、同源,承继的是相同的文化传统。台湾问题是中国国内战争遗留下来的问题,}\textbf{{台湾问题实质是中国的内政问题}}{。}

解决台湾问题的方针由\textbf{{武力解放}}到\textbf{{和平解放}}再到\textbf{{和平统一}}。

\textbf{{``和平统一、一国两制''构想的基本内容}}:第一,\textbf{{一个中国。这是"}{和平统一、一国两制"}{的核心}},是发展两岸关系和实现和平统一的\textbf{{基础}}。第二,两制并存。第三,高度自治。第四,尽最大努力争取和平统一,但不承诺放弃使用武力。第五,解决台湾问题,实现祖国的完全统一,寄希望于台湾人民。

\textbf{{``和平统一、一国两制''的意义}}{:}{``和平统一、一国两制''构想创造性地发展了马克思主义的国家学说}{;}{``和平统一、一国两制''构想有利于争取社会主义现代化建设事业所需要的和平的国际环境与国内环境}{;}{``和平统一、一国两制''构想为解决\textbf{{国际争端}}和历史遗留问题提供了新的思路}{。}

\subsection{08562-马克思主义的创立}
{经济社会根源}:\textbf{{资本主义经济的发展}}为马克思主义的产生提供了经济、社会历史条件。

{{实践基础:}\textbf{{无产阶级反对资产阶级的斗争}}日益激化。}

{{思想渊源:}\textbf{{德国古典哲学、英国古典政治经济学和法国、英国空想社会主义}}的合理成分,\textbf{{创立了唯物史观和剩余价值学说,把社会主义由空想变为科学。}}}

{马克思、恩格斯批判地继承了前人的成果,创立了唯物史观和剩余价值学说,实现了人类思想史上的伟大革命。马克思}{1845}{年春天写的\textbf{{《关于费尔巴哈的提纲》}}和马克思、恩格斯}{1844-1846}{年合写的\textbf{{《德意志意识形态》}},{标志着马克思主义的基本形成}。}

{1847}{年\textbf{{《哲学的贫困》}}和}{1848}{年}{2}{月\textbf{{《共产党宣言》}}的发表,{标志着马克思主义的公开问世}。}

\subsection{08563-马克思主义的鲜明特征}
马克思主义从产生到发展,表现出了强大的生命力,\textbf{{这种强大生命力}}的{根源}在于它的{\textbf{以实践为基础的科学性与革命性的统一}}。这种实践基础上的科学性与革命性的统一,是马克思主义基本的和{\textbf{最鲜明的特征}}。

{马克思主义具有{科学性},它是对客观世界特别是人类社会{本质和规律的正确反映}。其科学性表现在,坚持世界的物质性和真理的客观性,力求按照世界的本来面目如实地认识世界,力求全面地认识事物,并透过现象而深刻地揭示事物的本质和规律,自觉接受实践的检验,并在实践中不断丰富和发展。}

{马克思主义具有{革命性},它是无产阶级和广大人民群众推翻旧世界、建设新世界的理论。它的革命性表现在坚持唯物辩证法,具有彻底的批判精神。它具有鲜明的政治立场,毫不隐讳自己的阶级本质。}

{\textbf{{辩证唯物主义与历史唯物主义}}是马克思主义{最根本的世界观和方法论}。}

{马克思主义政党的一切理论和奋斗都应致力于实现\textbf{{以劳动人民为主体的最广大人民的根本利益}},这是马克思主义{最鲜明的政治立场}。}

{坚持一切从实际出发,理论联系实际,实事求是,在实践中检验真理和发展真理,是马克思主义最重要的理论品质。}

{实现物质财富极大丰富、人民精神境界极大提高、每个人自由而全面发展的\textbf{{共产主义社会}},是马克思主义{最崇高的社会理想。}}

\subsection{08564-马克思主义基本原理}
{\textbf{马克思主义基本原理}},是马克思主义理论体系中\textbf{{最基本、最核心}}的内容,是对马克思主义的\textbf{{立场、观点和方法}}的集中概括。它体现马克思主义的{根本性质}和{整体特征},体现马克思主义{{科学性和革命性的统一}}。

相对于特定历史条件下所作的个别理论判断和具体结论,马克思主义基本原理具有普遍的、根本的和长远的指导意义。可以从基本立场、基本观点和基本方法三个方面把握马克思主义的基本原理。

{马克思主义基本立场},是马克思主义\textbf{{观察.分析和解决}}问题的根本{立足点}和{出发点}。这就是始终站在人民大众的立场上,一切为了人民,一切相信人民,一切依靠人民,全心全意为人民谋利益。

{马克思主义的基本观点,是关于自然、社会和人类思维规律的科学认识,是对人类思想成果和社会实践经验的科学总结}。

{\textbf{这些基本观点主要包括}}:关于客观世界的本质和规律的观点,关于人的实践和认识活动的本质和规律的观点,关于社会形态和社会基本矛盾运动规律的观点,关于人民群众的历史主体作用的观点,关于商品经济和社会化大生产与一般规律的观点,关于劳动价值论、剩余价值和资本主义生产方式本质的观点,关于社会主义必然代替资本主义的观点,关于社会主义革命和无产阶级专政的观点,关于无产阶级政党建设的观点,关于社会主义本质特征和建设规律的观点,关于共产主义社会基本特征的观点,等等。

{\textbf{马克思主义的基本方法}},是建立在辩证唯物主义和历史唯物主义世界观.方法论基础上的思想方法和工作方法,主要包括实事求是的方法、辩证分析的方法、历史分析的方法、群众路线的方法等等。

\subsection{08565-自觉学习和运用马克思主义}
{学习和掌握马克思主义基本原理,是大学生个人成长和长远发展的客观需要,具有重要的现实意义。}

{马克思主义诞生以来,指引着各国无产阶级和广大劳动人民进行了艰苦卓绝的斗争,科学社会主义从理论~发展为亿万人民群众的实践,人类取得了历史性的进步。}

{中国共产党成立以来,把马克思主义基本原理同~中国具体实际相结合,带领中国人民取得了革命、建设和改革的卓越成就。}

{{\textbf{实践证明,}}{马克思主义是我们立党立国的根本指导思想,是全国各族人民团结奋斗的共同理论基础。马克思主义的基本原理任何时候都要坚持,否则我们的事业就会因为没有正确的理论基础和思想灵魂而迷失方向,就会归于失败。}}

{{这就是我们为什么必须始终学习和坚持马克思主义基本原理的道理所在}{。}}

{当前,在世界范围内和我国社会主义现代化建设中,经济、政治、思想、文化各个方面都有许多复杂的事物需要认识,许多重大问题需要回答,许多未曾认识的领域需要开拓。只有马克思主义才能引导我们深刻认识社会发展的客观规律,把握世界形势;变化的本质,提高解决建设和改革中各种实际问题的本领。同时,随着改革开放和社会主义市场经济的发展,我国社会正发生着深刻的历史变革,在社会主义主流意识形态得到坚持和发展的同时,社会生活多样、多变的特征日益凸显,各种思想观念相互交织、相互影响、相互激荡;现代社会生活多样化,人们的思想活动具有更多的独立性、选择性、多变性和差异性。面对这种情况,只有学习、掌握并坚持以马克思主义作为我们行动的指南,才能更好地坚持爱国主义、集体主义、社会主义思想的主旋律,才能有效地整合各种各样的利益诉求和价值观念,在全社会形成强大的凝聚力和共同的意志。}

{{学习马克思主义理论,}{\textbf{我们应该做到}}{:}}

{{第一}{,学习理论,武装头脑,要努力在掌握理论的科学体系上下功夫,在掌握基本原理及其精神实质上下功夫,在掌握马克思主义的立场、观点、方法并用以指导实践上下功夫。}}

{{第二}{,坚持和弘扬理论联系实际的学风,是学习马克思主义基本原理的根本方法。所谓理论联系实际,就是以马克思主义基本原理为指导,联系国际国内的大局,联系社会实际,去观察和分析问题。}}

{{第三}{,用科学的态度对待马克思主义。坚持马克思主义不动摇,这是就马克思主义的基本原理、基本观点和基本方法而言的。随着时代的发展和历史条件的变化,马克思主义创始人针对特定历史条件的一些具体论述可能不再适用,而新的实践又会提出新的问题,需要我们去认识、去解决,这就要求我们在坚持马克思主义基本原理的基础上,不断地在实践中丰富和发展马克思主义。在我国社会主义实践中,坚持和发展是统一的。}}

{{\textbf{因此}}{,我们要把马克思主义作为行动的指南,在思想上自觉地坚持以马克思主义为指导,确立马克思主义的坚定信念,树立和坚定共产主义远大理想;不断提高运用马克思主义的立场、观点和方法分析、解决问题的能力,自觉地辨别和抵制各种不良思想文化的影响;不断增强服务社会的本领,自觉投身中国特色社会主义实践。}}

\subsection{08566-唯物主义和唯心主义}
{马克思主义关于哲学基本问题的原理为研究哲学发展历史和现实提供了一条基本的指导线索,\textbf{{为划分哲学中的基本派别确定了科学标准}}。}

{根据对哲学基本问题第一方面的不同回答,哲学可划分为唯物主义和唯心主义两个对立的基本派别。}

{{\textbf{唯物主义}}{把世界的本质归结为物质,主张物质第一性,意识第二性,意识是物质的产物;}}

{{\textbf{唯心主义}}{把世界的本原归结为精神,主张意识第一性,物质第二性,物质是意识的产物。}}

{唯物主义和唯心主义这两个专门的哲学术语有着特定的含义和确定标准,不能随意乱用,也不能另立标准。~
~~}

\subsection{08604-全面依法治国的基本格局}
{党的十八大提出了``}\textbf{{科学立法、严格执法、公正司法}{和全民守法}}{''的十六字方针}\textbf{{}{}}{。}

{{科学立法}:以完善以宪法为核心的中国特色社会主义法律体系,加强宪法实施为目标。}

{{严格执法}:以深入推进依法行政,加快建设法治政府为目标。}

{{公正司法}:是维护社会公平正义的最后一道防线。}

{{全民守法}:以增强全民法治观念,推进法治社会建设为目标。}

\subsection{08605-唯物辩证法与“四个全面”战略思想}
{
党的十八大以来,党中央从坚持和发展中国特色社会主义全局出发,{提出并形成了全面建成小康社会、全面深化改革.全面依法治国、全面从严治党的战略布局。}}

{
习近平总书记在论述四者之间的关系时指出:{全面建成小康社会是我们的战略目标,全面深化改革、全面依法治国、全面从严治党是三大战略举措。}}

{要努力做到``四个全面''相辅相成、相互促进、相得益彰。唯物辩证法坚持用联系的、发展的、全面的观点看世界,认为\textbf{{发展的根本原因在于事物的内部矛盾性}}。}

{``四个全面''战略构想在各个方面都体现了唯物辩证法思想。}

{{第一:体现了事物联系和发展的思想。}联系和发展是唯物辩证法的总特征。联系是指事物内部各
要素之间和事物之间相互影响、相互制约和相互作用的关系。}

{``四个全面''不仅揭示了``建成小康社会''``深化改革''"依法治国''和``从严治党之间的联系,也揭示各自战略目标和举措之间的联系。}

{发展是前进的上升的运动,发展的实质是新事物的产生和旧事物的灭亡。``四个全面''思想也体现了事物是发展变化的这一辩证思想。}

{{第二,辩证法要求我们用整体的、全面的观点看问题。}四个``全面''思想贯彻了唯物辩证法全面看问题的方法。}

{{第三,在唯物辩证法的方法论体系中,矛盾分析方法居于核心地位,是根本的认识方法。}
如要求人们做到``两点论''和``重点论''相结合等,\textbf{``四个全面''思想也是矛盾分析方法的具体体现。}总之,
``四个全面''战略思想是唯物辩证法思中反映和深刻展现。}

\subsection{08623-法律权利}
{法律权利}{概括为,}{权利主体依法要求义务主体作出某种行为或者不作出某种行为的资格}{。}

{\textbf{法律权利四个特征}}{:}

{1)}{法律权利的内容、种类和实现程度受社会物质生活条件的制约。}{}

{2)}{法律权利的内容、分配和实现方式因社会制度和国家法律的不同而存在差异。}

{3)}{法律权利不仅由法律规定或认可,而且受法律维护或保障,具有不可侵犯性。}

{4)}{法律权利必须依法行使,不能不择手段地行使法律权利。}

{\textbf{法律权利的分类}}

{1)}{基本权利和普通权利}{}

{2)}{政治权利、人身权利、财产权利、社会经济权利、文化权利。}

{3)}{一般主体享有的权利和特定主体享有的权利。}

{4)}{实体性权利和程序性权利。}

{\textbf{法律权利与人权}}

{人权是法律权利的内容和来源,法律权利是对人权的确认和保障。}

{人权是个体人权和集体人权的统一}{;}

{人权是普遍性和特殊性的统一。有的人权是国际社会公认的权利;}

{{人权的评价标准也是多元的}{}{;}}

{{人权的保障水平和实现程度则取决于各国的经济社会发展水平。}{}\\
}

\subsection{08628-法律权利与法律义务的关系}
\textbf{{(1) 辩证统一。}}

{首先,法律权利和法律义务是相互依存的关系;}

{其次,法律权利与法律义务是目的与手段的关系;}

{最后,法律权利和法律
义务还具有二重性的关系,即一个行为可以同时是权利行为和义务行为。}

\textbf{{(2) ---律平等。}}

{首先,法律权利与法律义务平等表现为法律面前人人平等被确立为基本原则,这里的平等讲的就是权利和义务平等。}{}

{其次,在法律权利和法律义务的具体设定上要平等。}

{再次,权利与义务的实现要体现平等。}

\textbf{{(3) 互利互赢。}{}}

\subsection{08632-政治权利与义务}
\textbf{{(1)选举权利与义务~}}

{选举权利包括}{选举权与被选举权}{。}{选举义务是指公民在选举活动中应当承担的法律义务。}

\textbf{{(2)表达权利与义务~}}

{表达权利是指公民依法享有的表达自己对国家公共生活的看法、观点、意见的权利}{。}

{言论自由}{是指公民享有通过各种语言形式表达、传播自己的思想和观点的自由。}{}

{出版自由}{是指公民有权依法通过公开发行的出版物。}

{结社自由}{是指公民为了实现一定的目标而依法律规定的程序组织某种社会团体的自由。}

{集会、游行、示威是公民表达政治意愿的重要方式}{。}{}

{\textbf{{(3) 民主管理权利与义务}{}}\\
}

{民主管理权利是指公民根据宪法法律规定,管理国家事务、经济和文化事业以及社会事务的权利。}

{{\textbf{(4) 监督权利与义务}}{\textbf{}}\\
}

{监督权是指公民依据宪法法律规定监督国家机关及其工作人员活动的权利。一般认为,批评、建议、申诉、检举、控告是宪法法律賦予公民对国家机关和国家工作人员的一种监督权。\\
}

\subsection{08778-中国特色社会主义法律体系的意义和内容}
{\textbf{意义}}

{凝聚思想共识的法治航标;推进国家治理现代化的重要举措;全面依法治国的基础工程。}

{{中国特色社会主义法律体系是在中国共产党领导下,适应中国特色社会主义建设事业的历史进程而逐步形成的。经历了个从无到有、从初步形成到基本形成再到形成、然后经过不断完善趋于更加成熟的过程。}{一个}\textbf{立足中国国情和实际、适应改革开放和社会主义现代化建设需要、集中体现党和人民意志的,以宪法为统帅,以宪法相关法、民法商法等多个法律部门的法律为主干,由法律、行政法规、地方性法规等多个层次的法律规范构成的}{\textbf{中国特色社会主义法律体}}{\textbf{系}}{已经形成}{,国家经济建设、政治建设、文化建设、社会建设以及生态文明建设的各个方面实现有法可依。}\\
}

{\textbf{内容}}

{建设完备的法律规范体系。}

{建设高效的法治实施体系。}

{建设严密的法治监督体系。}

{建设有力的法制保障体系。}

{建设完善的党内法规体系。}

\subsection{08991-我国的实体法律部门}
{宪法相关法}{:宪法相关法是与宪法相配套、直接保障宪法实施和国家政权运作等方面的法律规范}

{民法商法}{:民法是调整平等主体的公民之间、法人之间、公民和法人之间的财产关系和人身关系的法律规范,}{遵循民事主体地位平等、意思自治、公平、诚实信用等基本原则}{。
商法调整商事主体之间的商事关系,}{遵循民法的基本原则,同时秉承保障商事交易自由、等价有偿、便捷安全等原则}{。}

{行政法}{:行政法是关于行政权的授予、行政权的行使以及对行政权的监督的法律规范,,遵循职权法定、程序法定、公正公开、有效监督等原则。}

{经济法}{:经济法是调整国家从社会整体利益出发,对经济活动实行干预、管理或者调控所产生的社会经济关系的法律规范}

{社会法}{:社会法是调整劳动关系、社会保障、社会福利和特殊群体权益保障等方面的法律规范。}

\textbf{{刑法}}{:}{刑法是规定犯罪与刑罚的法律规范}{。我国刑法确立了}{罪刑法定、法律面前人人平等、罪刑相适应等基本原则}{。我国刑法规定了刑罚的种类,包括}{管制、拘役、有期徒刑、无期徒刑、死刑五种主刑以及罚金、剥夺政治权利、没收财产三种附加刑}{。}

{诉讼与非诉讼程序法}{:诉讼与非诉讼程序法是规范解决社会纠纷的诉讼活动与非诉讼活动的法律规范。}

\subsection{08996-认识世界和改造世界必需勇于创新}
{创新是一个民族进步的灵魂,是一个国家兴旺发达的不竭动力,也是一个政党永葆生机的源泉。~}

{实践基础上的理论创新是社会发展和变革的先导,能够在更高层次上引领和推动实践活动的开展。~}

{{马克思主义生命力在于创新。}重视理论创新,是我们党的一个根本特点,也是我们党的一条重要政治经验。注重理论创新,是党的事业前进的重要保证。}

{创新是以坚持和继承为前提的。}

\subsection{08999-金融垄断资本的发展}
20世纪70年代以后,西方国家普遍走上了\textbf{{{金融自由化和金融创新}}}的道路。{金融自由化和金融创新是金融垄断资本得以形成和壮大的重要制度条件},推动着资本主义经济的金融化程度不断提高。

在金融垄断资本的推动下,垄断资本主义的金融化程度不断提高;金融业在国民经济中的地位大幅上升,金融资本在资本主义国家国民生产总值和利润总额中所占的比例越来越大;随着实体经济的资本利润率下降,面对激烈竞争,实体经济部门不得不把利润的一部分投向金融领域,导致金融资本的急剧膨胀;制造业人数严重减少,以金融为核心的服务业就业人数急剧增加。

\textbf{{虚拟经济越来越脱离实体经济。金融垄断资本的发展,一方面促进了资本主义的发展,另一方面也造成了经济过度虚拟化,导致金融危机频繁发生,不仅给资本主义经济,也给全球经济带来灾难。}}

\subsection{09003-实践的本质、基本特征和基本形式}
\textbf{第一:}实践是物质性的活动,具有直接现实性。

\textbf{第二:}实践是人类有意识的活动,体现了自觉的能动性。

\textbf{第三:}实践是社会的历史活动,具有社会历史性的特点。

{实践的基本形式包括物质生产实践,社会政治实践和科学文化实践等。}

物质生产实践是人类最基本的实践活动,社会政治实践是人们社会生活中的一个重要方面,科学文化实践是改造自然和社会的准备性和探索性的实践活动。~~~

\subsection{09006-全面深化改革的总目标}
{ }

全面深化改革的总目标:{\textbf{完善和发展中国特色社会主义制度,推进国家治理体系和治理能力现代化。}}

{总目标所包含的两个方面是一个整体:前者规定了改革的根本方向,就是中国特色社会主义道路,而不是其他什么道路;后者规定了在根本方向指引下完善和发展中国特色社会主义制度的鲜明指向。国家治理体系和治理能力是一个国家的制度和制度执行能力的集中体现。推进国家治理体系和治理能力现代化,是完善和发展中国特色社会主义制度的必然要求,是实现社会主义现代化的题中应有之义。}



{{国家治理体系}是在党领导下管理国家的制度体系,包括经济、政治、文化、社会、生态文明和党的建设等各领域体制机制、法律法规安排,也就是一整套紧密相连、相互协调的国家制度。~}

{{国家治理能力}则是运用国家制度管理社会各方面事务的能力,包括改革发展稳定、内政外交国防、治党治国治军等各个方面。~}

{{国家治理体系和治理能力}是一个有机整体,相辅相成,有了好的国家治理体系才能提高治理能力,提高了国家治理能力才能充分发挥国家治理体系的效能。推进国家治理体系和治理能力现代化必须完整理解和把握全面深化改革的总目标。完善和发展中国特色社会主义制度,推进国家治理体系和治理能力现代化,是两句话组成的一个整体,\textbf{{前一句规定了根本方向,后一句规定了所走路径}},我们是在中国特色社会主义道路这个方向上推进国家治理体系和治理能力现代化。}

\subsection{09015-完善生态文明制度体系}
{\textbf{实行最严格的生态环境保护制度:}}建设生态文明,是一场涉及生产方式、生活方式、思维方式和价值观念的革命性变革。实现这样的变革,必须依靠制度和法治。完善经济社会发展考核评价体系。建立责任追究制度。建立健全生态环境管理制度。~

{\textbf{加快生态文明体制改革,建设美丽中国:}}推进绿色发展。着力解决突出环境问题。加大生态系统保护力度。改革生态环境监管体制。

\subsection{14829-真理问题讨论与十一届三中全会}
{
关于真理标准问题的讨论的背景:当时主持中共中央工作的华国锋坚持``两个凡是''的错误方针。党和国家的工作处于在徘徊中前进的状态。}

{\textbf{{关于真理标准问题的讨论的意义:}{①}{真理标准问题的讨论是继五四运动和延安整风运动之后又一场马克思主义思想解放运动。}{②}{为党重新确立实事求是思想路线,纠正长期以来的}{``}{左}{''}{倾错误,实现历史性的转折做了思想理论准备}}{。}}

{\textbf{{中共十一届三中全会的内容}}{:}{①全会冲破长期``左''的错误的严重束缚,彻底否定了``两个凡是''的错误方针,高度评价了关于真理标准问题的讨论。②断然否定``以阶级斗争为纲''的指导思想,做出了把工作重点转移到社会主义现代化建设上来和实行改革开放的战略决策。③恢复了党的民主集中制的优良传统,审查解决了历史上遗留的一批重大问题和一些重要领导人的功过是非问题}{。}}

{\textbf{{中共十一届三中全会的意义}}{:}\textbf{{①中共十一届三中全会是新中国成立以来党的历史上具有深远意义的伟大转折。②重新确立了马克思主义的思想路线、政治路线和组织路线,会议形成了以邓小平为核心的党的中央领导集体。③揭开了社会主义改革开放的序幕,中国开始进入了改革开放和社会主义现代化建设的历史新时期。}}}

\subsection{14830-农村改革、四项基本原则和科学评价毛泽东思想}
\textbf{《建国以来党的若干历史问题》:}科学地评价了毛泽东的历史地位,充分论述了作为党的指导思想的伟大意义。毛泽东思想是马克思列宁主义在中国的运用和发展,是被实践证明了的关于中国革命和建设的正确的理论原则和经验总结,是中国共产党集体智慧的结晶。

《决议》从根本上否定了``文化大革命''的理论和实践,对新中国成立以来的重大历史事件做出了基本结论。这个决议还肯定了中共十一届三中全会以来逐步确立的适合中国情况的建设社会主义现代化强国的道路,进一步指明了中国社会主义事业和党的工作继续前进的方向。历史决议的通过,\textbf{标志着党和国家在指导思想上拨乱反正的胜利完成}。

农村家庭联产承包责任制是在土地公有制的基础上,实行集体经营和家庭联产承包经营相结合(即``统分结合'')的经营管理方式。

{邓小平在党的理论工作务虚会上发表讲话,指出:}坚持社会主义道路,坚持人民民主专政,坚持共产党的领导,坚持马克思列宁主义、毛泽东思想这四项基本原则{,``是}实现四个现代化的根本前提''{。}

\subsection{14831-中共十二大、十三大、十四大与南方谈话、十五大}
1982年9月,中国共产党\textbf{第十二次全国代表大会}在北京召开。这是\textbf{邓小平第一次明确提出建设有中国特色的社会主义的命题}。这次大会还制定了全面开创社会主义现代化建设新局面的正确纲领。

1987年10月,召开的中国共产党\textbf{第十三次全国代表大会}。中共十三大的中心任务是加快和深化改革。
大会最突出的贡献是\textbf{系统地阐明了社会主义初级阶段的理论和党在社会主义初级阶段的基本路线},并制定了下一步经济体制改革和政治体制改革的基本任务和奋斗目标。\textbf{中共十三大正式制定了社会主义现代化建设``三步走''的战略部署}。

1992年1月18日至2月21日,邓小平先后视察武昌、深圳、珠海、上海等地,发表重要谈话。即\textbf{{南方谈话}}。南方谈话的主要内容:①计划和市场都是经济手段。②阐明了社会主义本质。③提出了``发展才是硬道理''的重要论断。④提出判断改革开放和各项工作成败得失的``三个有利于''标准。⑤强调加强党的建设。⑥关于社会主义初级阶段的长期性和前途。

\textbf{{南方谈话}}①在重大历史关头,科学地总结了中共十一届三中全会以来党的基本实践和基本经验,明确回答了长期困扰和束缚人们思想的许多重大认识问题。②对整个社会主义现代化建设事业产生了重大而深远的影响。

\textbf{以邓小平南方谈话和中共十四大为标志,改革开放和现代化建设事业进入从计划经济体制向社会主义市场经济体制转变的新阶段,由此打开了中国经济、政治、文化发展的崭新局面}。

{1997年9月12日至18日,}中国共产党第十五次全国代表大会在北京召开{。}把邓小平理论,同马克思列宁主义、毛泽东思想一道确立为中国共产党的指导思想,并写入修改后的《中国共产党章程》{。并确立}党在社会主义初级阶段的基本纲领{。}

\subsection{14848-商品经济}
商品经济是作为自然经济的对立物而产生和发展起来的。

商品经济是指直接以交换为目的经济形式,包括商品生产和商品交换。

社会分工是商品经济存在的前提,是一切商品生产的一般基础。

\subsection{14849-商品经济产生的条件}
\textbf{{商品经济}}是作为\textbf{自然经济的对立物}而产生和发展起来的。商品经济产生和存在有两个基本条件:\textbf{第一是存在社会分工。第二是生产资料和产品属于不同的所有者}。

{由于资本主义经济是高度发达的商品经济,因此,分析以私有制为基础的商品经济的内在矛盾和运动规律,是提示资本主义生产方式矛盾的出发点。}

\subsection{14850-资本主义生产关系的产生}
一、资本主义生产的条件:货币转化为资本和劳动力成为商品

1、
货币转化为资本:作为资本的货币能够带来剩余价值,进而实现价值增值;而普通
货币在交换中不会发生价值增值。

2、劳动力成为商品------货币转化为资本的前提

(1)劳动力成为商品的条件:

第一,劳动者具有人身自由,能够把自己的劳动力当作自己的商品来支配;

第二,劳动者丧失一切生产资料和基本生活来源,除了自身的劳动力以外一无所有,只
有靠出卖劳动力为生。

(2)劳动力商品的特点

劳动不仅是创造价值的唯一源泉,同时也是创造剩余价值的唯一源泉。

\subsection{14851-资本主义的生产过程}
{\textbf{1、 资本主义的生产过程是劳动过程和价值增殖过程的统一}}

2、资本的不同部分在资本主义生产中的作用

\textbf{不变资本:}以生产资料形式存在,在生产过程中不发生价值增值

\textbf{可变资本:}以劳动力形式存在,在生产过程中发生价值增值

区分意义:一是揭露了剩余价值的源泉和资本主义剥削的实质;二是为揭示资本家对工人的剥削程度提供了科学依据。

3、生产剩余价值的两种基本方法------绝对剩余价值和相对剩余价值

(1)绝对剩余价值:在必要劳动时间不变的情况下,

(2)相对剩余价值:在劳动时间不变的前提下,

(3)超额剩余价值:个别企业首先提高劳动生产率

(4)相对剩余价值和超额剩余价值的关系联系:超额剩余价值属于相对剩余价值的范畴,因为它也是通过缩短必要劳动时间相应延长剩余劳动时间而产生的。

4、剩余价值规律的内容:{\textbf{资本主义生产的直接目的和根本动机是为了最大限度地榨取工人创造的剩余价值}}

\subsection{14854-其他理论成果}
{\textbf{{走中国工业化道路的思想}}:毛泽东在《论十大关系》中论述的第一大关系,便是重工业、轻工业和农业的关系。在《关于正确处理人民内部矛盾的问题》一文中,毛泽东明确提出要走一条有别于苏联的中国工业化道路。~{毛泽东提出了以农业为基础,以工业为主导,以农轻重为序发展国民经济的总方针}。}

{关于\textbf{{社会主义发展阶段}}。毛泽东提出,社{会主义又可分为两个阶段,第一个阶段是不发达的社会主义,第二个阶段是比较发达的社会主义。后一个阶段可能比前一阶段需要更长的时间。}}

{关于\textbf{{社会主义现代化建设的战略目标和步骤}}:毛泽东提出,社会主义现代化的战略目标,是要把中国建设成为一个具有{现代农业、现代工业、现代国防和现代科学技术的强国}。为了实现这个目标,应当采取``{两步走}''的发展战略,{第一步建成一个独立的比较完整的工业体系和国民经济体系,第二步全面实现工业、农业、国防和科学技术现代化,使中国走在世界前列}。}

{关于\textbf{{经济建设方针}}:\textbf{{党的八大提出了既反保守又反冒进、在综合平衡中稳步前进的方针。毛泽东多次阐述了统筹兼顾的方针}},强调正确处理国家、集体与个人的关系,生产两大部类的关系,中央与地方的关系,积累与消费的关系,长远利益与当前利益的关系;既要顾全大局,突出重点,也要统筹兼顾,全面安排,综合平衡。同时,也要在自力更生的基础上积极争取外援,开展与外国的经济交流。}

{\textbf{{关于所有制结构的调整}}:{朱德提出了要注意发展手工业和农业多种经营的思想}。\textbf{{陈云提出了``}{三个主体,三个补充''}{的设想}},即{在工商业经营方面,国家经济和集体经济是工商业的主体,一定数量的个体经济是国家经济和集体经济的补充;在生产计划方面,计划生产是工农业生产的主体,按照市场变化在国家计划许可范围内的自由生产是计划生产的补充;在社会主义的统一市场里,国家市场是它的主体,一定范围内的国家领导的自由市场是国家市场的补充}。}

{\textbf{{关于经济体制和运行机制改革}}:{毛泽东}提出了发展商品生产、利用价值规律的思想,认为商品生产在社会主义条件下,还是一个不可缺少的、有利的工具,要有计划地大大地发展社会主义的商品生产。{刘少奇}则提出了使社会主义经济既有计划性又有多样性和灵活性的主张,以及按经济办法管理经济的思想。{陈云}提出了要建立``适合于我国情况和人民需要的社会主义的市场''的思想。此外,{毛泽东还主张企业要建立合理的规章制度和严格的责任制},要{实行民主管理,实行干部参加劳动,工人参加管理,改革不合理的规章制度,工人群众、领导干部和技术人员三结合,即``两参一改三结合''}。{邓小平}提出了关于整顿工业企业,改善和加强企业管理,实行职工代表大会制等观点。}

{关于\textbf{{社会主义民主政治建设}}。{党的八大提出,要进一步扩大民主,健全法制}。毛泽东则进一步提出,``{我们的目标,是想造成一个又有集中又有民主,又有纪律又有自由,又有统一意志、又有个人心情舒畅、生动活泼,那样一种政治局面,以利于社会主义革命和社会主义建设}''。}

{关于\textbf{{科学和教育}}。党提出了``{向科学进军}''的口号,毛泽东强调,``我们的教育方针,应该使受教育者在德育、智育、体育几方面都得到发展,成为有社会主义觉悟的有文化的劳动者。''{刘少奇提出实行``两种劳动制度、两种教育制度'',一种是全日制的劳动制度,全日制的教育制度;一种是半日制劳动制,半日制的教育制度}。}

{{关于}\textbf{{知识分子工作}}{:毛泽东提出,知识分子在革命和建设中都具有重要作用,要建设一支宏大的工人阶级知识分子队伍。周恩来提出了}{知识分子是工人阶级一部分}{的观点,强调要加强和改善党对知识分子和科学文化工作的领导,更好地为社会主义服务。}}

\subsection{14860-中国特色社会主义总布局历史回顾}
{在过去,我们认为计划经济就是社会主义,市场经济就是资本主义。当然这是没有道理的,依据}\textbf{{马克思主义的历史观,看一个社会的社会形态要看其经济基础,看其经济基础就是看其主要的生产关系}}{,而生产关系中所有制才是关键的。故而,看是不是社会主义要看是不是坚持了公有制,
而非资源如何配置和流动。}

{法治不同于法制。法治是一种治理国家的理念,
与之相对应的是人治,法治优于人治。因为其集中人民意志,因为其稳定不易变动,因为其长期性和可靠性。法制只是指法律}{内部的建设情况。我们强调的依法治国,
不仅仅是法律内部完善情况,更}\textbf{{主要的是国家机关和人民都依据法律做事情,也就是让法律在治国理政中发挥出作用}}{。}

{2014年10月15日,习近平总书记在北京召开文艺座谈会,对文化和文艺工作做出了深远的指导。
这也是继毛泽东召开延安文艺座谈会后我党第二次高规格研究讨论文艺问题。
这次讲话涉及五个问题,其一,实现中华民族伟大复兴需要中华文化繁荣兴盛。
其二,
创作无愧于时代的优秀作品,文艺不能在市场中迷失方向,低俗不是通俗,欲望不是希望。其三,坚持以人民为中心的工作导向。}\textbf{{社会效益排在首位,经济效益服从社会效益}}{,市场价值服从社会价值。其四,中国精神是社会主义文艺的灵魂,
其五,加强和改进党对文艺工作的领导。}

{党的十八届五中全会认为,到2020年全面建成小康社会,需要在我国整体消除贫困。我国在扶贫攻坚工作中采取的重要举措,
就是实施精准扶贫方略,找到``贫根'',对症下药,
靶向治疗。我们注重抓}\textbf{{六个精准,即扶持对象精准、项目安排精准、资金使用精准、措施到户精准、因村派人精准、脱贫成效精准}}{,确保各项政策好处落到扶贫对象身上。~}

{党的十八届五中全会提出了实行能源与水资源消耗、建设用地等总量和强度双控行动,提出了探索实行耕地轮耕休耕制度试点,
~提出了实行省以下环保机构监测监察执法垂直管理制度。}

{坚持把节约优先、保护优先、自然恢复为主作为基本方针。坚持把绿色发展、循环发展、低碳发展作为基本途径。坚持把深化改革和创新驱动作为基本动力。坚持把培育生态文化作为重要支撑。坚持把重点突破和整体推进作为工作方式。}

{党的十九届五中全会指出,中国特色社会主义是全面发展的社会主义。进入新时代,要继续夺取中国特色社会主义伟大胜利,就必须按照党的十九大精神的要求,统筹推进``五位一体''总体布局。}

统筹推进``五位一体''总体布局:第一,要准确把握我国经济发展的大逻辑,主动适应把握引领新常态。第二,尊重人民主体地位,保证人民当家作主,是我们党的一贯主张。第三,坚持社会主义先进文化前进方向,坚定文化自信,增强文化自觉,加快文化改革发展,加强社会主义精神文明建设。培育和践行社会主义核心价值观,增强国家文化软实力,建设社会主义文化强国。第四,坚持以民为本,以人为本的执政理念,把民生工作和社会治理工作作为社会建设的两大根本任务,高度重视,大力推进,使改革发展成果更多更公平惠及全体人民。第五,建设生态文明是关系人民福祉,关乎民族未来的大计,是实现中华民族伟大复兴的中国梦的重要内容。\\



\section{[史纲]中国近现代史纲要}


\subsection{14861-帝国主义的侵略}
{\textbf{鸦片战争}}\\
\textbf{{性质:}}{侵略战争~~}\\
\textbf{{结果:}}{战败,《中英南京条约》~~}\\
\textbf{{原因:}}{社会制度的腐败是}\textbf{{{根本原因}}}{。经济技术的落后是近代中国反侵略战争失败的}\textbf{{{另一个重要原因}}}{。~~}\\
\textbf{{意义:}}{中国开始半殖民地半封建社会,}\textbf{{{中国近代史的起点}}}{。~~}\\
\textbf{{人物:}}{林则徐}\textbf{{{(睁眼看世界第一人)}}}{,魏源(《海国图志》,师夷长技以制夷)。~~}\\
\textbf{{相关内容:}}{虎门销烟,三元里抗英。~}\\
{~}\\
{\textbf{中日甲午战争}}{~}{~}\\
\textbf{{性质:}}{侵略~~}\\
\textbf{{结果:}}{战败,马关条约~~}\\
\textbf{{原因:}}{社会制度的腐败是根本原因。经济技术的落后是近代中国反侵略战争失败的另一个重要原因。~~}\\
\textbf{{意义:}}{半殖民地半封建社会程度加深;}\textbf{{{标志洋务运动失败}}}{,}\textbf{{{民族意识普遍觉醒}}}{,}\textbf{{{使西方瓜分中国达到高潮}}}{。~~}\\
{~~~}\\
{\textbf{八国联军}}{~~}\\
\textbf{{性质:}}{侵略战争~~}\\
\textbf{{结果:}}{战败,《辛丑条约》~~}\\
\textbf{{原因:}}{社会制度的腐败是根本原因。经济技术的落后是近代中国反侵略战争失败的另一个重要原因。~~}\\
\textbf{{意义:}}{中国彻底沦为半殖民地半封建社会。}

\subsection{14862-近代社会的性质}
{社会性质}{:鸦片战争以后,随着外国资本---帝国主义的入侵,中国逐渐沦为}{半殖民地半封建社会}{,这是}{近代中国最基本的国情}{。}{}

{主要矛盾}{:}{帝国主义和中华民族的矛盾,封建主义和人民大众的矛盾是近代以来的主要矛盾。其中最主要的矛盾是帝国主义和中华民族的矛盾}{。这两对主要矛盾及其斗争贯穿整个中国半殖民地半封建社会的始终,并对中国近代社会的发展变化起着决定性的作用。~}{}

{历史任务}{:近代以来中华民族面临的两大历史任务是:}{争取民族独立、人民解放}{;}{实现国家富强、人民富裕}{。两者的关系:前一个任务为后一个任务扫除障碍,创造必要的前提;后一个任务是前一个任务的最终目的和必然要求。}

\subsection{14863-反抗外族侵略的斗争}
{\textbf{{三元里人民的抗英斗争}}}{,是中国近代史上中国人民}{\textbf{{第一次}}}{大规模的反侵略武装斗争。}

{太平天国农民战争后期,太平军曾多次重创英、法侵略军和外国侵略者指挥的洋枪队``常胜军''、``常捷军''。}

{台湾人民与台湾总兵刘永福所率领的}{黑旗军}{共同抗击日本侵略。在中越边境镇南关,老将冯子材身先士卒,大败法军,取得}{镇南关大捷}{。}

{义和团运动在粉碎帝国主义列强瓜分中国的斗争中,发挥了重大的历史作用。}

{在}{1895}{年,严复就写了}{《救亡决论》}{一文,响亮地喊出了``救亡''的口号。在甲午战争后,严复翻译了}{《天演论》(}{1898}{年正式出版)。他用``物竞天择''、``适者生存''的社会进化论思想,为这种危机意识和民族意识提供了理论根据。}

{孙中山1894年11月在创立革命团体兴中会时就喊出了``}{振兴中华}{''这个时代的最强音。}

\subsection{14864-反侵略战争的失败和民族意识的觉醒}
{第一,社会制度的腐败是根本原因。}

{第二,经济技术的落后是近代中国反侵略}战争失败的另一个重要原因。

{中日甲午战争以后,中国人才开始有了普遍的民族意识的觉醒。}

\subsection{14866-洋务运动}
{\textbf{{性质}:}地主阶级探索。}

{\textbf{{事件}:}{兴办近代企业;建立新式海陆军;创办新式学堂,派遣留学生}。}

{\textbf{{结果}:}失败(\textbf{{甲午战争战败为标志}})。}

{\textbf{{原因}:{封建性,依赖性,腐朽性}}。}

{\textbf{{意义}:}促进民族资本主义发展,培养人才,社会风气变化。}

{\textbf{{人物}:}{奕訢、曾国藩、李鸿章、左宗棠、张之洞、冯桂芬(最早)。}}

{\textbf{{相关内容}:{中学为体,西学为用}}{;}\textbf{{自强求富}}{;}{稍分洋商之利。}}

\subsection{14867-维新运动}
{\textbf{{性质}:}{资产阶级维新派改良},探索。}

{\textbf{{事件}:}论战:\textbf{{要不要变法}}。\textbf{{要不要兴民权、设议院,实行君主立宪}}。\textbf{{要不要废八股、改科举和兴西学}}。}

{\textbf{{结果}:}在守旧势力打压下失败。}

{\textbf{{原因}:}民族资产阶级\textbf{{力量弱小}};\textbf{{维新派的局限性}}:{一是不敢否定封建主义。二是对帝国主义抱有幻想。三是惧怕人民群众};\textbf{{以慈禧太后为首的强大的守旧势力的反对}}。}

{\textbf{{意义}:{爱国救亡}}运动;资产阶级性质的\textbf{{政治改良运动}};\textbf{{思想启蒙运动}}。}

{\textbf{{人物}:}{康有为(《新学伪经考》、《孔子改制考》)、梁启超(《变法通议》)、谭嗣同(《仁学》)、严复(《天演论》)}}

\subsection{14868-辛亥革命的性质与早期工作}
{资产阶级革命派的骨干是一批{资产阶级、小资产阶级知识分子}。这些青年知识分子,成为了辛亥革命的中坚力量。}

{孙中山到檀香山组织第一个\textbf{{资产阶级革命团体}{------}{兴中会}},提出了``{驱除鞑虏,恢复中华,创立合众政府}''的革命纲领,并筹划发动反清起义。}

{{章炳麟发表了《驳康有为论革命书》,邹容出版了《革命军》,陈天华出版了《警世钟》、《猛回头》两本小册子}{,号召人民奋起革命。在资产阶级革命思想的传播过程中,资产阶级革命团体也在各地次第成立,其中重要的有}{华兴会、科学补习所、光复会、岳王会}{等。}\textbf{{孙中山和黄兴、宋教仁等人在日本东京成立中国同盟会}}{,同盟会以}\textbf{{《民报》为机关报}}{,并确定了革命纲领。这是近代中国}\textbf{{第一个领导资产阶级革命的政党}}{,它的成立标志着中国资产阶级民主革命进入了一个新的阶段。}}

\subsection{14869-理论工作:三民主义和革命改良的辩论}
{三民主义学说和资产阶级共和国方案:同盟会的政治纲领是``\textbf{{驱除鞑虏,恢复中华,创立民国,平均地权}}''。孙中山将同盟会的纲领概括为\textbf{{三大主义}},即\textbf{{民族主义、民权主义、民生主义}},后被称为\textbf{{三民主义}}。}

{\textbf{{民族主义}}{,即}\textbf{{民族革命}}{,包括``}\textbf{{驱除鞑虏,恢复中华}}{''两项内容。}}

{\textbf{{民权主义}}{即}\textbf{{政治革命}}{,内容是``}\textbf{{创立民国}}{'',即建立资产阶级民主共和国。}}

{\textbf{{民生主义}}{即}\textbf{{社会革命}}{,指的是``}\textbf{{平均地权}}{''。}}

{{它初步描绘出中国还不曾有过的资产阶级共和国方案,是一个比较完整而明确的资产阶级民主革命纲领;对推动革命的发展产生了重大而积极的影响;但它并不是一个彻底的资产阶级民主革命纲领}{。}}

{关于革命与改良的辩论 :}

{\textbf{{论战内容}}{:}{①要不要以革命手段推翻清王朝。②要不要推翻帝制,实行共和。③要不要社会革命}{。}}

{\textbf{{论战意义}}{:}{通过这场论战,划清了革命与改良的界限,传播了民主革命思想,促进了革命形势的发展}{。但这场论战也暴露了革命派在思想理论方面的弱点。}}

\subsection{14870-武昌起义与封建帝制的覆灭}
{{1906年12月,萍、浏、醴起义是同盟会成立后发动的第一次武装起义}。{辛亥革命前影响最大的,是广州起义}。\textbf{{武昌起义吹响了辛亥革命的号角}}。1912年1月1日,{孙中山在南京宣誓就任临时大总统,改国号为``中华民国''},定1912年为民国元年,并正式成立中华民国临时政府。\textbf{{南京临时政府是一个资产阶级共和国性质的革命政权}}。从政权的组成人员看,资产阶级革命派在这个政权中占有领导和主体的地位。从南京临时政府制定的政策看,各项政策措施集中代表和反映了中国民族资产阶级的愿望和利益,在相当程度上也符合广大中国人民的利益。}

{1912年3月,临时参议院颁布\textbf{{《中华民国临时约法》}}。\textbf{{这是中国历史上第一部具有资产阶级共和国宪法性质的法典}}。}

{\textbf{{辛亥革命是一次比较完全意义上的资产阶级民主革命。它成了}{20}{世纪中国第一次历史性巨变}}。}

{\textbf{{历史意义}}{:}\textbf{{辛亥革命是一次比较完全意义上的资产阶级民主革命}}{。}\textbf{{①沉重打击了中外反动势力。②结束了统治中国两千多年的封建君主专制制度,建立了中国历史上第一个资产阶级共和政府,使民主共和的观念开始深入人心。③给人们带来一次思想上的解放。④辛亥革命促使社会经济、思想习惯和社会风俗等方面发生了新的积极变化。⑤推动了亚洲各国民族解放运动的高涨。}}}

\subsection{14878-土地改革与第二条战线}
{中共中央发出\textbf{{《关于清算、减租及土地问题的指示》}}(史称《\textbf{{五四指示}}》),\textbf{{标志着党把在抗日战争时期实行的减租减息政策改变为实现}{``}{耕者有其田}{''}{的政策}}。}

{\textbf{{《中国土地法大纲》:}{``}{废除封建性及半封建性的土地制度,实现耕者有其田的制度}{''}},毛泽东总结正反两方面的经验,提出了\textbf{{完整的土地改革的总路线}},即\textbf{{依靠贫雇农,团结中农,有步骤、有分别地消灭封建剥削制度,发展农业生产}}。同时强调土地改革必须注意的两个基本原则:\textbf{{第一,必须满足贫雇农的土地要求;第二,必须坚决地团结中农,不要损害中农的利益}}。}

{\textbf{{1.~}{土地制度改革,是从根本上摧毁中国封建制度根基的社会大变革。}{2.~}{经过土地改革运动,人民解放战争获得了源源不断的人力、物力的支援。3.~}{为打败蒋介石、建立新中国奠定了深厚的群众基础。}}}

{{第二条战线是在国民党统治区,以学生运动为先导的人民民主运动}。}

{昆明学生发动了以``反对内战,争取自由''为主要口号的一二·一运动。}

{为抗议驻华美军强暴北京大学先修班一女学生,抗议驻华美军暴行的运动(史称{抗暴运动}、``{一二·三〇运动}'')由此掀起。}

{南京、北平等地爆发了反饥饿、反内战运动(史称``{五二〇运动}'')。}

{第二条战线虽然是辅助性的,但仍然是十分重要的战线。}

\subsection{14880-新民主主义革命胜利的原因和基本经验}
{\textbf{{中共七届二中全会}}:
①提出了迅速夺取全国胜利的方针。②党的工作重心必须由乡村转移到城市。
③规定了党在全国胜利后在政治、经济、外交方面应当采取的基本政策。④他提出了``两个务必''的思想,即``{务必使同志们继续地保持谦虚、谨慎.不骄、不躁的作风,务必使同志们继续地保持艰苦奋斗的作风}''。}

{1949年9月21日,{中国人民政治协商会议第一届全体会议}在北平隆重开幕。会议通过了\textbf{{《中国人民政治协商会议共同纲领》}}。\textbf{{《共同纲领》在当时是全国人民的大宪章,起着临时宪法的作用}}。}

{新民主主义革命\textbf{{胜利原因}}:{1.}~{中国革命的发生,有着深刻的社会根源和雄厚的群众基础。2.~中国革命之所以能够走上胜利发展的道路,是由于有了中国工人阶级的先锋队------中国共产党的领导。3.~中国革命之所以能够赢得胜利,同国际无产阶级和人民群众的支持也是分不开的}。}

{\textbf{{基本经验}}:``\textbf{{统一战线,武装斗争,党的建设,是中国共产党在中国革命中战胜敌人的三个法宝}},三个主要的法宝。''}

{\textbf{{统一战线中存在着两个联盟}}:{一个是劳动者的联盟},主要是工人、农民和城市小资产阶级的联盟;{一个是劳动者与非劳动者的联盟},主要是劳动者与民族资产阶级的联盟,有时还包括与一部分大资产阶级的暂时的联盟。前者是基本的、主要的;后者是辅助的、同时又是重要的。{必须坚决依靠第一个联盟,争取建立和扩大第二个联盟}。~{巩固和扩大统一战线的关键,是坚持工人阶级及其政党的领导权}。为此,必须率领同盟者向共同的敌人作坚决的斗争并取得胜利;必须对被领导者给以物质福利,至少不损害其利益,同时对被领导者给以政治教育;必须对同工人阶级争夺领导权的资产阶级采取又联合、又斗争的政策。}

{\textbf{{中国的武装斗争实质上是工人阶级领导的农民战争}}。\textbf{{武装斗争是中国革命的特点和优点之一}}。{坚持党对军队的绝对领导是建设新型人民军队的根本原则}。{全心全意为人民服务是人民军队的唯一宗旨。它集中体现了人民军队的本质,是人民军队立于不败之地的根本所在}。}

{{中国共产党要领导革命取得胜利,必须不断加强党的思想建设、组织建设和作风建设}{。}{党内无产阶级思想和非无产阶级思想之间的矛盾成为党内思想上的主要矛盾}{。要建设一个广大群众性的、马克思主义的无产阶级政党,是一项艰巨的任务,也}{是一项伟大的工程}{。}\textbf{{加强党的建设,必须把思想建设始终放在首位}}{,克服党内的非无产阶级思想。党在领导新民主主义革命的过程中,}\textbf{{逐步形成了理论和实践相结合的作风、和人民群众紧密地联系在一起的作风以及自我批评的作风}}{,这是}\textbf{{中国共产党区别于其他任何政党的显著标志}}{。}}

\subsection{14881-中共八大}
{\textbf{主要内容}:\textbf{}}

{\textbf{1.~中共八大正确分析了社会主义改造完成后中国社会的主要矛盾和主要任务。\\
}}

{\textbf{2.~在经济建设上,大会坚持既反保守又反冒进即在综合平衡中稳步前进的方针。陈云提出``三个主体、三个补充''的思想。}}

{\textbf{3.~在政治建设上,提出要扩大社会主义民主、健全社会主义法制。\\
}}

{\textbf{4.~在执政党建设上,健全党内民主集中制,坚持集体领导制度,反对个人崇拜,发展党内民主和人民民主}。}

{\textbf{{意义}}{:}}

{1.~{中共八大提出了许多创造性的正确思想,反映了党对社会主义的初步认识。}}

{2.~{事实证明,中共八大所制定的政治路线和组织路线都是正确的,它为新的历史时期社会主义事业的发展和党的建设指明了方向。}}

\subsection{14882-《论十大关系》和《关于正确处理人民内部矛盾的问题》}
{\textbf{{《论十大关系》背景}}:以苏为鉴,走中国自己的社会主义建设道路。}

{\textbf{{《论十大关系》主要内容}}{:毛泽东在1956年作了《论十大关系》的报告,概括提出了十大关系。这十大关系围绕}\textbf{{一个基本方针}}{,即}\textbf{{调动国内外一切积极因素,为社会主义服务}}{。}}

{\textbf{{意义}}{:}}

{\textbf{{1.~}{这是以毛泽东为主要代表的中国共产党人开始探索中国自己的社会主义建设道路的标志。}}{\\
}}

{2.~它在新的历史条件下从经济方面(这是主要的)和政治方面提出了新的指导方针,为中共八大的召开作了理论准备。}

{\textbf{{《关于正确处理人民内部矛盾的问题》的主要内容}}:}

{1. 关于社会主义社会两类不同性质的社会矛盾。}

{2. 关于社会主义社会的基本矛盾。~}

{\textbf{{《关于正确处理人民内部矛盾的问题》的意义}}:}

{1. 这是一篇重要的马克思主义文献。}

{2.
它创造性地阐述了社会主义社会矛盾学说,是对科学社会主义理论的重要发展。}

{3. 对中国社会主义事业具有长远的指导意义。}

{\textbf{{第二次结合}:}{随着苏共二十大对于苏联模式弊端的进一步披露,中国共产党人决心走自己的路,}{开始探索适合中国国情的社会主义建设道路}{。}{毛泽东提出的关于实行马克思主义同中国实际的``第二次结合''的任务}{,为探索适合中国情况的社会主义建设道路,提供了基本的指导原则。}}

\subsection{14883-整风与反右斗争}
{1957年4月27日,中共中央下发《关于整风运动的指示》,指出:由于党在全国范围内处于执政地位,有必要在全党进行一次{反对官僚主义、宗派主义和主观主义}的整风运动。}

{这场运动采取开门整风的形式。毛泽东在1957年7月写的《一九五七年夏季的形势》一文中提出:{``要造成一个又有集中又有民主,又有纪律又有自由,又有统一意志又有个人心情舒畅、生动活泼,那样一种政治局面}''({``六又''政治局面})。进6月8日,中共中央发出组织力量反击右派分子进攻的党内指示,《人民日报》同日发表题为《这是为什么?》的社论。一场全国规模的群众性{反右派运动}全面展开。}

{{对反右运动的评价}{:①对极少数右派分子的进攻实行坚决反击,是完全正确和必要的。但是反右派斗争被严重地扩大化了。}}



\section{[思修法基]思想道德修养与法律基础}


\subsection{14886-树立理想信念:信仰与理想}
{\textbf{{理想的含义与特征}}:理想作为一种社会意识和精神现象,是人类社会实践的产物。理想是一定社会关系的产物。{\textbf{理想源于现实,又超越现实}}。理想是多方面和多类型的。理想不仅具有现实性,而且具有{\textbf{预见性}}。}

{\textbf{{信念的含义与特征}}:信念同理想一样,也是人类特有的一种精神现象。信念具有高于一般认识的稳定性。信仰是信念最集中、最高的表现形式。}

{\textbf{{理想信念的作用}}:指引人生的{\textbf{奋斗目标}}。提供人生的\textbf{{前进动力}}。提高人生的{\textbf{精神境界}}。}

{\textbf{{树立马克思主义的科学信仰:}}马克思主义作为我们党和国家的根本指导思想,是由马克思主义严密的科学体系、鲜明的阶级立场和巨大的实践指导作用决定的。马克思主义是科学的又是崇高的。马克思主义具有持久的生命力。马克思主义以改造世界为己任。}

{\textbf{{树立中国特色社会主义的共同理想}}:{\textbf{建设和发展中国特色社会主义、实现中华民族伟大复兴}},是现阶段我国各族人民的共同理想。}

{\textbf{{实现共同理想需要}}:实现共同理想需坚定对中国共产党的\textbf{{信任}}。坚定走中国特色社会主义道路的\textbf{{信念}}。坚定实现中华民族伟大复兴的\textbf{{信心}}。}

{\textbf{{中国梦}}:{\textbf{实现中华民族伟大复兴}},就是中华民族近代以来最伟大的梦想。中国梦的内涵是实现\textbf{{国家富强、民族振兴、人民幸福}}。实现中国梦,必需走\textbf{{中国道路}},弘扬\textbf{{中国精神}},凝聚\textbf{{中国力量}}。}

{\textbf{{个人理想与社会理想}}{:个人理想与社会理想的关系实质上是个人与社会的关系在理想层面上的反映。个人与社会有机地联系在一起,二者相互依存,相互制约,共同发展。社会理想与个人理想也不是互相孤立的存在,它们之间既相互联系、相互影响,又相互区别、相互制约。}\textbf{{社会理想决定、制约着个人理想;社会理想又是个人理想的凝炼和升华}}{。}}

\subsection{14887-在实践中化理想为现实}
\textbf{{正确理解理想与现实的关系}}

{{辩证看待理想与现实的矛盾}}{,现实是理想的基础,理想是未来的现实。}

{{实现理想的}{长期性、艰巨性和曲折性}}{:需正确对待实现理想过程中的顺境与逆境。}

{艰苦奋斗是实现理想的重要条件。}

\textbf{{坚持个人理想与社会理想的统一}}

{社会理想规定、指引着个人理想。个人理想的实现,必须以社会理想的实现为前提和基础。}

{社会理想是对社会成员个人理想的凝练和升华。}

{\textbf{为实现中国梦注入青春能量}}

{立志当高远。}

{立志做大事。}

{立志需躬行。}

{伟大出平凡。}

\subsection{14889-新时期的爱国主义与爱国主义的时代价值}
\textbf{{{爱国主义的基本要求}}}{:爱祖国的大好河山;爱自己的骨肉同胞(对人民群众感情的深浅程度,是检验一个人对祖国忠诚程度的}\textbf{{{试金石}}}{)。爱祖国的灿烂文化。}

\textbf{{{爱国主义的时代价值}}}{:维护祖国统一和民族团结的纽带;实现中华民族伟大复兴的动力;实现人生价值的源泉。}

{\textbf{{{新时期的爱国主义}}}{:对于大学生来说,在如何把握经济全球化趋势与爱国主义的相互关系问题上,需要看重树立以下三个观念。}}

{1.人有地域和信仰的不同,但报效祖国之心不应有差别。}

{2.科学没有国界,但科学家有祖国。}

{3.经济全球化是世界经济发展的必然趋势,但不等于全球政治、文化一体化。}

{{爱国主义与爱社会主义具有一致性;爱国主义与拥护祖国统一也是一致的}。}

\subsection{14890-民族精神与时代精神}
{\textbf{{民族精神}{(中国精神)}:}民族精神是一个民族赖以生存和发展的精神支柱。中华民族形成了\textbf{{以爱国主义为核心}}的{\textbf{团结统一、爱好和平、勤劳勇敢、自强不息}}的伟大民族精神。中华民族精神是社会主义核心价值体系的重要组成部分。}

{\textbf{{时代精神}}{:}\textbf{{改革创新是时代精神的核心}}{。改革创新是中华民族进步的灵魂,是我国兴旺发达的不竭动力,是中国共产党永葆生机的源泉。}}

\subsection{14897-道德及其作用}
{{\textbf{道德是上层建筑的一部分,决定于经济基础,并且带有阶级属性}}。}

{{道德的功能集中表现为,它是处理个人与他人、个人与社会之间关系的行为规范及实现自我完善的一种重要精神力量。}道德的主要的功能是{\textbf{认识功能、规范功能(2016新增)和调节功能}}{。道德}\textbf{{最突出也是最重要}}{的社会功能是}\textbf{{调节功能}}{。道德还具有}其他方面的功能,如{\textbf{导向功能、激励功能、辩护功能、沟通功能}}{等。在阶级社会中,}道德是{\textbf{阶级斗争的重要工具}}{。}道德的发展和进步成为\textbf{{衡量社会文明程度}}的{\textbf{重要尺度}}{。}}



\section{形势与政策以及当代世界经济与政治}


\subsection{14902-世界总体格局的演变}
{{一、}{\textbf{两极格局解体~}}
\textbf{{世界经济格局}}{:}
{第一阶段:从战后初期到20世纪60年代末,主要表现为美国称霸世界经济领域。}
{第二阶段:20世纪70年代后世界经济向多极化方向发展。}
{第三阶段:自20世纪80年代末期开始,三大区域经济集团化加快发展。~}
}

{世界政治格局:雅尔塔体制的形成、冷战的开始、两极政治格局的动摇、两极格局的终结。}

{{}
{二、}{\textbf{世界多极化}}
{\textbf{世界多极化长期性的原因}}{:}
{第一,美国的霸权主义和构建单极世界的图谋,是多极化趋势发展的最大障碍。}
{第二,世界上冷战思维的继续、南北贫富差距的扩大,以及民族分裂和宗教纠纷等,也会对多极化趋势产生各种干扰和冲击。}
{第三,多极化格局的形成是世界各种力量重新组合和利益重新分配的过程,由此将产生多种不确定因素,世界多极化进程将充满矛盾和斗争。}

{三、}{\textbf{经济全球化}}
{表现:}
{第一,国际贸易已成为世界经济发展中不可缺少的组成部分,是国际交往中最活跃的一环。}
{第二,国际投资,特别是发达国家间的相互投资越来越频繁,资本流动已经国际化。}
{第三,国际金融活动规模空前,大大超过了全世界生产和商品交易。}
{第四,跨国公司遍布全球,产品的国际化水平越来越高。}
{第五,全球贸易规则日趋统一。}
{经济全球化有利于促成各国之间生产要素的合理流动,形成优势互补,推动世界经济的发展。经济全球化对各国的作用是不一样的。}
{\textbf{对于发展中国家来说,它既是机遇,又是挑战}}
{。发展中国家既要适应经济全球化趋势,又要趋利避害。当今世界需要的是各国``共赢''、平等、公平、共存的经济全球化。~}
{经济全球化与区域经济集团化的关系}{:两者是并行不悖(对立统一)的。}



{四、}{\textbf{区域经济一体化~}}
{四个阶段:第一,贸易一体化;第二,要素一体化;第三,政策一体化;第四,完全一体化。}
{类型:第一,优惠贸易安排。第二,自由贸易区。第三,关税同盟。第四,共同市场。第五,经济同盟。第六,完全经济一体化。}
{主要的组织:欧洲联盟、北美自由贸易区、亚太经济合作组织等。}
\textbf{2014年11月,亚太经合组织第22次领导人非正式会议在北京举行,主题是``共建面向未来的亚太伙伴关系''
这是自2001年后, 中国再一次成为APEC
的东道主。习近平指出,中国将以此次会议为契机,面向未来,谋求建主更紧密的伙伴关系,深化务实合作,推动亚太经合组织发挥更大引领作用,勾画亚太长远发展愿景。)~~}}
\subsection{16509-加强和规范党内政治生活加强和规范党内政治生活(新增)}
考点名称:加强和规范党内政治生活加强和规范党内政治生活(新增)

难度:0.4

重要性:★☆☆☆☆

常考题型:选择题

考点内容:

办好中国的事情关键在党,关键在党要管党,从严治党。党要管党从严治党必须从党内政治生活管起,从严治党必须从党内政治生活严起。

新形势下加强和规范党内政治生活,必须以党章为根本遵循,重点是各级领导机关和领导干部。

坚定理想信念是开展党内政治生活的首要任务。

坚决维护党中央权威,保证全党令行禁止

纪律严明,是全党统一意志,统一行动步调一致前进的重要保障,是党内政治生活的重要内容纪律严明,是全党统一意志,统一行动步调一致前进的重要保障,是党内政治生活的重要内容。

坚持全心全意为人民服务的根本宗旨,保持党同人民群众的血肉联系是加强和规范党内政治生活的根本要求。

民主集中制是党的根本组织原则,是党内政治生活正常开展的重要制度保障。

党内民主是党的生命,是党内政治生活积极健康的重要基础。

坚持正确选人用人导向,是严肃党内政治生活的组织保证。

建设廉洁政治,坚决反对腐败,是加强和规范党内政治生活的重要任务。

\subsection{16515-“两个一百年”,“中国梦”(补充)}
考点名称:实现两个一百年奋斗目标,实现中华民族伟大复兴的中国梦(补充)

难度:0.7

重要性:★★★☆☆

常考题型:选择题,分析题

曾命题年份:2013年选择题

考点内容:

党的十八大以来,以习近平同志为核心的党中央,团结带领全国各族人民,紧紧围绕实现两个一百年奋斗目标和中华民族伟大复兴的中国梦,举旗定向谋篇布局,攻坚克难,强基固本,开辟了治国理政新境界,开创了党和国家事业发展新局面,得到广大干部群众的衷心拥护,在国际社会产生重大影响。

习近平总书记系列重要讲话,作为中国特色社会主义理论体系最新成果,作为马克思主义中国化最新理论成果,作为指导具有许多新的历史特点的伟大斗争的鲜活的马克思主义,是新的历史条件下我们党治国理政的行动纲领,是我们凝聚力量攻坚克难的强大思想武器,是实现两个100年奋斗目标,实现中华民族伟大复兴中国梦的行动指南

\subsection{18794-十月国内时政}
{1、{9月30日晚,国务院在人民大会堂举行国庆招待会,}\textbf{{热烈庆祝中华人民共和国成立六十八周年}}{。}}

{2、}{{}{10月2日电,}\textbf{{国家质检总局近日派出首批赴安哥拉传染病监测哨点工作小组,标志着境外传染病监测哨点开始常态化运作}}{。}}

{{3、}{习近平10月3日致信祝贺中国人民大学建校80周年,向全体师生员工和广大校友致以热烈的祝贺。习近平强调,}\textbf{{当前,党和国家事业正处在一个关键时期,我们对高等教育的需要比以往任何时候都更加迫切,对科学知识和卓越人才的渴求比以往任何时候都更加强烈。希望中国人民大学以建校80周年为新的起点,围绕解决好为谁培养人、培养什么样的人、怎样培养人这个根本问题,坚持立德树人,遵循教育规律,弘扬优良传统,扎根中国大地办大学,努力建设世界一流大学和一流学科,为我国高等教育事业繁荣发展,为实现``两个一百年''奋斗目标、实现中华民族伟大复兴的中国梦作出新的更大贡献}}{。}}

{{4、}{历经50多天,我国``深海勇士''号载人潜水器在南海完成全部海上试验任务后,10月3日随``探索一号''母船顺利返航三亚港。通过本次海试,进一步全面检验和验证了4500米载人潜水器的各项功能和性能,}\textbf{{海试的成功标志着研制工作取得圆满成功}}{。}}

{{5、10月6日,比亚迪北美工厂三期扩建竣工仪式在美国南加州沙漠小城兰卡斯特举行。这座比亚迪纯电动大巴工厂,}\textbf{{是美国首家由中国独资的大巴工厂,也是北美地区最大的电动大巴工厂}}{,占地近4.1万平方米,相当于6个标准足球场大小。}}

{6、}\textbf{{人力资源和社会保障部10月8日召开第四十四届世界技能大赛行前动员会}}{。10月14日至19日,第四十四届世界技能大赛将在阿联酋阿布扎比举办,中国代表团将派出52名选手参加47个项目的比赛。}

{{7、北京时间10月9日12时13分,}\textbf{{我国在酒泉卫星发射中心用长征二号丁运载火箭,将委内瑞拉遥感卫星二号送入预定轨道,发射任务获得圆满成功}}{。}}

{{8、中国科学院国家天文台10月10日召开新闻发布会宣布,被誉为``中国天眼''的500米口径球面射电望远镜(FAST)经过一年紧张调试,已实现指向、跟踪、漂移扫描等多种观测模式的顺利运行,并确认了多颗新发现的脉冲星。}\textbf{{这是我国天文望远镜首次发现脉冲星}}{。}}

{{9、在10月10日召开的国新办新闻发布会上,国家统计局主要负责同志介绍,党的十八大以来,经济社会发展取得新的辉煌成就。经济运行``一枝独秀''------2013年至2016年,我国GDP年均增长7.2\%,高于同期世界2.6\%和发展中经济体4\%的平均增长水平。综合国力不断增强------2016年,国内生产总值达到74万亿元;今年8月末国家外汇储备达3.09万亿美元,}\textbf{{继续保持世界首位,一批具有标志性意义的科技成果举世瞩目}}{。}}

{{10、经中央军委批准,中央军委办公厅日前印发《中国人民解放军军营开放办法》(以下简称《办法》)。}\textbf{{这是新形势下发挥军队资源优势推动全民国防教育普及深入的重要举措,为各部队规范有序组织军营向社会开放提供了基本遵循。}}{《办法》依据国防教育法和党中央、国务院、中央军委《关于加强新形势下国防教育工作的意见》,认真贯彻落实习主席关于加强国防教育的重要指示精神,总结近年来一些部队组织军营向社会开放的实践经验,对军营向社会开放的组织领导、审批程序和权限、开放时机、开放内容等作了明确规范。}}

{{11、10月13日,}\textbf{{在阿联酋阿布扎比举行的世界技能组织全体成员大会一致决定,2021年第46届世界技能大赛在中国上海举办}}{。}}

{{12、10月13日上午,}\textbf{{中央军委给航天员大队记一等功庆功大会在京举行。会上,宣读了中央军委主席习近平签署的通令,并向航天员大队颁发了奖状}}{。}}

{{13、从中国商飞公司获悉:10月10日至14日,在山东东营机场,}\textbf{{我国自主研发的北斗卫星导航系统首次在我国完全自主设计并制造的支线客机ARJ21---700飞机103架机上进行了测试试飞,试验取得圆满成功}}{。}}

{14、}{\textbf{{第四十四届世界技能大赛10月14日晚在阿联酋首都阿布扎比拉开帷幕。来自68个世界技能组织成员国家和地区的1260余名选手将在运输与物流、结构与建筑技术、制造与工程技术、信息与通信技术、创意艺术与时尚、社会与个人服务六大类的52个比赛项目中展开角逐}}{。}}

{{15、10月15日,从国家杂交水稻工程技术研究中心获悉,}\textbf{{中国工程院院士袁隆平团队选育的超级杂交稻品种``湘两优900(超优千号)'',15日在河北省硅谷农科院超级杂交稻示范基地,通过了该省科技厅组织的测产验收。平均亩产1149.02公斤,即每公顷17.2吨。创造了世界水稻单产的最新、最高纪录}}{。}}

{{16、10月16日,由中车长春轨道客车股份有限公司研制的首批美国波士顿橙线地铁车顺利下线,预计将在今年12月份运抵美国。}\textbf{{这是我国轨道交通装备首次出口美国。该车是国内首批具有完全自主知识产权的美标地铁车,在材料、空间等方面有诸多创新}}{。}}

{{17、从国家邮政局获悉:}\textbf{{十八大以来,我国快递业务量、业务收入分别增长5.5倍和3.8倍,快递业务量年均增长53\%,已占全球四成份额,连续三年稳居世界第一。目前,我国快递从业法人企业达2万家,从业人数超过200万}}{。}}

{{18、10月16日,}\textbf{{我国西北地区首条高速公路螺旋隧道------卧龙沟一号隧道全线贯通,标志着高速公路螺旋隧道施工技术首次在我国高寒、高海拔地区运用成功}}{。}}

{{19、10月18日,}\textbf{{中国共产党第十九次全国代表大会在北京人民大会堂开幕。习近平代表第十八届中央委员会向大会作报告。习近平指出,经过长期努力,中国特色社会主义进入了新时代,这是我国发展新的历史方位。这标志着我国社会主要矛盾已经转化为人民日益增长的美好生活需要和不平衡不充分的发展之间的矛盾。新时代中国特色社会主义思想明确坚持和发展中国特色社会主义,总任务是实现社会主义现代化和中华民族伟大复兴,在全面建成小康社会的基础上,分两步走在本世纪中叶建成富强民主文明和谐美丽的社会主义现代化强国}}{。}}

{{20、10月19日,}\textbf{{中国商飞公司在山东东营向成都航空公司交付第三架ARJ21新支线喷气客机,这是ARJ21-700飞机首次通过售后回租的形式交付的第一架飞机,也是ARJ21-700飞机获得首张生产许可证(PC证)。中国商飞公司总装制造中心批产后交付首架飞机,标志着我国支线喷气客机正向批产化稳步迈进}}{。}}

{{21、}{央行货币政策司发布《人民币国际化报告》(2017年,下称《报告》)指出,将进一步完善人民币汇率市场化形成机制,逐步增强汇率弹性,保持人民币在全球货币体系中的稳定地位。而根据上述报告,}\textbf{{越来越多的央行和货币当局把人民币作为其储备资产,人民币``入篮''满一年之际,已有60多个国家和地区将人民币纳入外汇储备。纲领,具有划时代的里程碑意义}}{。}}

{{22、}}{\textbf{{第四十四届世界技能大赛10月19日在阿联酋阿布扎比闭幕}}{。平均年龄不到21岁的52名中国``年轻工匠''共在本次大赛中参加了47个项目比赛,获得了15枚金牌、7枚银牌、8枚铜牌和12个优胜奖,中国队位列金牌榜首位,取得历史最好成绩。}}

{{23、10月20日从国家食品药品监督管理总局获悉,}\textbf{{由我国独立研发、具有完全自主知识产权的创新性重组疫苗产品``重组埃博拉病毒病疫苗(腺病毒载体)''的新药注册申请19日获食药监总局批准。这是我国首个重组埃博拉病毒病疫苗获批注册}}{。}}

{{24、从中国国家航天局和法国驻华大使馆10月20日联合举行的媒体见面会上获悉,}\textbf{{我国和法国航天合作的首颗卫星------中法海洋卫星(CFOSAT)计划于2018年下半年由长征运载火箭在中国发射}}{。}}

{{25、}\textbf{{香港特区政府10月21日在礼宾府举行2017年度勋衔颁授典礼}}{,特区行政长官林郑月娥颁授勋衔及奖状予出席的312名受勋人士。其中,10人获颁大紫荆勋章,25人获颁金紫荆星章,35人获颁银紫荆星章,66人获铜紫荆星章,1人获追授金英勇勋章。}}

{{26、10月22日,}\textbf{{由中交二航局施工的世界最大跨度公铁两用钢拱桥------沪通长江大桥天生港专用航道桥主拱合龙}}{。该桥主跨1092米,为目前世界上最大跨度的公铁两用斜拉桥。}}

{{27、2017年10月22日,首批10部新能源通道车投入1路车运营,}\textbf{{标志着67岁``大1路''走进纯电动时代}}{。并率先配置了PM2.5自动过滤净化系统和360度主动安全预警系统。}}

{{28、10月24日,中国共产党第十九次全国代表大会在北京人民大会堂胜利闭幕。}\textbf{{选举产生新一届中央委员会和中央纪律检查委员会。通过关于十八届中央委员会报告的决议、关于中央纪律检查委员会工作报告的决议、关于《中国共产党章程(修正案)》的决议。习近平主持大会并发表重要讲话,习近平强调,中国共产党人的初心和使命,就是为中国人民谋幸福,为中华民族谋复兴。这个初心和使命是激励中国共产党人不断前进的根本动力。全党同志一定要永远与人民同呼吸、共命运、心连心,永远把人民对美好生活的向往作为奋斗目标,以永不懈怠的精神状态和一往无前的奋斗姿态,继续朝着实现中华民族伟大复兴的宏伟目标奋勇前进。全党要紧密团结在党中央周围,高举中国特色社会主义伟大旗帜,解放思想,改革创新,锐意进取,埋头苦干,带领全国各族人民为实现党的十九大确定的目标任务而奋斗}}{。}}

{{29、10月23日从WAPI产业联盟获悉,}\textbf{{中国自主研发的物联网安全协议关键技术TRAIS-X,被国际标准组织正式发布,成为国际标准技术规范}}{。}}

{{30、10月23日,}\textbf{{国内首单房企租赁住房REITs(房地产信托投资基金)、首单储架发行REITs------中联前海开源---保利地产租赁住房一号资产支持专项计划获得上海证券交易所审核通过,开创了租赁住房资产证券化新篇章,对于加快推进租赁住房市场建设具有积极的示范效应}}{。}}

{{31、中国国家卫星气象中心主任杨军10月24日在美国首都华盛顿说,}\textbf{{中国新一代静止轨道气象卫星风云四号和首颗全球二氧化碳监测科学实验卫星(简称碳卫星)的数据产品将对全球用户免费开放}}{。}}

{{32、中共中央总书记、国家主席、中央军委主席习近平10月26日下午出席了在京召开的军队领导干部会议并发表重要讲话。习近平强调,}\textbf{{中国特色社会主义进入了新时代,国防和军队建设也进入了新时代。人民军队要不忘初心、牢记使命,认真学习贯彻党的十九大精神,深入学习贯彻新时代党的强军思想,坚定不移走中国特色强军之路,全面推进国防和军队现代化,为实现党在新时代的强军目标、到本世纪中叶把人民军队全面建成世界一流军队、实现中华民族伟大复兴的中国梦而努力奋斗}}{。}}

{{33、}\textbf{{中国与古巴10月25日在此间签署了关于向古巴提供优惠贷款的框架协议,用于实施双方商定的50兆瓦光伏组件自动生产线项目和家电工业园改造项目}}{。光伏组件项目旨在对古巴唯一一家组装生产太阳能光伏板的厂商埃内斯托·切·格瓦纳的生产线进行扩大升级。}}

{{34、10月26日,由中科院工程热物理研究所和朗星无人机公司作为总体单位,}\textbf{{联合航空工业618所、中电54所、航天773所、西工大等单位研发的大型货运无人机AT200完成首飞,标志着全球首款吨位级货运无人机的诞生}}{。}}

{{35、10月27日电,国务院总理李克强日前签署国务院令,}\textbf{{公布《机关团体建设楼堂馆所管理条例》,自2017年12月1日起施行,1988年9月22日发布施行的《楼堂馆所建设管理暂行条例》同时废止}}{。}}

{36、10月29日电,\textbf{{湖南省近日发放全国首批有色行业排污许可证------五矿铜业(湖南)有限公司、郴州市金贵银业股份有限公司分别获得衡阳市环保局、郴州市环保局颁发的由环保部统一印发的排污许可证}}。}

\subsection{20114-2018年1月国际时政}
1、巴基斯坦国家银行(央行)1月2日晚发表声明,{\textbf{批准贸易商在与中国的双边贸易中使用人民币作为结算货币}}。

2、在特朗普政府宣布暂停对巴基斯坦的安全援助后,{\textbf{巴基斯坦外交部长阿西夫于1月5日称,自己国家与美国的``盟友''关系宣告结束。尽管美国防长马蒂斯在同日表示,不会完全``放弃''巴基斯坦,但有诸多学者认为,伊斯兰堡在地缘政治上将越来越向中国靠拢}}。\\
3、欧盟委员会1月8日发布的报告显示,{\textbf{去年12月,反映经济信心的欧元区经济景气指数进一步延续自2016年秋季以来的上升趋势,创下2000年10月份以来最高水平}}。

4、美国知情人士1月9日说,{\textbf{唐纳德·特朗普政府有意松绑核武器使用限制、研发新型核弹头,已就此拟定新版《核态势评估》报告。预计特朗普本月晚些时候发表国情咨文后将正式发布这份文件}}。

5、1月15日电,英国、欧盟和美国的金融机构监管者日前先后发布关于金融数据共享的法规或指导意见,计划今年开始实行。毫不夸张地说,金融数据共享虽然看起来更像是监管变化而不是技术突破,但它带给金融领域的冲击和改变,一点都不逊于目前的任何金融科技。{\textbf{英国《经济学人》将其称之为欧洲银行业的``地震''。}}\\
6、{\textbf{目前,54个非洲国家驻联合国代表共同要求美国总统特朗普收回在移民问题上的不当言论并正式道歉。有分析指出,特朗普的言论会使美国与非洲国家的关系受到严重负面影响,美国在非洲的形象和影响力将再次遭受冲击}}。

7、{\textbf{由于美国国会参议院未能通过联邦政府临时拨款法案,美国联邦政府非核心部门当地时间1月20日零时开始被迫``关门''。这是美联邦政府2013年10月以来再次``关门''。在美国总统特朗普执政周年之际,华盛顿以这种方式再次震惊世界}}。\\
8、{\textbf{第三十届非盟峰会系列会议1月22日在埃塞俄比亚首都亚的斯亚贝巴的非盟总部举行,会议将重点关注非洲国家的反腐败问题}}。

9、1月22日,{\textbf{美国媒体报道,国会参议院当天将通过临时拨款法案,这意味着联邦政府部门将重新``开张''。}}\\
10、1月23日,{\textbf{美国国会参议院以85票赞成、12票反对的结果通过杰罗姆·鲍威尔美国联邦储备委员会(美联储)主席一职的任命。鲍威尔将在2月接替珍妮特·耶伦,成为美联储近40年来首位没有经济学学位的主席}}。

11、1月22日,{\textbf{美国彭博社发布2018年世界创新指数,中国位居第十九位,比前一年提升两位。这个指数综合测算研发投入、研究人员集中度、专利申请等指标,从一个侧面反映出中国创新能力的稳步提高}}。\\
12、巴勒斯坦解放组织(巴解组织)执行委员会秘书长埃雷卡特1月24日强调,在美国收回有关耶路撒冷归属问题的表态前,巴方拒绝美国任何有关巴以和平的方案。

13、{\textbf{新年伊始,保加利亚接替爱沙尼亚担任欧盟轮值主席国}}。保加利亚总统拉德夫日前呼吁欧盟各成员国继续推进欧洲一体化进程,构建``团结更强大''的欧洲。在全球保护主义、民粹主义、反全球化``逆风''抬头的大背景下,欧洲内部分化严重,``反欧''``疑欧''右翼民粹政党借机乘势崛起,加之难民问题持续发酵,排外主义思潮涌动,欧洲一体化进程步履维艰。

14、{\textbf{1月25日,新一轮叙利亚和谈在奥地利首都维也纳召开}}。由于叙利亚政府及反对派拒绝直接进行谈判,联合国秘书长叙利亚问题特使德米斯图拉当天分别与双方进行了会谈。\\
15、美国财政部1月26日宣布对9家实体和21名个人实施制裁,{\textbf{从而在乌克兰和克里米亚问题上对俄罗斯继续施压}}。

16、据捷克统计办公室消息,根据对捷克总统选举第二轮投票96%的选票统计,{\textbf{现任总统泽曼1月27日在第二轮投票中击败对手捷克科学院前院长德拉霍什,赢得连任。泽曼总统在当天发表的胜选演讲中表示,他将不辜负捷克民众对他的信任}}。\\
17、日本前众议院议长河野洋平1月26日在东京都举办的共同通信加盟社研讨会上发表演讲,公开批评日本首相安倍晋三计划在和平宪法第九条中加入有关自卫队内容的修宪企图。

18、近日,德国财政部长彼得·阿尔特迈尔和法国财政部长布鲁诺·勒梅尔在巴黎联合举行新闻发布会,表示将在今年阿根廷举行的二十国集团峰会中联合推动全球对比特币的监管,将警告这一世界最流行的加密货币正在被非法团体利用。\\
19、{\textbf{1月28日芬兰举行总统大选,现任总统绍利·尼尼斯托获得62.7\%的选票,顺利连任总统}}。

20、喀布尔消息:据当地媒体报道,阿富汗首都喀布尔一军事学院1月29日遭武装分子袭击。阿富汗国防部官员证实,袭击造成至少5名士兵丧生,另有10人受伤。参与袭击的5名武装分子,其中2人引爆身上炸药死亡、2人被安全人员击毙,还有1人被捕。极端组织``伊斯兰国''已宣布对该起袭击负责。\\
21、{\textbf{1月30日,由俄罗斯、土耳其和伊朗倡议的叙利亚全国对话大会在俄南部城市索契开幕}}。这次会议是迄今为止涉及派别最多、覆盖面最广的一次叙利亚全国对话大会。分析认为,俄罗斯力图通过此次大会,借助推动叙利亚危机的政治解决,持续提升其在中东地区的外交影响力。

22、{\textbf{北美自由贸易协定第六轮谈判1月29日在加拿大蒙特利尔结束}}。尽管谈判``向前迈出了一步'',与会的加拿大、美国和墨西哥开始触及某些核心问题,但``进展极为缓慢'',仍存在巨大分歧。

% \subsection{21580-2016年张修齐政治真题}
% 1

% \subsection{21583-2013年张修齐政治真题}
% 1

% \subsection{22196-资本主义的发展及趋势}
% 1

% \subsection{22197-社会主义社会的发展及其规律}
% 1

% \subsection{22200-新民主主义革命理论}
% 1

% \subsection{22201-社会主义改造理论}
% 1

% \subsection{22204-社会主义本质和建设中国特色社会主义总任务}
% 1

% \subsection{22205-社会主义改革开放理论}
% 1

% \subsection{22206-建设中国特色社会主义总布局}
% 1

% \subsection{22208-中国特色社会主义外交和国际战略}
% 1

% \subsection{22209-建设中国特色社会主义的根本目的和依靠力量理论}
% 1

\subsection{22532-2018年1月国内时事}
1、新年前夕,国家主席习近平通过中国国际广播电台、中央人民广播电台、中央电视台、中国国际电视台(中国环球电视网)和互联网,发表了二○一八年新年贺词{\textbf{。``我为中国人民迸发出来的创造伟力喝彩''``千千万万普通人最伟大''``幸福都是奋斗出来的''``逢山开路,遇水架桥''``将改革进行到底''``不驰于空想、不骛于虚声''``以造福人民为最大政绩''\ldots{}\ldots{}习近平主席2018年的新年贺词,激荡光荣与梦想,充满信心与斗志,见证情怀}}。\\
2、2018年北京市将推进市级机关和市属行政事业单位向城市副中心搬迁工作,完善副中心与中心城区快速交通体系迫在眉睫。{\textbf{国家首批市郊铁路试点项目------北京城市副中心线已于2017年最后一天开通运营}},设计时速达200公里,从北京站至通州只需28分钟,将大大缓解市区至城市副中心间的通勤压力,有效满足中心城区与城市副中心之间日益增强的长距离快速出行需求。

3、外交部发言人耿爽1月2日在例行记者会上表示,{\textbf{根据中非合作论坛非方成员的强烈愿望,着眼于中非关系发展的现实需要,中方决定2018年在中国举办中非合作论坛峰会}}。\\
4、截至2017年底,我国铁路营业里程达12.7万公里,其中高铁2.5万公里,占世界高铁总量的66.3\%,{\textbf{铁路电气化率和复线率分别居世界第一和第二位}}。

5、{\textbf{2017年全面推行河长制取得重大进展,省市县乡四级工作方案全部出台,6项配套制度基本建立,设立乡级及以上河长31万名、村级河长62万名,湖长制全面启动实施}}。\\
6、新年伊始,万象更新,全军上下厉兵秣马,练兵正当时。1月3日上午,中央军委隆重举行2018年开训动员大会,中共中央总书记、国家主席、中央军委主席习近平向全军发布训令,{\textbf{号召全军贯彻落实党的十九大精神和新时代党的强军思想,全面加强实战化军事训练,全面提高打赢能力。习近平向全军发布训令。他命令:全军各级要强化练兵备战鲜明导向,坚定不移把军事训练摆在战略位置、作为中心工作,抓住不放,抓出成效。要坚持领导带头、以上率下,坚持实战实训、联战联训,坚持按纲施训、从严治训。要端正训练作风、创新训练方法、完善训练保障、严格训练监察,开展群众性练兵比武活动,加强针对性对抗性训练,提高军事训练实战化水平,牢牢掌握能打仗、打胜仗的过硬本领。全军指战员要坚决贯彻党中央和中央军委决策指示,发扬一不怕苦、二不怕死的战斗精神,刻苦训练、科学训练,勇于战胜困难,勇于超越对手,锻造召之即来、来之能战、战之必胜的精兵劲旅,坚决完成党和人民赋予的新时代使命任务}}。

7、中共中央总书记、国家主席、中央军委主席习近平1月3日视察中部战区陆军某师,{\textbf{强调要认真贯彻党的十九大精神,贯彻新时代党的强军思想,大抓实战化军事训练,深入推进数字化部队建设管理和作战运用创新,聚力打造精锐作战力量}}。\\
8、中国人民银行官网发布``关于百行征信有限公司(筹)相关情况的公示''称,{\textbf{已受理百行征信有限公司(筹)的个人征信业务申请,根据《征信业管理条例》《征信机构管理办法》等规定,现将百行征信有限公司(筹)相关情况予以公示。业内期盼已久的``信联''终于揭开神秘面纱。公示显示,百行征信主要股东及所持股份为:中国互联网金融协会持股36\%,芝麻信用管理有限公司持股8\%,腾讯征信有限公司持股8\%,深圳前海征信中心股份有限公司持股8\%,鹏元征信有限公司持股8\%,中诚信征信有限公司持股8\%,考拉征信有限公司持股8\%,中智诚征信有限公司持股8\%,北京华道征信有限公司持股8\%。注册资本人民币10亿元。据了解,百行征信主要在银、证、保等传统金融机构以外的网络借贷等领域开展个人征信活动,与人民银行征信中心运维的国家金融信用信息基础数据库形成错位发展、功能互补的市场格局}}。

9、新进中央委员会的委员、候补委员和省部级主要领导干部学习贯彻习近平新时代中国特色社会主义思想和党的十九大精神研讨班1月5日在中央党校开班。中共中央总书记、国家主席、中央军委主席习近平在开班式上发表重要讲话强调,{\textbf{建设好我们这样的大党,领导好我们这样的大国,中央委员会成员和省部级主要领导干部至关重要,必须提高政治站位、树立历史眼光、强化理论思维、增强大局观念、丰富知识素养、坚持问题导向,从历史和现实相贯通、国际和国内相关联、理论和实际相结合的宽广视角,对一些重大理论和实践问题进行思考和把握,做到坚持和发展中国特色社会主义要一以贯之,推进党的建设新的伟大工程要一以贯之,习近平强调,``备豫不虞,为国常道'',增强忧患意识、防范风险挑战要一以贯之,以时不我待、只争朝夕的精神投入工作,推动全党全国各族人民把思想统一到党的十九大精神上来,把力量凝聚到实现党的十九大确定的目标任务上来,不断开创新时代中国特色社会主义事业新局面}}。\\
10、2017年12月28日,石家庄至济南高铁正式开通运营,两地旅行时间从原来最快约4个小时缩短到约2个小时,{\textbf{标志着我国``四纵四横''高铁网中的``四横''完美收官。\\
}}11、{\textbf{中央军委日前印发《军队互联网媒体管理规定》,自2018年2月1日起施行}}。《规定》深入贯彻党的十九大精神,明确军队互联网媒体管理基本原则和总体要求,涵盖军队互联网媒体资质准入、审批备案、传播运行、建设保障等方面,就军队互联网媒体的开办范围、资格条件、审批程序、信息发布、保密要求、主体责任等作了统一规范,并对平台建设、技术监管、人才培养等工作进行明确,对军队互联网媒体违反国家和军队法律法规的各种情形作出追究责任的规定。

12、近日,中共中央办公厅、国务院办公厅印发了《关于推进城市安全发展的意见》,{\textbf{随着我国城市化进程明显加快,城市人口、功能和规模不断扩大,发展方式、产业结构和区域布局发生了深刻变化,新材料、新能源、新工艺广泛应用,新产业、新业态、新领域大量涌现,城市运行系统日益复杂,安全风险不断增大。该意见强化城市运行安全保障,有效防范事故发生}}。\\
13、中国人民银行公布的最新外汇储备规模数据显示,{\textbf{2017年12月末,我国外汇储备规模为31399亿美元,较11月末上升207亿美元,升幅为0.66\%,连续第十一个月出现回升}}。

14、从国家林业局获悉:{\textbf{我国将启动大规模国土绿化行动,力争2018年完成造林1亿亩以上,到2020年森林覆盖率达到23.04\%、到2035年达到26\%、到本世纪中叶达到世界平均水平}}。\\
15、{\textbf{中共中央、国务院1月8日上午在北京隆重举行国家科学技术奖励大会。}}党和国家领导人习近平、李克强、张高丽、王沪宁出席大会并为获奖代表颁奖。李克强代表党中央、国务院在大会上讲话。张高丽主持大会。习近平首先向获得2017年度国家最高科学技术奖的南京理工大学王泽山院士和中国疾病预防控制中心病毒病预防控制所侯云德院士颁发奖励证书,并同他们热情握手,表示祝贺。随后,习近平等党和国家领导人向获得国家自然科学奖、国家技术发明奖、国家科学技术进步奖和中华人民共和国国际科学技术合作奖的代表颁奖。2017年度国家科学技术奖共评选出271个项目和9名科技专家。其中,国家最高科学技术奖2人;国家自然科学奖35项,其中一等奖2项、二等奖33项;国家技术发明奖66项,其中一等奖4项、二等奖62项;国家科学技术进步奖170项,其中特等奖3项、一等奖21项(含创新团队3项)、二等奖146项;授予7名外籍科技专家中华人民共和国国际科学技术合作奖。

16、国家主席习近平1月8日在钓鱼台国宾馆会见来华进行国事访问的法国总统马克龙。习近平强调,{\textbf{当今世界存在很多不确定性,中方主张构建人类命运共同体,法方也持相似的理念。两国可以超越社会制度、发展阶段、文化传统差异,增进政治互信,充分挖掘合作潜力。中方愿继续本着合作共赢的原则,密切同法方各领域合作,加强``一带一路''框架下合作。中方重视同法方密切在重大国际问题上的沟通合作,共同努力促进世界稳定与繁荣。新时代中法关系大有作为}}。\\
17、1月8日在厦门举行的全国旅游工作会议上,{\textbf{国家旅游局局长李金早表示,国家旅游局将坚决纠正一些地方搞形式主义、搞所谓``五星级厕所''的做法,扎实推进``厕所革命''新三年行动计划}}。

18、国家主席习近平1月9日在人民大会堂同法国总统马克龙举行会谈。{\textbf{两国元首一致同意,秉承友好传统,推动紧密持久的中法全面战略伙伴关系行稳致远}}。\\
19、{\textbf{云南省日前发布《关于贯彻落实湿地保护修复制度方案的实施意见》,明确对全省湿地资源实行面积总量管控,确保湿地面积不减少,特别是自然湿地面积不减少,合理划定纳入生态保护红线的湿地范围,实现湿地资源管理``一张图''。到2020年,全省湿地面积不低于845万亩,其中自然湿地面积不低于588万亩,湿地保护率不低于52\%}}。

20、1月10日,{\textbf{我国首个拥有完全自主知识产权的``云轨''无人驾驶系统发布,首条搭载这一系统的``云轨''线路也在银川通车运行。``云轨''是一种中、小运量轨道交通系统,采用跨座式单轨技术,可应用于中、小城市的骨干线和大中城市的加密线等}}。\\

21、1月15日电,{\textbf{从全国标准化工作会上获悉:去年全年共发布国家标准3811项,立项2042项。在冶金、材料、建筑等领域新提交国际标准提案161项,参与制定国际标准数量超过新增数量的50\%,国际标准化贡献率跃居世界第五}}。

22、中共中央总书记、国家主席、中央军委主席习近平1月11日上午在中国共产党第十九届中央纪律检查委员会第二次全体会议上发表重要讲话。他强调,{\textbf{在中国特色社会主义新时代,完成伟大事业必须靠党的领导,党一定要有新气象新作为。要全面贯彻党的十九大精神,重整行装再出发,以永远在路上的执着把全面从严治党引向深入,开创全面从严治党新局面。习近平强调,党的建设新的伟大工程,是引领伟大斗争、伟大事业、最终实现伟大梦想的根本保证。全面从严治党,必须坚持和加强党的全面领导。坚持党的领导,最根本的是坚持党中央权威和集中统一领导。习近平强调,全面从严治党必须持之以恒、毫不动摇。习近平指出,要坚持以党的政治建设为统领,坚决维护党中央权威和集中统一领导。习近平强调,要锲而不舍落实中央八项规定精神,保持党同人民群众的血肉联系。要继续在常和长、严和实、深和细上下功夫,密切关注享乐主义、奢靡之风新动向新表现,坚决防止回潮复燃。纠正形式主义、官僚主义,一把手要负总责。习近平强调,要深化标本兼治,夺取反腐败斗争压倒性胜利。标本兼治,既要夯实治本的基础,又要敢于用治标的利器。要坚持无禁区、全覆盖、零容忍,坚持重遏制、强高压、长震慑,坚持受贿行贿一起查,坚决减存量、重点遏增量。``老虎''要露头就打,``苍蝇''乱飞也要拍。要推动全面从严治党向基层延伸,严厉整治发生在群众身边的腐败问题。要把扫黑除恶同反腐败结合起来,既抓涉黑组织,也抓后面的``保护伞''。要加强反腐败综合执法国际协作,强化对腐败犯罪分子的震慑}}。\\
23、{\textbf{我国首个等离子体危废处理示范项目------10吨/天等离子体危废处理项目,近日在广东清远通过竣工验收,正式进入工程应用阶段,为国内医疗垃圾、生活垃圾、废矿物油等危废物的处理探索出了一条新路。}}\\
24、1月18日,国家统计局对外公布,初步核算,{\textbf{全年国内生产总值(GDP)827122亿元,我国经济总量首次站上80万亿元的历史新台阶。2017年中国经济的名义增量约8.4万亿元,这意味着中国一年干出的经济增量相当于2016年全球第十四大经济体的经济总量}}。

25、{\textbf{福建省首个县级路长办公室日前在三明永安市正式挂牌,县长当``路长''的农村公路管理制度将在全省范围内推行。福建省也将成为全国首个全省性推广农村公路``路长制''的省份}}。\\
26、中国共产党第十九届中央委员会第二次全体会议,于2018年1月18日至19日在北京举行。出席这次全会的有,中央委员203人,候补中央委员172人。中央纪律检查委员会常务委员会委员和有关方面负责同志列席会议。{\textbf{党的十九大代表中部分基层同志和专家学者也列席会议。中央政治局主持会议,中央委员会总书记习近平作重要讲话,全会通过《中共中央关于修改宪法部分内容的建议》,全会号召,全党同志要更加紧密地团结在以习近平同志为核心的党中央周围,以习近平新时代中国特色社会主义思想为指导,全面深入贯彻党的十九大精神和本次全会精神,牢固树立政治意识、大局意识、核心意识、看齐意识,坚定不移走中国特色社会主义法治道路,自觉维护宪法权威、保证宪法实施,为新时代推进全面依法治国、建设社会主义法治国家而努力奋斗}}。

27、十九届二次全会认为,宪法是国家的根本法,是治国安邦的总章程,是党和人民意志的集中体现。维护宪法尊严和权威,是维护国家法制统一、尊严、权威的前提,也是维护最广大人民根本利益、确保国家长治久安的重要保障。{\textbf{全会认为,我们党高度重视宪法在治国理政中的重要地位和作用,明确坚持依法治国首先要坚持依宪治国,坚持依法执政首先要坚持依宪执政,我国宪法必须随着党领导人民建设中国特色社会主义实践的发展而不断完善发展。这是我国宪法发展的一个显著特点,也是一条基本规律。从1954年我国第一部宪法诞生至今,我国宪法一直处在探索实践和不断完善过程中。1982年宪法公布施行后,分别于1988年、1993年、1999年、2004年进行了4次修改。全会强调,习近平新时代中国特色社会主义思想是马克思主义中国化最新成果,是当代中国马克思主义、21世纪马克思主义,是党和国家必须长期坚持的指导思想。中国共产党领导是中国特色社会主义最本质的特征,是中国特色社会主义制度最大的优势,必须坚持和加强党对一切工作的领导。经济建设、政治建设、文化建设、社会建设、生态文明建设``五位一体''总体布局,创新、协调、绿色、开放、共享的新发展理念,到2020年全面建成小康社会、到2035年基本实现社会主义现代化、到本世纪中叶建成社会主义现代化强国的奋斗目标,实现中华民族伟大复兴,对激励和引导全党全国各族人民团结奋斗具有重大引领意义。坚持和平发展道路,坚持互利共赢开放战略,推动构建人类命运共同体,对促进人类和平发展的崇高事业具有重大意义。国家监察体制改革是事关全局的重大政治体制改革,是强化党和国家自我监督的重大决策部署,要依法建立党统一领导的反腐败工作机构,构建集中统一、权威高效的国家监察体系,实现对所有行使公权力的公职人员监察全覆盖。宪法是国家各项制度和法律法规的总依据,充实宪法的重大制度规定,对完善和发展中国特色社会主义制度具有重要作用。\\
中共中央2017年12月15日在中南海召开党外人士座谈会,就中共中央关于修改宪法部分内容的建议听取各民主党派中央、全国工商联负责人和无党派人士代表的意见和建议。中共中央总书记习近平主持会议并发表重要讲话。习近平强调,我国宪法是一部好宪法。各民主党派和统一战线为我国宪法制度的形成和发展作出了重要贡献。宪法只有不断适应新形势才能具有持久生命力。中共中央决定对宪法进行适当修改,是经过反复考虑、综合方方面面情况作出的,目的是通过修改使我国宪法更好体现人民意志,更好体现中国特色社会主义制度的优势,更好适应提高中国共产党长期执政能力、推进全面依法治国、推进国家治理体系和治理能力现代化的要求。宪法修改,既要顺应党和人民事业发展要求,又要遵循宪法法律发展规律}}。\\
28、在全国个体劳动者第五次代表大会开幕之际,{\textbf{中共中央总书记、国家主席、中央军委主席习近平致信大会,代表党中央,向全体与会代表及全国广大个体私营企业经营者致以诚挚的问候。习近平强调,中国特色社会主义进入新时代,深化供给侧结构性改革,实施区域协调发展战略,发展实体经济,推进精准扶贫,对个体私营经济发展提出了新的更高的要求。广大个体私营企业经营者要认真学习贯彻党的十九大精神,弘扬企业家精神,发挥企业家作用,坚守实体经济,落实高质量发展,在全面建成小康社会、全面建设社会主义现代化国家新征程中作出新的更大贡献}}。

29、习近平近日就政法工作作出重要指示,对党的十八大以来政法战线取得的成绩给予充分肯定,对新时代政法工作提出明确要求。习近平强调,{\textbf{希望全国政法战线深入学习贯彻党的十九大精神,强化``四个意识'',坚持党对政法工作的绝对领导,坚持以人民为中心的发展思想,增强工作预见性、主动性,深化司法体制改革,推进平安中国、法治中国建设,加强过硬队伍建设,深化智能化建设,严格执法、公正司法,履行好维护国家政治安全、确保社会大局稳定、促进社会公平正义、保障人民安居乐业的主要任务,努力创造安全的政治环境、稳定的社会环境、公正的法治环境、优质的服务环境,增强人民群众获得感、幸福感、安全感}}。\\
30、从中国科学技术大学获悉:该校潘建伟教授及其同事彭承志等组成的研究团队,最近与合作者利用``墨子号''量子科学实验卫星,在中国和奥地利之间首次实现距离达7600公里的洲际量子密钥分发,并利用共享密钥实现加密数据传输和视频通信。{\textbf{该成果标志着``墨子号''已具备洲际量子保密通信能力}}。

31、中共中央总书记、国家主席、中央军委主席、中央全面深化改革领导小组组长习近平1月23日下午主持召开中央全面深化改革领导小组第二次会议并发表重要讲话。{\textbf{他强调,2018年是贯彻党的十九大精神的开局之年,也是改革开放40周年,做好改革工作意义重大。要弘扬改革创新精神,推动思想再解放改革再深入工作再抓实,凝聚起全面深化改革的强大力量,在新起点上实现新突破}}。\\
32、从国防科工局、国家航天局获悉:{\textbf{1月23日,我国首颗高通量通信卫星实践十三号在轨交付,正式投入使用。实践十三号卫星投入使用后,将纳入``中星''卫星系列,命名为``中星十六号''卫星}}。

33、{\textbf{在1月26日召开的福建省第十三届人民代表大会第一次会议上,一份``成绩单''公布:福建省GDP继2016年首次进入全国前十后,去年又首次突破3万亿元;人均GDP则上升至全国第六。更令人欣喜的是,去年全省民生相关支出比重已连续五年超过七成,省级财政累计投入超过800亿元,完成了110件为民办实事项目}}。

34、{\textbf{2017年我国社会保险覆盖范围进一步扩大}}。截至2017年底,基本养老、基本医疗、失业、工伤、生育保险参保人数分别达到9.15亿人、11.77亿人、1.88亿人、2.27亿人、1.92亿人;五项基金总收入6.64万亿元,同比增长23.9\%,总支出5.69万亿元,同比增长21.4\%;全民参保登记信息库已基本建设成型,社会保障卡持卡人数达10.88亿人。\\
35、{\textbf{近日,中国地质调查局在京发布《中国地质调查局科学技术普及规划(2017---2020年)》,这是我国首个地学科普规划}}。

36、{\textbf{我国首个通过公开招标中标国际``人造太阳''项目的核压力设备完成制造}}。1月28日,由中广核工程有限公司牵头组成联合体采用国际核二级标准共同研制、用于国际热核聚变项目的4台不锈钢蒸汽冷凝罐(VST)顺利装船,运往法国。这是我国企业首次成功研制的核聚变关键设备。\\
37、{\textbf{中共中央政治局1月30日召开会议,听取和审议《中央政治局常委会听取和研究全国人大常委会、国务院、全国政协、最高人民法院、最高人民检察院党组工作汇报和中央书记处工作报告的综合情况报告》。}}中共中央总书记习近平主持会议。会议认为,党的领导是中国特色社会主义最本质的特征,是全党全国各族人民共同意志和根本利益的体现,是决胜全面建成小康社会、夺取新时代中国特色社会主义伟大胜利的根本保证。坚持党的领导,首先要坚持维护党中央权威和集中统一领导。全党同志要牢固树立政治意识、大局意识、核心意识、看齐意识,把维护党中央权威和集中统一领导作为最高政治原则和根本政治规矩来执行,始终在思想上政治上行动上同以习近平同志为核心的党中央保持高度一致。

\subsection{22535-2018年2月国际时事}
1、{\textbf{叙利亚全国对话大会1月30日在俄罗斯索契举行。}}与会各方代表在会后发表联合声明和与会各方呼吁书,并决定成立叙利亚宪法委员会,初步确定150人宪法委员会候选人名单。\\
2、巴基斯坦瓜达尔港自由区开园仪式暨首届瓜达尔国际商品展销会日前落下帷幕。{\textbf{这标志着瓜达尔港建设进入新阶段,中巴经济走廊通往印度洋的门户已经开启。}}

{\textbf{3、2月2日,作为第二十七届古巴国际书展中国主宾国活动的一项重要内容,中国和古巴出版机构签署了10项出版合作协议}}。\\
4、《马尔代夫时报》2月6日报道,在马尔代夫总统亚明5日宣布进入为期15天的紧急状态后数小时,{\textbf{前总统加尧姆6日凌晨被马尔代夫特警队以腐败和蓄谋推翻政府的名义逮捕。同时,警方以涉嫌腐败逮捕了马尔代夫最高法院首席大法官阿卜杜拉·赛义德和法官阿里·哈米德}}。

5、据叙利亚通讯社2月5日报道,{\textbf{土耳其在叙利亚北部阿夫林地区军事行动已持续两周,共造成近500人伤亡}}。\\
6、{\textbf{2月7日电,继6日成功试射``烈火-1''型改进型弹道导弹后,印度7日在东部奥迪沙邦昌迪普尔基地又成功试射一枚``大地-2''型地对地短程弹道导弹}}。

7、\textbf{{韩国第二大在野党国民之党的部分成员6日另起炉灶,组建了一个名为``民主和平党''的新政党,所占议席数量在国会排名第四。韩国政坛角力风云变幻,受到外界密切关注。}}至此,韩国国会299个议席中,执政党共同民主党占有121席,第一大在野党自由韩国党117席,国民之党24席,民主和平党15席。韩联社报道,四届国会议员赵培淑当选民主和平党党首,三届国会议员张秉浣当选该党的国会领袖。\\
8、2月7日,{\textbf{由德国总理默克尔领导的联盟党(基民盟与基社盟)与德国社会民主党(社民党)达成联合组阁协议,并确定了部长职务分配。}}据德新社等媒体报道,联盟党在谈判中作出了重大让步,社民党在新一任内阁中将收获外交部长、财政部长和劳工部长等重要职务。

9、{\textbf{2月6日,美国私营公司太空探索(SpaceX)在佛罗里达州肯尼迪航天中心成功发射重型猎鹰火箭,把一辆桃红色特斯拉跑车送入地球至火星的椭圆形轨道。}}此次发射的重型猎鹰火箭是目前人类现役火箭中运载能力最强的一款,也是自美国航空航天局土星5号火箭在45年前发射、将宇航员送上月球以来最强大的宇宙飞船。

10\textbf{、}{\textbf{``激情平昌,和谐世界''。2月9日20时,第二十三届冬季奥林匹克运动会在韩国平昌开幕。习近平主席特别代表、中共中央政治局常委韩正应邀出席开幕式}}。\\
11、{\textbf{朝鲜高级别代表团2月9日乘专机抵达韩国仁川机场,并出席当晚举行的平昌冬奥会开幕式。}}

12、据德国媒体2月8日报道,随着北约抵制俄罗斯的军事活动升级,{\textbf{德国联邦国防军将在科隆---波恩地区设立新的北约军事指挥中心}}。\\
13、俄罗斯一架载有71人的客机当地时间2月11日下午在莫斯科州坠毁。救援人员已在现场搜索到两具遗体,并找到了黑匣子。俄媒体报道说,机上人员恐难有生还。

14、英国《泰晤士高等教育》杂志日前公布的2018年亚洲大学排名显示,东京大学排名比去年下降一位,名列第八,是日本唯一进入前十名的大学,此外跻身前100名之内的日本大学也仅有11所,比去年减少1所。日本民众纷纷就此发表评论,表达担忧。\\
15、美国比尔和梅琳达·盖茨基金会(简称基金会)主席比尔·盖茨及其妻子梅琳达·盖茨2月13日发布了2018年度公开信,自1990年以来,全球每年的儿童死亡人数已经减半,且在过去短短20年时间里,全球极端贫困人口数量下降了近一半,并高度评价中国的减贫成就,称早在基金会参与之前,中国已经在扶贫工作上取得显著成就。

16、{\textbf{南非国民议会2月15日选举非洲人国民大会(非国大)主席、代总统西里尔·拉马福萨为南非新总统}}。\\
17、{\textbf{由国际货币基金组织、经济合作与发展组织、联合国和世界银行联合举办的首届税收合作平台全球会议2月14日至16日在纽约联合国总部召开。}}400多名来自世界各国税收政策与征管部门的高级官员及重要国际组织、学术界、经济界和民间组织代表,围绕如何把税收作为政策工具,实现可持续发展目标进行深入探讨。

18、{\textbf{为期三天的第54届慕尼黑安全会议(慕安会)于2月18日下午在德国慕尼黑闭幕。}}会议主席沃尔夫冈·伊申格尔在闭幕致辞时说,在会议中,他听到了当前世界面临的朝核问题、中东乱局等挑战,也听到了促进欧洲防务联盟、``一带一路''倡议等高明见解,以及世界对于避免军备竞赛、保护主义的期待,但帮助世界远离安全危机仍缺少具体的措施建议。慕尼黑安全会议始于1963年,前身系以跨大西洋伙伴关系为重点议题的``国际防务大会''。近年来,慕尼黑安全会议已逐步成为全球高规格安全政策论坛。\\
19、{\textbf{联合国安理会2月20日在纽约联合国总部举行会议审议巴勒斯坦问题。}}联合国秘书长古特雷斯首先在会上发言。他重申联合国通过``两国方案''、在联合国有关决议和国际法框架下解决巴以问题的决心。他指出,当前巴以局势面临严峻考验,国际社会关于``两国方案''的共识有削弱的危险,加沙地带局势动荡不安,数百万巴勒斯坦难民的生活受到威胁。他呼吁各方立即采取一致行动,通过对话解决当前问题。

20、{\textbf{欧盟边境与海岸警卫局2月20日发布年度报告说,2017年共有近20.5万难民进入欧盟,数量降至4年来的最低}}。\\
21、美国海关和边境保护局2月21日宣布,将对加利福尼亚州南部卡莱克西科市市中心附近的边境隔离墙进行更换。{\textbf{这标志着美国联邦政府启动了美国与墨西哥边境隔离墙的修建工程}}。

22、{\textbf{第三十届联合国粮农组织非洲区域会议日前在苏丹首都喀土穆召开,来自非洲38个国家的农业部长和400余名农业专家参会,会议主要聚焦近年来日益严峻的非洲粮食安全和人口营养不良问题}}。\\
23、以色列国防军2月22日表示,{\textbf{以色列和美国将于3月举行大规模反导联合军演}}。

24、联合国安理会2月24日通过了敦促叙利亚全境停火至少30天的2401号决议。\\
25、{\textbf{第二}}{\textbf{十三届冬季奥林匹克运动会2月25日晚在平昌奥林匹克体育场闭幕。中国作为下届冬奥会主办国,在闭幕式上奉献了《2022相约北京》8分钟文艺表演。国家主席习近平通过视频,向全世界发出诚挚邀请------2022年相约北京!}}

26、{\textbf{201}}{\textbf{8年世界移动通信大会2月26日在西班牙巴塞罗那开幕。}}世界移动通信大会一直是移动通信产业的风向标,在为期4天的展会中,来自全世界的2300多家企业将发布和展示其最新的通信产品和技术,本届大会的与会人数预计将超过10万。

\subsection{23910-2004年考研政治真题解析}
2004年考研政治真题解析

\subsection{23918-毛泽东思想的形成与发展}
{{毛泽东思想是马克思列宁主义在中国的运用和发展,是被实践证明了}{的关于中国革命和建设的正确的理论原则和经验总结,是}\textbf{{中国共产党集\textbf{体}}{\textbf{智慧的结晶。}}}}

{帝国主义战争与无产阶级革命的\textbf{时代主题},是毛泽东思想形成的时代背景。}{
    {
        % \\
}{{中国共产党领导的革命和建设的实践,是毛泽东思想形成的实践基础。\\
}{开始形成}{ }{:第一次国内革命战争时期,通过调查研究论证阶级关系
;土地革命战争时期开辟了农村包围城市、武装夺取政权的道路,与教条主义作斗争,论证中国革命新道路。\\
}{走向成熟}{
}{:遵义会议后到抗日战争时期,完成新民主主义革命理论和政策是成熟的标志。
在中共七大上毛泽东思想被确立为党的指导思想。新民主主义理论的系统阐明,
标志着毛泽东思想得到多方面展开而达到成熟。\\
}{继续发展}{
:解放战争时期和新中国成立后,提出人民民主专政理论、社会主义改造理论及``第二次结合''等。}}\\
}

\subsection{24667-2018年4月国内时事}
1、3月31日,{\textbf{上海首条胶轮路轨全自动无人驾驶APM线(旅客自动运输系统)------浦江线开通试运营,吸引众多市民前来体验}}。\\
2、习近平4月2日主持召开中央财经委员会第一次会议,研究打好三大攻坚战的思路和举措,研究审定《中央财经委员会工作规则》。{\textbf{习近平在会上发表重要讲话强调,防范化解金融风险,事关国家安全、发展全局、人民财产安全,是实现高质量发展必须跨越的重大关口。精准脱贫攻坚战已取得阶段性进展,只能打赢打好。环境问题是全社会关注的焦点,也是全面建成小康社会能否得到人民认可的一个关键,要坚决打好打胜这场攻坚战。会议审议通过了《中央财经委员会工作规则》,强调要加强党中央对经济工作的集中统一领导,做好经济领域重大工作的顶层设计、总体布局、统筹协调、整体推进、督促落实}}。

3、{\textbf{习近平4月2日上午在参加首都义务植树活动时强调,绿化祖国要坚持以人民为中心的发展思想,广泛开展国土绿化行动,人人出力,日积月累,让祖国大地不断绿起来美起来}}。\\
4、{\textbf{外交部4月3日举行中外媒体吹风会。国务委员兼外交部长王毅介绍习近平主席出席博鳌亚洲论坛2018年年会开幕式并举行有关活动相关情况,并回答记者提问。}}王毅表示,博鳌亚洲论坛2018年年会将在海南拉开帷幕,中国国家主席习近平将应邀出席论坛年会开幕式并发表重要主旨演讲,会见与会外国国家元首、政府首脑和国际组织负责人,集体会见论坛理事,并同与会中外企业家代表座谈。王毅强调,2018年是中国改革开放40周年,也是贯彻落实十九大精神的开局之年。在这一重要历史时刻,习近平主席出席年会开幕式并举行一系列活动,对于深入推进新时代中国特色大国外交、推动构建亚洲和人类命运共同体、促进人类和平与发展事业具有重大意义。这也是今年中国首个重大主场活动,将为本届年会增添一系列亮点。

5、{\textbf{近日,民政部和最高人民法院正式签署《关于开展部门间信息共享的合作备忘录》,建立当事人婚姻登记信息和涉婚姻案由的案件信息的共享机制}}。\\
6、{\textbf{中国科学技术大学郭光灿院士团队开发的全球首款量子计算云平台APP``本源量子计算云服务平台''上线}},旨在为手机用户打造量子计算在线演示、教育科普及模拟服务的掌上平台。科研院所、高校、专业用户及有需求的公众可通过手机操作量子计算,让量子``门外汉''也能快速上手,进入神奇的量子编程世界。

7、{\textbf{国务院关税税则委员会发布公告,决定对原产于美国的大豆、汽车、化工品等14类106项商品加征25\%的关税}}。\\
8、4月5日,{\textbf{中国就美国进口钢铁和铝产品232措施,在世界贸易组织争端解决机制项下向美方提出磋商请求,正式启动争端解决程序}}。

9、4月的博鳌,草木滴翠,春潮拍岸。春雨把碧空洗得更为澄净。{\textbf{博鳌亚洲论坛2018年年会4月8日拉开帷幕,世界再度进入``博鳌时间''。}}2000多位嘉宾,近2000名记者,60多场正式讨论\ldots{}\ldots{}观点交集,智慧碰撞,共识汇聚。\\
10、4月10日上午,博鳌亚洲论坛2018年年会在海南省博鳌开幕。{\textbf{国家主席习近平出席开幕式并发表题为《开放共创繁荣
创新引领未来》的主旨演讲,强调各国要顺应时代潮流,坚持开放共赢,勇于变革创新,向着构建人类命运共同体的目标不断迈进;中国将坚持改革开放不动摇,继续推出扩大开放新的重大举措,同亚洲和世界各国一道,共创亚洲和世界的美好未来。习近平指出,实践证明,过去40年中国经济发展是在开放条件下取得的,未来中国经济实现高质量发展也必须在更加开放条件下进行。中国开放的大门不会关闭,只会越开越大。这是中国基于发展需要作出的战略抉择,也是在以实际行动推动经济全球化造福世界各国人民}}。

11、{\textbf{庆祝海南建省办经济特区30周年大会4月13日下午在海南省人大会堂举行。中共中央总书记、国家主席、中央军委主席习近平出席大会并发表重要讲话}}。他强调,在决胜全面建成小康社会、夺取新时代中国特色社会主义伟大胜利的征程上,经济特区不仅要继续办下去,而且要办得更好、办出水平。经济特区要不忘初心、牢记使命,把握好新的战略定位,继续成为改革开放的重要窗口、改革开放的试验平台、改革开放的开拓者、改革开放的实干家。\\
12、中共中央总书记、国家主席、中央军委主席习近平近日在海南考察时强调,{\textbf{全面贯彻党的十九大和十九届二中、三中全会精神,统筹推进``五位一体''总体布局、协调推进``四个全面''战略布局,以更高的站位、更宽的视野、更大的力度谋划和推进改革开放,充分发挥生态环境、经济特区、国际旅游岛的优势,真抓实干加快建设美好新海南}}。

13、斯德哥尔摩4月13日电,{\textbf{中国科学家陈竺和法国科学家安娜·德尚、于克·德戴13日下午在瑞典首都斯德哥尔摩获颁2018年舍贝里奖}}。\\
14、财政部等五部门近日发出通知,决定自5月1日起,{\textbf{在上海、福建(含厦门)和苏州工业园区实施个人税收递延型商业养老保险试点}}。试点期限暂定一年。

15、4月16日,{\textbf{生态环境部、退役军人事务部、应急管理部等国务院新组建部门作为第二批挂牌单位分别举行挂牌仪式,正式对外履行职责}}。\\
16、4月16日电,从华为获悉:华为将于今年全球商用的5G
NR(第五代移动通信新无线接入技术,以下简称5G)产品,获得全球第一张5G产品欧盟无线设备指令型式认证(CE---TEC)证书,\textbf{{这标志着华为5G产品正式获得市场商用许可,向规模商用又迈出了关键一步}}。

17、经中共中央批准,{\textbf{4月19日,中央统战部组织各民主党派中央负责人和无党派人士代表赴西柏坡、李家庄参观学习并举行座谈会,纪念中共中央发布``五一口号''70周年}}。\\
18、在首届数字中国建设峰会开幕之际,中共中央总书记、国家主席、中央军委主席习近平发来贺信,向峰会的召开表示衷心的祝贺,向出席会议的各界人士表示热烈的欢迎。习近平在贺信中指出,{\textbf{当今世界,信息技术创新日新月异,数字化、网络化、智能化深入发展,在推动经济社会发展、促进国家治理体系和治理能力现代化、满足人民日益增长的美好生活需要方面发挥着越来越重要的作用。习近平说,2000年我在福建工作时,作出了建设数字福建的部署,经过多年探索和实践,福建在电子政务、数字经济、智慧社会等方面取得了长足进展。习近平强调,党的十九大描绘了决胜全面建成小康社会、开启全面建设社会主义现代化国家新征程、实现中华民族伟大复兴的宏伟蓝图,对建设网络强国、数字中国、智慧社会作出战略部署。加快数字中国建设,就是要适应我国发展新的历史方位,全面贯彻新发展理念,以信息化培育新动能,用新动能推动新发展,以新发展创造新辉煌。习近平指出,本届峰会以``以信息化驱动现代化,加快建设数字中国''为主题,展示我国电子政务和数字经济发展最新成果,交流数字中国建设体会和看法,进一步凝聚共识,必将激发社会各界建设数字中国的积极性、主动性、创造性,推动信息化更好造福社会、造福人民}}。

19、4月21日下午,{\textbf{首届数字中国建设成果展览会在福建省福州市海峡国际会展中心正式启幕}}。首届数字中国建设成果展览会涵盖数字福建、电子政务、数字经济、数字社会体验等4个展馆,总面积超过4万平方米,共有293个参展单位。\\
20、{\textbf{全国网络安全和信息化工作会议4月20日至21日在北京召开}}。中共中央总书记、国家主席、中央军委主席、中央网络安全和信息化委员会主任习近平出席会议并发表重要讲话。他强调,信息化为中华民族带来了千载难逢的机遇。我们必须敏锐抓住信息化发展的历史机遇,加强网上正面宣传,维护网络安全,推动信息领域核心技术突破,发挥信息化对经济社会发展的引领作用,加强网信领域军民融合,主动参与网络空间国际治理进程,自主创新推进网络强国建设,为决胜全面建成小康社会、夺取新时代中国特色社会主义伟大胜利、实现中华民族伟大复兴的中国梦作出新的贡献。

21、{\textbf{4月22日是第四十九个世界地球日}},今年地球日活动的主题是``珍惜自然资源
呵护美丽国土------讲好我们的地球故事''。自然资源部聘任10位专家为第二批首席科学传播专家,授予《地学真好玩儿》等38种图书``2018年优秀科普图书''称号。\\
22、在4月22日开幕的首届数字中国建设峰会上,农业农村部发布《农业农村信息化发展前景及政策导向》,相关数据显示,{\textbf{到2017年底我国农村地区网民线下消费使用手机网上支付的比例已提升至47.1\%。据介绍,按照``有场所、有人员、有设备、有宽带、有网页、有持续运营能力''的``六有''标准,农业农村部在试点地区的每个行政村建设益农信息社,实现了公益服务、便民服务、电子商务和培训体验服务``一社综合、一站解决''}}。

23、4月23日电,中共中央总书记、国家主席、中央军委主席习近平近日作出重要指示强调,{\textbf{要结合实施农村人居环境整治三年行动计划和乡村振兴战略,进一步推广浙江好的经验做法,建设好生态宜居的美丽乡村}}。\\
24、4月23日上午,中国载人航天庆祝第三个``中国航天日''主题活动暨曾宪梓载人航天基金会颁奖大会在京举行,中国载人航天工程办公室主任杨利伟代表中国载人航天工程办公室宣布,{\textbf{第三批预备航天员选拔工作正式启动}}。

25、4月23日电,国家版权局网络版权产业研究基地今天在京发布《中国网络版权产业发展报告(2018)》。报告显示,{\textbf{当前我国网络版权产业继续保持快速增长趋势。根据测算,2017年中国网络版权产业的市场规模为6365亿元,较2016年增长27.2\%。其中,中国网络版权产业用户付费规模为3184亿元,占比规模突破50\%}}。\\
26、4月24日电,日前,国务院办公厅印发《2018年政务公开工作要点》,部署全国政务公开年度重点工作。《要点》指出,{\textbf{做好2018年政务公开工作,要全面贯彻党的十九大和十九届二中、三中全会精神,以习近平新时代中国特色社会主义思想为指导,深入落实党中央、国务院关于全面推进政务公开工作的系列部署,大力推进决策、执行、管理、服务、结果公开,不断提升政务公开的质量和实效}}。

27、4月24日,首届数字中国建设峰会在福州闭幕。{\textbf{本届峰会以``以信息化驱动现代化,加快建设数字中国''为主题}},打造信息化发展政策发布、电子政务和数字经济发展成果展示、数字中国建设理论经验和实践交流三大平台。闭幕式上,``数字中国''研究院成立仪式和``数字中国''核心技术产业联盟发起仪式分别举行。\\
28、{\textbf{国际奥委会主席托马斯·巴赫24日在瑞士洛桑国际奥委会总部向中国艺术家韩美林颁发``顾拜旦奖章'',表彰他为奥林匹克运动发展作出的杰出贡献}}。

29、{\textbf{2018年度全国五一劳动奖状、奖章和全国工人先锋号评选4月25日出炉}}。中华全国总工会表示,今年将颁发全国五一劳动奖状99个、全国五一劳动奖章697个,表彰全国工人先锋号799个,其中首次单独列出获表彰比例的产业工人占比超过四成。据全总劳动和经济工作部部长王俊治介绍,从今年起,全国总工会将产业工人在全国五一劳动奖章中所占比例单列。\\
30、4月25日,从西藏自治区那曲地区撤地设市新闻发布会上获悉:{\textbf{经国务院批复,我国国土面积最大的地级市那曲市将挂牌成立}}。\\
31、4月26日12时42分,{\textbf{我国在酒泉卫星发射中心用长征十一号固体运载火箭,采用``一箭五星''的方式成功将``珠海一号''02组卫星发射升空,卫星进入预定轨道}}。\\
32、4月26日电,记者近日从中建材蚌埠玻璃院获悉:{\textbf{蚌埠中建材信息显示材料有限公司0.12毫米超薄电子触控玻璃日前成功下线,继0.15毫米之后,又一次创造了浮法技术工业化生产的世界最薄玻璃纪录}}。

33、4月26日电,记者从六盘山国家级自然保护区管理局获悉:该局经过半年资料比对,\textbf{{已确认去年采用红外相机在六盘山国家级自然保护区王化南林场拍摄发现的新鹿种为赤麂。这是宁夏兽类分布新记录种,也是该物种在我国北方地区分布的首次记录}}。\\
34、{\textbf{首届中国国际低碳科技博览会4月22日---26日在上海世博展览馆举行,博览会以``低碳科技,点亮未来''为主题,成为低碳技术装备展示交易平台,促进低碳技术产业化}}。

35、4月27日电,十三届全国人大常委会第二次会议27日下午在北京人民大会堂闭幕。会议经表决,{\textbf{通过了人民陪审员法、英雄烈士保护法、关于修改国境卫生检疫法等六部法律的决定,国家主席习近平分别签署第4、5、6号主席令予以公布。栗战书委员长主持会议}}。

36、4月27日,{\textbf{中国人民银行、中国银行保险监督管理委员会、中国证券监督管理委员会联合发布《关于加强非金融企业投资金融机构监管的指导意见》,}}规范非金融企业投资金融机构行为,强化对非金融企业投资金融机构的监管,促进非金融企业与金融机构良性互动发展。

\subsection{24668-2018年4月国际时事}
1、埃及全国选举委员会主席拉欣·易卜拉欣4月2日宣布,{\textbf{现任总统塞西在2018年埃及总统选举中赢得97.08%的有效选票,成功获得连任}}。

2、日本政府4月3日召开内阁会议,{\textbf{正式批准2019年天皇退位和皇太子即位相关仪式的基本方针}}。\\
3、中国常驻世界贸易组织代表张向晨4月4日发表书面声明说,美国根据301调查报告公布拟对华采取的关税措施,蓄意严重违反世贸组织最基本、最核心的规则和精神,是典型的单边主义和贸易保护主义行径。对此,中方强烈谴责,坚决反对,并准备对美产品采取对等措施。

4、俄罗斯总统普京4月3日抵达土耳其首都安卡拉,开启为期两天的访问,{\textbf{这是普京再度当选俄罗斯总统后的首次出访。土耳其总统埃尔多安与普京当天在安卡拉出席土俄高级别合作理事会会议,双方就加强国防、能源等领域合作达成多项共识}}。\\
5、{\textbf{因``亲信干政''事件被逮捕的韩国前总统朴槿惠4月6日被韩国首尔中央地方法院一审判处有期徒刑24年,并处罚金180亿韩元(约合1.06亿元人民币)。}}

6、菲律宾住建署4月6日表示,{\textbf{由5家中国公司和3家菲律宾公司组成的联营体提交的马拉维市重建方案已通过专家初步评审,成为拟中标方}}。\\
7、针对美国发布的对华征税产品建议清单,中国政府4月4日作出回应,将对原产自美国的106项进口商品加征25\%的关税。其中,原产自美国的大豆、玉米等农产品首当其冲。{\textbf{作为中国农产品的重要进口来源国之一,巴西对此密切关注,相关行业期待以此为契机,继续扩大在中国的出口份额}}。

8、4月9日,根据博鳌亚洲论坛会员大会选举结果,博鳌亚洲论坛产生新一届理事会。在随后召开的新一届理事会会议上,{\textbf{第八任联合国秘书长潘基文当选理事长}}。\\
9、人民币原油期货日前成功上市,{\textbf{首周总成交额近1160亿元,成交量超过27万手,日均5.4万手,合5400万桶原油。这一日均数量超过800万桶的迪拜商品交易所(DME),跃居世界第三,仅次于伦敦布伦特原油期货和美国西得克萨斯轻质(WTI)原油期货。值得指出的是,人民币原油期货的看点其实是人民币,这是继加入特别提款权(SDR)之后人民币国际化的又一个里程碑,将增强人民币的国际影响力}}。

10、{\textbf{上海合作组织成员国元首理事会第十八次会议将于今年6月在山东青岛举行}}。4月10日,峰会新闻中心网站(\href{http://www.scochn2018.cn/}{www.scochn2018.cn})正式开通上线,媒体注册系统(registration.scochn2018.cn)同步开通。\\
11、新华社记者4月14日凌晨在叙利亚首都大马士革听到空中传来巨大爆炸声,{\textbf{叙利亚国家电视台说美英法三国对叙利亚``发动了侵略''。}}

12、4月17日,{\textbf{联合国教科文组织正式批准中国提交申报的四川光雾山---诺水河地质公园、湖北黄冈大别山地质公园成为联合国教科文组织世界地质公园。这也是我国第三十六、三十七个世界地质公园}}。\\
13、4月19日,古巴第九届全国人民政权代表大会在哈瓦那闭幕,{\textbf{大会选举迪亚斯---卡内尔为新一任古巴国务委员会主席兼部长会议主席,接替86岁的劳尔·卡斯特罗,成为新一任古巴国家元首兼政府首脑}}。

14、据叙利亚国家电视台4月19日报道,盘踞在大马士革北部加拉蒙地区的``伊斯兰军''等反政府武装已从杜迈尔镇全部撤离,叙政府军当天已进驻杜迈尔镇。\\

15、俄罗斯外交部4月21日表示,{\textbf{欢迎朝鲜自21日起中止核试验与洲际弹道导弹发射试验的决定}}。\\
16、韩国总统府青瓦台一名官员4月23日说,韩国和朝鲜代表当天在板门店举行实务会谈,商定首脑会晤流程。会晤将于27日上午开始,韩国总统文在寅将为朝鲜最高领导人金正恩举行正式欢迎仪式和宴会。

17、关于``森友学园''出售国有土地审批文件篡改问题,曾任日本财务省理财局长的前国税厅长官佐川宣寿向周边人士承认参与篡改。日本大阪地检特搜部日前对佐川进行了询问。据分析,检方可能主要确认了佐川是否下达篡改指示及其动机。检方今后将判断能否以涉嫌伪造公文等立案。\\
18、近日,墨西哥和欧盟宣布双方已达成原则性协议,将对2000年签署的墨欧自由贸易协定进行更新升级,双方专业团队还将继续就协议细节进行磋商。\\
19、{\textbf{第三十二届东盟峰会及系列会议4月25日在新加坡拉开帷幕}},东盟10国领导人和代表将讨论如何推动创新发展、建立智慧城市网络以及加强网络安全合作等议题。

20、4月27日电(夏雪)当地时间27日上午9时30分许(北京时间8时30分),{\textbf{朝鲜最高领导人金正恩在板门店跨过军事分界线,与韩国总统文在寅会晤。两位领导人在军事分界线握手,开始历史性的会面}}。\\
21、外交部发言人华春莹26日表示,{\textbf{印度和巴基斯坦加入上海合作组织后,上合组织已成为人口最多、地域最广、潜力巨大的综合性区域组织,将在地区和国际事务中发挥更加积极的作用}}。\\
22、4月27日电,{\textbf{朝鲜国务委员会委员长金正恩和韩国总统文在寅27日在板门店举行会晤并签署《板门店宣言》,宣布双方将为实现朝鲜半岛无核化和停和机制转换而共同努力}}。

23、据菲律宾媒体报道,{\textbf{菲旅游胜地长滩岛26日如期封岛。}}长滩岛将关闭半年,以修复环境并进行基础设施修缮升级。封岛后,长滩岛将禁止游客上岛,只有当地居民和持有许可证的人能留在岛上。为弥补封岛给岛上居民带来的损失,菲政府计划发放救济金

\subsection{24670-2018年5月国际时事}
1、据朝中社4月30日报道,{\textbf{朝鲜最高人民会议常任委员会为统一北南时间,于30日发布《关于修改平壤时间》的政令,将平壤时间改为以东经135度为基准子午线的东9区标准时间(比现在的时间早30分钟),并从5月5日开始使用修改后的时间。}}\\
2、{\textbf{朝鲜劳动党委员长、国务委员会委员长金正恩5月3日在党中央总部会见了正在朝鲜访问的国务委员兼外交部长王毅。}}金正恩请王毅转达他对习近平主席的亲切问候。金正恩表示,朝中友谊是两国老一辈领导人留下的宝贵遗产,弥足珍贵。巩固和发展朝中友好合作是朝方坚定不移的战略方针。不久前我对中国进行了历史性访问,同习近平主席广泛深入交流,达成重要共识,取得丰硕成果。朝方愿同中方一道,推动朝中友好关系迈向新的更高阶段。朝方高度评价中方为朝鲜半岛和平稳定所作出的积极贡献,愿同中方加强战略沟通。金正恩说,实现半岛无核化是朝方的坚定立场。一段时间以来,半岛局势出现的积极变化是有意义的,有利于半岛问题的和平解决。朝方愿通过恢复对话,建立互信,探讨消除威胁半岛和平的根源。

3、{\textbf{亚洲基础设施投资银行5月2日在北京宣布其理事会已批准2个意向成员加入,成员总数增至86个。这一轮包括域内成员巴布亚新几内亚和域外成员肯尼亚。}}\\
4、中国地震台网正式测定:{\textbf{05月05日06时32分在夏威夷群岛(北纬19.39度,西经155.10度)发生6.9级地震,震源深度10千米。}}

5、{\textbf{美国航天局5月5日凌晨从加利福尼亚州中部发射``洞察''号火星无人着陆探测器,首次探索这颗红色星球``内心深处''的奥秘。}}\\
6、5月7日,{\textbf{俄罗斯联邦新一届总统普京在克里姆林宫宣誓就职,正式开始其第四个总统任期,这也是俄罗斯现代历史上的第七次总统就职仪式。}}随着俄联邦宪法法院院长佐尔金将象征总统权力的徽标授予普京,宣布他正式就任总统,人们对普京的第四任期以及俄罗斯未来6年内政和外交走向充满了期待。

7、5月8日电,本格拉消息:{\textbf{当地时间5月5日,中铁二十局创办的安哥拉首家铁路职业技能培训学校在本格拉揭牌成立,安哥拉国内铁路的本地化发展迈出重要一步。}}\\
8、据人社部消息,{\textbf{中国和日本近日签署社会保障协定。中日政府间社会保障协定谈判于2011年正式启动,双方于2018年1月共同对外宣布实质性结束谈判。}}日本将免除中国在日本投资企业的派遣员工、船员、空乘人员、外交领事机构人员和公务员缴纳厚生年金和国民年金的义务,中国将免除日本上述人员缴纳职工基本养老保险的义务。\\
9、俄罗斯著名民意调查机构``社会舆论''基金会10日发布的民调结果显示,该机构于今年4月29日在俄104个居民区,对1500名年龄不低于18岁的俄公民进行了问卷调查,48%的俄受访民众认为,在世界各国当中,{\textbf{与中国开展经济合作对俄罗斯最为重要。选择中国的这一比例在排行榜上位列第一。}}

10、联合国秘书长古特雷斯近日通过发言人发表声明,{\textbf{对也门冲突骤然升级表示严重关切,呼吁冲突各方克制,避免冲突进一步升级。他提醒各方,通过也门内部对话找到政治解决方案是结束冲突和解决人道主义危机的唯一出路。乱象丛生的也门局势日前再趋紧张,持续多年的内战延绵不绝,未来形势的发展令人担忧。}}\\
11、德国总理默克尔日前表示,目前欧盟的防务合作已经取得巨大进展,{\textbf{去年签署了``永久结构性合作''防务协议,这是在北约防务之外的``补充''。德国在防务问题上既需要同北约保持合作,也要加强同欧盟的联系。}}\\
12、5月14日至16日,{\textbf{联合国亚太经社会第七十四届年会在泰国曼谷举行。}}第七十二届联大主席莱恰克、亚太经社会执行秘书阿赫塔尔、东道主泰国外长敦和成员代表等约500人与会。

13、5月18日中午12时08分,古巴航空公司从墨西哥一家航空公司租用的一架波音737客机在古巴首都哈瓦那何塞·马蒂国际机场起飞不久后坠毁,机上114人中仅3人幸免于难。事故发生后,现场救出的4名幸存者被送往当地医院救治,截至记者发稿时,已有1名男子伤重去世,3名女子伤势严重。\\
14、{\textbf{刚果(金)卫生部5月20日发布公报显示,该国自本月初发生埃博拉疫情以来疫区已发现有26人死亡。}}公报称,本轮埃博拉疫情确诊病例已达21例,其中2人确诊死于埃博拉出血热。令人担忧的是,该国一座拥有上百万人口的城市首次出现埃博拉出血热病例,显示此次埃博拉疫情正从农村地区向大城市蔓延。\\
15、{\textbf{中国北京市与柬埔寨金边市5月21日在金边签订建立友好城市关系协议书}}。根据协议,北京和金边将本着互利原则,在经济、贸易、环保、旅游、科技、文化、教育、体育、卫生、人才等方面开展多种形式的交流与合作,促进共同繁荣发展。

16、委内瑞拉国家选举委员会5月20日晚宣布,{\textbf{执政党统一社会主义党候选人、现任总统马杜罗再次当选委内瑞拉总统。}}\\
17、{\textbf{上海合作组织成员国安全会议秘书第十三次会议5月22日在北京举行,国务委员、公安部部长赵克志主持会议并作主旨发言。}}印度、哈萨克斯坦、吉尔吉斯斯坦、巴基斯坦、俄罗斯、塔吉克斯坦、乌兹别克斯坦等上合组织成员国代表,上合组织秘书处和地区反恐怖机构执委会代表参加会议。

18、5月22日上午8时,由6艘执法艇组成的联合巡逻执法编队在中国西双版纳关累港鸣笛启航,{\textbf{标志着第七十次中老缅泰湄公河联合巡逻执法正式启动。}}\\
利用设在智利的陆基天文观测站,日本名古屋大学等机构研究人员发现了宇宙中迄今已知最遥远的氧元素,再次打破相关纪录。这项发现有望为研究星系演化提供重要线索。\\
19、5月21日,美国国务卿蓬佩奥发表其上任后首个对伊朗政策演讲,向伊朗提出12项要求,作为停止制裁伊朗的条件。伊朗方面随即做出回应,表示美国提出的``苛刻要求''不可接受。22日,美国财政部宣布制裁5名伊朗人,{\textbf{自美国退出伊核协议后,美财政部已经发起多轮针对伊朗的制裁。分析认为,美国单方面退出伊核协议,并不断威胁对伊朗实行``最严厉制裁'',给日趋紧张的美伊关系蒙上了一层阴影。}}

20、{\textbf{朝鲜5月24日在位于朝鲜东北部吉州郡的丰溪里核试验场的多条坑道进行爆破,并拆除相关设施,正式宣布废弃这座核试验场。}}\\
21、5月24日,{\textbf{法国总统马克龙开始对俄罗斯进行为期两天的正式访问。}}此前,俄罗斯总统普京已经与到访的德国总理默克尔在索契举行了工作会晤,双方就伊核协议前途、叙利亚局势、推进``北溪---2''输气项目等广泛议题进行了讨论。分析认为,德法领导人在一周之内相继访问俄罗斯,显示俄欧关系的``坚冰''有望消融。最近一段时间,欧盟在伊核协议和钢铝关税问题上均对美国展现强硬姿态,在美欧关系裂痕加大的背景下,欧盟和俄罗斯都展现出改善双边关系的强烈意愿。

22、5月24日,{\textbf{为期3天的第二十二届圣彼得堡国际经济论坛在俄罗斯圣彼得堡开幕。}}来自70多个国家和地区的1万多名政府、工商及企业代表齐聚一堂,探讨俄罗斯以及全球和地区经济领域的议题。即将于6月在中国青岛举行的上合组织成员国元首理事会会议,成为此次论坛的热门话题。就上合组织未来如何进一步提高经济合作水平,各方出谋划策、凝聚共识。\\
23、5月26日,{\textbf{中国共产党与世界政党高层对话会框架下的第二届中拉政党论坛在广东深圳开幕。}}本届论坛由中共中央对外联络部主办,以``改革、创新与党的建设''为主题,来自近30个国家的60多个政党和地区性组织的80余名外方代表参加。会议现场,与会嘉宾期望深入了解中国共产党治国理政的成功经验,借鉴中国智慧和中国方案。\\
24、{\textbf{第七十一届世界卫生大会5月26日在日内瓦落下帷幕。}}大会通过了以``3个10亿''健康目标为核心的未来5年战略计划,规划了新的行动方向,旨在使全球享有更好医疗保健和健康福利的人口显著增加。

25、{\textbf{2018年亚太经合组织(APEC)贸易高官会议5月26日在巴布亚新几内亚首都莫尔兹比港闭幕。}}会议通过了《第二十四届APEC贸易部长会议声明》和《支持多边贸易体系主席声明》,就促进多边贸易达成广泛共识。

\subsection{24672-2018年6月国际时事}
1、{\textbf{欧盟委员会5月29日发布2021年至2027年欧盟财政预算案细则,计划增加对南欧国家的支出,削减对东欧国家的补贴。尤其在占欧盟预算1/3、旨在弥补欧盟成员国间贫富差距的团结基金上,波兰和匈牙利将分别面临23\%和24\%的削减}}。

2、6月1日,{\textbf{意大利新政府在意大利总统府宣誓就职,孔特任新政府总理}}。

3、{\textbf{西班牙议会6月1日以180票赞成、169票反对和1票弃权通过对首相拉霍伊的弹劾案,拉霍伊被弹劾下台,工人社会党总书记桑切斯接任西班牙首相}}。

4、{\textbf{6月3日,为期3天的第十七届亚洲安全会议暨香格里拉对话会在新加坡落下帷幕}},共有来自40多个国家和地区的600多名代表参会。由中国人民解放军军事科学院副院长何雷中将率领的中方代表团参加了会议全部主要活动。会议期间,中国代表团重点宣介关于中国构建人类命运共同体、新型国际关系等重要理念,积极阐述实现亚太地区长治久安的``中国方案'',引发强烈反响。

5、{\textbf{6月6日在法国巴黎联合国教科文组织总部举行的《保护非物质文化遗产公约》缔约国大会第七届会议上,中国以123票高票当选保护非物质文化遗产政府间委员会委员国,本届新当选委员国的任期从2018年到2022年}}。

6、总部位于日内瓦的联合国贸易和发展会议(贸发会议)6月6日发布报告表示,{\textbf{2017年中国已成为全球第二大外资流入国和第三大对外投资国,并继续成为发展中国家中最大的外资流入国和对外投资国。}}

7、\textbf{{第72届联合国大会6月8日选举南非、比利时、多米尼加、德国、印度尼西亚5国为2019年和2020年安理会非常任理事国}}。

8、当地时间6月9日,\textbf{{美国总统特朗普登上``空军一号''前往新加坡,预计周二(12日)在新加坡会见朝鲜最高领导人金正恩}}。

9、``七国集团''(G7)峰会6月8日在加拿大揭幕。峰会前夕,美、法、加三国领导人在社交媒体上率先唇枪舌剑,美国总统特朗普和法国总统马克龙更上演一场``兄弟反目''。目前,特朗普没有等到G7峰会结束,就于9日离开了加拿大魁北克,加拿大总理特鲁多表示,各国已经商定了峰会最终公报。

10、希腊总理齐普拉斯和马其顿总理扎埃夫6月12日宣布,{\textbf{两国就国名问题达成历史性协议,马其顿将改名为``北马其顿共和国''}}。

11、6月14日晚,{\textbf{第二十一届世界杯足球赛在俄罗斯首都莫斯科开幕}}。国家主席习近平特使、国务院副总理孙春兰应邀出席开幕式。国际足联世界杯是世界足坛竞技水平最高、知名度最高的一项赛事。这是俄罗斯历史上首次承办世界杯足球赛,也是世界杯首次在东欧国家举行。未来32天时间里,32支球队将通过64场赛事争夺代表世界足球最高荣誉的``大力神杯''。除东道主队外,来自亚洲、欧洲、非洲、北中美及加勒比地区、南美洲五个赛区的31支队伍在历时两年多的预选赛后锁定世界杯决赛圈名额。当日,在莫斯科进行的2018俄罗斯世界杯揭幕战中,俄罗斯队以5∶0大胜沙特阿拉伯队。

12、韩国国防部6月14日发表声明说,{\textbf{韩朝双方当天在板门店举行高级别军事会谈,就恢复朝鲜半岛东西海域军事通信线路达成一致。此次会谈是双方时隔10年6个月重新启动将军级军事会谈}}。

13、国家主席习近平6月14日在人民大会堂会见美国国务卿蓬佩奥。习近平指出,{\textbf{中美两国在维护世界和平稳定、促进全球发展繁荣方面拥有广泛共同利益、肩负重要责任。中美合作可以办成有利于两国和世界的大事,希望双方团队按照我同特朗普总统北京会晤达成的共识,加强沟通,增进互信,管控分歧,扩大合作,推动中美关系沿着正确轨道向前发展,更好造福两国人民和世界各国人民}}。

14、据日本气象厅消息,当地时间6月18日上午7时58分,日本大阪府发生里氏6.1级地震。地震已造成4人死亡、370多人受伤。

15、联合国也门问题特使马丁·格里菲思6月16日抵达也门首都萨那,就也门港口城市荷台达发生的战事与胡塞武装进行磋商。截至18日,格里菲思两周内在也门政府和胡塞武装代表之间的第二次斡旋仍毫无进展。在沙特阿拉伯领导的多国联军支援下,也门政府军16日收复荷台达国际机场,随后巩固了对荷台达周边地区的控制。

16、当地时间6月19日,{\textbf{美国宣布退出联合国人权理事会(UNHRC),指责人权理事会``长久以来对以色列存有偏见''}}。联合国对美国做出该决定表示失望。

17、{\textbf{2018年美国中期选举已经揭开两党大规模初选的序幕}}。截至6月12日,美国50州已有半数举行党内预选。尽管距离11月正式的选举日尚有数月,但共和党、民主党两党的``烧钱战''已打得如火如荼。美国媒体统计,过去四届中期选举的花费不断飙升,本届选举花费有望超过上届的38.4亿美元,创造新的纪录。

18、{\textbf{欧盟6月20日正式批准对美国产品加征报复性关税的实施条例,22日起将对美国产品加征关税}}。

19、{\textbf{第72届联合国大会主席莱恰克的发言人瓦尔马6月20日说,美国退出联合国人权理事会后,联大将从地区国家中新选出一个国家填补这一空缺}}。

20、{\textbf{6月23日是英国``脱欧''公投两周年}}。20日,英国议会表决通过了政府提出的《退出欧盟法案》,确立明年3月29日正式退出欧盟后的法律框架。这一关键性法案的通过,标志着议会正式放弃了在``脱欧''进程中的最终决定权,为英国首相特雷莎·梅领导的政府主导同欧盟谈判``脱欧''扫清了最后的障碍。

21、{\textbf{联合国秘书长古特雷斯6月22日在美国纽约会见出席第二届联合国警长峰会的中国代表团时,高度评价中国为联合国维和事业作出的重要贡献}}。

22、{\textbf{2018年6月26日是第三十一个国际禁毒日,今年禁毒日主题是``抵制毒品,参与禁毒''}}。连日来,泰国、缅甸、柬埔寨、老挝等东南亚国家都积极通过新闻媒体开展禁毒宣传,并举行毒品销毁活动。

23、2018年6月26日电,亚洲基础设施投资银行(亚投行)理事会26日在此间举行的年会上宣布,{\textbf{已批准黎巴嫩作为意向成员加入,其成员总数将增至87个}}。成立两年半以来的成绩单,与会人士纷纷表示赞赏,认为亚投行持续扩容是国际社会投给中国的一张信任票,作为21世纪成立的多边发展机构,中国展示了令人敬佩的姿态。

24、国家主席习近平6月27日在人民大会堂会见了来访的美国国防部长马蒂斯。{\textbf{中美关系是世界上最重要的双边关系之一。}}中美建交近40年的历史和现实表明,中美关系发展得好,可以造福两国人民和各国人民,有利于世界和地区的和平、稳定、繁荣。中美在广泛领域存在共同利益,双方的共同点远远大于分歧。宽广的太平洋可以容纳中美两国和其他国家。中美双方应该本着相互尊重、合作共赢的原则推进两国关系发展。

25、{\textbf{6月28日,中国企业在欧加登盆地为埃塞俄比亚产出了第一桶原油,标志着该国油气产业发展进入新阶段}}。

\subsection{24673-2018年7月国内时事}
1、中共中央政治局6月29日下午就加强党的政治建设举行第六次集体学习。中共中央总书记习近平在主持学习时强调,{\textbf{马克思主义政党具有崇高政治理想、高尚政治追求、纯洁政治品质、严明政治纪律。如果马克思主义政党政治上的先进性丧失了,党的先进性和纯洁性就无从谈起。这就是我们把党的政治建设作为党的根本性建设的道理所在。党的政治建设是一个永恒课题。要把准政治方向,坚持党的政治领导,夯实政治根基,涵养政治生态,防范政治风险,永葆政治本色,提高政治能力,为我们党不断发展壮大、从胜利走向胜利提供重要保证。习近平强调,政治方向是党生存发展第一位的问题,事关党的前途命运和事业兴衰成败。我们所要坚守的政治方向,就是共产主义远大理想和中国特色社会主义共同理想、``两个一百年''奋斗目标,就是党的基本理论、基本路线、基本方略。加强党的政治建设就是要发挥政治指南针作用,引导全党坚定理想信念、坚定``四个自信'',把全党智慧和力量凝聚到新时代坚持和发展中国特色社会主义伟大事业中来;就是要推动全党把坚持正确政治方向贯彻到谋划重大战略、制定重大政策、部署重大任务、推进重大工作的实践中去,经常对表对标,及时校准偏差,坚决纠正偏离和违背党的政治方向的行为,确保党和国家各项事业始终沿着正确政治方向发展;就是要把各级党组织建设成为坚守正确政治方向的坚强战斗堡垒,教育广大党员、干部坚定不移沿着正确政治方向前进}}。

2、为贯彻落实习近平总书记在博鳌亚洲论坛年会上重要讲话精神,经党中央、国务院同意,国家发展改革委、商务部于6月30日以第19号令,{\textbf{发布了《自由贸易试验区外商投资准入特别管理措施(负面清单)(2018年版)》,自2018年7月30日起施行,适用于所有自由贸易试验区。修订后,自由贸易试验区负面清单由2017年版95条措施减至2018年版45条措施,在全国负面清单开放措施基础上,在更多领域试点取消或放宽外资准入限制}}。

3、习近平7月2日下午在中南海同团中央新一届领导班子成员集体谈话并发表重要讲话,他强调,{\textbf{青年一代有理想、有本领、有担当,国家就有前途、民族就有希望。代表广大青年、赢得广大青年、依靠广大青年是我们党不断从胜利走向胜利的重要保证。中华民族伟大复兴的中国梦终将在一代代青年的接力奋斗中变为现实。新时代的青年工作要毫不动摇坚持党的领导,坚定不移走中国特色社会主义群团发展道路,紧紧围绕、始终贯穿为实现中国梦而奋斗的主题,让广大青年敢于有梦、勇于追梦、勤于圆梦}}。

4、当地时间7月2日11时36分,{\textbf{在巴林麦纳麦举行的第四十二届世界遗产大会上,经联合国教科文组织世界遗产委员会同意,中国贵州梵净山获准列入《世界遗产名录》。至此,我国世界遗产增至53处,世界自然遗产增至13处。世界自然遗产总数超越之前并列的澳大利亚和美国,居世界第一}}。

5、6月30日晚间,{\textbf{国内首个大型商业化槽式光热电站------中广核新能源德令哈50兆瓦光热项目一次带电并网成功,填补了我国大规模槽式光热发电技术的空白,使我国正式成为世界上第八个拥有大规模光热电站的国家}}。

6、全国组织工作会议7月3日至4日在北京召开。中共中央总书记、国家主席、中央军委主席习近平出席会议并发表重要讲话。他强调,{\textbf{中国特色社会主义进入新时代,我们党一定要有新气象新作为,关键是党的建设新的伟大工程要开创新局面。伟大斗争、伟大工程、伟大事业、伟大梦想,其中起决定性作用的是党的建设新的伟大工程。要把新时代坚持和发展中国特色社会主义这场伟大社会革命进行好,我们党必须勇于进行自我革命,把党建设得更加坚强有力。新时代党的组织路线是:全面贯彻新时代中国特色社会主义思想,以组织体系建设为重点,着力培养忠诚干净担当的高素质干部,着力集聚爱国奉献的各方面优秀人才,坚持德才兼备、以德为先、任人唯贤,为坚持和加强党的全面领导、坚持和发展中国特色社会主义提供坚强组织保证。新时代党的组织路线是理论的也是实践的,要在推进党的建设新的伟大工程、落实全面从严治党的实践中切实贯彻落实}}。

7、中共中央总书记、国家主席、中央军委主席习近平近日对实施乡村振兴战略作出重要指示强调,{\textbf{实施乡村振兴战略,是党的十九大作出的重大决策部署,是新时代做好``三农''工作的总抓手。各地区各部门要充分认识实施乡村振兴战略的重大意义,把实施乡村振兴战略摆在优先位置,坚持五级书记抓乡村振兴,让乡村振兴成为全党全社会的共同行动。习近平指出,要坚持乡村全面振兴,抓重点、补短板、强弱项,实现乡村产业振兴、人才振兴、文化振兴、生态振兴、组织振兴,推动农业全面升级、农村全面进步、农民全面发展。要尊重广大农民意愿,激发广大农民积极性、主动性、创造性,激活乡村振兴内生动力,让广大农民在乡村振兴中有更多获得感、幸福感、安全感。要坚持以实干促振兴,遵循乡村发展规律,规划先行,分类推进,加大投入,扎实苦干,推动乡村振兴不断取得新成效}}。\\

8、习近平7月6日下午主持召开中央全面深化改革委员会第三次会议并发表重要讲话。他强调,{\textbf{党的十九大以来,党中央在深化党的十八大以来改革成果的基础上,不失时机推进重大全局性改革,全面深化改革取得新的重大进展。继续推进改革,要把更多精力聚焦到重点难点问题上来,集中力量打攻坚战,激发制度活力,激活基层经验,激励干部作为,扎扎实实把全面深化改革推向深入}}。

9、7月5日18时45分许,两艘载有中国游客的游船在泰国普吉岛附近海域突遇特大暴风雨发生倾覆事故。截至目前,船上127名中国游客中,16人死亡、33人失踪,另有78人获救,受伤游客已送当地医院治疗。事故发生后,中共中央总书记、国家主席、中央军委主席习近平作出重要指示,{\textbf{外交部和我驻泰国使领馆要加大工作力度,要求泰国政府及有关部门全力搜救失踪人员,积极救治受伤人员。文化和旅游部要配合做好相关工作。习近平强调,目前正值暑期,外出旅游人员较多,一些地方雨情汛情等情况突出,各地区各有关部门要及时进行风险提示,提醒旅行社和游客增强风险防范意识,消除安全隐患,加强安全检查监测和应急工作,切实保障人民群众生命财产安全}}。

10、7月7日,生态文明贵阳国际论坛2018年年会在贵州省贵阳市开幕。国家主席习近平向论坛年会致贺信。习近平指出,{\textbf{生态文明建设关乎人类未来,建设绿色家园是各国人民的共同梦想。国际社会需要加强合作、共同努力,构建尊崇自然、绿色发展的生态体系,推动实现全球可持续发展。中国高度重视生态环境保护,秉持绿水青山就是金山银山的理念,倡导人与自然和谐共生,坚持走绿色发展和可持续发展之路。我们愿同国际社会一道,全面落实2030年可持续发展议程,共同建设一个清洁美丽的世界}}。\\

11、国家``十三五''集成电路重大生产力布局规划的重点项目------晋华存储器集成电路生产线已进入全面竣工倒计时,{\textbf{国内首个拥有自主技术的千亿级内存制造产业呼之欲出}}。福建晋江,这座素以运动鞋、纺织服装等轻工产品闻名的城市,眼下正以集成电路、石墨烯、高效光伏等一批新兴产业来撬动自身发展。40年改革开放积聚的能量,如泉州湾的潮水澎湃激荡;16年前时任福建省省长的习近平同志总结提炼的``晋江经验'',给这片发展热土以思想的引领和精神的滋养。

12、中阿合作论坛第八届部长级会议7月10日在人民大会堂开幕。{\textbf{国家主席习近平出席开幕式并发表题为《携手推进新时代中阿战略伙伴关系》的重要讲话}},宣布中阿双方一致同意,建立全面合作、共同发展、面向未来的中阿战略伙伴关系。习近平强调,中方倡议共建``一带一路'',得到包括阿拉伯世界在内的国际社会广泛支持和积极参与。作为历史上丝路文明的重要参与者和缔造者之一,阿拉伯国家身处``一带一路''交汇地带,是共建``一带一路''的天然合作伙伴。双方携手同行,把``一带一路''同地区实际结合起来,把集体行动同双边合作结合起来,把促进发展同维护和平结合起来,优势互补,合作共赢,造福地区人民和世界人民。``一带一路''建设全面带动中阿关系发展,中阿全方位合作进入新阶段。中方愿同阿方加强战略和行动对接,携手推进``一带一路''建设,共同做中东和平稳定的维护者、公平正义的捍卫者、共同发展的推动者、互学互鉴的好朋友。\\

13、为深入贯彻习近平新时代中国特色社会主义思想和党的十九大精神,全面贯彻习近平强军思想,延揽社会优秀人才为军队建设服务,{\textbf{经中央军委批准,中央军委政治工作部近日部署展开深化国防和军队改革以来全军首次面向社会公开招考文职人员工作}}。

14、{\textbf{中共中央政治局常委、国务院副总理韩正7月10日在国家医疗保障局调研并主持召开座谈会。}}韩正表示,要坚持以人民为中心的发展思想,完善统一的城乡居民基本医疗保险制度和大病保险制度,着力解决医疗保障领域发展不平衡不充分问题。要千方百计保基本,坚持尽力而为、量力而行,聚焦基本医疗需求,满足人民群众最迫切的愿望和要求。要始终做到可持续,健全医保筹资机制,强化医保基金监管,确保医保资金合理使用、安全可控。要发挥好医保的基础性、引导性作用,实行医疗、医保、医药``三医联动'',形成协同推进医改的良好格局。

15、外交部发言人华春莹7月11日在回应美方公布拟对中国2000亿美元输美产品加征关税清单时说,这是一场单边主义与多边主义,保护主义与自由贸易,强权与规则之战。中方将和国际社会一道,站在历史正确一边,共同维护多边贸易体制和规则。\\

16、习近平近日对中央和国家机关推进党的政治建设作出重要指示强调,{\textbf{中央和国家机关首先是政治机关,必须旗帜鲜明讲政治,坚定不移加强党的全面领导,坚持不懈推进党的政治建设。希望中央和国家机关各级党组织和广大党员干部牢固树立``四个意识'',坚定``四个自信'',带头维护党中央权威和集中统一领导,在深入学习贯彻新时代中国特色社会主义思想上作表率,在始终同党中央保持高度一致上作表率,在坚决贯彻落实党中央各项决策部署上作表率,建设让党中央放心、让人民群众满意的模范机关}}。

17、近日,中共中央办公厅印发了《关于党的基层组织任期的意见》,{\textbf{党的基层委员会每届任期一般为5年,党的总支部委员会、支部委员会每届任期一般为3年,其中,村和社区党的委员会、总支部委员会、支部委员会每届任期为5年。}}

18、{\textbf{2018年是习近平同志提出``晋江经验''16周年。}}从1985年到2002年,习近平同志在福建工作了17年,进行了大量的实践探索和理论思考,取得了极其丰硕的成果,成为新时代中国特色社会主义思想的重要来源。2002年,时任福建省省长的习近平同志在深入调研的基础上,将晋江经济社会持续快速发展的成功经验提炼概括为``六个始终坚持''和``处理好五大关系''的``晋江经验''。习近平同志指出,``晋江经验''是晋江人民对中国特色社会主义发展道路的大胆探索和成功实践。

19、习近平7月13日主持召开中央财经委员会第二次会议并发表重要讲话。他强调,{\textbf{关键核心技术是国之重器,对推动我国经济高质量发展、保障国家安全都具有十分重要的意义,必须切实提高我国关键核心技术创新能力,把科技发展主动权牢牢掌握在自己手里,为我国发展提供有力科技保障}}。

20、7月13日,中共中央总书记习近平在人民大会堂会见中国国民党前主席连战率领的台湾各界人士参访团时强调,{\textbf{大道之行、人心所向,势不可挡。我们有充分的信心和足够的能力,牢牢把握正确方向,坚定不移推动两岸关系和平发展、推进祖国和平统一进程。希望两岸同胞共同努力,坚持体现一个中国原则的``九二共识'',坚决反对和遏制``台独'',扩大深化两岸各领域交流合作,增进同胞亲情福祉,在新时代携手同心书写中华民族伟大复兴新篇章。习近平指出,``不畏浮云遮望眼,自缘身在最高层。''只要大家登高望远,就能看清主流、把握大势,共同推动两岸关系克难前行。习近平指出,``不忘初心,方得始终。''正确道路要坚持走下去。特别是在当前台海形势下,两岸同胞更要坚定信心,团结前行}}。

21、国家主席习近平7月16日在钓鱼台国宾馆会见来华出席第二十次中国欧盟领导人会晤的欧洲理事会主席图斯克和欧盟委员会主席容克。习近平指出,{\textbf{中国和欧盟同为世界和平的建设者、全球发展的贡献者、国际秩序的维护者。中方愿在相互尊重、公平正义、合作共赢的基础上,同欧方共同努力,推动中欧全面战略伙伴关系百尺竿头更进一步,促进双方经济社会发展和民众福祉}}。

22、7月16日,中国在世贸组织就美国301调查项下对我2000亿美元输美产品征税建议措施追加起诉。

23、15年前,时任浙江省委书记的习近平同志聚焦``如何发挥优势,如何补齐短板''两个关键问题,全面系统阐释浙江的八个优势,提出改革发展的八项举措。{\textbf{作为谋篇布局、开篇破题的大文章,``八八战略''是引领浙江发展的总纲领、推进浙江各项工作的总方略,为推动浙江经济社会发展提供了科学指南,其精神实质对全国各地区的改革发展也有着重大指导意义}}。

24、{\textbf{7月以来,我国多地出现大到暴雨,长江发生2次编号洪水,嘉陵江上游、涪江上游、沱江上游发生特大洪水,大渡河上中游发生大洪水,黄河发生1次编号洪水,部分中小河流发生超警以上洪水。}}今年截至7月18日,我国已有27个省(区、市)遭受洪涝灾害,造成2053万人、1759千公顷农作物受灾,因灾死亡54人、失踪8人,倒塌房屋2.3万间,直接经济损失约516亿元。习近平对当前汛情高度重视并作出重要指示,习近平强调,当前,正值洪涝、台风等自然灾害多发季节,相关地区党委和政府要牢固树立以人民为中心的思想,全力组织开展抢险救灾工作,最大限度减少人员伤亡,妥善安排好受灾群众生活,最大程度降低灾害损失。要加强应急值守,全面落实工作责任,细化预案措施,确保灾情能够快速处置。要加强气象、洪涝、地质灾害监测预警,紧盯各类重点隐患区域,开展拉网式排查,严防各类灾害和次生灾害发生。

25、中共中央总书记、国家主席、中央军委主席习近平日前对毕节试验区工作作出重要指示指出,{\textbf{30年来,在党中央坚强领导下,在社会各方面大力支持下,广大干部群众艰苦奋斗、顽强拼搏,推动毕节试验区发生了巨大变化,成为贫困地区脱贫攻坚的一个生动典型。}}在这一过程中,统一战线广泛参与、倾力相助,作出了重要贡献。

26、党的十八大以来,我国高度重视草原保护建设,全面推动草原事业发展,2017年全国天然草原鲜草总产量10.65亿吨,较上年增加2.53\%;全国天然草原鲜草总产量连续7年超过10亿吨,实现稳中有增。2017年草原综合植被盖度达55.3\%,较2011年提高4.3个百分点。{\textbf{我国天然草原面积达3.928亿公顷,约占全球草原面积12\%,居世界第一。}}

27、国家主席习近平7月20日在阿布扎比同阿联酋副总统兼总理穆罕默德、阿布扎比王储穆罕默德举行会谈。双方一致决定,建立中阿全面战略伙伴关系,加强各领域深度合作,推动两国关系在更高水平、更宽领域、更深层次上不断发展。

29、\textbf{{全国县乡国税地税机构7月20日正式合并,所有县级和乡镇新税务机构统一挂牌。}}经过36天的努力,全国省市县乡四级税务机构分步合并和相应挂牌工作全部完成,国税地税征管体制改革第一场攻坚战圆满收官,下一步改革将向逐级制定和落实``三定''规定、逐级接收社会保险费和非税收入征管职责划转等领域纵深推进。

30、为促进统一资产管理产品监管标准,推动银行理财业务规范健康发展,银保监会近日起草了《商业银行理财业务监督管理办法》,并向社会公开征求意见。《办法》要求商业银行理财业务实行分类管理,{\textbf{区分公募和私募理财产品。公募理财产品面向不特定社会公众公开发行,私募理财产品面向不超过200名合格投资者非公开发行;同时,将单只公募理财产品的销售起点由目前的5万元降至1万元}}。

31、2018年7月21日至22日,中国国家主席习近平对塞内加尔进行国事访问。{\textbf{这是习近平主席此行访问的首个西非国家,塞内加尔也成为第一个同中国签署``一带一路''合作文件的西非国家。塞内加尔各界纷纷表示,习近平主席访问塞内加尔是对塞中关系、非中关系的极大鼓舞与提振,塞中合作必将迈向更高水平}}。

32、国家药监局负责人2018年7月22日通报长春长生生物科技有限责任公司违法违规生产冻干人用狂犬病疫苗案件有关情况。{\textbf{现已查明,企业编造生产记录和产品检验记录,随意变更工艺参数和设备。上述行为严重违反了《中华人民共和国药品管理法》《药品生产质量管理规范》有关规定,国家药监局已责令企业停止生产,收回药品GMP证书,召回尚未使用的狂犬病疫苗。国家药监局会同吉林省局已对企业立案调查,涉嫌犯罪的移送公安机关追究刑事责任}}。

33、{\textbf{正在国外访问的中共中央总书记、国家主席、中央军委主席习近平对吉林长春长生生物疫苗案件作出重要指示指出,长春长生生物科技有限责任公司违法违规生产疫苗行为,性质恶劣,令人触目惊心。有关地方和部门要高度重视,立即调查事实真相,一查到底,严肃问责,依法从严处理。要及时公布调查进展,切实回应群众关切。习近平强调,确保药品安全是各级党委和政府义不容辞之责,要始终把人民群众的身体健康放在首位,以猛药去疴、刮骨疗毒的决心,完善我国疫苗管理体制,坚决守住安全底线,全力保障群众切身利益和社会安全稳定大局。根据习近平指示和李克强要求,国务院建立专门工作机制,并派出调查组进驻长春长生生物科技有限责任公司进行立案调查。调查组将抓紧完成案件查办、责任追查、风险隐患排查等工作。吉林省成立省市两级案件查处领导小组,配合国务院调查组做好相关工作,并结合此案件全面排查高风险药品企业。吉林省食品药品监督管理局已收回长春长生狂犬病疫苗药品GMP证书,停止该企业狂犬病疫苗生产及销售,暂停该企业所有产品批签发}}。

34、{\textbf{国家主席习近平7月24日在比勒陀利亚同南非总统拉马福萨举行会谈。}}两国元首高度评价中南传统友好,就推进新时期中南全面战略伙伴关系达成重要共识,一致同意加强高层往来,深化政治互信,对接发展战略,推进务实合作,密切人文交流,让两国人民更多享受中南合作成果。

35、国家药监局决定从严查处吉林长春长生疫苗案件。{\textbf{一是在前期工作基础上,进一步增加人员,充实案件查处工作领导小组力量,全力配合国务院调查组工作。二是对长春长生所有疫苗生产、销售全流程、全链条进行彻查,尽快查清事实真相,锁定证据线索。三是坚持重拳出击,对不法分子严惩不贷、以儆效尤;对失职渎职的,从严处理、严肃问责。四是针对人民群众关切的热点问题,做好解疑释惑工作。五是对全国疫苗生产企业全面开展飞行检查。六是对疫苗全生命周期监管制度进行系统分析,逐一解剖问题症结,研究完善我国疫苗管理体制。从吉林省纪委监委获悉:将对长春长生生物科技有限责任公司疫苗事件涉及的责任者依纪依法,依照工作职能坚决查处,严肃追责。}}

36、{\textbf{首届中国国际进口博览会倒计时100天誓师动员大会7月27日在上海举行。}}中共中央政治局委员、国务院副总理胡春华出席大会并讲话。胡春华强调,举办中国国际进口博览会,是习近平总书记亲自谋划、亲自提出、亲自部署推动的,是以习近平同志为核心的党中央着眼新一轮高水平对外开放作出的重大决策,是我们坚定支持贸易自由化、主动向世界开放市场的重大举措。要以习近平新时代中国特色社会主义思想为指导,深入学习领会贯彻总书记重要批示指示精神,按照党中央、国务院决策部署,扎扎实实做好各项筹备工作。

37、国务院总理李克强7月30日主持召开国务院常务会议,{\textbf{听取吉林长春长生公司违法违规生产狂犬病疫苗案件调查进展汇报,要求坚决严查重处并建立保障用药安全长效机制;部署优化教育经费使用结构和落实义务教育教师工资待遇,办好人民满意的教育。}}

\subsection{24674-2018年7月国际时事}
1、{\textbf{经过彻夜马拉松式的紧张谈判,6月29日清晨,欧盟各国领导人终于在难民问题上达成一致,表示将在欧盟境外建立难民``地区登陆平台'',在欧盟境内设置难民``安全中心'',以解决不断恶化的难民问题。}}意大利总理孔特对达成的协议表示满意,此前他曾威胁说,如果意大利的相关要求得不到满足,就不签协议。评论指出,该协议缺乏细节,在实施层面将面临很多挑战,不过却可能帮助德国总理默克尔解燃眉之急,她的政治生命不会就此结束。

2、6月30日---7月1日,{\textbf{区域全面经济伙伴关系协定(RCEP)第五次部长级会间会在东京举行。}}东盟10国、中国、澳大利亚、印度、日本、韩国、新西兰等16方经贸部长或代表出席会议。商务部副部长兼国际贸易谈判副代表王受文代表钟山部长参会。发展改革委、工业和信息化部、财政部、农业农村部和海关总署派员参会。

3、\textbf{{中国常驻联合国日内瓦办事处和瑞士其他国际组织代表俞建华2日在联合国人权理事会第三十八次会议上,代表近140个国家发表题为``坚持以人民为中心,促进和保护人权''的联合声明。}}

4、{\textbf{7月2日至3日,2018澜沧江---湄公河合作媒体峰会在老挝首都万象举行。}}本次峰会由人民日报社和老挝新闻文化和旅游部共同主办,来自中国、老挝、柬埔寨、泰国、缅甸、越南6个澜湄流域国家的40家媒体人士齐聚一堂,就进一步深化媒体合作、促进民心相通深入研讨建言献策。

5、{\textbf{由中国政府出资、联合国难民署和世界伊斯兰救济组织支持的巴勒斯坦难民社区活动中心重建项目7月2日在伊拉克巴格达举行落成交接仪式}}。

6、{\textbf{第三十一届首脑会议7月2日在毛里塔尼亚首都努瓦克肖特闭幕。}}本届非盟峰会系列会议沿用上届``赢得反腐败斗争的胜利:一条非洲转型的可持续之路''的主题。峰会期间,各国首脑对非盟改革、非洲一体化、打击腐败及打击恐怖主义和地区安全局势展开讨论。

7、7月2日,位于莫斯科红场的俄罗斯国家百货商店的金色大厅,在中俄各界200多名嘉宾见证下,``俄罗斯---中国国家品牌合作中心''正式揭牌。当天``一带一路''中俄国家品牌合作论坛正式开幕。{\textbf{这是继去年11月``莫斯科中国品牌商品境外营销中心''项目启动后,中俄两国在品牌建设方面的又一重要举动。}}

8、{\textbf{墨西哥左翼的国家复兴运动党候选人洛佩斯日前赢得总统大选。}}墨西哥舆论认为,洛佩斯当选总统不仅意味着墨西哥左翼执掌政权,还将带来广泛的地缘政治影响。洛佩斯在今后的执政中将面临内政外交多重挑战,兑现竞选承诺之路并不平坦。

9、{\textbf{据朝中社7月7日报道,朝鲜外务省发言人当天发表谈话说,历史性的首次朝美首脑会谈举行后,国际社会的期待和关心集中于履行朝美峰会联合声明的朝美高级别会谈。}}朝方原本期待美方本着朝美首脑会谈的精神,带来有助于构筑信任的建设性方案,还曾考虑做些相应的行动。但美方在6日和7日举行的首次朝美高级别会谈中采取的态度和立场,令人遗憾至极。

10、2018年7月7日,第七次中国---中东欧国家领导人会晤在保加利亚索非亚举行。与会各方认为中国---中东欧国家合作(以下简称``16+1合作'')取得积极进展和众多成果,{\textbf{《中国---中东欧国家合作布达佩斯纲要》得到有效落实,确信16+1各领域合作蓬勃发展,日益成为务实的跨区域合作机制,惠及各方。}}

11、{\textbf{第二届东盟地区论坛城市应急救援研讨班9日在广西南宁开幕。}}该研讨班由中国外交部、中国应急管理部和马来西亚国家灾害管理局共同主办,来自中国、马来西亚、越南等12个东盟地区论坛成员国和4个国际组织代表近600人将围绕``城市应急救援''等相关话题展开为期4天的研讨。

12、{\textbf{习近平主席7月10日出席中阿合作论坛第八届部长级会议开幕式并发表题为《携手推进新时代中阿战略伙伴关系》的重要讲话,引发国际人士广泛关注和热烈反响。}}他们认为,习近平主席的重要讲话着眼中阿双方的长远利益,为中阿关系的未来发展指明了方向,中阿互利共赢的未来一定愈发光明。站在新的历史起点上,中阿全方位合作进入新阶段。

13、7月10日,世界知识产权组织、美国康奈尔大学、欧洲工商管理学院在纽约联合发布2018年全球创新指数报告。报告显示,{\textbf{中国首次跻身最具创新力经济体20强,位居第十七位,成为首个也是唯一进入前20名的中等收入经济体。}}这从一个侧面反映出中国创新驱动发展战略和高质量发展的可喜成就。

14、{\textbf{世界贸易组织(世贸组织)对中国第七次贸易政策审议会议7月11日在瑞士日内瓦开幕。}}中国代表团团长、商务部副部长兼国际贸易谈判副代表王受文在会上表示,中国是多边贸易体制的坚定支持者。中国积极参与世贸组织各项工作,认真履行成员义务,努力确保国内相关立法和政策与世贸组织规则相一致。

15、日前,欧元集团主席马里奥·森特诺宣布,欧元区财长同意希腊在今年8月第三轮救助计划到期后如期退出该计划。{\textbf{这意味着希腊将加入爱尔兰、西班牙、塞浦路斯和葡萄牙的行列,像这些国家一样实现经济稳定增长。这一消息对欧元区和整个欧盟来说都具有重要意义,表明``退欧''风险退潮,欧元区完整性得以保持。欧洲经济正在努力走出欧债危机阴影,但挑战依然存在}}。

16、{\textbf{世界贸易组织对中国第七次贸易政策审议7月13日在日内瓦结束。}}率团参加本次审议的中国商务部副部长、国际贸易谈判副代表王受文在审议结束后对记者表示,在这次审议中,世贸组织成员对中国这两年的经济发展和贸易政策方向给予了充分肯定,认为中国认真履行成员义务,为其他成员带来了机会和好处,中国的改革开放没有止步。

17、中国首支维和直升机分队授勋仪式7月15日在位于苏丹达尔富尔法希尔的营区举行,{\textbf{全体140名维和官兵荣获联合国``和平荣誉勋章''}}。

18、{\textbf{在莫斯科进行的2018俄罗斯世界杯足球赛决赛中,法国队以4∶2战胜克罗地亚队,获得本届世界杯冠军,这也是继1998年世界杯之后,法国队第二次夺得世界杯冠军。克罗地亚队获得本届世界杯亚军}}。

19、7月16日,美国总统特朗普和俄罗斯总统普京在芬兰首都赫尔辛基会晤,这是继去年G20汉堡峰会(实现会晤)、APEC越南峰会(公开场合见面)之后,两国领导人的第三次会面。在目前美俄关系历史最差的情况下,此次会晤被公众期待,希望它将有助于改善美俄两大国之间的关系,也将有助于世界和平。

20、{\textbf{为期6天的缅甸第三届21世纪彬龙会议暨联邦和平大会7月16日顺利闭幕。}}各方代表在闭幕式上签署了将在未来达成、旨在实现永久和平的《联邦和平协议》中的部分条款。

21、{\textbf{卢旺达总统办公室主任朱蒂斯·乌维茨瓦近日表示,该国准备在2019年1月开始签发非洲统一护照,首批发放对象是高级外交官和部分政府公务人员。}}乌维茨瓦强调,新护照是非洲联盟(非盟)《2063年议程》的一部分,将为地区团结和非洲一体化带来益处。《2063年议程》的目标是要建设一个一体化、团结、无边界的非洲。

22、{\textbf{在对阿拉伯联合酋长国进行国事访问前夕,国家主席习近平7月18日在阿联酋《联邦报》《国民报》发表题为《携手前行,共创未来》的署名文章。}}文章发表后,在阿联酋引发广泛关注和强烈共鸣。阿联酋各界人士纷纷表示,习近平主席在署名文章中总结了中阿友好合作成果,表达了携手打造中阿共建``一带一路''命运共同体,更好造福两国人民的期望,为未来中阿关系发展指明了方向。

23、7月23日晚,{\textbf{老挝阿速坡省沙南赛县突发重大水坝决堤事故}},造成人员伤亡和数百人失踪,经济损失严重,目前具体伤亡人数和损失尚在统计之中。据了解,该项目业主为老挝Xepian-Xenamnoy能源公司,无中国公司参与持股或建设。

24、巴基斯坦大选投票于7月25日结束,自当晚7时起,选举便进入紧张的计票阶段。{\textbf{截至发稿时为止,已有194个国民议会选区的结果被公布,正义运动党领导人、前板球明星伊姆兰·汗26日傍晚已宣布在本次大选中获胜,并称要建立一个``新巴基斯坦''。}}

25、{\textbf{7月27日,朝鲜在平壤友谊塔举行祭奠活动,深切缅怀中国人民志愿军烈士。}}朝鲜最高人民会议常任委员会副委员长杨亨燮、内阁副总理李龙男等朝党政军干部,中国驻朝鲜大使李进军及使馆外交人员、旅朝华侨、在朝留学生、驻朝机构和媒体等共同凭吊中国人民志愿军烈士。正在朝鲜访问的中国外交部副部长孔铉佑也率团参加祭奠活动。友谊塔前,朝鲜最高领导人金正恩敬献的花圈庄重地摆放在塔基上,缎带上写着``光荣属于中国人民志愿军烈士''。

\subsection{24675-2018年8月国内时事}
1、{\textbf{中共中央政治局7月31日召开会议,分析研究当前经济形势,部署下半年经济工作,审议《中国共产党纪律处分条例》。中共中央总书记习近平主持会议。}}会议指出,当前经济运行稳中有变,面临一些新问题新挑战,外部环境发生明显变化。要抓住主要矛盾,采取针对性强的措施加以解决。下半年,要保持经济社会大局稳定,深入推进供给侧结构性改革,打好``三大攻坚战'',加快建设现代化经济体系,推动高质量发展,任务艰巨繁重。要坚持稳中求进工作总基调,保持经济运行在合理区间,加强统筹协调,形成政策合力,精准施策,扎实细致工作。会议要求,第一,保持经济平稳健康发展,坚持实施积极的财政政策和稳健的货币政策,提高政策的前瞻性、灵活性、有效性。财政政策要在扩大内需和结构调整上发挥更大作用。要把好货币供给总闸门,保持流动性合理充裕。要做好稳就业、稳金融、稳外贸、稳外资、稳投资、稳预期工作。保护在华外资企业合法权益。第二,把补短板作为当前深化供给侧结构性改革的重点任务,加大基础设施领域补短板的力度,增强创新力、发展新动能,打通去产能的制度梗阻,降低企业成本。要实施好乡村振兴战略。第三,把防范化解金融风险和服务实体经济更好结合起来,坚定做好去杠杆工作,把握好力度和节奏,协调好各项政策出台时机。要通过机制创新,提高金融服务实体经济的能力和意愿。第四,推进改革开放,继续研究推出一批管用见效的重大改革举措。要落实扩大开放、大幅放宽市场准入的重大举措,推动共建``一带一路''向纵深发展,精心办好首届中国国际进口博览会。第五,下决心解决好房地产市场问题,坚持因城施策,促进供求平衡,合理引导预期,整治市场秩序,坚决遏制房价上涨。加快建立促进房地产市场平稳健康发展长效机制。第六,做好民生保障和社会稳定工作,把稳定就业放在更加突出位置,确保工资、教育、社保等基本民生支出,强化深度贫困地区脱贫攻坚工作,做实做细做深社会稳定工作。

2、中共中央政治局7月31日下午就全面停止军队有偿服务举行第七次集体学习。{\textbf{中共中央总书记习近平在主持学习时强调,全面停止军队有偿服务,是党中央和中央军委着眼于强军兴军作出的重大决策,是深化国防和军队改革的重要内容。要坚定决心意志,加强创新突破,坚持积极稳妥,增强工作合力,不开口子、不打折扣、不搞变通,坚决做好全面停止军队有偿服务工作,为新时代强军事业创造良好条件}}。

3、{\textbf{2018年全国征兵工作从8月1日开始,至9月30日结束}}。

4、中央宣传部8月2日在北京向全社会公开发布{\textbf{海军海口舰的先进事迹,授予他们``时代楷模''称号}}。

5、中国主要薯类作物年种植面积超过1.5亿亩,占全国可用耕地8\%左右。其中,马铃薯和甘薯的种植面积和总产量均居世界第一位。薯类作物是我国粮食作物的重要组成部分,其产业发展对促进我国种植业调整,支持农业发展具有重要意义。

6、{\textbf{日前,人社部公布第一批拖欠农民工工资``黑名单''}},30个部委单位将进行联合惩戒。从提出构想到政策落地不到一年,``黑名单''制度的实行,如同给农民工工资上了一份保险。

7、国务院关税税则委员会发布公告,\textbf{{决定对原产于美国的部分进口商品(第二批)加征关税}}。

8、8月1日至3日,{\textbf{由联合国开发计划署与中华全国青年联合会共同主办的亚太青年领导力与创新创业论坛在北京举行}}。来自30多个国家的300多名政府官员、联合国代表、研究人员和青年创业者齐聚一堂,共商如何用实际行动落实联合国2030年可持续发展议程中的各项目标。

9、中国人民银行近日发布消息称,{\textbf{自今年8月6日起,将远期售汇业务的外汇风险准备金率从0调整为20\%}}。

10、由中宣部、中央党史和文献研究院、中国文联共同主办,中央编译局、中国美术家协会、中国国家博物馆承办的{\textbf{``真理的力量------纪念马克思诞辰200周年主题展览''于8月5日落下帷幕}}。

11、习近平近日对王继才同志先进事迹作出重要指示强调,{\textbf{王继才同志守岛卫国32年,用无怨无悔的坚守和付出,在平凡的岗位上书写了不平凡的人生华章。我们要大力倡导这种爱国奉献精神,使之成为新时代奋斗者的价值追求}}。

12、8月3日至8日,受党中央、国务院邀请,62位专家来到北戴河进行暑期休假。{\textbf{自2001年以来,党中央、国务院先后邀请18批共计1000多位专家学者参加休假活动。本次受邀休假专家都是各行业作出突出贡献的人才,专业涵盖载人航天、青藏铁路、高分卫星、集成电路、生物医药、食品安全、农业技术等,与核心技术相关、与民生福祉相连}}。

13、为贯彻落实党中央、国务院有关要求,促进我国制种行业长期可持续发展,从源头上保障国家粮食安全,财政部、农业农村部、银保监会近日发布通知,{\textbf{将水稻、玉米、小麦三大粮食作物制种纳入中央财政农业保险保险费补贴目录}}。

14、{\textbf{2018年5月,全国有16个地区已经开展工程建设项目审批制度改革试点}}。真正让``放管服''发挥效用,除了各个审批环节、审批事项``瘦身'',还需要整个审批体系打破梗阻,实现``血脉相通''。此次工程建设项目审批制度改革,通过``一张蓝图、一个系统、一个窗口、一张表单、一套机制''来构建完善整个审批体系。此次改革还强调要对中介和公共服务机构加强管理。中介服务和市政公用服务虽然不属于政府行政审批,但却是工程建设项目审批流程中的重要部分,也是企业和群众反映比较集中的环节。住建部表示,要下大力气全面整顿规范中介和市政公用市场,要求所有中介机构和市政公用服务单位明确服务标准和办事流程,规范服务收费和时限,用信息化手段对中介服务办理时限、服务质量、收费情况进行全方位、全过程监督。

15、中共中央总书记习近平向柬埔寨人民党主席洪森致贺电,{\textbf{祝贺其领导柬埔寨人民党在第六届国会选举中获胜}}。

16、从国家医保局了解到,{\textbf{全国跨省异地就医定点医疗机构数已超过1万家,直接结算近60万人次}}。

17、{\textbf{中共中央政治局常务委员会8月16日召开会议,听取关于吉林长春长生公司问题疫苗案件调查及有关问责情况的汇报。中共中央总书记习近平主持会议并发表重要讲话。}}会议指出,这起问题疫苗案件发生以来,习近平总书记高度重视,多次作出重要指示,要求立即查清事实真相,严肃问责,依法从严处理,坚决守住安全底线,全力保障群众切身利益和社会稳定大局。在党中央坚强领导下,国务院多次召开会议研究,派出调查组进行调查,目前已基本查清案件情况和有关部门及干部履行职责情况。 会议强调,疫苗关系人民群众健康,关系公共卫生安全和国家安全。这起问题疫苗案件是一起疫苗生产者逐利枉法、违反国家药品标准和药品生产质量管理规范、编造虚假生产检验记录、地方政府和监管部门失职失察、个别工作人员渎职的严重违规违法生产疫苗的重大案件,情节严重,性质恶劣,造成严重不良影响,既暴露出监管不到位等诸多漏洞,也反映出疫苗生产流通使用等方面存在的制度缺陷。要深刻汲取教训,举一反三,重典治乱,去疴除弊,加快完善疫苗药品监管长效机制,坚决守住公共安全底线,坚决维护最广大人民身体健康。

18、国务院总理李克强8月16日主持召开国务院常务会议,听取吉林长春长生公司问题疫苗案件调查情况汇报并作出相关处置决定;部署以改革举措破除民间投资和民营经济发展障碍,激发经济活力和动力。

19、8月17日,国家医保局发布2018年抗癌药医保准入专项谈判药品范围的通告,\textbf{{18种抗癌药纳入2018年医保准入谈判范围}}。

20、8月17日电,国家知识产权局和军委装备发展部共同确定江苏省、福建省、山东省、湖南省、广东省、重庆市、四川省、陕西省、甘肃省、上海市闵行区、山东省烟台市、湖南省长沙市、四川省成都市为首批知识产权军民融合试点地方,试点期限为期3年。

21、8月17日电,{\textbf{十三届全国政协第八次双周协商座谈会17日在京召开}},中共中央政治局常委、全国政协主席汪洋主持会议并讲话。他强调,培养造就一支懂农业、爱农村、爱农民的``三农''工作队伍,是以习近平同志为核心的中共中央作出的重大决策部署,是实施乡村振兴战略的重要支撑。

22、8月18日电,{\textbf{2年期国债期货17日在中国金融期货交易所成功挂牌上市,标志着我国已基本形成覆盖短中长期的国债期货产品体系}}。

23、{\textbf{中央军委党的建设会议8月17日至19日在北京召开。}}中共中央总书记、国家主席、中央军委主席习近平出席会议并发表重要讲话。他强调,全面加强新时代我军党的领导和党的建设工作,是推进党的建设新的伟大工程的必然要求,是推进强国强军的必然要求。全军要全面贯彻新时代中国特色社会主义思想和党的十九大精神,深入贯彻新时代党的强军思想,落实新时代党的建设总要求,落实新时代党的组织路线,坚持党对军队绝对领导,坚持全面从严治党,坚持聚焦备战打仗,全面提高我军加强党的领导和党的建设工作质量,为实现党在新时代的强军目标、完成好新时代军队使命任务提供坚强政治保证。

24、{\textbf{国家主席习近平8月20日在钓鱼台国宾馆会见马来西亚总理马哈蒂尔。}}习近平欢迎马哈蒂尔访华,赞赏马来西亚新政府和马哈蒂尔本人高度重视中马关系,赞赏马哈蒂尔多次表示视中国为发展机遇并支持``一带一路''倡议,赞赏马哈蒂尔为推动亚洲区域合作作出的重要贡献。

25、外交部发言人陆慷8月20日宣布:2018年中非合作论坛北京峰会将于9月3日至4日在北京举行。{\textbf{本次峰会主题为``合作共赢,携手构建更加紧密的中非命运共同体''。中国国家主席习近平将主持峰会并举行相关活动。中非合作论坛非方成员领导人将应邀与会,有关非洲地区组织和国际组织代表将出席峰会有关活动}}。

26、今年夏粮呈现``面积稳定、单产略减、总体丰收''的局面,产量达到2774亿斤,同比减少61亿斤,夏粮总产量仍处于历史高位。眼下,南方早稻收获已全部结束,早稻呈现丰收趋势。据农业农村部农情调度,目前全国秋粮作物长势总体正常,东北粳稻和南方晚稻长势好于上年,东北局部地区前期因旱迟播的玉米,后期光温水匹配较好,生育过程已与常年同期相当。另据农业农村部数据,全国畜禽水产品产销总体宽松,上半年猪牛羊禽肉产量3995万吨,同比增长0.9\%。其中,生猪存栏4.09亿头,同比下降1.8\%;生猪出栏3.34亿头,同比增长1.2\%;猪肉产量2614万吨,同比增长1.4\%。水产品总产量达到2740万吨,同比增长0.86\%。总体上,农业生产呈现五谷丰登、六畜兴旺的态势。

27、{\textbf{全国宣传思想工作会议8月21日至22日在北京召开。}}中共中央总书记、国家主席、中央军委主席习近平出席会议并发表重要讲话。他强调,完成新形势下宣传思想工作的使命任务,必须以新时代中国特色社会主义思想和党的十九大精神为指导,增强``四个意识''、坚定``四个自信'',自觉承担起举旗帜、聚民心、育新人、兴文化、展形象的使命任务,坚持正确政治方向,在基础性、战略性工作上下功夫,在关键处、要害处下功夫,在工作质量和水平上下功夫,推动宣传思想工作不断强起来,促进全体人民在理想信念、价值理念、道德观念上紧紧团结在一起,为服务党和国家事业全局作出更大贡献。习近平指出,中国特色社会主义进入新时代,必须把统一思想、凝聚力量作为宣传思想工作的中心环节。当前,我国发展形势总的很好,我们党要团结带领人民实现党的十九大确定的战略目标,夺取中国特色社会主义新胜利,更加需要坚定自信、鼓舞斗志,更加需要同心同德、团结奋斗。我们必须把人民对美好生活的向往作为我们的奋斗目标,既解决实际问题又解决思想问题,更好强信心、聚民心、暖人心、筑同心。我们必须既积极主动阐释好中国道路、中国特色,又有效维护我国政治安全和文化安全。我们必须坚持以立为本、立破并举,不断增强社会主义意识形态的凝聚力和引领力。我们必须科学认识网络传播规律,提高用网治网水平,使互联网这个最大变量变成事业发展的最大增量。

28、{\textbf{首届中国国际智能产业博览会8月23日在重庆市开幕,国家主席习近平向会议致贺信。}}习近平强调,中国高度重视创新驱动发展,坚定贯彻新发展理念,加快推进数字产业化、产业数字化,努力推动高质量发展、创造高品质生活。中国愿积极参与数字经济国际合作,同各国携手推动数字经济健康发展,为世界经济增长培育新动力、开辟新空间。本次会议以``智能化:为经济赋能,为生活添彩''为主题,体现了世界经济发展的趋势,体现了各国人民对美好生活的期盼。希望与会代表深化交流合作,智汇八方、博采众长,共同推动数字经济发展,为构建人类命运共同体贡献智慧和力量。

29、{\textbf{习近平8月24日主持召开中央全面依法治国委员会第一次会议并发表重要讲话。}}他强调,全面依法治国具有基础性、保障性作用,在统筹推进伟大斗争、伟大工程、伟大事业、伟大梦想,全面建设社会主义现代化国家的新征程上,要加强党对全面依法治国的集中统一领导,坚持以全面依法治国新理念新思想新战略为指导,坚定不移走中国特色社会主义法治道路,更好发挥法治固根本、稳预期、利长远的保障作用。

30、8月25日7时52分,{\textbf{我国在西昌卫星发射中心用长征三号乙运载火箭以``一箭双星''方式成功发射第三十五、三十六颗北斗导航卫星,两颗卫星属于中圆地球轨道卫星,也是我国北斗三号全球系统第十一、十二颗组网卫星。}}

31、{\textbf{近日,中共中央印发了修订后的《中国共产党纪律处分条例》}},通知强调,《条例》全面贯彻习近平新时代中国特色社会主义思想和党的十九大精神,以党章为根本遵循,将党的纪律建设的理论、实践和制度创新成果,以党规党纪形式固定下来,着力提高纪律建设的政治性、时代性、针对性。严明政治纪律和政治规矩,把坚决维护习近平总书记党中央的核心、全党的核心地位,坚决维护党中央权威和集中统一领导作为出发点和落脚点,将党章和《关于新形势下党内政治生活的若干准则》等党内法规的要求细化具体化。坚持问题导向,针对管党治党存在的突出问题扎紧笼子,实现制度的与时俱进,使全面从严治党的思路举措更加科学、更加严密、更加有效。

32、中共中央总书记、国家主席、中央军委主席习近平近日作出重要指示指出,我国学生近视呈现高发、低龄化趋势,严重影响孩子们的身心健康,这是一个关系国家和民族未来的大问题,必须高度重视,不能任其发展。习近平指示有关方面,要结合深化教育改革,拿出有效的综合防治方案,并督促各地区、各有关部门抓好落实。习近平强调,全社会都要行动起来,共同呵护好孩子的眼睛,让他们拥有一个光明的未来。

33、8月25日7时52分,{\textbf{我国在西昌卫星发射中心用长征三号乙运载火箭以``一箭双星''方式成功发射第三十五、三十六颗北斗导航卫星,两颗卫星属于中圆地球轨道卫星,也是我国北斗三号全球系统第十一、十二颗组网卫星}}。

34、{\textbf{8月29日,2018中国民营企业500强峰会在沈阳举办。}}峰会以``提振发展信心实现高质量发展''为主题,揭晓了``2018中国民营企业500强''系列榜单,并发布了关于中国民营企业500强的调研分析报告。分析报告显示,此次中国民营企业500强入围门槛达156.84亿元。我国民营经济发展呈现出产业结构持续优化,自主创新能力不断增强,营商环境明显改善等亮点。

35、习近平对黄群等3名同志壮烈牺牲作出重要指示指出,\textbf{{黄群、宋月才、姜开斌三位同志面对台风和巨浪,挺身而出、英勇无惧,为保护国家重点试验平台壮烈牺牲,用实际行动诠释了共产党员对党忠}}

\textbf{{诚、恪尽职守、不怕牺牲的优秀品格,用宝贵生命践行了共产党员``随时准备为党和人民牺牲一切''的初心和誓言,他们是共产党员的优秀代表、时代楷模。习近平强调,广大党员干部要以黄群、}}

\textbf{{宋月才、姜开斌同志为榜样,坚定理想信念,不忘初心、牢记使命,履职尽责、许党报国,为实现``两个一百年''奋斗目标、实现中华民族伟大复兴的中国梦贡献智慧和力量}}。

36、中共中央总书记、国家主席、中央军委主席习近平8月27日在北京人民大会堂出席推进``一带一路''建设工作5周年座谈会并发表重要讲话强调,{\textbf{共建``一带一路''顺应了全球治理体系变革的内在要求,彰显了同舟共济、权责共担的命运共同体意识,为完善全球治理体系变革提供了新思路新方案。我们要坚持对话协商、共建共享、合作共赢、交流互鉴,同沿线国家谋求合作的最大公约数,推动各国加强政治互信、经济互融、人文互通,一步一个脚印推进实施,一点一滴抓出成果,推动共建``一带一路''走深走实,造福沿线国家人民,推动构建人类命运共同体}}。

37、国家主席习近平8月30日在人民大会堂同科特迪瓦总统瓦塔拉举行会谈。{\textbf{两国元首一致同意,推动中科关系迈向更高水平,实现互利共赢}}。

38、国家主席习近平8月30日在人民大会堂同塞拉利昂总统比奥举行会谈。{\textbf{两国元首一致同意,巩固友好互信,扩大务实合作,将中塞全面战略合作伙伴关系不断推向前进,更好造福两国人民}}。

\subsection{24676-2018年8月国际时事}
1、7月30日至31日,{\textbf{俄伊土三方索契会谈在俄罗斯索契召开,俄罗斯、土耳其、伊朗等国家政府代表团,叙利亚政府和反政府武装代表团参加会谈}}。联合国和约旦以观察员身份出席会谈。俄罗斯、伊朗和土耳其政府代表团7月31日发表会谈声明,决心继续在叙利亚反恐,推动政治解决叙问题,呼吁加强对叙人道救援和重建的支持力度。

2、{\textbf{津巴布韦选举委员会8月3日宣布,现任总统埃默森·姆南加古瓦赢得7月30日举行的总统选举,获得连任}}。同时,执政党津巴布韦非洲民族联盟---爱国阵线(民盟)也赢得国民议会多数。

3、8月4日,{\textbf{国务委员兼外交部长王毅出席在新加坡举行的东盟与中日韩(10+3)外长会议}}。

4、{\textbf{上一轮疫情宣布结束仅一周后,非洲中部国家刚果(金)再次暴发埃博拉出血热疫情。}}该国卫生部8月5日凌晨发布公报说,在新一轮疫情中已报告有33人死亡。

5、美国政府8月7日重启对伊朗包括金融、金属、矿产、汽车等一系列产业在内的非能源领域制裁。有分析指出,美国不顾各方反对重启对伊制裁,将导致美伊关系进一步恶化,地区局势以及美欧关系也将受到影响。

6、据报道,{\textbf{埃及目前正在建设全球最大的太阳能发电场。}}根据计划,这座发电场将由30个独立的太阳能发电厂组成,其中第一个发电厂将于2019年12月正式运行。同样的太阳能发电场还将在埃及其他地区普及建设,以更好地满足埃及人民生活、生产和发展经济的需要。

7、世界贸易组织8月9日发布最新一期全球贸易景气指数,预计今年三季度全球贸易增速将继续放缓。

8、{\textbf{8月11日,纪念中日和平友好条约缔结40周年国际学术研讨会在北京举行。}}来自中日双方的友好人士、专家学者及媒体代表等百余人与会。与会代表就两国关系发展的历史经验以及未来合作前景进行

总结与探讨,为推动中日关系重回正轨并长期健康稳定发展建言献策。

9、{\textbf{有史以来飞得最快的航天器美国``帕克''太阳探测器8月12日升空,正式开启人类历史上首次穿越日冕``触摸''太阳的逐日之旅,这也将成为迄今最``热''的太空探测任务}}。

10、8月12日,第五届里海沿岸国家领导人会议在哈萨克斯坦西部城市阿克套举行。哈总统纳扎尔巴耶夫、俄罗斯总统普京、阿塞拜疆总统阿利耶夫、伊朗总统鲁哈尼、土库曼斯坦总统别尔德穆哈梅多夫出席会议,{\textbf{并共同签署了《里海法律地位公约》,为该水域的资源开发以及相关合作奠定了法律基础}}。

11、韩国和朝鲜8月13日在板门店朝方一侧统一阁举行高级别会谈,商定9月在朝鲜首都平壤再次举行南北首脑会晤。

12、尽管遭到国际社会普遍反对,尽管国内相关产业受到重创,但白宫依然对单边威胁和讹诈执迷不悟。近日,美方在此前公布对中方2000亿美元输美产品加征10\%关税清单的基础上,又提出要将征税税率由10\%提高到25\%。对此,中方决定,将依法对自美进口的约600亿美元产品按照四档不同税率加征关税,实施日期将视美方行动而定。

13、北京时间8月14日凌晨在加拿大萨斯卡通召开的国际灌排委员会第六十九届国际执行理事会全体会议上,{\textbf{公布了2018年(第五批)世界灌溉工程遗产名录,我国的都江堰、灵渠、姜席堰、长渠4个项目全部申报成功。截至目前,我国已有17处世界灌溉工程遗产项目,是拥有遗产工程类型最丰富、灌溉效益最突出、分布范围最广泛的国家}}。

14、近日,英国国际贸易大臣福克斯表示,欧盟毫不妥协的态度正把英国推向``无协议''脱欧的境地,``概率高达60\%''。英国央行行长卡尼也表示,英国可能``无协议''脱欧,即在没有达成任何正式协议的情况下退出欧盟。

15、孟加拉国中央银行近日出台新政,{\textbf{允许相关银行开设人民币结算账户与央行进行结算}}。

16、8月17日电,国务院总理李克强8月17日致电伊姆兰·汗,祝贺他当选巴基斯坦伊斯兰共和国总理。

17、8月17日电,{\textbf{独联体成员国国防部理事会在白俄罗斯首都明斯克举行第四十九次国防问题协调委员会会议}},制订了打造独联体国家统一反导系统的行动计划。8月15日,俄罗斯、白俄罗斯两国还在明斯克举行了一年一度的反导演习。

18、8月17日至18日,塔吉克斯坦总统拉赫蒙对乌兹别克斯坦进行正式访问,两国签署了战略伙伴关系条约及20多项合作协议,为两国关系发展注入新活力。

19、委内瑞拉8月20日启用新货币,同时委内瑞拉政府宣布一系列经济改革举措,应对恶性通货膨胀,试图令委经济重回正轨。

20、{\textbf{第六十七届联合国新闻部非政府组织会议8月22日在纽约联合国总部开幕。}}会议将探讨如何加强联合国与非政府组织之间合作,并提倡用多边主义理念解决当前全球面临的重大问题。

21、``和平使命---2018''上海合作组织联合反恐军事演习开幕式,当地时间8月24日上午在俄罗斯切巴尔库尔训练基地举行。

22、8月26日,随着X8044次中欧班列汉堡---武汉)顺利到达武汉吴家山铁路集装箱中心站,{\textbf{标志着中欧班列累计开行数量达到10000列}}。

23、{\textbf{第十八届亚洲运动会(亚运会)正在印度尼西亚如火如荼地举行}},来自亚洲45个国家和地区约1.13万名运动员在雅加达和巨港等地展开角逐。据印尼官方估计,亚运会期间预计约70万名游客到访该国并创造3万亿印尼盾(1万印尼盾约合4.7元人民币)的外汇收入。为了筹办亚运会,印尼前期投入大量资金用于基础设施建设,仅基建一项就将为印尼经济贡献约12.7亿美元的额外增长。印尼媒体普遍认为,作为四年一度的亚洲体育盛会,亚运会不仅为印尼吸引了大量资金流和人员流,更有望为印尼带来新的发展机遇。

24、{\textbf{美国国务院8月27日发布公告,宣布美方因俄罗斯前特工在英国中毒事件而对俄方实施的制裁于当天正式生效,相关制裁措施将执行至少一年}}。分析认为,近期美国频繁对俄罗斯实施制裁措施,在7月的赫尔辛基美俄元首会晤后,美俄关系并没有得到相应改善,双方分歧仍在继续加大。美国彭博社认为,随着美国11月中期选举临近,面对``通俄门''调查的压力,美国政府预计将展示出更强硬的对俄制裁手段,美俄对抗还将不断升级。

25、韩国统计厅日前公布的2017年人口普查数据显示,截至2017年11月1日,韩国总人口为5142万人,其中韩国65岁及以上人口占总人口的14.2\%,{\textbf{这标志着韩国正式进入``老龄社会''}}。

\subsection{24677-2018年9月国内时事}
1、8月27日至9月1日,中央扫黑除恶专项斗争9个督导组完成对山西、辽宁、福建、山东、河南、湖北、广东、重庆、四川等9省(市)的进驻工作,{\textbf{标志着中央扫黑除恶专项斗争第一轮督导工作全面启动}}。

2、9月2日,中国企业联合会、中国企业家协会连续第十七年发布``中国企业500强''榜单及报告,{\textbf{国家电网、中国石化、中国石油继续分列前三。}}``2018中国企业500强''入围门槛首次突破300亿元大关,实现了16连升;企业营业收入总额首次突破70万亿元大关,达到71.17万亿元,迈上新的台阶,营收较上年增长了11.20\%,增速加快3.56个百分点,重回两位数增速区间。\\

3、{\textbf{9月3日,中非合作论坛北京峰会在人民大会堂隆重开幕}}。中国国家主席习近平出席开幕式并发表主旨讲话,强调中非要携起手来,共同打造责任共担、合作共赢、幸福共享、文化共兴、安全共筑、和谐共生的中非命运共同体,重点实施好产业促进、设施联通、贸易便利、绿色发展、能力建设、健康卫生、人文交流、和平安全``八大行动''。习近平强调,中国是世界上最大的发展中国家,非洲是发展中国家最集中的大陆,中非早已结成休戚与共的命运共同体。我们愿同非洲人民共筑更加紧密的中非命运共同体,为推动构建人类命运共同体树立典范。

4、{\textbf{国家主席习近平9月3日在国家会议中心出席中非领导人与工商界代表高层对话会暨第六届中非企业家大会开幕式并发表题为《共同迈向富裕之路》的主旨演讲}},强调中国支持非洲国家参与共建``一带一路'',愿同非洲加强全方位对接,打造符合国情、包容普惠、互利共赢的高质量发展之路,共同走上让人民生活更加美好的幸福之路。

5、{\textbf{纪念中国人民抗日战争暨世界反法西斯战争胜利73周年座谈会9月3日在北京举行}},中共中央政治局委员、中宣部部长黄坤明出席。中国人民抗日战争胜利,是近代以来中国抗击外敌入侵的第一次完全胜利。这一伟大胜利,开辟了中华民族伟大复兴的光明前景,开启了古老中国凤凰涅槃、浴火重生的新征程。

6、{\textbf{中非合作论坛北京峰会圆桌会议9月4日在人民大会堂举行。}}国家主席习近平和论坛共同主席国南非总统拉马福萨分别主持第一阶段和第二阶段会议。会议通过《关于构建更加紧密的中非命运共同体的北京宣言》和《中非合作论坛---北京行动计划(2019---2021年)》。习近平强调,我们一致同意秉持共商共建共享原则,将中非合作论坛建设成为中非团结合作的品牌、国际对非合作的旗帜。我们将加强政策协调,推进落实论坛峰会成果,并把中非共建``一带一路''、非洲联盟《2063年议程》、联合国2030年可持续发展议程、非洲各国发展战略紧密结合起来,为非洲发展振兴提供更多机遇和有效平台,为中非合作提供不竭动力和更大空间。\\

7、{\textbf{国务院总理李克强9月6日主持召开国务院常务会议}},确定落实新修订的个人所得税法的配套措施,为广大群众减负;决定完善政策确保创投基金税负总体不增;部署打造``双创''升级版,增强带动就业能力、科技创新力和产业发展活力;通过《专利代理条例(修订草案)》。会议指出,全面落实全国人大常委会审议通过的新修订的个人所得税法,建立综合与分类相结合的个人所得税制,是我国前所未有的重大税制改革。要在确保10月1日起如期将个税基本减除费用标准由3500元提高到5000元并适用新税率表的同时,抓紧按照让广大群众得到更多实惠的要求,明确子女教育、继续教育、大病医疗、普通住房贷款利息、住房租金、赡养老人支出6项专项附加扣除的具体范围和标准,使群众应纳税收入在减除基本费用标准的基础上,再享有教育、医疗、养老等多方面附加扣除,确保扣除后的应纳税收入起点明显高于5000元,进一步减轻群众税收负担,增加居民实际收入、增强消费能力。专项附加扣除范围和标准在向社会公开征求意见后依法于明年1月1日起实施。今后随着经济社会发展和人民生活水平提高,专项附加扣除范围和标准还将动态调整。会议强调,目前全国养老金累计结余较多,可以确保按时足额发放,在社保征收机构改革到位前,各地要一律保持现有征收政策不变,同时抓紧研究适当降低社保费率,确保总体上不增加企业负担,以激发市场活力,引导社会预期向好。 为促进创业创新,会议决定,保持地方已实施的创投基金税收支持政策稳定,由有关部门结合修订个人所得税法实施条例,按照不溯及既往、确保总体税负不增的原则,抓紧完善进一步支持创投基金发展的税收政策。

8、{\textbf{2018年9月7日,纪念``一带一路''倡议在哈萨克斯坦提出5周年商务论坛在哈萨克斯坦首都阿斯塔纳举行}}。国家主席习近平通过视频表示祝贺。习近平表示,哈萨克斯坦是``一带一路''倡议的坚定支持者和积极参与者。5年来,在双方共同努力下,中哈共建``一带一路''合作取得丰硕成果。中国愿同哈萨克斯坦及其他有关各国一道,秉持共商共建共享理念,以开放包容姿态致力于共同发展和繁荣,把``一带一路''建设成为和平之路、繁荣之路、开放之路、创新之路、文明之路,为造福各国人民、推动构建人类命运共同体作出更大贡献。

9、{\textbf{9月7日11时15分,我国在太原卫星发射中心用长征二号丙运载火箭成功发射海洋一号C星}}。该星将进一步提升我国海洋遥感技术水平,对我国研究海气相互作用、提高防灾减灾能力、开展全球气候变化研究、解决人类共同面临的全球气候变暖等问题具有重要意义,将开启我国自然资源卫星陆海统筹发展新局面,助力海洋强国建设。

10、9月8日,{\textbf{为期两天的第二届``中拉文明对话''研讨会在江苏南京召开,会议主题是``一带一路:中拉文明对话之路''}}。

11、9月8日电,{\textbf{第二十届中国国际投资贸易洽谈会8日在福建省厦门市开幕,国家主席习近平向投洽会致贺信}}。

12、{\textbf{9月9日,中共中央总书记、国家主席习近平特别代表,中共中央政治局常委、全国人大常委会委员长栗战书在平壤会见了朝鲜劳动党委员长、国务委员会委员长金正恩}}。栗战书首先转达习近平对金正恩的亲切问候并转交亲署函。习近平在亲署函中指出,朝鲜建国70年以来,在金日成同志、金正日同志和委员长同志坚强领导下,朝鲜党和人民奋力推进社会主义建设事业,取得了不平凡的成就。当前,委员长同志正带领朝鲜党和人民,全面贯彻落实新战略路线,致力于发展经济、改善民生,在社会主义建设各个领域不断取得新的成就。

13、{\textbf{全国教育大会9月10日在北京召开}}。中共中央总书记、国家主席、中央军委主席习近平出席会议并发表重要讲话。他强调,在党的坚强领导下,全面贯彻党的教育方针,坚持马克思主义指导地位,坚持中国特色社会主义教育发展道路,坚持社会主义办学方向,立足基本国情,遵循教育规律,坚持改革创新,以凝聚人心、完善人格、开发人力、培育人才、造福人民为工作目标,培养德智体美劳全面发展的社会主义建设者和接班人,加快推进教育现代化、建设教育强国、办好人民满意的教育。全党全社会要弘扬尊师重教的社会风尚,努力提高教师政治地位、社会地位、职业地位,让广大教师享有应有的社会声望,在教书育人岗位上为党和人民事业作出新的更大的贡献。习近平指出,要深化教育体制改革,健全立德树人落实机制,扭转不科学的教育评价导向,坚决克服唯分数、唯升学、唯文凭、唯论文、唯帽子的顽瘴痼疾,从根本上解决教育评价指挥棒问题。要深化办学体制和教育管理改革,充分激发教育事业发展生机活力。要提升教育服务经济社会发展能力,调整优化高校区域布局、学科结构、专业设置,建立健全学科专业动态调整机制,加快一流大学和一流学科建设,推进产学研协同创新,积极投身实施创新驱动发展战略,着重培养创新型、复合型、应用型人才。要扩大教育开放,同世界一流资源开展高水平合作办学。

14、{\textbf{我国第一艘自主建造的极地科学考察破冰船9月10日在上海下水,并正式命名为``雪龙2''号,标志着我国极地考察现场保障和支撑能力取得新的突破}}。

15、{\textbf{国家主席习近平9月11日在符拉迪沃斯托克同俄罗斯总统普京举行会谈}}。两国元首一致认为,今年以来,中俄关系呈现更加积极的发展势头,进入更高水平、更21快发展的新时期。一致同意,无论国际形势如何变化,中俄都将坚定发展好两国关系,坚定维护好世界和平稳定。习近平强调,中俄双方要深化共建``一带一路''和欧亚经济联盟对接合作,扩大能源、农业、科技创新、金融等领域合作,推动重点项目稳步实施,加强前沿科学技术共同研发,利用好今明两年中俄地方合作交流年契机,调动两国更多地方积极性,开展更加广泛合作。

16、教育是国之大计、党之大计。{\textbf{9月10日,习近平总书记在全国教育大会发表重要讲话}},从党和国家事业发展全局的战略高度,系统总结了我国教育事业发展的成就与经验,深刻分析了教育工作面临的新形势新任务,对加快推进教育现代化、建设教育强国、办好人民满意的教育作出了全面部署。教育是民族振兴、社会进步的重要基石,是功在当代、利在千秋的德政工程。党的十九大从新时代坚持和发展中国特色社会主义的战略高度,作出了优先发展教育事业、加快教育现代化、建设教育强国的重大部署。

17、{\textbf{从中央军委训练管理部了解到,我军司号制度恢复和完善工作正有序展开,拟从10月1日起全面恢复播放作息号,下达日常作息指令。明年8月1日起,全军施行新的司号制度}}。

18、{\textbf{第四届东方经济论坛全会9月12日在符拉迪沃斯托克举行}}。中国国家主席习近平、俄罗斯总统普京、蒙古国总统巴特图勒嘎、日本首相安倍晋三、韩国总理李洛渊等出席。习近平发表了题为《共享远东发展新机遇
开创东北亚美好新未来》的致辞,强调中方愿同地区国家一道,维护地区和平安宁,实现各国互利共赢,巩固人民传统友谊,实现综合协调发展,促进本地区和平稳定和发展繁荣。

19、国家主席习近平9月12日在符拉迪沃斯托克会见日本首相安倍晋三。习近平强调,{\textbf{中日双方要始终恪守和遵循中日间四个政治文件,巩固政治基础,把握正确方向,建设性管控分歧,特别是日方要妥善处理好历史、台湾等敏感问题,积极营造良好气氛,不断扩大共同利益。我们欢迎日本继续积极参与中国改革开放进程,实现共同发展繁荣。``一带一路''倡议为中日深化互利合作提供了新平台和试验田。中方愿同日方一道,着眼新形势,为两国务实合作开辟新路径,打造新亮点。中日双方应共同推进区域一体化进程,建设和平、稳定、繁荣的亚洲。双方要坚定维护多边主义,维护自由贸易体制和世界贸易组织规则,推动建设开放型世界经济。双方要弘扬民间友好传统,赋予其新的时代内涵,夯实两国关系的社会和民意基础}}。

20、国台办发言人安峰山9月12日在例行新闻发布会上介绍,{\textbf{《港澳台居民居住证申领发放办法》公布后,受到广大台湾同胞普遍欢迎和肯定。据不完全统计,到9月10日止,短短10天已有超过2.2万名台胞申领了居住证。这充分说明,这是一项真正造福于民、广受台胞欢迎的好政策}}。

21、{\textbf{中国残疾人联合会第七次全国代表大会9月14日上午在北京人民大会堂开幕}}。习近平、李克强、栗战书、汪洋、王沪宁、赵乐际等党和国家领导人到会祝贺,韩正代表党中央、国务院致词。韩正在致词中说,党的十八大以来,我国残疾人事业取得历史性进展和显著成就。习近平总书记对残疾人和残疾人事业发展提出了一系列明确要求,为新时代中国特色残疾人事业发展指明了前进方向,提供了根本遵循。我们要以习近平新时代中国特色社会主义思想为指引,坚持树立正确的价值理念,坚守弱有所扶的原则立场,完成决胜全面建成小康社会的关键任务,促进残疾人全面发展和共同富裕,把推进残疾人事业当作分内责任,在实现中国梦的伟大征程中创造残疾人更加幸福美好的新生活。\\

22、国家主席习近平9月14日在人民大会堂同委内瑞拉总统马杜罗举行会谈。习近平强调,{\textbf{双方要筑牢政治互信,保持高层交往势头,让中委友好成为两国各界政治共识。中方赞赏委方在涉及中方核心利益和重大关切问题上给予中方理解和支持,将一如既往支持委内瑞拉政府谋求国家稳定发展的努力,支持委内瑞拉探索符合本国国情的发展道路,愿同委方加强治国理政经验交流。双方要优化创新务实合作,以签署共建``一带一路''谅解备忘录为契机,加紧对接、推进落实双方业已达成的合作共识,提升委方自主发展能力,推动两国合作可持续发展。双方要积极促进民心相通,扩大人文领域交流合作和地方交往,夯实两国友好社会根基。双方要加强多边协调配合,继续在联合国等国际和地区组织内加强沟通,共同参与全球治理体系改革和建设,维护发展中国家正当权益}}。

23、目前国内人工智能商业落地的百强企业中,22家在上海,国内1/3左右的人工智能人才也在上海。科技实力和人才基础丰厚的上海,近日将迎来2018世界人工智能大会。上海先后出台科创中心建设22条、促进科技成果转化条例、人才政策30条、扩大开放100条等政策法规,勇于改革,激发自主创新活力。{\textbf{连续5年,上海在``魅力中国------外籍人才眼中最具吸引力的中国城市''评选中拔得头筹}}。

24、9月16日12时,北纬35度至26度30分之间的黄海和东海海域正式开渔,标志着今年我国伏季休渔全面结束。

25、{\textbf{2018世界人工智能大会9月17日在上海开幕}}。国家主席习近平致信,向大会的召开表示热烈祝贺,向出席大会的各国代表、国际机构负责人和专家学者、企业家等各界人士表示热烈欢迎。习近平强调,中国正致力于实现高质量发展,人工智能发展应用将有力提高经济社会发展智能化水平,有效增强公共服务和城市管理能力。中国愿意在技术交流、数据共享、应用市场等方面同各国开展交流合作,共享数字经济发展机遇。希望与会嘉宾围绕``人工智能赋能新时代''这一主题,深入交流、凝聚共识,共同推动人工智能造福人类。

26、{\textbf{由工业和信息化部、科技部和江苏省政府共同主办、主题为``数字新经济
物联新时代''的2018世界物联网博览会日前在无锡开幕}}。会上发布的《2017---2018年中国物联网发展年度报告》显示,2017年以来,我国物联网市场进入实质性发展阶段。全年市场规模突破1万亿元,年复合增长率超过25%,其中物联网云平台成为竞争核心领域,预计2021年我国物联网平台支出将位居全球第一。

27、{\textbf{9月18日,国务院关税税则委员会发布公告,决定对美国原产的约600亿美元进口商品实施加征关税。9月18日,中国在世贸组织追加起诉美国301调查项下对华2000亿美元输美产品实施的征税措施}}。

28、近日,中国科学技术大学教授潘建伟及其同事张强、范靖云、马雄峰等与中科院上海微系统与信息技术研究所和日本NTT基础科学实验室合作,在国际上首次成功实现器件无关的量子随机数。相关研究成果于北京时间9月20日凌晨在线发表在《自然》杂志上。这项突破性成果有望形成新的随机数国际标准。

29、首届世界语言资源保护大会19日在长沙召开,来自40个参会国的代表及联合国教科文组织嘉宾、部分国家驻华使(领)馆嘉宾等参会。

30、{\textbf{中共中央政治局9月21日召开会议,审议《中国共产党支部工作条例(试行)》和《2018---2022年全国干部教育培训规划》}}。中共中央总书记习近平主持会议。会议指出,党支部是党的基础组织,是党的组织体系的基本单元。党的十八大以来,以习近平同志为核心的党中央高度重视党支部建设,要求把全面从严治党落实到每个支部、每名党员,推动全党形成大抓基层、大抓支部的良好态势,取得明显成效。会议指出,干部教育培训是干部队伍建设的先导性、基础性、战略性工程,在进行伟大斗争、建设伟大工程、推进伟大事业、实现伟大梦想中具有不可替代的重要地位和作用。制定实施好干部教育培训规划是全党的一件大事,对贯彻落实新时代党的建设总要求和新时代党的组织路线、培养造就忠诚干净担当的高素质专业化干部队伍、确保党的事业后继有人具有重大而深远的意义。

31、{\textbf{中共中央政治局9月21日下午就实施乡村振兴战略进行第八次集体学习}}。中共中央总书记习近平在主持学习时强调,乡村振兴战略是党的十九大提出的一项重大战略,是关系全面建设社会主义现代化国家的全局性、历史性任务,是新时代``三农''工作总抓手。我们要加深对这一重大战略的理解,始终把解决好``三农''问题作为全党工作重中之重,明确思路,深化认识,切实把工作做好,促进农业全面升级、农村全面进步、农民全面发展。

32、{\textbf{9月23日是秋分日,我国将迎来第一个中国农民丰收节}}。中共中央总书记、国家主席、中央军委主席习近平代表党中央,向全国亿万农民致以节日的问候和良好的祝愿。习近平指出,设立中国农民丰收节,是党中央研究决定的,进一步彰显了``三农''工作重中之重的基础地位,是一件影响深远的大事。秋分时节,全国处处五谷丰登、瓜果飘香,广大农民共庆丰年、分享喜悦,举办中国农民丰收节正当其时。习近平强调,我国是农业大国,重农固本是安民之基、治国之要。广大农民在我国革命、建设、改革等各个历史时期都作出了重大贡献。今年是农村改革40周年,40年来我国农业农村发展取得历史性成就、发生历史性变革。希望广大农民和社会各界积极参与中国农民丰收节活动,营造全社会关注农业、关心农村、关爱农民的浓厚氛围,调动亿万农民重农务农的积极性、主动性、创造性,全面实施乡村振兴战略、打赢脱贫攻坚战、加快推进农业农村现代化,在促进乡村全面振兴、实现``两个一百年''奋斗目标新征程中谱写我国农业农村改革发展新的华彩乐章!

33、广深港高铁香港段开通仪式9月22日在香港西九龙站举行,粤港各界人士约400人参加仪式。全国政协副主席董建华和梁振英、香港特区行政长官林郑月娥、广东省省长马兴瑞、国务院港澳办主任张晓明、香港中联办主任王志民等担任主礼嘉宾。{\textbf{自此,香港正式接入国家高铁大网络}}。

34、在中央电视台建台暨新中国电视事业诞生60周年之际,中共中央总书记、国家主席、中央军委主席习近平发来贺信,代表党中央表示热烈的祝贺,向中央广播电视总台全体干部职工、全国广大电视工作者致以诚挚的问候。习近平在贺信中表示,{\textbf{电视事业是党的新闻舆论工作的重要组成部分。60年来,广大电视工作者在党的领导下,坚持正确政治方向和舆论导向,围绕中心,服务大局,宣传党的主张,反映人民心声,唱响主旋律,传播正能量,为党和人民事业作出了积极贡献}}。

35、{\textbf{日前,中共中央决定,追授黄群、宋月才、姜开斌、王继才同志``全国优秀共产党员''称号}}。

36、为期3天的第四届中国(国际)商业航天高峰论坛9月26日在武汉拉开帷幕。来自中、俄、美、法、日等11个国家的近400位专家学者、企业代表,共话商业航天产业的发展现状与未来趋势。论坛还设立了占地5000平方米的商业航天产业主题成果展,展示全球商业航天领域发展硕果。

37、{\textbf{世界经济论坛日前发布2018年度青年科学家榜单,3位来自生物医药领域的中国科学家入围}}。本次评选中,世界范围内共有36位科学家入选,其中包括3位中国科学家,分别是专门研究心血管疾病病因的天津医科大学教授艾玎、研发用于早期疾病诊断传感器的天津大学教授段学欣以及研究方向为环境因素和遗传性疾病关系的南开大学药物化学生物学国家重点实验室教授杨娜。

38、习近平近日在东北三省考察,主持召开深入推进东北振兴座谈会并发表重要讲话。{\textbf{他强调,要认真贯彻新时代中国特色社会主义思想和党的十九大精神,落实党中央关于东北振兴的一系列决策部署,坚持新发展理念,解放思想、锐意进取,瞄准方向、保持定力,深化改革、破解矛盾,扬长避短、发挥优势,以新气象新担当新作为推进东北振兴。习近平强调,东北地区是我国重要的工业和农业基地,维护国家国防安全、粮食安全、生态安全、能源安全、产业安全的战略地位十分重要,关乎国家发展大局。新时代东北振兴,是全面振兴、全方位振兴,要从统筹推进``五位一体''总体布局、协调推进``四个全面''战略布局的角度去把握,瞄准方向、保持定力,扬长避短、发挥优势,一以贯之、久久为功,撸起袖子加油干,重塑环境、重振雄风,形成对国家重大战略的坚强支撑。习近平强调,坚持和加强党的全面领导是东北振兴的坚强保证。要加强东北地区党的政治建设,全面净化党内政治生态,营造风清气正、昂扬向上的社会氛围。要加快建设一支高素质干部队伍,提高领导能力专业化水平。领导干部要带头转变作风、真抓实干,出真招、办实事、求实效,防止和克服形式主义、官僚主义。政治生态同自然生态一样,污染容易,治理不易。要坚持无禁区、全覆盖、零容忍,坚决查处各类腐败案件,始终保持党同人民的血肉联系}}。

\subsection{24678-2018年9月国际时事}
1、俄罗斯国防部8月30日发布消息称,俄海军与空天军将于9月1日至8日在地中海举行大规模军事演习,将有超过25艘舰船和约30架飞机参加演习,俄海军总司令科罗廖夫负责指挥。{\textbf{这是俄罗斯自2015年参与在叙利亚境内打击极端武装以来,在地中海部署的最强大海军力量。俄波罗的海舰队前司令弗拉基米尔·瓦卢耶夫表示,俄正在释放明确信号:反对别国侵略叙利亚,反对美国及北约国家可能以``化武''为借口对叙利亚政府军实施军事打击}}。

2、燃烧了16天的亚运圣火,在朋加诺体育场缓缓熄灭,{\textbf{第十八届亚洲运动会于9月2日晚在印度尼西亚雅加达落幕}}。来自亚洲45个国家和地区的1.13万名运动员参加了本届亚运会40个大项、465个小项的角逐。中国体育代表团在闭幕式上的旗手是轮滑运动员郭丹。中国体育代表团在本次亚运会上共获得132金、92银、65铜,共计289枚奖牌,在金牌榜和奖牌榜均位列第一。日本代表团以75金、56银、74铜的成绩位列第二。韩国代表团以49金、58银、70铜的成绩位列第三。第十九届亚运会将于2022年在中国杭州举办,在主题为``庆祝''与``感谢''的闭幕式中,杭州市市长徐立毅接过了亚奥理事会会旗,标志着亚运会正式进入``杭州时间''。一场短暂而精彩的文艺表演将杭州这座凝聚了古老与现代的城市完美展现在亚洲各地的观众面前。``绿色、智能、节俭、文明'',这是杭州对2022年亚运会的承诺。亚运会告别雅加达,2022年,杭州见。

3、{\textbf{9月1日从中非发展基金获悉:该基金规模已达到100亿美元}}。截至目前,中非发展基金累计决定对非洲36个国家的90多个项目投资超过46亿美元,投资遍及基础设施、产能装备、农业民生、能源资源开发等各个领域。

4、{\textbf{位于里约热内卢市北区、拥有2000万件藏品的巴西国家博物馆当地时间9月2日晚发生火灾。该馆今年6月刚刚举行成立200周年庆典,是巴西历史最悠久的博物馆,也是拉丁美洲最大的自然历史博物馆之一}}。

5、我国与毛里求斯于9月2日结束{\textbf{中毛自由贸易协定谈判,这一协定是我国与非洲国家商签的首个自贸协定,下一步双方将为最终签署协定做好准备}}。

6、{\textbf{日前,德国政府通过了《2018年度税法》草案。该法案将于2019年正式生效}},旨在确保在线零售商尤其是境外商户履行其在德国的增值税义务,杜绝偷税漏税行为。

7、9月9日,{\textbf{朝鲜在平壤金日成广场举行盛大阅兵式和群众花车游行,热烈庆祝国庆70周年}}。

8、9月9日,为期6天的2018年联合国气候变化曼谷谈判在泰国落下帷幕,此次会议是今年12月在波兰卡托维兹召开的第二十四届联合国气候变化大会的筹备会议,共有178个缔约方和140个非政府组织参加。{\textbf{《联合国气候变化框架公约》秘书处执行秘书埃斯皮诺萨在会后表示,此次会议在应对气候变化的一些议题上,取得了不同程度的进展,各方还需在未来几周加快谈判进度}}。

9、{\textbf{亚洲金融合作协会(亚金协)2018年度主论坛------东京金融高峰论坛9月10日在东京举行}},本次论坛由亚金协与三菱日联银行、瑞穗银行、三井住友银行联合举办。来自近20个国家和地区的金融机构、金融监管机构代表300余人围绕主题``新兴产业发展背景下的金融创新与合作''畅所欲言,为亚洲乃至全球经济金融的稳健发展开拓新思路、探索新路径。

10、9月11日上午,一场规模空前的中俄战略级联合战役演练,在俄罗斯后贝加尔边疆区楚戈尔训练场拉开帷幕。至此,备受关注的``东方-2018''战略演习渐入高潮阶段,中俄两军官兵将在未来几天里密切协同完成预定演习任务。``东方-2018''战略演习是自1981年苏联``西方-81''演习以来俄罗斯规模最大的军事演习,俄军参演人员超过30万,参演装备3.6万台(辆)、各种飞机1000余架、舰船近80艘,堪称``史无前例''。

11、{\textbf{9月10日,第七十二届联大通过了77国集团提交的非洲发展新伙伴关系和非洲冲突起因决议,重申了``合作共赢''和``人类命运共同体''理念}}。

12、在欧洲的智能手机市场趋于饱和之际,中国智能手机销售实现了意想不到的逆势增长。国际数据公司(IDC)发布的最新数据显示,{\textbf{华为凭借24.8\%的市场份额,成为欧洲市场上第二大手机供应商,超过了苹果公司的22.5\%。而小米手机登陆欧洲不到一年,迅速占领了3.8\%的市场份额,成为欧洲第四大手机供应商}}。短短8年间,以华为、小米、一加、欧珀等为代表的中国品牌智能手机受到了欧洲消费者的热捧。\\

13、9月12日晚,南苏丹冲突各方在埃塞俄比亚首都亚的斯亚贝巴签署最终和平协议,标志着南苏丹自2013年底爆发的内战宣告结束。当天,{\textbf{南苏丹总统基尔和苏丹各反对派领导人分别在最终和平协议上签字,苏丹总统巴希尔、乌干达总统穆塞维尼等地区国家领导人作为协议担保方同时签署协议}}。

14、英国时装协会近日宣布,{\textbf{2018年伦敦时装周将全面禁止使用皮草,成为全球首个弃用皮草的知名时装周}}。

15、{\textbf{9月13日,德国宣布与意大利在难民问题上取得突破,双方就签署难民返还协议达成一致}}。协议规定,今后德国将有权把在德国与奥地利边境截获的所谓``二次移民''(已在意大利登记过避难申请的难民)遣送回意大利。作为交换条件,德国每遣送一人回意大利,便有义务从意大利接收一名其从海上救起的新难民。此前,德国已经与希腊和西班牙达成了内容相近的协议。

16、{\textbf{第十五届中国---东盟博览会、中国---东盟商务与投资峰会15日在广西南宁闭幕}}。据悉,本届中国---东盟博览会展览总面积12.4万平方米,总展位数6600个,参展企业2780家,比上届增长2.6\%,
8个东盟国家包馆。会期举办贸易投资促进活动91场,投资合作项目涵盖中国20多个省区与东盟及欧美等10多个国家。东盟国家的农产品、生活消费品、轻工工艺品,中国的工程机械、食品包装机械、电力设备、卫生洁具等商品成交踊跃。

17、{\textbf{首届中国---巴西研讨会近日在巴西圣保罗市开幕}}。与会的巴西专家及商界人士表示,期待以首届中国国际进口博览会为契机,与中国合作伙伴签订长效合作协议,扩大向中国市场出口优质产品,推进两国经贸合作向纵深发展。

18、{\textbf{第七十五届威尼斯电影节日前在意大利水城威尼斯举行}}。威尼斯电影节创办于1932年,是世界上历史最悠久的电影节,与戛纳电影节、柏林电影节并称为欧洲三大电影节。

19、9月17日,生态环境部与联合国环境署联合召开2018年中国国际保护臭氧层日纪念大会。{\textbf{生态环境部有关负责人表示,通过不懈努力,中国累计淘汰消耗臭氧层物质(ODS)约28万吨,占发展中国家淘汰量一半以上}}。

20、9月16日,厄立特里亚总统伊萨亚斯与埃塞俄比亚总理阿比在沙特阿拉伯国王萨勒曼主持下,在沙特海滨城市吉达签署了和平协议。{\textbf{联合国官网评论称,该协议结束了双方持续数十年的冲突,是一份``历史性的和平协议''}}。

21、{\textbf{9月18日至20日,韩国总统文在寅访问朝鲜,与朝鲜国务委员会委员长金正恩在平壤举行今年以来的第三次会晤。朝韩两国领导人19日签署《9月平壤共同宣言》}},就改善发展相互关系、缓和地区军事紧张、推动半岛无核化与和谈进程达成新的重要共识。两国同时签署《〈板门店宣言〉军事领域履行协议》,决定将积极采取实质性措施,消除朝鲜半岛战争威胁,把朝鲜半岛建设成``永久的和平地带''。朝韩领导人还就军事、经济、文化、体育等领域交流合作达成诸多共识。金正恩委员长表示,《9月平壤共同宣言》使北南关系更上台阶,促使半岛成为稳固的和平安全地带、让半岛和平繁荣的时代更早到来。文在寅总统认为,南北首次就无核化方案达成协议,``这是很有意义的成果''。

22、9月21日,中国外交部副部长郑泽光召见美国驻华大使布兰斯塔德,就美方援引美国国内法,对中国中央军委装备发展部及其负责人实施制裁提出严正交涉和抗议。9月22日晚,中央军委国际军事合作办公室副主任黄雪平召见美国驻华使馆代理国防武官孟绩伟,就美方宣布对中国中央军委装备发展部及该部负责人实施制裁提出严正交涉和抗议。

23、{\textbf{中国驻瑞典大使馆9月22日就瑞典电视台21日播出辱华节目提出强烈抗议}}。中国大使馆发言人当天发表谈话指出,21日晚,瑞典电视台``瑞典新闻''栏目播出恶劣辱华节目,该栏目主持人耶斯佩尔·伦达尔``发表恶毒侮辱攻击中国和中国人的言论,我们对此予以强烈谴责,已向瑞典电视台提出强烈抗议''。

24、由中国交通建设股份有限公司承建的内罗毕---马拉巴标轨铁路(内马铁路)项目第一期恩贡隧道9月24日实现贯通。据悉,恩贡隧道全长约4.5公里,是目前东非地区最长的铁路隧道。{\textbf{它的建成标志着内马铁路项目第一期取得了突破性进展}}。

25、{\textbf{土耳其总统埃尔多安9月27日至29日对德国进行国事访问,这是其2014年连任总统后首次访问德国}}。德国总理默克尔28日与埃尔多安在会晤后的新闻发布会上表示,双方在很多问题上``达成一致''。双方均表达了希望进一步加强双边关系的意愿。

