矛盾的同一性和斗争性:

\textbf{{同一性}}是指矛盾双方相互依存、相互贯通的性质和趋势。

\textbf{{斗争性}}是矛盾着的对立面之间相互排斥、相互分离的性质和趋势。

矛盾的同一性和斗争性相互联结、相互制约的原理,要求我们{在分析和解决矛盾时,必须从对立中把握同一,从同一中把握对立。}矛盾的同一性和斗争性不仅揭示了事物联系的实在内容,而且\textbf{{揭示了事物发展的内在动力}}。\textbf{{矛盾是事物发展的动力和源泉}}。矛盾推动事物的发展,事物发展的根本原因不在事物的外部,而在于事物内部的矛盾。

\textbf{{和谐}}作为矛盾的一种特殊表现形式,体现着矛盾双方的相互依存、相互促进、共同发展。

矛盾的\textbf{{普遍性和特殊性}}的关系:矛盾普遍性和特殊性相互区别、相互联系,并在一定条件下相互转化。任何事物都是共性和个性的统一,共性寓于个性之中,个性与共性相联系而存在。

{事物的}\textbf{{主要矛盾和非主要矛盾}}{相互区别,相互作用,并在一定条件下相互转化。}\textbf{{矛盾的主要方面与非主要方面}}{相互区别、相互作用,并在一定条件下相互转化。}
