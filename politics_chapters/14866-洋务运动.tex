{\textbf{{性质}:}地主阶级探索。}

{\textbf{{事件}:}{兴办近代企业;建立新式海陆军;创办新式学堂,派遣留学生}。}

{\textbf{{结果}:}失败(\textbf{{甲午战争战败为标志}})。}

{\textbf{{原因}:{封建性,依赖性,腐朽性}}。}

{\textbf{{意义}:}促进民族资本主义发展,培养人才,社会风气变化。}

{\textbf{{人物}:}{奕訢、曾国藩、李鸿章、左宗棠、张之洞、冯桂芬(最早)。}}

{\textbf{{相关内容}:{中学为体,西学为用}}{;}\textbf{{自强求富}}{;}{稍分洋商之利。}}
