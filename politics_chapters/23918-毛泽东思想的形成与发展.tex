{{毛泽东思想是马克思列宁主义在中国的运用和发展,是被实践证明了}{的关于中国革命和建设的正确的理论原则和经验总结,是}\textbf{{中国共产党集\textbf{体}}{\textbf{智慧的结晶。}}}}

{帝国主义战争与无产阶级革命的\textbf{时代主题},是毛泽东思想形成的时代背景。}{
    {
        % \\
}{{中国共产党领导的革命和建设的实践,是毛泽东思想形成的实践基础。\\
}{开始形成}{ }{:第一次国内革命战争时期,通过调查研究论证阶级关系
;土地革命战争时期开辟了农村包围城市、武装夺取政权的道路,与教条主义作斗争,论证中国革命新道路。\\
}{走向成熟}{
}{:遵义会议后到抗日战争时期,完成新民主主义革命理论和政策是成熟的标志。
在中共七大上毛泽东思想被确立为党的指导思想。新民主主义理论的系统阐明,
标志着毛泽东思想得到多方面展开而达到成熟。\\
}{继续发展}{
:解放战争时期和新中国成立后,提出人民民主专政理论、社会主义改造理论及``第二次结合''等。}}\\
}
