\textbf{主要错误}:1.
在革命性质和统一战线问题上,混淆民主革命与社会主义革命的界限。2.
在革命道路问题上,继续坚持以城市为中心。3.
在土地革命问题上,提出坚决打击富农和``地主不分田,富农分坏田''的主张。4.
在军事斗争问题上,实行进攻中的冒险主义、防御中的保守主义、退却中的逃跑主义。5.
在党内斗争和组织问题上,推行宗派主义和``残酷斗争,无情打击''的方针。

\textbf{形成原因}{:1.
八七会议以后党内一直存在着的浓厚的``左''倾情绪始终没有得到认真的清理。2.
共产国际对中国共产党内部事务的错误干预和瞎指挥。3.
全党不善于把马克思列宁主义与中国实际全面地、正确地结合起来。}

\textbf{危害}{:1.
其最大的恶果就是,使红军在第五次反``围剿''作战中遭到失败,不得不退出南方根据地实行长征。2.
这次错误使红军和根据地损失了90%。3.
国民党统治区党的力量几乎损失了100%。}
