{\textbf{量}}:事物存在和发展的规模、程度、速度以及它的构成成分在空间上的排列组合等可以用数量表示的规定性。

质:一事物成为自身并区别于它事物的规定性。事物质的规定性是由事物内部矛盾的特殊性所决定的。

\textbf{{度}}:\textbf{{事物保持自己质的量的界限}},即事物的范围、幅度和限度。它的极限叫\textbf{{关节点}}。认识度才能为实践活动提供正确的准则即适度原则,防止``过''或``不及''。

\textbf{{区分量变和质变的根本标志是事物的变化是否超出度}},在度的范围内的变化是量变,超出度的变化是质变。

{量变与质变是事物变化和发展的状态与形式;在实践上,即要重视量变,又要重视质变。}
