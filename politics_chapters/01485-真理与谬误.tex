{真理的}\textbf{{具体性}}{是指任何真理都是在特定条件、范围的限定下主体的认识同特定对象的一致或者符合。}如果超出这些限定``只要再多走一小步,真理就会变成错误''。

{真理是有}\textbf{{条件的}}{,真理是}\textbf{{历史的}}{,}\textbf{{具体的}}{,而不是抽象的。同一个真理不能因人而异。}

\textbf{真理和谬误}{是性质不同的两种认识,它们是对立的。但是,真理和谬误又是统一的,它们相互依存、相互转化。}\textbf{{真理和谬误在同一条件和范围下对立,超出这个范围就可以相互转化。}}

{在人们的认识和实践活动中,正确的认识往往会导致成功的实践,而由于主客观条件的限制,人们的实践活动也会达不到自身所期待的结果,导致失败。}\textbf{{错误往往是正确的先导}}{,失败常常是成功之母。要人们分析失败的原因,化不利条件为有利因素,就能从失败中吸取教训,变失败为成功。}

{}

实践是检验真理的唯一标准。

