原因:建立革命统一战线是中国革命发展的客观需要。1.
中国处于帝国主义与北洋军阀的残酷统治之下,社会矛盾日益加深,人民生活更趋恶化。2.
京汉铁路罢工遭血腥镇压,中国共产党由此认识到要胜利,必须组织革命的统一战线。{\textbf{3.}
}\textbf{中共三大正式决定全体共产党员以个人名义加入国民党,同孙中山领导的国民党建立统一战线}{。}{4.
孙中山的转变。}\textbf{国民党一大的成功召开标志着第一次国共合作的正式形成}{。}

国民党一大通过的宣言对三民主义作出了新的解释{。这个新三民主义的政纲同中共在民主革命阶段的纲领基本一致,因而成为}\textbf{{国共合作的政治基础}}{。在}民族主义中突出了反帝的内容,强调对外实现中华民族的独立,同时主张国内各民族一律平等{;}在民权主义中强调了民主权利应``为一般平民所共有'',不应为``少数人所得而私''{;}把民生主义概括为``平均地权''和``节制资本''两大原则{(后来又提出了``}耕者有其田{''的主张)。大会实际上确定了}\textbf{{联俄、联共、扶助农工}}{三大革命政策。}

{国共合作的形成,极大地推动了国民革命的迅猛发展。主要表现在:黄埔军校的创建;广东革命根据地的统一;工农运动的发展;北伐战争的胜利进展。}
