现象和本质是揭示客观事物的外部表现和内部联系相互关系的范畴。

{\textbf{现象}}:事物的外部联系和表面特征,人们可通过感官感知。

\textbf{{本质}}:是事物的内在联系和根本性质,只有靠人的理性思维才能把握。

{\textbf{区别}}:现象是个别的、具体的,本质是一般的、共同的;现象是多变的,本质是相对稳定的;现象是生动、丰富的,本质是比较深刻、单纯的。

\textbf{{联系}}:任何本质都是通过现象表现出来,任何现象都从一定的方面表现着本质。

\textbf{现象和本质的区别决定了认识的必要性,现象和本质的联系决定了认识的可能性。}
