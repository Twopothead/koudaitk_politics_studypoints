{三民主义学说和资产阶级共和国方案:同盟会的政治纲领是``\textbf{{驱除鞑虏,恢复中华,创立民国,平均地权}}''。孙中山将同盟会的纲领概括为\textbf{{三大主义}},即\textbf{{民族主义、民权主义、民生主义}},后被称为\textbf{{三民主义}}。}

{\textbf{{民族主义}}{,即}\textbf{{民族革命}}{,包括``}\textbf{{驱除鞑虏,恢复中华}}{''两项内容。}}

{\textbf{{民权主义}}{即}\textbf{{政治革命}}{,内容是``}\textbf{{创立民国}}{'',即建立资产阶级民主共和国。}}

{\textbf{{民生主义}}{即}\textbf{{社会革命}}{,指的是``}\textbf{{平均地权}}{''。}}

{{它初步描绘出中国还不曾有过的资产阶级共和国方案,是一个比较完整而明确的资产阶级民主革命纲领;对推动革命的发展产生了重大而积极的影响;但它并不是一个彻底的资产阶级民主革命纲领}{。}}

{关于革命与改良的辩论 :}

{\textbf{{论战内容}}{:}{①要不要以革命手段推翻清王朝。②要不要推翻帝制,实行共和。③要不要社会革命}{。}}

{\textbf{{论战意义}}{:}{通过这场论战,划清了革命与改良的界限,传播了民主革命思想,促进了革命形势的发展}{。但这场论战也暴露了革命派在思想理论方面的弱点。}}
