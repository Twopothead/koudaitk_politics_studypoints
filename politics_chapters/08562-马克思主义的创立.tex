{经济社会根源}:\textbf{{资本主义经济的发展}}为马克思主义的产生提供了经济、社会历史条件。

{{实践基础:}\textbf{{无产阶级反对资产阶级的斗争}}日益激化。}

{{思想渊源:}\textbf{{德国古典哲学、英国古典政治经济学和法国、英国空想社会主义}}的合理成分,\textbf{{创立了唯物史观和剩余价值学说,把社会主义由空想变为科学。}}}

{马克思、恩格斯批判地继承了前人的成果,创立了唯物史观和剩余价值学说,实现了人类思想史上的伟大革命。马克思}{1845}{年春天写的\textbf{{《关于费尔巴哈的提纲》}}和马克思、恩格斯}{1844-1846}{年合写的\textbf{{《德意志意识形态》}},{标志着马克思主义的基本形成}。}

{1847}{年\textbf{{《哲学的贫困》}}和}{1848}{年}{2}{月\textbf{{《共产党宣言》}}的发表,{标志着马克思主义的公开问世}。}
