{资产阶级革命派的骨干是一批{资产阶级、小资产阶级知识分子}。这些青年知识分子,成为了辛亥革命的中坚力量。}

{孙中山到檀香山组织第一个\textbf{{资产阶级革命团体}{------}{兴中会}},提出了``{驱除鞑虏,恢复中华,创立合众政府}''的革命纲领,并筹划发动反清起义。}

{{章炳麟发表了《驳康有为论革命书》,邹容出版了《革命军》,陈天华出版了《警世钟》、《猛回头》两本小册子}{,号召人民奋起革命。在资产阶级革命思想的传播过程中,资产阶级革命团体也在各地次第成立,其中重要的有}{华兴会、科学补习所、光复会、岳王会}{等。}\textbf{{孙中山和黄兴、宋教仁等人在日本东京成立中国同盟会}}{,同盟会以}\textbf{{《民报》为机关报}}{,并确定了革命纲领。这是近代中国}\textbf{{第一个领导资产阶级革命的政党}}{,它的成立标志着中国资产阶级民主革命进入了一个新的阶段。}}
