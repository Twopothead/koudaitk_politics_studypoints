1、8月27日至9月1日,中央扫黑除恶专项斗争9个督导组完成对山西、辽宁、福建、山东、河南、湖北、广东、重庆、四川等9省(市)的进驻工作,{\textbf{标志着中央扫黑除恶专项斗争第一轮督导工作全面启动}}。

2、9月2日,中国企业联合会、中国企业家协会连续第十七年发布``中国企业500强''榜单及报告,{\textbf{国家电网、中国石化、中国石油继续分列前三。}}``2018中国企业500强''入围门槛首次突破300亿元大关,实现了16连升;企业营业收入总额首次突破70万亿元大关,达到71.17万亿元,迈上新的台阶,营收较上年增长了11.20\%,增速加快3.56个百分点,重回两位数增速区间。\\

3、{\textbf{9月3日,中非合作论坛北京峰会在人民大会堂隆重开幕}}。中国国家主席习近平出席开幕式并发表主旨讲话,强调中非要携起手来,共同打造责任共担、合作共赢、幸福共享、文化共兴、安全共筑、和谐共生的中非命运共同体,重点实施好产业促进、设施联通、贸易便利、绿色发展、能力建设、健康卫生、人文交流、和平安全``八大行动''。习近平强调,中国是世界上最大的发展中国家,非洲是发展中国家最集中的大陆,中非早已结成休戚与共的命运共同体。我们愿同非洲人民共筑更加紧密的中非命运共同体,为推动构建人类命运共同体树立典范。

4、{\textbf{国家主席习近平9月3日在国家会议中心出席中非领导人与工商界代表高层对话会暨第六届中非企业家大会开幕式并发表题为《共同迈向富裕之路》的主旨演讲}},强调中国支持非洲国家参与共建``一带一路'',愿同非洲加强全方位对接,打造符合国情、包容普惠、互利共赢的高质量发展之路,共同走上让人民生活更加美好的幸福之路。

5、{\textbf{纪念中国人民抗日战争暨世界反法西斯战争胜利73周年座谈会9月3日在北京举行}},中共中央政治局委员、中宣部部长黄坤明出席。中国人民抗日战争胜利,是近代以来中国抗击外敌入侵的第一次完全胜利。这一伟大胜利,开辟了中华民族伟大复兴的光明前景,开启了古老中国凤凰涅槃、浴火重生的新征程。

6、{\textbf{中非合作论坛北京峰会圆桌会议9月4日在人民大会堂举行。}}国家主席习近平和论坛共同主席国南非总统拉马福萨分别主持第一阶段和第二阶段会议。会议通过《关于构建更加紧密的中非命运共同体的北京宣言》和《中非合作论坛---北京行动计划(2019---2021年)》。习近平强调,我们一致同意秉持共商共建共享原则,将中非合作论坛建设成为中非团结合作的品牌、国际对非合作的旗帜。我们将加强政策协调,推进落实论坛峰会成果,并把中非共建``一带一路''、非洲联盟《2063年议程》、联合国2030年可持续发展议程、非洲各国发展战略紧密结合起来,为非洲发展振兴提供更多机遇和有效平台,为中非合作提供不竭动力和更大空间。\\

7、{\textbf{国务院总理李克强9月6日主持召开国务院常务会议}},确定落实新修订的个人所得税法的配套措施,为广大群众减负;决定完善政策确保创投基金税负总体不增;部署打造``双创''升级版,增强带动就业能力、科技创新力和产业发展活力;通过《专利代理条例(修订草案)》。会议指出,全面落实全国人大常委会审议通过的新修订的个人所得税法,建立综合与分类相结合的个人所得税制,是我国前所未有的重大税制改革。要在确保10月1日起如期将个税基本减除费用标准由3500元提高到5000元并适用新税率表的同时,抓紧按照让广大群众得到更多实惠的要求,明确子女教育、继续教育、大病医疗、普通住房贷款利息、住房租金、赡养老人支出6项专项附加扣除的具体范围和标准,使群众应纳税收入在减除基本费用标准的基础上,再享有教育、医疗、养老等多方面附加扣除,确保扣除后的应纳税收入起点明显高于5000元,进一步减轻群众税收负担,增加居民实际收入、增强消费能力。专项附加扣除范围和标准在向社会公开征求意见后依法于明年1月1日起实施。今后随着经济社会发展和人民生活水平提高,专项附加扣除范围和标准还将动态调整。会议强调,目前全国养老金累计结余较多,可以确保按时足额发放,在社保征收机构改革到位前,各地要一律保持现有征收政策不变,同时抓紧研究适当降低社保费率,确保总体上不增加企业负担,以激发市场活力,引导社会预期向好。 为促进创业创新,会议决定,保持地方已实施的创投基金税收支持政策稳定,由有关部门结合修订个人所得税法实施条例,按照不溯及既往、确保总体税负不增的原则,抓紧完善进一步支持创投基金发展的税收政策。

8、{\textbf{2018年9月7日,纪念``一带一路''倡议在哈萨克斯坦提出5周年商务论坛在哈萨克斯坦首都阿斯塔纳举行}}。国家主席习近平通过视频表示祝贺。习近平表示,哈萨克斯坦是``一带一路''倡议的坚定支持者和积极参与者。5年来,在双方共同努力下,中哈共建``一带一路''合作取得丰硕成果。中国愿同哈萨克斯坦及其他有关各国一道,秉持共商共建共享理念,以开放包容姿态致力于共同发展和繁荣,把``一带一路''建设成为和平之路、繁荣之路、开放之路、创新之路、文明之路,为造福各国人民、推动构建人类命运共同体作出更大贡献。

9、{\textbf{9月7日11时15分,我国在太原卫星发射中心用长征二号丙运载火箭成功发射海洋一号C星}}。该星将进一步提升我国海洋遥感技术水平,对我国研究海气相互作用、提高防灾减灾能力、开展全球气候变化研究、解决人类共同面临的全球气候变暖等问题具有重要意义,将开启我国自然资源卫星陆海统筹发展新局面,助力海洋强国建设。

10、9月8日,{\textbf{为期两天的第二届``中拉文明对话''研讨会在江苏南京召开,会议主题是``一带一路:中拉文明对话之路''}}。

11、9月8日电,{\textbf{第二十届中国国际投资贸易洽谈会8日在福建省厦门市开幕,国家主席习近平向投洽会致贺信}}。

12、{\textbf{9月9日,中共中央总书记、国家主席习近平特别代表,中共中央政治局常委、全国人大常委会委员长栗战书在平壤会见了朝鲜劳动党委员长、国务委员会委员长金正恩}}。栗战书首先转达习近平对金正恩的亲切问候并转交亲署函。习近平在亲署函中指出,朝鲜建国70年以来,在金日成同志、金正日同志和委员长同志坚强领导下,朝鲜党和人民奋力推进社会主义建设事业,取得了不平凡的成就。当前,委员长同志正带领朝鲜党和人民,全面贯彻落实新战略路线,致力于发展经济、改善民生,在社会主义建设各个领域不断取得新的成就。

13、{\textbf{全国教育大会9月10日在北京召开}}。中共中央总书记、国家主席、中央军委主席习近平出席会议并发表重要讲话。他强调,在党的坚强领导下,全面贯彻党的教育方针,坚持马克思主义指导地位,坚持中国特色社会主义教育发展道路,坚持社会主义办学方向,立足基本国情,遵循教育规律,坚持改革创新,以凝聚人心、完善人格、开发人力、培育人才、造福人民为工作目标,培养德智体美劳全面发展的社会主义建设者和接班人,加快推进教育现代化、建设教育强国、办好人民满意的教育。全党全社会要弘扬尊师重教的社会风尚,努力提高教师政治地位、社会地位、职业地位,让广大教师享有应有的社会声望,在教书育人岗位上为党和人民事业作出新的更大的贡献。习近平指出,要深化教育体制改革,健全立德树人落实机制,扭转不科学的教育评价导向,坚决克服唯分数、唯升学、唯文凭、唯论文、唯帽子的顽瘴痼疾,从根本上解决教育评价指挥棒问题。要深化办学体制和教育管理改革,充分激发教育事业发展生机活力。要提升教育服务经济社会发展能力,调整优化高校区域布局、学科结构、专业设置,建立健全学科专业动态调整机制,加快一流大学和一流学科建设,推进产学研协同创新,积极投身实施创新驱动发展战略,着重培养创新型、复合型、应用型人才。要扩大教育开放,同世界一流资源开展高水平合作办学。

14、{\textbf{我国第一艘自主建造的极地科学考察破冰船9月10日在上海下水,并正式命名为``雪龙2''号,标志着我国极地考察现场保障和支撑能力取得新的突破}}。

15、{\textbf{国家主席习近平9月11日在符拉迪沃斯托克同俄罗斯总统普京举行会谈}}。两国元首一致认为,今年以来,中俄关系呈现更加积极的发展势头,进入更高水平、更21快发展的新时期。一致同意,无论国际形势如何变化,中俄都将坚定发展好两国关系,坚定维护好世界和平稳定。习近平强调,中俄双方要深化共建``一带一路''和欧亚经济联盟对接合作,扩大能源、农业、科技创新、金融等领域合作,推动重点项目稳步实施,加强前沿科学技术共同研发,利用好今明两年中俄地方合作交流年契机,调动两国更多地方积极性,开展更加广泛合作。

16、教育是国之大计、党之大计。{\textbf{9月10日,习近平总书记在全国教育大会发表重要讲话}},从党和国家事业发展全局的战略高度,系统总结了我国教育事业发展的成就与经验,深刻分析了教育工作面临的新形势新任务,对加快推进教育现代化、建设教育强国、办好人民满意的教育作出了全面部署。教育是民族振兴、社会进步的重要基石,是功在当代、利在千秋的德政工程。党的十九大从新时代坚持和发展中国特色社会主义的战略高度,作出了优先发展教育事业、加快教育现代化、建设教育强国的重大部署。

17、{\textbf{从中央军委训练管理部了解到,我军司号制度恢复和完善工作正有序展开,拟从10月1日起全面恢复播放作息号,下达日常作息指令。明年8月1日起,全军施行新的司号制度}}。

18、{\textbf{第四届东方经济论坛全会9月12日在符拉迪沃斯托克举行}}。中国国家主席习近平、俄罗斯总统普京、蒙古国总统巴特图勒嘎、日本首相安倍晋三、韩国总理李洛渊等出席。习近平发表了题为《共享远东发展新机遇
开创东北亚美好新未来》的致辞,强调中方愿同地区国家一道,维护地区和平安宁,实现各国互利共赢,巩固人民传统友谊,实现综合协调发展,促进本地区和平稳定和发展繁荣。

19、国家主席习近平9月12日在符拉迪沃斯托克会见日本首相安倍晋三。习近平强调,{\textbf{中日双方要始终恪守和遵循中日间四个政治文件,巩固政治基础,把握正确方向,建设性管控分歧,特别是日方要妥善处理好历史、台湾等敏感问题,积极营造良好气氛,不断扩大共同利益。我们欢迎日本继续积极参与中国改革开放进程,实现共同发展繁荣。``一带一路''倡议为中日深化互利合作提供了新平台和试验田。中方愿同日方一道,着眼新形势,为两国务实合作开辟新路径,打造新亮点。中日双方应共同推进区域一体化进程,建设和平、稳定、繁荣的亚洲。双方要坚定维护多边主义,维护自由贸易体制和世界贸易组织规则,推动建设开放型世界经济。双方要弘扬民间友好传统,赋予其新的时代内涵,夯实两国关系的社会和民意基础}}。

20、国台办发言人安峰山9月12日在例行新闻发布会上介绍,{\textbf{《港澳台居民居住证申领发放办法》公布后,受到广大台湾同胞普遍欢迎和肯定。据不完全统计,到9月10日止,短短10天已有超过2.2万名台胞申领了居住证。这充分说明,这是一项真正造福于民、广受台胞欢迎的好政策}}。

21、{\textbf{中国残疾人联合会第七次全国代表大会9月14日上午在北京人民大会堂开幕}}。习近平、李克强、栗战书、汪洋、王沪宁、赵乐际等党和国家领导人到会祝贺,韩正代表党中央、国务院致词。韩正在致词中说,党的十八大以来,我国残疾人事业取得历史性进展和显著成就。习近平总书记对残疾人和残疾人事业发展提出了一系列明确要求,为新时代中国特色残疾人事业发展指明了前进方向,提供了根本遵循。我们要以习近平新时代中国特色社会主义思想为指引,坚持树立正确的价值理念,坚守弱有所扶的原则立场,完成决胜全面建成小康社会的关键任务,促进残疾人全面发展和共同富裕,把推进残疾人事业当作分内责任,在实现中国梦的伟大征程中创造残疾人更加幸福美好的新生活。\\

22、国家主席习近平9月14日在人民大会堂同委内瑞拉总统马杜罗举行会谈。习近平强调,{\textbf{双方要筑牢政治互信,保持高层交往势头,让中委友好成为两国各界政治共识。中方赞赏委方在涉及中方核心利益和重大关切问题上给予中方理解和支持,将一如既往支持委内瑞拉政府谋求国家稳定发展的努力,支持委内瑞拉探索符合本国国情的发展道路,愿同委方加强治国理政经验交流。双方要优化创新务实合作,以签署共建``一带一路''谅解备忘录为契机,加紧对接、推进落实双方业已达成的合作共识,提升委方自主发展能力,推动两国合作可持续发展。双方要积极促进民心相通,扩大人文领域交流合作和地方交往,夯实两国友好社会根基。双方要加强多边协调配合,继续在联合国等国际和地区组织内加强沟通,共同参与全球治理体系改革和建设,维护发展中国家正当权益}}。

23、目前国内人工智能商业落地的百强企业中,22家在上海,国内1/3左右的人工智能人才也在上海。科技实力和人才基础丰厚的上海,近日将迎来2018世界人工智能大会。上海先后出台科创中心建设22条、促进科技成果转化条例、人才政策30条、扩大开放100条等政策法规,勇于改革,激发自主创新活力。{\textbf{连续5年,上海在``魅力中国------外籍人才眼中最具吸引力的中国城市''评选中拔得头筹}}。

24、9月16日12时,北纬35度至26度30分之间的黄海和东海海域正式开渔,标志着今年我国伏季休渔全面结束。

25、{\textbf{2018世界人工智能大会9月17日在上海开幕}}。国家主席习近平致信,向大会的召开表示热烈祝贺,向出席大会的各国代表、国际机构负责人和专家学者、企业家等各界人士表示热烈欢迎。习近平强调,中国正致力于实现高质量发展,人工智能发展应用将有力提高经济社会发展智能化水平,有效增强公共服务和城市管理能力。中国愿意在技术交流、数据共享、应用市场等方面同各国开展交流合作,共享数字经济发展机遇。希望与会嘉宾围绕``人工智能赋能新时代''这一主题,深入交流、凝聚共识,共同推动人工智能造福人类。

26、{\textbf{由工业和信息化部、科技部和江苏省政府共同主办、主题为``数字新经济
物联新时代''的2018世界物联网博览会日前在无锡开幕}}。会上发布的《2017---2018年中国物联网发展年度报告》显示,2017年以来,我国物联网市场进入实质性发展阶段。全年市场规模突破1万亿元,年复合增长率超过25%,其中物联网云平台成为竞争核心领域,预计2021年我国物联网平台支出将位居全球第一。

27、{\textbf{9月18日,国务院关税税则委员会发布公告,决定对美国原产的约600亿美元进口商品实施加征关税。9月18日,中国在世贸组织追加起诉美国301调查项下对华2000亿美元输美产品实施的征税措施}}。

28、近日,中国科学技术大学教授潘建伟及其同事张强、范靖云、马雄峰等与中科院上海微系统与信息技术研究所和日本NTT基础科学实验室合作,在国际上首次成功实现器件无关的量子随机数。相关研究成果于北京时间9月20日凌晨在线发表在《自然》杂志上。这项突破性成果有望形成新的随机数国际标准。

29、首届世界语言资源保护大会19日在长沙召开,来自40个参会国的代表及联合国教科文组织嘉宾、部分国家驻华使(领)馆嘉宾等参会。

30、{\textbf{中共中央政治局9月21日召开会议,审议《中国共产党支部工作条例(试行)》和《2018---2022年全国干部教育培训规划》}}。中共中央总书记习近平主持会议。会议指出,党支部是党的基础组织,是党的组织体系的基本单元。党的十八大以来,以习近平同志为核心的党中央高度重视党支部建设,要求把全面从严治党落实到每个支部、每名党员,推动全党形成大抓基层、大抓支部的良好态势,取得明显成效。会议指出,干部教育培训是干部队伍建设的先导性、基础性、战略性工程,在进行伟大斗争、建设伟大工程、推进伟大事业、实现伟大梦想中具有不可替代的重要地位和作用。制定实施好干部教育培训规划是全党的一件大事,对贯彻落实新时代党的建设总要求和新时代党的组织路线、培养造就忠诚干净担当的高素质专业化干部队伍、确保党的事业后继有人具有重大而深远的意义。

31、{\textbf{中共中央政治局9月21日下午就实施乡村振兴战略进行第八次集体学习}}。中共中央总书记习近平在主持学习时强调,乡村振兴战略是党的十九大提出的一项重大战略,是关系全面建设社会主义现代化国家的全局性、历史性任务,是新时代``三农''工作总抓手。我们要加深对这一重大战略的理解,始终把解决好``三农''问题作为全党工作重中之重,明确思路,深化认识,切实把工作做好,促进农业全面升级、农村全面进步、农民全面发展。

32、{\textbf{9月23日是秋分日,我国将迎来第一个中国农民丰收节}}。中共中央总书记、国家主席、中央军委主席习近平代表党中央,向全国亿万农民致以节日的问候和良好的祝愿。习近平指出,设立中国农民丰收节,是党中央研究决定的,进一步彰显了``三农''工作重中之重的基础地位,是一件影响深远的大事。秋分时节,全国处处五谷丰登、瓜果飘香,广大农民共庆丰年、分享喜悦,举办中国农民丰收节正当其时。习近平强调,我国是农业大国,重农固本是安民之基、治国之要。广大农民在我国革命、建设、改革等各个历史时期都作出了重大贡献。今年是农村改革40周年,40年来我国农业农村发展取得历史性成就、发生历史性变革。希望广大农民和社会各界积极参与中国农民丰收节活动,营造全社会关注农业、关心农村、关爱农民的浓厚氛围,调动亿万农民重农务农的积极性、主动性、创造性,全面实施乡村振兴战略、打赢脱贫攻坚战、加快推进农业农村现代化,在促进乡村全面振兴、实现``两个一百年''奋斗目标新征程中谱写我国农业农村改革发展新的华彩乐章!

33、广深港高铁香港段开通仪式9月22日在香港西九龙站举行,粤港各界人士约400人参加仪式。全国政协副主席董建华和梁振英、香港特区行政长官林郑月娥、广东省省长马兴瑞、国务院港澳办主任张晓明、香港中联办主任王志民等担任主礼嘉宾。{\textbf{自此,香港正式接入国家高铁大网络}}。

34、在中央电视台建台暨新中国电视事业诞生60周年之际,中共中央总书记、国家主席、中央军委主席习近平发来贺信,代表党中央表示热烈的祝贺,向中央广播电视总台全体干部职工、全国广大电视工作者致以诚挚的问候。习近平在贺信中表示,{\textbf{电视事业是党的新闻舆论工作的重要组成部分。60年来,广大电视工作者在党的领导下,坚持正确政治方向和舆论导向,围绕中心,服务大局,宣传党的主张,反映人民心声,唱响主旋律,传播正能量,为党和人民事业作出了积极贡献}}。

35、{\textbf{日前,中共中央决定,追授黄群、宋月才、姜开斌、王继才同志``全国优秀共产党员''称号}}。

36、为期3天的第四届中国(国际)商业航天高峰论坛9月26日在武汉拉开帷幕。来自中、俄、美、法、日等11个国家的近400位专家学者、企业代表,共话商业航天产业的发展现状与未来趋势。论坛还设立了占地5000平方米的商业航天产业主题成果展,展示全球商业航天领域发展硕果。

37、{\textbf{世界经济论坛日前发布2018年度青年科学家榜单,3位来自生物医药领域的中国科学家入围}}。本次评选中,世界范围内共有36位科学家入选,其中包括3位中国科学家,分别是专门研究心血管疾病病因的天津医科大学教授艾玎、研发用于早期疾病诊断传感器的天津大学教授段学欣以及研究方向为环境因素和遗传性疾病关系的南开大学药物化学生物学国家重点实验室教授杨娜。

38、习近平近日在东北三省考察,主持召开深入推进东北振兴座谈会并发表重要讲话。{\textbf{他强调,要认真贯彻新时代中国特色社会主义思想和党的十九大精神,落实党中央关于东北振兴的一系列决策部署,坚持新发展理念,解放思想、锐意进取,瞄准方向、保持定力,深化改革、破解矛盾,扬长避短、发挥优势,以新气象新担当新作为推进东北振兴。习近平强调,东北地区是我国重要的工业和农业基地,维护国家国防安全、粮食安全、生态安全、能源安全、产业安全的战略地位十分重要,关乎国家发展大局。新时代东北振兴,是全面振兴、全方位振兴,要从统筹推进``五位一体''总体布局、协调推进``四个全面''战略布局的角度去把握,瞄准方向、保持定力,扬长避短、发挥优势,一以贯之、久久为功,撸起袖子加油干,重塑环境、重振雄风,形成对国家重大战略的坚强支撑。习近平强调,坚持和加强党的全面领导是东北振兴的坚强保证。要加强东北地区党的政治建设,全面净化党内政治生态,营造风清气正、昂扬向上的社会氛围。要加快建设一支高素质干部队伍,提高领导能力专业化水平。领导干部要带头转变作风、真抓实干,出真招、办实事、求实效,防止和克服形式主义、官僚主义。政治生态同自然生态一样,污染容易,治理不易。要坚持无禁区、全覆盖、零容忍,坚决查处各类腐败案件,始终保持党同人民的血肉联系}}。
