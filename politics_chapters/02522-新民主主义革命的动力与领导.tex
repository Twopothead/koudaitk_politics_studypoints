认清这个革命的动力问题,才能正确地解决中国革命的基本策略问题。\textbf{{新民主主义革命的动力}{是工人阶级、农民阶级、城市小资产阶级和民族资产阶级}},而\textbf{{根本的动力是工人和农民}}。

\textbf{{中国无产阶级是中国革命最基本的动力}}。{无产阶级的领导权是中国革命的中心问题,也是新民主主义革命理论的核心问题}。{革命的领导权是区别新旧两种不同范畴的民主主义革命的}\textbf{{根本标志}}。中国工人阶级是新的社会生产力的代表,是近代中国最进步的阶级。中国工人阶级具有{自身的特点和优点}。1.
它从诞生之日起,就身受外国资本主义、本国封建势力和资产阶级的三重压迫,而这些压迫的严重性和残酷性,是世界各民族中少见的,这就形成了中国无产阶级\textbf{{坚强的斗争性和彻底的革命性}}。2.
它\textbf{{分布集中}},有利于无产阶级队伍的组织和团结,有利于革命思想的传播和形成强大的革命力量。3.
它大部分出身于破产的农民,\textbf{{和农民有着天然的联系}},使无产阶级便于和农民结成亲密的联盟,共同团结战斗。无产阶级及其政党的领导,是中国革命取得胜利的根本保证。

\textbf{{农民是中国革命的主力军}},其中的贫{农是无产阶级最可靠的同盟军,而中农是无产阶级可靠的同盟军}。{农民问题是中国革命的基本问题},{新民主主义革命实质上就是党领导下的农民革命,中国革命战争实质上就是党领导下的农民战争}。

{城市小资产阶级,包括广大的知识分子、小商人、手工业者和自由职业者,城市小资产阶级同样是中国革命的动力之一},{是中国革命可靠的同盟军}。

\textbf{{民族资产阶级是一个带有两面性的阶级}}。中国民族资产阶级由一部分买办、地主官僚、商人、手工工场主转变而来。中国共产党对民族资产阶级在政治上争取它,对其动摇性和妥协性进行批评和斗争,在经济上实行保护民族工商业的政策,中国共产党对民族资产阶级采取{既联合又斗争}的方针。

{无产阶级及其政党对中国革命的领导权不是自然而然得来的,而是在与资产阶级争夺领导权的斗争中实现的}{。毛泽东指出:``领导的阶级和政党,要实现自己对于被领导的阶级、阶层、政党和人民团体的领导,必须具备}\textbf{{两个条件}}{:(甲)率领被领导者(同盟者)向着共同敌人作坚决的斗争,并取得胜利;(乙)对被领导者给以物质福利,至少不损害其利益,同时对被领导者给以政治教育。''}
