\textbf{{向社会主义过渡原因}:}第一,社会主义性质的国营经济力量相对来说比较强大,它是实现国家工业化的主要基础。第二,资本主义经济力量弱小,发展困难,不可能成为中国工业起飞的基础。第三,对个体农业进行社会主义改造,是保证工业发展、实现国家工业化的一个必要条件。第四,当时的国际环境也促使中国选择社会主义。

\textbf{{必然性}:}第一,实现社会主义工业化,是国家独立和富强的必然要求和必要条件。第二,对个体经济和私营资本主义工商业进行社会主义改造,是实现社会主义工业化的客观需要。

\textbf{{可能性}}:第一,我国已经有了相对强大和迅速发展的社会主义国营经济。第二,土地改革完成后,为发展生产、抵御自然灾害,广大农民具有走互助合作道路的要求。第三,新中国成立初期,党和国家在合理调整工商业的过程中,出现了加工订货、经销代销、统购包销、公私合营等一系列从低级到高级的国家资本主义形式。第四,当时的国际形势也有利于中国向社会主义过渡。

\textbf{{计划}:}{党和毛泽东对于何时向社会主义过渡、怎样过渡的问题,}\textbf{{经历了一个从先搞工业化建设再一举过渡,到建设和改造同时并举}}{、从中华人民共和国成立起即逐步过渡的发展变化过程。}
