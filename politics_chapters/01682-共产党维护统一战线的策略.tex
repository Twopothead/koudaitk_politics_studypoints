\textbf{统一战线中的独立自主原则}:中国共产党在统一战线中坚持独立自主原则,既统一,又独立。这样做的目的,是为了保持共产党领导的革命力量已经取得的阵地,尤其是为了发展这些阵地,以动员千百万群众进入抗日民族统一战线,实质上就是力争中国共产党对抗日战争的领导权,使自己成为团结全民族抗战的中坚力量。这是把抗日战争引向胜利的中心一环。

\textbf{坚持抗战、团结、进步,反对妥协、分裂、倒退}{:抗日战争相持阶段到来以后,由于以蒋介石为代表的国民党亲英美派开始推行消极抗日、积极反共的政策,团结抗战的局面逐步发生严重危机,出现了中途妥协和内部分裂两大危险。针对这种情况,中国共产党明确提出}\textbf{``坚持抗战到底,反对中途妥协''、``巩固国内团结,反对内部分裂''、``力求全国进步,反对向后倒退''}{三大口号,在抗战情况下,民族斗争高于阶级斗争,民族斗争与阶级斗争表现出一致性。}

{\textbf{巩固抗日民族统一战线的策略总方针:}}

为了抗日民族统一战线的坚持、扩大和巩固,中国共产党制定了``\textbf{发展进步势力,争取中间势力,孤立顽固势力}''的策略总方针。

进步势力主要的是指工人、农民和城市小资产阶级。他们是统一战线的基础,抗日战争的主要依靠力量。{}

中间势力主要是指民族资产阶级、开明绅士和地方实力派。争取中间势力,是中国共产党在抗日民族统一战线中的一项极严重的任务。争取中间势力需要一定的条件:一是共产党要有充足的力量;二是尊重他们的利益;三是要同顽固派作坚决的斗争,并能一步一步地取得胜利。

顽固势力是指大地主大资产阶级的抗日派,即以蒋介石集团为代表的国民党亲英美派{。他们采取两面政策,既主张团结抗日,又限共、溶共、反共并摧残进步势力。为此,共产党必须以革命的两面政策来对付他们,}即贯彻又联合又斗争的政策{,斗争不忘统一,统一不忘斗争,二者不可偏废,而以统一为主。}同顽固派斗争的策略原则是``有理''、``有利''、``有节''{。}{}
