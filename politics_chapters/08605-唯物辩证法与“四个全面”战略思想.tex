{
党的十八大以来,党中央从坚持和发展中国特色社会主义全局出发,{提出并形成了全面建成小康社会、全面深化改革.全面依法治国、全面从严治党的战略布局。}}

{
习近平总书记在论述四者之间的关系时指出:{全面建成小康社会是我们的战略目标,全面深化改革、全面依法治国、全面从严治党是三大战略举措。}}

{要努力做到``四个全面''相辅相成、相互促进、相得益彰。唯物辩证法坚持用联系的、发展的、全面的观点看世界,认为\textbf{{发展的根本原因在于事物的内部矛盾性}}。}

{``四个全面''战略构想在各个方面都体现了唯物辩证法思想。}

{{第一:体现了事物联系和发展的思想。}联系和发展是唯物辩证法的总特征。联系是指事物内部各
要素之间和事物之间相互影响、相互制约和相互作用的关系。}

{``四个全面''不仅揭示了``建成小康社会''``深化改革''"依法治国''和``从严治党之间的联系,也揭示各自战略目标和举措之间的联系。}

{发展是前进的上升的运动,发展的实质是新事物的产生和旧事物的灭亡。``四个全面''思想也体现了事物是发展变化的这一辩证思想。}

{{第二,辩证法要求我们用整体的、全面的观点看问题。}四个``全面''思想贯彻了唯物辩证法全面看问题的方法。}

{{第三,在唯物辩证法的方法论体系中,矛盾分析方法居于核心地位,是根本的认识方法。}
如要求人们做到``两点论''和``重点论''相结合等,\textbf{``四个全面''思想也是矛盾分析方法的具体体现。}总之,
``四个全面''战略思想是唯物辩证法思中反映和深刻展现。}
