马克思主义从产生到发展,表现出了强大的生命力,\textbf{{这种强大生命力}}的{根源}在于它的{\textbf{以实践为基础的科学性与革命性的统一}}。这种实践基础上的科学性与革命性的统一,是马克思主义基本的和{\textbf{最鲜明的特征}}。

{马克思主义具有{科学性},它是对客观世界特别是人类社会{本质和规律的正确反映}。其科学性表现在,坚持世界的物质性和真理的客观性,力求按照世界的本来面目如实地认识世界,力求全面地认识事物,并透过现象而深刻地揭示事物的本质和规律,自觉接受实践的检验,并在实践中不断丰富和发展。}

{马克思主义具有{革命性},它是无产阶级和广大人民群众推翻旧世界、建设新世界的理论。它的革命性表现在坚持唯物辩证法,具有彻底的批判精神。它具有鲜明的政治立场,毫不隐讳自己的阶级本质。}

{\textbf{{辩证唯物主义与历史唯物主义}}是马克思主义{最根本的世界观和方法论}。}

{马克思主义政党的一切理论和奋斗都应致力于实现\textbf{{以劳动人民为主体的最广大人民的根本利益}},这是马克思主义{最鲜明的政治立场}。}

{坚持一切从实际出发,理论联系实际,实事求是,在实践中检验真理和发展真理,是马克思主义最重要的理论品质。}

{实现物质财富极大丰富、人民精神境界极大提高、每个人自由而全面发展的\textbf{{共产主义社会}},是马克思主义{最崇高的社会理想。}}
