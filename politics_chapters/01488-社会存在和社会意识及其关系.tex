\textbf{{社会存在}:}社会存在属于社会生活的物质方面,是社会实践和物质生活条件的总和,包括\textbf{{物质生活资料的生产}}以及\textbf{{生产方式}}、\textbf{{地理环境}}和\textbf{{人口因素}}。

{\textbf{社会意识}}\textbf{:}社会生活的精神方面,它既包括社会意识的各种形式,也包括社会心理与自发形成的风俗、习惯。\textbf{{属于上层建筑的社会意识形式称为社会意识形态}},主要包括政治法律思想、道德、艺术、宗教、哲学等。

社会意识具有\textbf{{相对独立性}}主要表现在:社会意识和社会存在发展的\textbf{{不平衡性}};社会意识的\textbf{{历史继承性}};社会意识内部各种形式的\textbf{{相互作用性}};\textbf{{社会意识对社会存在的反作用性}}。

{社会存在决定社会意识,社会意识是社会存在的反映,并反作用于社会存在。}
