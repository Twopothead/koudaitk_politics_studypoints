毛泽东根据中国社会的性质和新民主主义革命的特点,认为\textbf{{中国革命必须分两步走}}:\textbf{{第一步是新民主主义革命}},反对帝国主义、封建主义和官僚资本主义,改变中国半殖民地半封建社会形态,使中国成为一个独立的新民主主义的国家;\textbf{{第二步是社会主义革命}}。即在新民主主义国家的基础上,使革命继续向前发展,建立社会主义制度,使中国成为一个社会主义国家。\textbf{{新民主主义革命与社会主义革命又是互相联系、紧密衔接的,中间不容横插一个资产阶级专政}}。民主主义革命是社会主义革命的必要准备,社会主义革命是民主主义革命的必然趋势。

{中共党内在革命前途问题上曾经有过}\textbf{{两种错误倾向}}{:一是陈独秀的``}\textbf{{二次革命论}}{'',把中国革命过程中两个紧密联系的阶段割裂开来,只看到两者之间的区别,没有看到两者之间的联系,要在两个阶段之间硬插一个资产阶级专政的和发展资本主义的阶段;二是以王明为代表的}\textbf{{``左''倾教条主义}}{,主张把民主革命和社会主义革命``毕其功于一役''的``一次革命论'',混淆了民主革命和社会主义革命的界限,企图把两种不同性质的革命阶段并作一步走,一举取得社会主义革命的胜利。这种观点只看到两者之间的联系,而忽视了两者之间的区别。}
