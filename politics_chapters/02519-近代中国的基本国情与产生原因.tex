{认清国情,是认清和解决革命问题的基本依据}。近代中国,已经沦为一个\textbf{{半殖民地半封建性质}}的社会,这是\textbf{{最基本的国情}}。

鸦片战争以后,随着外国资本---帝国主义的入侵,中国社会发生了\textbf{{两个根本性的变化}}:

1.
中国由一个领土完整、主权独立的国家沦为表面上独立、实际上受帝国主义列强共同支配的\textbf{{半殖民地国家}}。

2.
中国由一个完全的封建社会变为有了一定程度资本主义成分的\textbf{{半封建社会}}。

\textbf{{中国逐步变成半殖民地的原因}}:

1.
鸦片战争以后,西方列强通过发动侵略战争,强迫中国签订一系列不平等条约,一步一步地控制中国的政治、经济、外交和军事。中国已经丧失了完全独立的地位,在\textbf{{相当程度上被殖民地化}}了。

2.
西方列强侵略中国的目的,是要把它变成自己的殖民地。但是由于中国长期以来一直是一个统一的大国,特别是\textbf{{中国人民顽强、持久的反抗}},同时也由于\textbf{{帝国主义列强间争夺中国的矛盾无法协调}},使得它们中的任何一个国家都无法单独征服中国,也使得它们不可能共同瓜分中国。因此,近代中国尽管在实际上已经丧失拥有完整主权的独立国的地位,但是仍然维持着独立国家和政府的名义,还有一定的主权,\textbf{{因此被称作半殖民地}}。

\textbf{{中国逐步变成半封建社会的原因}}{。}

{1.
外国资本主义列强用武力打开中国的门户,把中国卷入世界资本主义经济体系和世界市场之中。随着外国资本主义的入侵,一方面,}{破坏了中}{国自给自足的自然经济的基础,破坏了城市的手工业和农民的家庭手工业;另一方面,促进了中国城乡商品经济的发展,给中国资本主义的产生提供了某些客观条件。}{破产的农民和手工业者成为产业工人的后备军。一批官僚、买办、地主、商人投资兴办新式工业。}\textbf{{中国出现了资本主义生产关系}}{。}

{2.
}\textbf{{西方列强并不愿意中国成为独立的资本主义国家}}{。中国的民族资本主义经济虽然有了某些发展,但是并没有也不可能成为中国社会经济的主要形式。在中国农村中,地主剥削农民的封建生产关系,在社会经济生活中依然占着明显的优势。这样,中国的经济就成为半殖民地半封建的经济了。}
