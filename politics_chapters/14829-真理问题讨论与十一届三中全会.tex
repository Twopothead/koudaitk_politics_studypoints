{
关于真理标准问题的讨论的背景:当时主持中共中央工作的华国锋坚持``两个凡是''的错误方针。党和国家的工作处于在徘徊中前进的状态。}

{\textbf{{关于真理标准问题的讨论的意义:}{①}{真理标准问题的讨论是继五四运动和延安整风运动之后又一场马克思主义思想解放运动。}{②}{为党重新确立实事求是思想路线,纠正长期以来的}{``}{左}{''}{倾错误,实现历史性的转折做了思想理论准备}}{。}}

{\textbf{{中共十一届三中全会的内容}}{:}{①全会冲破长期``左''的错误的严重束缚,彻底否定了``两个凡是''的错误方针,高度评价了关于真理标准问题的讨论。②断然否定``以阶级斗争为纲''的指导思想,做出了把工作重点转移到社会主义现代化建设上来和实行改革开放的战略决策。③恢复了党的民主集中制的优良传统,审查解决了历史上遗留的一批重大问题和一些重要领导人的功过是非问题}{。}}

{\textbf{{中共十一届三中全会的意义}}{:}\textbf{{①中共十一届三中全会是新中国成立以来党的历史上具有深远意义的伟大转折。②重新确立了马克思主义的思想路线、政治路线和组织路线,会议形成了以邓小平为核心的党的中央领导集体。③揭开了社会主义改革开放的序幕,中国开始进入了改革开放和社会主义现代化建设的历史新时期。}}}
