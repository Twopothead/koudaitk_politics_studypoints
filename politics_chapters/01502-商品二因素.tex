\textbf{{商品是用来交换的能满足人们某种需要的劳动产品}},具有\textbf{{使用价值和价值}}两个因素,是\textbf{{使用价值和价值的矛盾统一体}}。

\textbf{{使用价值}}{是指}\textbf{{商品能满足人们某种需要的属性}}{,即}\textbf{{商品的有用性}}{,反映}\textbf{{人与自然之间的物质关系}}{,是商品的}\textbf{{自然属性}}{,}\textbf{{是一切劳动产品共有的属性}}{。}\textbf{{使用价值构成社会财富的物质内容}}{。是商品价值和交换价值的}\textbf{{物质承担者}}{。}

\textbf{{价值}}是凝结在商品中的\textbf{{无差别的一般人类劳动}},即\textbf{{人类脑力和体力的耗费}}。价值是商品特有的\textbf{{社会属性}}。商品的价值实体是凝结在商品中的无差别的人类劳动,它\textbf{{本质上体现生产者之间的一定社会关系}}。\textbf{{价值是交换价值的基础和内容。}}

\textbf{{交换价值}}{:是一种使用价值和另一种使用价值相交换的量的关系或比例。}\textbf{{交换价值是价值的表现形式}}{。}

\textbf{{任何商品都是使用价值和价值的统一体}}{。而同时,}\textbf{{对买卖双方当事人来说,使用价值和价值不能兼得}}{。}{使用价值和价值的这种矛盾对立,只有通过交换才能得到解决}{。}
