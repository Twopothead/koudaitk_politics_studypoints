{\textbf{马克思主义基本原理}},是马克思主义理论体系中\textbf{{最基本、最核心}}的内容,是对马克思主义的\textbf{{立场、观点和方法}}的集中概括。它体现马克思主义的{根本性质}和{整体特征},体现马克思主义{{科学性和革命性的统一}}。

相对于特定历史条件下所作的个别理论判断和具体结论,马克思主义基本原理具有普遍的、根本的和长远的指导意义。可以从基本立场、基本观点和基本方法三个方面把握马克思主义的基本原理。

{马克思主义基本立场},是马克思主义\textbf{{观察.分析和解决}}问题的根本{立足点}和{出发点}。这就是始终站在人民大众的立场上,一切为了人民,一切相信人民,一切依靠人民,全心全意为人民谋利益。

{马克思主义的基本观点,是关于自然、社会和人类思维规律的科学认识,是对人类思想成果和社会实践经验的科学总结}。

{\textbf{这些基本观点主要包括}}:关于客观世界的本质和规律的观点,关于人的实践和认识活动的本质和规律的观点,关于社会形态和社会基本矛盾运动规律的观点,关于人民群众的历史主体作用的观点,关于商品经济和社会化大生产与一般规律的观点,关于劳动价值论、剩余价值和资本主义生产方式本质的观点,关于社会主义必然代替资本主义的观点,关于社会主义革命和无产阶级专政的观点,关于无产阶级政党建设的观点,关于社会主义本质特征和建设规律的观点,关于共产主义社会基本特征的观点,等等。

{\textbf{马克思主义的基本方法}},是建立在辩证唯物主义和历史唯物主义世界观.方法论基础上的思想方法和工作方法,主要包括实事求是的方法、辩证分析的方法、历史分析的方法、群众路线的方法等等。
