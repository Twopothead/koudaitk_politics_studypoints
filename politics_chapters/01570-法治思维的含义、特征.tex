\textbf{{\textbf{含义:}法治思维方式是指人们按照法治的理念、原则和标准判断、分析和处理问题的理性思维方式{。}\\
}}

{\textbf{{1、法律至上。}}}{}{法律至上尤其指宪法至上,因为宪法具有最高的法律效力。法律至上具体表现为法律的}{\textbf{{普遍适用性、优先适用性和不可违抗}}{性}}{。}{法律的普遍适用性}{,是}{指法律在本国主权范围内对所有人具有普遍的约束力。}{法律的优先适用}{性}{,}{是指同一项社会关系同时受到多种社会规范的调整时,要优先考虑法律规范的适用}{。}{法律的不可违抗性}{,}{是指法律必须遵守,违反法律要受到惩罚}{。}

{\textbf{{2、权力制约。}}}{}{是指国家机关的权力必须受到法律的规制和约束,也就是要把权力关进制度的笼子里。}

\textbf{{3、公平正义。}}{公平正义是指社会的政治利益、经济利益和其他利益在全体社会成员之间合理、平等分配和占有。公平正义包括:权利公平、机会公平、规则公平和救济公平。}

{\textbf{4、人权保障。}}{人权的法律保障包括:宪法保障、立法保障、行政保护和司法保障。}

\textbf{{5、正当程序。}}{程序的正当,表现在程序的合法性、中立性、参与性、公开性、时限性等方面。}
