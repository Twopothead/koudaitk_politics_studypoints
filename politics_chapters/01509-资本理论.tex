预付资本在生产中以生产资料和劳动力这两部分存在。根据这两部分资本在生产剩余价值中的作用不同,把它们区别为\textbf{不变资本和可变资本}。

\textbf{不变资本}:以生产资料形式存在的资本,在生产过程中只是把价值转移到新产品中去,并不改变自己的价值量。

\textbf{可变资本}:以劳动力形式存在的资本,它的价值在生产过程中不会转移到新产品中,而要由工人劳动再创造出来,而且还要\textbf{生产出剩余价值},改变了原来的价值量。

{区分不变资本和可变资本的意义:第一,提示了}\textbf{剩余价值的源泉}{和}\textbf{资本主义剥削的实质}{。第二,}\textbf{为确定资本家对工人的剥削程度}{,提供了科学依据。体现资本家对工人剥削程度的概念是}\textbf{剩余价值率}{。}
