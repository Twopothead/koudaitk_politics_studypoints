{\textbf{阶级斗争}}是阶级社会发展的\textbf{{直接动力}}(重要动力)

{\textbf{革命}}的根本问题是国家政权问题,\textbf{{重要动力}}。

{\textbf{改革}}是推动社会发展的又一\textbf{{重要动力}}。

{\textbf{科学技术革命}}是社会动力体系中的一种重要动力:

\textbf{首先},对生产方式产生了深刻影响。一是\textbf{{改变了社会生产力的构成要素}},二是\textbf{改变了人们的劳动形式},三是{\textbf{改变了社会经济结构}},特别是导致了产业结构发生变革。

\textbf{其次},{\textbf{对生活方式产生了巨大影响}}。现代科学技术革命直接或间接地作用于人们生活方式的四个基本要素,即生活主体、生活资料、生活时间和生活空间,从而引起生活方式发生新的变革。

\textbf{最后},{\textbf{促进了思维方式的变革}}。
