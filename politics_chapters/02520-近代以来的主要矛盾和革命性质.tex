{近代中国社会是半殖民地半封建社会},这是一个特殊的过渡性社会。近代中国社会的性质,即是中国特殊的国情。\textbf{{近代中国半殖民地半封建的社会性质,决定了社会主要矛盾是帝国主义和中华民族的矛盾、封建主义和人民大众的矛盾}}。\textbf{{而帝国主义和中华民族的矛盾,又是各种矛盾中最主要的矛盾}}。

近代中国社会的性质和\textbf{{主要矛盾}},\textbf{{决定}}了近代中国革命的\textbf{{根本任务是推翻帝国主义、封建主义和官僚资本主义的统治}},从根本上推翻反动腐朽的政治上层建筑,变革阻碍生产力发展的生产关系,为建设富强民主的国家、改善人民的生活、确立人民当家作主的政治地位扫清障碍,创造必要的前提。

\textbf{{俄国革命的胜利}},改变了整个世界历史的方向,划分了整个世界历史的时代,\textbf{{开辟了世界无产阶级社会主义革命的新纪元,标志着人类历史开始了由资本主义向社会主义转变的进程}}。这使\textbf{{中国的资产阶级民主主义革命}},从\textbf{{原来属于旧的世界资产阶级民主主义革命}}的范畴,属于旧的世界资产阶级民主主义革命的一部分,\textbf{{转变为}}属于新的资产阶级民主主义革命的范畴,\textbf{{属于世界无产阶级社会主义革命的一部分}}。

近代中国革命\textbf{{以五四运动为开端}},进入新民主主义革命阶段。

\textbf{{革命的性质是由社会性质和革命任务等因素决定的}}。\textbf{{中国革命的性质,就不是无产阶级社会主义革命,而是资产阶级民主主义革命}}。

{新民主主义革命与旧民主主义革命相比有其}\textbf{{新的内容和特点}}{,集中表现在:1.
中国新民主主义革命处于世界无产阶级社会主义革命的}\textbf{{时代条件}}{,是世界无产阶级社会主义革命的一部分;2.
革命的}\textbf{{领导力量}}{是中国无产阶级及其先锋队------中国共产党;(}\textbf{{根本标志}}{)③革命的}\textbf{{指导思想}}{是马克思列宁主义;3.
革命的}\textbf{{前途}}{不再是资本主义,而是经过新民主主义实现社会主义。}
