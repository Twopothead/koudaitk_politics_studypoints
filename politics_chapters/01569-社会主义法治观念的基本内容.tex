\textbf{\textbf{}{坚持走中国特色社会主义}}

{党的领导是中国特色社会主义最本质的特征,是社会主义法治最根本的保证。}

{中国特色社会主义制度是中国特色社会主义法治体系的根本制度基础,是全面依法治国的根本制度保障。}

{中国特色社会主义法治理论是中国特色社会主义法治体系的理论指导和学理支撑,是全面依法治国的行动指南。}

\textbf{{坚持党的领导、人民当家作主与依法治国相统一}}

{党的领导是人民当家作主和依法治国}{的根本保证;}

{人民当家作主是党的领导和依法治国的本质要求;}

{依法治国是党领导人民当家作主的治国方略。}

\textbf{{坚持依法治国和以德治国相结合}}

{正确认识法治和德治的地位;}

{正确认识法治和德治的作用;}

{正确认识法治和德治的实现途径。}

\textbf{{加强宪法实施,落实依宪治国}}

{深刻认识宪法实施和依宪治国的重大意义;}

{全面实施宪法的基本要求:要在全社会树立宪法意识,弘扬宪法精神;要加强宪法实施;要加强你宪法实施,要坚持党的依宪执政,自觉在宪法法律范围内活动;}

{准确把握宪法实施的正确方向。}
