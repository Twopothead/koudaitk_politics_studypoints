{社会性质}{:鸦片战争以后,随着外国资本---帝国主义的入侵,中国逐渐沦为}{半殖民地半封建社会}{,这是}{近代中国最基本的国情}{。}{}

{主要矛盾}{:}{帝国主义和中华民族的矛盾,封建主义和人民大众的矛盾是近代以来的主要矛盾。其中最主要的矛盾是帝国主义和中华民族的矛盾}{。这两对主要矛盾及其斗争贯穿整个中国半殖民地半封建社会的始终,并对中国近代社会的发展变化起着决定性的作用。~}{}

{历史任务}{:近代以来中华民族面临的两大历史任务是:}{争取民族独立、人民解放}{;}{实现国家富强、人民富裕}{。两者的关系:前一个任务为后一个任务扫除障碍,创造必要的前提;后一个任务是前一个任务的最终目的和必然要求。}
