必然性和偶然性是揭示客观事物发生、发展和灭亡的不同趋势的范畴。

{{\textbf{必然性}}:指事物联系和发展过程中一定要发生、确定不移的趋势。}

\textbf{{偶然性}}:指事物联系和发展过程中并非确定发生的,可以出现,也可以不出现,可以这样出现,也可以那样出现的不确定的趋势。

{{关系}:必然性产生于事物内部的根本矛盾,偶然性产生于非根本矛盾和外部条件}
;必然性存在于偶然性之中,通过大量的偶然性表现出来,并为自己开辟道路;偶然性背后隐藏着必然性,受必然性的支配,偶然\textbf{}性是必然性的表现形式和补充。\textbf{{{必然是偶然的,偶然是必然的}{。}}}
