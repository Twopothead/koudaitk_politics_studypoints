\textbf{{历史观}}是人们在认识社会历史现象、解决社会问题时所采取的根本观点。

\textbf{{社会存在与社会意识的关系问题,是社会历史观的基本问题}}。

在对待社会历史发展及其规律问题上,存在着两种根本对立的历史观:一种是\textbf{{唯物史观}},另一种是\textbf{{唯心史观}}。

在马克思主义产生之前,唯心史观一直占据统治地位,它的主要缺陷是:一是至多考察人们活动的思想动机,而\textbf{{没有进一步追究思想动机背后的物质动因}};二是只看到个人在历史上的作用,而\textbf{{忽视人民群众创造历史的决定作用}}。

马克思正确地解决了社会存在与社会意识的关系问题,发现了人类社会发展的客观规律,创立了唯物史观。

\textbf{{唯物史观认为}{:社会历史发展是有规律的而不是无序的,社会存在决定社会意识,人民群众是历史的创造者。}}
