{空想社会主义是科学社会主义的}\textbf{{{理论先驱}}}{。19世纪初,以{{\textbf{圣西门、傅立叶、欧文}}}为代表的空想社会主义是科学社会主义的}\textbf{{{直接思想来源}}}{。}

空想社会主义\textbf{{对资本主义制度进行批判}};对{\textbf{社会主义制度进行了天才描绘}}。

{空想社会主义的局限性:认为资本主义必然灭亡,却}\textbf{{不能揭示其必然灭亡的经济根源}}{;要求埋葬资本主义,却}\textbf{{找不到埋葬资本主义的社会力量}}{;主张建立社会主义理想社会,却}\textbf{{找不到通往理想社会的现实道路}}{。}
