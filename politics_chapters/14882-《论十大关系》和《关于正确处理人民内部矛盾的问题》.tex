{\textbf{{《论十大关系》背景}}:以苏为鉴,走中国自己的社会主义建设道路。}

{\textbf{{《论十大关系》主要内容}}{:毛泽东在1956年作了《论十大关系》的报告,概括提出了十大关系。这十大关系围绕}\textbf{{一个基本方针}}{,即}\textbf{{调动国内外一切积极因素,为社会主义服务}}{。}}

{\textbf{{意义}}{:}}

{\textbf{{1.~}{这是以毛泽东为主要代表的中国共产党人开始探索中国自己的社会主义建设道路的标志。}}{\\
}}

{2.~它在新的历史条件下从经济方面(这是主要的)和政治方面提出了新的指导方针,为中共八大的召开作了理论准备。}

{\textbf{{《关于正确处理人民内部矛盾的问题》的主要内容}}:}

{1. 关于社会主义社会两类不同性质的社会矛盾。}

{2. 关于社会主义社会的基本矛盾。~}

{\textbf{{《关于正确处理人民内部矛盾的问题》的意义}}:}

{1. 这是一篇重要的马克思主义文献。}

{2.
它创造性地阐述了社会主义社会矛盾学说,是对科学社会主义理论的重要发展。}

{3. 对中国社会主义事业具有长远的指导意义。}

{\textbf{{第二次结合}:}{随着苏共二十大对于苏联模式弊端的进一步披露,中国共产党人决心走自己的路,}{开始探索适合中国国情的社会主义建设道路}{。}{毛泽东提出的关于实行马克思主义同中国实际的``第二次结合''的任务}{,为探索适合中国情况的社会主义建设道路,提供了基本的指导原则。}}
