\textbf{{经济基础}:}指由社会一定发展阶段的生产力所决定的生产关系的总和。

\textbf{{上层建筑}:}指建立在一定经济基础之上的意识形态以及相应的制度、组织和设施。

上层建筑由意识形态和政治法律制度及设施、政治组织等两部分构成。\textbf{{意识形态又称观念上层建筑}},包括政治{法律思想、道德、艺术、宗教、哲学}等思想观点,政治法律制度及设施和政治组织又称政治上层建筑,包括:{国家政治制度、立法司法制度和行政制度;国家政权机构、政党、军队、警察、法庭、监狱等政治组织形态和设施}。

政治上层建筑是在一定意识形态指导下建立起来的,是统治阶级意志的体现;政治上层建筑一旦形成,就成为一种现实的力量,影响并制约着人们的思想理论观点。\textbf{{在整个上层建筑中,政治上层建筑居主导地位,国家政权是它的核心。}}

\textbf{{经济基础决定上层建筑,上层建筑反作用于经济基础}}{。}
