\textbf{我国社会的性质是新民主主义社会}。新民主主义\textbf{五种经济成分}中,\textbf{主要的是三种},即\textbf{社会主义经济、个体经济和私人资本主义经济}。通过没收官僚资本而形成的社会主义国营经济,掌握了主要经济命脉,居于\textbf{领导地位}。而以农业和手工业为主体的个体经济,则在国民经济中占\textbf{绝对优势}。工人阶级、农民阶级和其他小资产阶级、民族资产阶级等是新民主主义社会基本的阶级力量。三种基本的经济成分及与之相应的三种基本的阶级力量(工人阶级、农民及其他小资产阶级、资产阶级)之间的矛盾,就\textbf{{集中地表现为无产阶级与资产阶级的矛盾、社会主义与资本主义的矛盾}}{。}{}

\textbf{中共七届二中全会}决议分析了新民主主义社会的经济状况和基本矛盾,提出中国从农业国转变为工业国并解决了土地问题以后,中国还存在着\textbf{两种基本的矛盾}:\textbf{国际上是新中国同帝国主义的矛盾,国内是工人阶级和资产阶级的矛盾}。

\textbf{{阶级构成}:工人阶级、农民阶级和其他小资产阶级、民族资产阶级等是新民主主义社会基本的阶级力量}{。}由于农民和手工业者的个体经济既可以自发地走向资本主义,也可以被引导走向社会主义,其本身并不代表一种独立的发展方向{。随着土地改革的基本完成,}工人阶级和资产阶级的矛盾逐步成为国内的主要矛盾。而解决这一矛盾,必然使中国社会实现向社会主义的转变{。}这一时期的民族资产阶级仍然是一个具有两面性的阶级:既有剥削工人的一面,又有接受工人阶级及其政党领导的一面{。{因此},}民族资产阶级与工人阶级的矛盾也具有两重性,既有剥削者与被剥削者的阶级利益相互对立的对抗性的一面,又有相互合作、具有相同利益的非对抗性的一面{。对于工人阶级和社会主义革命来说,民族资产阶级作为一个剥削阶级是被消灭的对象,作为可以接受共产党和工人阶级领导的社会力量,又是团结和改造的对象。}
