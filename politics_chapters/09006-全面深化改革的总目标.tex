{ }

全面深化改革的总目标:{\textbf{完善和发展中国特色社会主义制度,推进国家治理体系和治理能力现代化。}}

{总目标所包含的两个方面是一个整体:前者规定了改革的根本方向,就是中国特色社会主义道路,而不是其他什么道路;后者规定了在根本方向指引下完善和发展中国特色社会主义制度的鲜明指向。国家治理体系和治理能力是一个国家的制度和制度执行能力的集中体现。推进国家治理体系和治理能力现代化,是完善和发展中国特色社会主义制度的必然要求,是实现社会主义现代化的题中应有之义。}



{{国家治理体系}是在党领导下管理国家的制度体系,包括经济、政治、文化、社会、生态文明和党的建设等各领域体制机制、法律法规安排,也就是一整套紧密相连、相互协调的国家制度。~}

{{国家治理能力}则是运用国家制度管理社会各方面事务的能力,包括改革发展稳定、内政外交国防、治党治国治军等各个方面。~}

{{国家治理体系和治理能力}是一个有机整体,相辅相成,有了好的国家治理体系才能提高治理能力,提高了国家治理能力才能充分发挥国家治理体系的效能。推进国家治理体系和治理能力现代化必须完整理解和把握全面深化改革的总目标。完善和发展中国特色社会主义制度,推进国家治理体系和治理能力现代化,是两句话组成的一个整体,\textbf{{前一句规定了根本方向,后一句规定了所走路径}},我们是在中国特色社会主义道路这个方向上推进国家治理体系和治理能力现代化。}
