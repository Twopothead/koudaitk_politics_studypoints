\textbf{{人民群众}}:人民群众是一个历史范畴,是指一切对社会历史起推动作用的人。在不同的历史时期,人民群众有着不同的作用,{主体部分始终是劳动群众及其知识分子}。

{人民群众的作用:在社会历史发展过程中,人民群众起着}\textbf{{决定性作用}}{,人民群众是}\textbf{{历史的主体}}{,是历史的}\textbf{{创造者}}{。主要表现:人民群众是社会}\textbf{{物质财富的创造者}}{,人民群众是}\textbf{{精神财富的创造者}}{,}\textbf{{人民群众是人类社会变革的决定力量}}{。}

{我党的}\textbf{{群众路线}}{:坚持一切为了群众,一切依靠群众,从群众中来,到群众中去的路线。}

{历史人物,特别是杰出人物在社会发展过程中起着特殊的作用,主要表现在历史人物是历史事件的发起者、当事者,历史人物是实现一定历史任务的组织者、领导者;历史人物是历史进程的影响者,它可以加速或延缓历史任务的解决。
历史人物对历史发展的具体过程始终起着一定的作用,有时甚至对历史事件的进程和结局发生决定性的影响,但}\textbf{{不能决定历史发展的基本趋势}}{。}

{所以,社会历史发展是}\textbf{{无数个人合力作用的结果}}{。历史人物在历史发展中起着特殊的作用。历史人物的产生是}\textbf{{必然性和偶然性的统一}}{。坚持时势造英雄的观点,既要承认杰出人物的历史作用,又要反对个人崇拜。}
