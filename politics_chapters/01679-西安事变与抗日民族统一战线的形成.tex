{1936}年{12}月{12}日,张学良、杨虎城发动了西安事变。西安事变的和平解决的意义:1.
它成为了时局转换的枢纽,十年内战的局面结束了,国内和平基本实现。2.
这是中国共产党抗日民族统一战线策略和逼蒋抗日方针的重大胜利,它为国共两党的合作抗日创造了有利条件。{}

中共关于抗日民族统一战线政策的制定:毛泽东在瓦窑堡会议上系统地解决了党的政治路线上的问题。(1)阐明建立抗日民族统一战线的可能性。(2)批判了``左''倾关门主义错误。(3)规定了建立广泛的抗日民族统一战线的具体政策。

{1}{936}年{5}月,中共中央发布《停战议和一致抗日》通电,放弃了``反蒋抗日''的口号。后正式确定了``逼蒋抗日''的策略方针。国民党五届三中全会在会议文件上第一次写上了``抗日''的字样。\textbf{国民党中央通讯社发表《中共中央为公布国共合作宣言》;23日,蒋介石发表实际承认共产党合法地位的谈话。标志着以国共两党第二次合作为基础的抗日民族统一战线正式形成。}

\textbf{抗日民族统一战线的特点}{:}1. 广泛的民族性和复杂的阶级矛盾。2.
国共双方有政权有军队的合作。国民党领导全国政权和军队,共产党领导局部政权和军队。3.
没有正式的固定的组织形式和协商一致的具体的共同纲领{。}

抗日民族统一战线的巩固、发展和壮大,是夺取抗日战争最后胜利的根本保证{。}
