\textbf{{马克思主义中国化}}的提出:1938年,毛泽东在党的\textbf{{六届六中全会}}上作了题为\textbf{{《论新阶段》}}的政治报告。这是在全党范围内,\textbf{{最早明确提出}}``马克思主义中国化''的命题。经过\textbf{{延安整风}},马克思主义中国化的思想成为\textbf{{全党的共识}}。1945年,\textbf{{刘少奇}}代表党中央在党的\textbf{{七大}}上作的关于修改党章的报告。中共七大通过的党章指出,{毛泽东思想是马克思主义中国化的第一个重大理论成果,是}{``}{中国化的马克思主义''}。

\textbf{{马克思主义中国化的科学内涵}}:{马克思主义中国化,就是将马克思主义基本原理与中国具体实际相结合}。具体地说,就是把马克思主义的基本原理同中国革命、建设和改革的实践结合起来,同中国的优秀历史传统和优秀文化结合起来,既坚持马克思主义,又发展马克思主义。

\textbf{{马克思主义中国化就是}}{:}{运用马克思主义指导中国革命、建设和改革的实践}{;}{就是把中国革命、建设和改革的实践经验和历史经验上升为马克思主义理论}{;}{就是把马克思主义植根于中国的优秀文化之中}{。}
