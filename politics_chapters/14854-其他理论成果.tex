{\textbf{{走中国工业化道路的思想}}:毛泽东在《论十大关系》中论述的第一大关系,便是重工业、轻工业和农业的关系。在《关于正确处理人民内部矛盾的问题》一文中,毛泽东明确提出要走一条有别于苏联的中国工业化道路。~{毛泽东提出了以农业为基础,以工业为主导,以农轻重为序发展国民经济的总方针}。}

{关于\textbf{{社会主义发展阶段}}。毛泽东提出,社{会主义又可分为两个阶段,第一个阶段是不发达的社会主义,第二个阶段是比较发达的社会主义。后一个阶段可能比前一阶段需要更长的时间。}}

{关于\textbf{{社会主义现代化建设的战略目标和步骤}}:毛泽东提出,社会主义现代化的战略目标,是要把中国建设成为一个具有{现代农业、现代工业、现代国防和现代科学技术的强国}。为了实现这个目标,应当采取``{两步走}''的发展战略,{第一步建成一个独立的比较完整的工业体系和国民经济体系,第二步全面实现工业、农业、国防和科学技术现代化,使中国走在世界前列}。}

{关于\textbf{{经济建设方针}}:\textbf{{党的八大提出了既反保守又反冒进、在综合平衡中稳步前进的方针。毛泽东多次阐述了统筹兼顾的方针}},强调正确处理国家、集体与个人的关系,生产两大部类的关系,中央与地方的关系,积累与消费的关系,长远利益与当前利益的关系;既要顾全大局,突出重点,也要统筹兼顾,全面安排,综合平衡。同时,也要在自力更生的基础上积极争取外援,开展与外国的经济交流。}

{\textbf{{关于所有制结构的调整}}:{朱德提出了要注意发展手工业和农业多种经营的思想}。\textbf{{陈云提出了``}{三个主体,三个补充''}{的设想}},即{在工商业经营方面,国家经济和集体经济是工商业的主体,一定数量的个体经济是国家经济和集体经济的补充;在生产计划方面,计划生产是工农业生产的主体,按照市场变化在国家计划许可范围内的自由生产是计划生产的补充;在社会主义的统一市场里,国家市场是它的主体,一定范围内的国家领导的自由市场是国家市场的补充}。}

{\textbf{{关于经济体制和运行机制改革}}:{毛泽东}提出了发展商品生产、利用价值规律的思想,认为商品生产在社会主义条件下,还是一个不可缺少的、有利的工具,要有计划地大大地发展社会主义的商品生产。{刘少奇}则提出了使社会主义经济既有计划性又有多样性和灵活性的主张,以及按经济办法管理经济的思想。{陈云}提出了要建立``适合于我国情况和人民需要的社会主义的市场''的思想。此外,{毛泽东还主张企业要建立合理的规章制度和严格的责任制},要{实行民主管理,实行干部参加劳动,工人参加管理,改革不合理的规章制度,工人群众、领导干部和技术人员三结合,即``两参一改三结合''}。{邓小平}提出了关于整顿工业企业,改善和加强企业管理,实行职工代表大会制等观点。}

{关于\textbf{{社会主义民主政治建设}}。{党的八大提出,要进一步扩大民主,健全法制}。毛泽东则进一步提出,``{我们的目标,是想造成一个又有集中又有民主,又有纪律又有自由,又有统一意志、又有个人心情舒畅、生动活泼,那样一种政治局面,以利于社会主义革命和社会主义建设}''。}

{关于\textbf{{科学和教育}}。党提出了``{向科学进军}''的口号,毛泽东强调,``我们的教育方针,应该使受教育者在德育、智育、体育几方面都得到发展,成为有社会主义觉悟的有文化的劳动者。''{刘少奇提出实行``两种劳动制度、两种教育制度'',一种是全日制的劳动制度,全日制的教育制度;一种是半日制劳动制,半日制的教育制度}。}

{{关于}\textbf{{知识分子工作}}{:毛泽东提出,知识分子在革命和建设中都具有重要作用,要建设一支宏大的工人阶级知识分子队伍。周恩来提出了}{知识分子是工人阶级一部分}{的观点,强调要加强和改善党对知识分子和科学文化工作的领导,更好地为社会主义服务。}}
