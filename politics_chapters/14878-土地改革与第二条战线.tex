{中共中央发出\textbf{{《关于清算、减租及土地问题的指示》}}(史称《\textbf{{五四指示}}》),\textbf{{标志着党把在抗日战争时期实行的减租减息政策改变为实现}{``}{耕者有其田}{''}{的政策}}。}

{\textbf{{《中国土地法大纲》:}{``}{废除封建性及半封建性的土地制度,实现耕者有其田的制度}{''}},毛泽东总结正反两方面的经验,提出了\textbf{{完整的土地改革的总路线}},即\textbf{{依靠贫雇农,团结中农,有步骤、有分别地消灭封建剥削制度,发展农业生产}}。同时强调土地改革必须注意的两个基本原则:\textbf{{第一,必须满足贫雇农的土地要求;第二,必须坚决地团结中农,不要损害中农的利益}}。}

{\textbf{{1.~}{土地制度改革,是从根本上摧毁中国封建制度根基的社会大变革。}{2.~}{经过土地改革运动,人民解放战争获得了源源不断的人力、物力的支援。3.~}{为打败蒋介石、建立新中国奠定了深厚的群众基础。}}}

{{第二条战线是在国民党统治区,以学生运动为先导的人民民主运动}。}

{昆明学生发动了以``反对内战,争取自由''为主要口号的一二·一运动。}

{为抗议驻华美军强暴北京大学先修班一女学生,抗议驻华美军暴行的运动(史称{抗暴运动}、``{一二·三〇运动}'')由此掀起。}

{南京、北平等地爆发了反饥饿、反内战运动(史称``{五二〇运动}'')。}

{第二条战线虽然是辅助性的,但仍然是十分重要的战线。}
