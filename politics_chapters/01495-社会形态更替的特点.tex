{社会形态更替归根结底是社会基本矛盾运动的结果};生产力的发展具有\textbf{{最终的决定意义}}。

第一,社会发展的客观必然性造成了一定阶级阶段社会发展的基本趋势,为人们的历史选择提供了{基础、范围和可能性空间}。第二,社会形态更替的过程也是一个{和目的性与合规律性相统一}的过程。第三,人们的历史选择性,归根到底是人民群众的选择性

社会形态更替具有\textbf{必然性}与人们的历史\textbf{选择性}:一个民族之所以做出这种或那种选择,有其特定的原因:{一是取决于民族利益。二是取决于交往。三是取决于对历史必然性以及本民族特点的把握程度。}人们的历史选择性,归根结底是人民群众的选择性。

社会形态更替具有{统一性和多样性};

{社会形态更替的}{前进性与曲折性、}{顺序性与跨越性的统一}{。}
