{{1906年12月,萍、浏、醴起义是同盟会成立后发动的第一次武装起义}。{辛亥革命前影响最大的,是广州起义}。\textbf{{武昌起义吹响了辛亥革命的号角}}。1912年1月1日,{孙中山在南京宣誓就任临时大总统,改国号为``中华民国''},定1912年为民国元年,并正式成立中华民国临时政府。\textbf{{南京临时政府是一个资产阶级共和国性质的革命政权}}。从政权的组成人员看,资产阶级革命派在这个政权中占有领导和主体的地位。从南京临时政府制定的政策看,各项政策措施集中代表和反映了中国民族资产阶级的愿望和利益,在相当程度上也符合广大中国人民的利益。}

{1912年3月,临时参议院颁布\textbf{{《中华民国临时约法》}}。\textbf{{这是中国历史上第一部具有资产阶级共和国宪法性质的法典}}。}

{\textbf{{辛亥革命是一次比较完全意义上的资产阶级民主革命。它成了}{20}{世纪中国第一次历史性巨变}}。}

{\textbf{{历史意义}}{:}\textbf{{辛亥革命是一次比较完全意义上的资产阶级民主革命}}{。}\textbf{{①沉重打击了中外反动势力。②结束了统治中国两千多年的封建君主专制制度,建立了中国历史上第一个资产阶级共和政府,使民主共和的观念开始深入人心。③给人们带来一次思想上的解放。④辛亥革命促使社会经济、思想习惯和社会风俗等方面发生了新的积极变化。⑤推动了亚洲各国民族解放运动的高涨。}}}
