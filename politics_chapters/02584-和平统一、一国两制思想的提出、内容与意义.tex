{台湾问题的}\textbf{{由来和实质}}{:台湾自古以来是中国领土不可分割的重要组成部分,台湾人民同大陆人民同根、同宗、同源,承继的是相同的文化传统。台湾问题是中国国内战争遗留下来的问题,}\textbf{{台湾问题实质是中国的内政问题}}{。}

解决台湾问题的方针由\textbf{{武力解放}}到\textbf{{和平解放}}再到\textbf{{和平统一}}。

\textbf{{``和平统一、一国两制''构想的基本内容}}:第一,\textbf{{一个中国。这是"}{和平统一、一国两制"}{的核心}},是发展两岸关系和实现和平统一的\textbf{{基础}}。第二,两制并存。第三,高度自治。第四,尽最大努力争取和平统一,但不承诺放弃使用武力。第五,解决台湾问题,实现祖国的完全统一,寄希望于台湾人民。

\textbf{{``和平统一、一国两制''的意义}}{:}{``和平统一、一国两制''构想创造性地发展了马克思主义的国家学说}{;}{``和平统一、一国两制''构想有利于争取社会主义现代化建设事业所需要的和平的国际环境与国内环境}{;}{``和平统一、一国两制''构想为解决\textbf{{国际争端}}和历史遗留问题提供了新的思路}{。}
