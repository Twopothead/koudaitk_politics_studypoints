{{一、}{\textbf{两极格局解体~}}
\textbf{{世界经济格局}}{:}
{第一阶段:从战后初期到20世纪60年代末,主要表现为美国称霸世界经济领域。}
{第二阶段:20世纪70年代后世界经济向多极化方向发展。}
{第三阶段:自20世纪80年代末期开始,三大区域经济集团化加快发展。~}
}

{世界政治格局:雅尔塔体制的形成、冷战的开始、两极政治格局的动摇、两极格局的终结。}

{{}
{二、}{\textbf{世界多极化}}
{\textbf{世界多极化长期性的原因}}{:}
{第一,美国的霸权主义和构建单极世界的图谋,是多极化趋势发展的最大障碍。}
{第二,世界上冷战思维的继续、南北贫富差距的扩大,以及民族分裂和宗教纠纷等,也会对多极化趋势产生各种干扰和冲击。}
{第三,多极化格局的形成是世界各种力量重新组合和利益重新分配的过程,由此将产生多种不确定因素,世界多极化进程将充满矛盾和斗争。}

{三、}{\textbf{经济全球化}}
{表现:}
{第一,国际贸易已成为世界经济发展中不可缺少的组成部分,是国际交往中最活跃的一环。}
{第二,国际投资,特别是发达国家间的相互投资越来越频繁,资本流动已经国际化。}
{第三,国际金融活动规模空前,大大超过了全世界生产和商品交易。}
{第四,跨国公司遍布全球,产品的国际化水平越来越高。}
{第五,全球贸易规则日趋统一。}
{经济全球化有利于促成各国之间生产要素的合理流动,形成优势互补,推动世界经济的发展。经济全球化对各国的作用是不一样的。}
{\textbf{对于发展中国家来说,它既是机遇,又是挑战}}
{。发展中国家既要适应经济全球化趋势,又要趋利避害。当今世界需要的是各国``共赢''、平等、公平、共存的经济全球化。~}
{经济全球化与区域经济集团化的关系}{:两者是并行不悖(对立统一)的。}



{四、}{\textbf{区域经济一体化~}}
{四个阶段:第一,贸易一体化;第二,要素一体化;第三,政策一体化;第四,完全一体化。}
{类型:第一,优惠贸易安排。第二,自由贸易区。第三,关税同盟。第四,共同市场。第五,经济同盟。第六,完全经济一体化。}
{主要的组织:欧洲联盟、北美自由贸易区、亚太经济合作组织等。}
\textbf{2014年11月,亚太经合组织第22次领导人非正式会议在北京举行,主题是``共建面向未来的亚太伙伴关系''
这是自2001年后, 中国再一次成为APEC
的东道主。习近平指出,中国将以此次会议为契机,面向未来,谋求建主更紧密的伙伴关系,深化务实合作,推动亚太经合组织发挥更大引领作用,勾画亚太长远发展愿景。)~~}}