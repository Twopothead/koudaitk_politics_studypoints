资本家凭借对生产资料的占有,在等价交换原则的掩盖下雇佣工人从事劳动,占有雇佣工人的剩余价值,这就是\textbf{{资本主义所有制的实质}}。

在资本主义社会,工人在市场上卖给资本家的是劳动力,而不是劳动。

在资本主义制度下,\textbf{工资的本质是劳动力的价值或价格}。但是,在资本主义经济的现象中,工资却表现为劳动的价格,因此,资本主义工资掩盖了剥削。

资本主义工资包括计时工资和计件工资两种主要形式。

只要资本和雇佣劳动的基本经济关系不变,资本主义工资的本质就不会发生根本变化{。}
