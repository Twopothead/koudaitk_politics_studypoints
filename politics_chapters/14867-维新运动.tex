{\textbf{{性质}:}{资产阶级维新派改良},探索。}

{\textbf{{事件}:}论战:\textbf{{要不要变法}}。\textbf{{要不要兴民权、设议院,实行君主立宪}}。\textbf{{要不要废八股、改科举和兴西学}}。}

{\textbf{{结果}:}在守旧势力打压下失败。}

{\textbf{{原因}:}民族资产阶级\textbf{{力量弱小}};\textbf{{维新派的局限性}}:{一是不敢否定封建主义。二是对帝国主义抱有幻想。三是惧怕人民群众};\textbf{{以慈禧太后为首的强大的守旧势力的反对}}。}

{\textbf{{意义}:{爱国救亡}}运动;资产阶级性质的\textbf{{政治改良运动}};\textbf{{思想启蒙运动}}。}

{\textbf{{人物}:}{康有为(《新学伪经考》、《孔子改制考》)、梁启超(《变法通议》)、谭嗣同(《仁学》)、严复(《天演论》)}}
