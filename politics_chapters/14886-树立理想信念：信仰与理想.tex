{\textbf{{理想的含义与特征}}:理想作为一种社会意识和精神现象,是人类社会实践的产物。理想是一定社会关系的产物。{\textbf{理想源于现实,又超越现实}}。理想是多方面和多类型的。理想不仅具有现实性,而且具有{\textbf{预见性}}。}

{\textbf{{信念的含义与特征}}:信念同理想一样,也是人类特有的一种精神现象。信念具有高于一般认识的稳定性。信仰是信念最集中、最高的表现形式。}

{\textbf{{理想信念的作用}}:指引人生的{\textbf{奋斗目标}}。提供人生的\textbf{{前进动力}}。提高人生的{\textbf{精神境界}}。}

{\textbf{{树立马克思主义的科学信仰:}}马克思主义作为我们党和国家的根本指导思想,是由马克思主义严密的科学体系、鲜明的阶级立场和巨大的实践指导作用决定的。马克思主义是科学的又是崇高的。马克思主义具有持久的生命力。马克思主义以改造世界为己任。}

{\textbf{{树立中国特色社会主义的共同理想}}:{\textbf{建设和发展中国特色社会主义、实现中华民族伟大复兴}},是现阶段我国各族人民的共同理想。}

{\textbf{{实现共同理想需要}}:实现共同理想需坚定对中国共产党的\textbf{{信任}}。坚定走中国特色社会主义道路的\textbf{{信念}}。坚定实现中华民族伟大复兴的\textbf{{信心}}。}

{\textbf{{中国梦}}:{\textbf{实现中华民族伟大复兴}},就是中华民族近代以来最伟大的梦想。中国梦的内涵是实现\textbf{{国家富强、民族振兴、人民幸福}}。实现中国梦,必需走\textbf{{中国道路}},弘扬\textbf{{中国精神}},凝聚\textbf{{中国力量}}。}

{\textbf{{个人理想与社会理想}}{:个人理想与社会理想的关系实质上是个人与社会的关系在理想层面上的反映。个人与社会有机地联系在一起,二者相互依存,相互制约,共同发展。社会理想与个人理想也不是互相孤立的存在,它们之间既相互联系、相互影响,又相互区别、相互制约。}\textbf{{社会理想决定、制约着个人理想;社会理想又是个人理想的凝炼和升华}}{。}}
