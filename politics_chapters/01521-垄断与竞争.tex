资本主义的发展经历{\textbf{自由竞争资本主义和垄断资本主义}}两个阶段。

垄断取代自由竞争在资本主义经济中占据统治地位,垄断资本主义的发展包括\textbf{{私人垄断资本主义和国家垄断资本主义}}两种形式。

{垄断形成的一般过程:自由竞争引起生产集中,生产集中发展到一定阶段就自然形成垄断。}{\textbf{{生产集中}}是指生产资料、劳动力和商品的生产日益集中于少数大企业的过程}{,其结果是大企业在社会生产中所占约比重不断增加。资本集中是指大资本吞并小资本,或由许多小资本合并而成大资本的过程,其结果是越来越多的资本为少数大资本所支配。}

垄断形成的原因:\\
1. 少数大企业容易达成联合起来,进行垄断的协议;\\
2. 大企业形成对竞争的限制,也会产生垄断;\\

3. 大企业为了避免竞争造成两败俱伤也会联合起来,进行垄断。

{\textbf{垄断没有消灭竞争}}:\\
1. 垄断没有消灭商品经济,因而就不可能消除竞争;\\
2. 垄断形成后,垄断组织之间和内部仍存在竞争;\\
3. 任何垄断组织不可能垄断一切。\\

{同自由竞争相比,垄断条件下的竞争,规模大、时间长、手段残酷、程度更加激烈,具有更大的破坏性。}
