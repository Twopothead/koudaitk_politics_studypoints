社会主义从理论到实践的飞跃是通过无产阶级革命实现的,无产阶级革命有暴力与和平两种形式。其中,{\textbf{暴力革命}}\textbf{{是主要的基本形式}}。

社会主义革命{\textbf{首先在一个或几个国家取得胜利}}{:进入垄断资本主义阶段,}{\textbf{经济和政治发展不平衡是资本主义的绝对规律}}{,社会主义可能在少数甚至单独一个资本主义国家获得胜利。}

列宁领导的苏维埃俄国对社会主义道路的探索经历了{\textbf{三个时期}}:{{\textbf{第一个时期}}}{}进步{\textbf{巩固苏维埃政权时期}}。{{\textbf{第二个时期}}}{},外国武装干涉和国内战争时期,即{\textbf{战时共产主义时期}}。{{\textbf{第三个时期}}}{}由战时共产主义转变为\textbf{{新经济政策时期}}。

{列宁关于社会主义建设提出了许多{\textbf{精辟的论述}}:}首先,把{\textbf{建设社会主义作为一个长期探索、不断实践的{\textbf{过程}}}}。其次,把{\textbf{大力发展生产力、提高劳动生产率放在首要地位}}\textbf{。}再次,在多种经济成分并存的条件下,\textbf{{利用商品、货币和市场发展经济}}。最后,{\textbf{利用资本主义,建设社会主义}}{。}
