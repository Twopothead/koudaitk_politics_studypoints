{\textbf{1、 资本主义的生产过程是劳动过程和价值增殖过程的统一}}

2、资本的不同部分在资本主义生产中的作用

\textbf{不变资本:}以生产资料形式存在,在生产过程中不发生价值增值

\textbf{可变资本:}以劳动力形式存在,在生产过程中发生价值增值

区分意义:一是揭露了剩余价值的源泉和资本主义剥削的实质;二是为揭示资本家对工人的剥削程度提供了科学依据。

3、生产剩余价值的两种基本方法------绝对剩余价值和相对剩余价值

(1)绝对剩余价值:在必要劳动时间不变的情况下,

(2)相对剩余价值:在劳动时间不变的前提下,

(3)超额剩余价值:个别企业首先提高劳动生产率

(4)相对剩余价值和超额剩余价值的关系联系:超额剩余价值属于相对剩余价值的范畴,因为它也是通过缩短必要劳动时间相应延长剩余劳动时间而产生的。

4、剩余价值规律的内容:{\textbf{资本主义生产的直接目的和根本动机是为了最大限度地榨取工人创造的剩余价值}}
