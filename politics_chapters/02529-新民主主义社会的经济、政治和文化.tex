我国从中华人民共和国成立到社会主义改造基本完成时期,是一个\textbf{{过渡时期}}。这一时期,\textbf{{我国社会的性质是新民主主义社会}}。新民主主义社会不是一个独立的社会形态,而是由新民主主义到社会主义转变的过渡性的社会形态,它\textbf{{属于社会主义体系}}。

在\textbf{{经济上}},\textbf{{五种经济成分并存}}。即:社会主义性质的国营经济、半社会主义性质的合作社经济、农民和手工业者的个体经济、私人资本主义经济和国家资本主义经济。\textbf{{主要的经济成分是三种}}:\textbf{{社会主义经济、个体经济和资本主义经济}}。在这些经济成分中,\textbf{{通过没收官僚资本而形成的社会主义的国营经济}},\textbf{{掌握了主要经济命脉}},居于\textbf{{领导地位}}。

在\textbf{{政治上}},\textbf{{新民主主义国家实行工人阶级领导的,以工农联盟为基础的,包括工人阶级、农民阶级、城市小资产阶级和民族资产阶级在内的人民民主专政}}。新中国建立之初,我国人民民主专政,是属于新民主主义政权性质。

{在}\textbf{{文化上}}{,}\textbf{{发展以马克思主义指导下的新民主主义的文化,即民族的、科学的、大众的文化}}{。}
