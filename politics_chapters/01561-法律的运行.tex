法律的\textbf{运行:}法律的运行是一个从\textbf{创制、实施到实现}的过程。这个过程主要包括\textbf{法律制定(立法)、法律遵守(守法)、法律执行(执法)、法律适用(司法)}等环节。{}

法律\textbf{制定}:{\textbf{全国人民代表大会及其常务委员会}}行使国家立法权。大体包括以下四个环节:法律案的\textbf{提出},法律案的\textbf{审议},法律案的\textbf{表决},法律的\textbf{公布(国家主席)}。{}

法律\textbf{遵守}:依法办事包括两层含义:一是\textbf{依法享有并行使}权利,二是\textbf{依法承担并履行}义务。一切组织和个人都是守法的\textbf{主体}。{}

法律\textbf{执行}:在法律运行中,{\textbf{行政执法}}\textbf{是最大量、最经常的工作,是实现国家职能和法律价值的重要环节}。

法律\textbf{适用}{:在我国,}司法机关是指\textbf{国家检察机关(人民检察院)和审判机关(人民法院)}{。
}\textbf{{司法的基本}{要求}}{}{}{}{}是\textbf{正确、合法、合理、及时}{。}\textbf{{司法原则}}{主要有:}司法公正;公民在法律面前一律平等;以事实为依据,以法律为准绳;司法机关依法独立行使职权{。}
