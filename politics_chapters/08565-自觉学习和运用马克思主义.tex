{学习和掌握马克思主义基本原理,是大学生个人成长和长远发展的客观需要,具有重要的现实意义。}

{马克思主义诞生以来,指引着各国无产阶级和广大劳动人民进行了艰苦卓绝的斗争,科学社会主义从理论~发展为亿万人民群众的实践,人类取得了历史性的进步。}

{中国共产党成立以来,把马克思主义基本原理同~中国具体实际相结合,带领中国人民取得了革命、建设和改革的卓越成就。}

{{\textbf{实践证明,}}{马克思主义是我们立党立国的根本指导思想,是全国各族人民团结奋斗的共同理论基础。马克思主义的基本原理任何时候都要坚持,否则我们的事业就会因为没有正确的理论基础和思想灵魂而迷失方向,就会归于失败。}}

{{这就是我们为什么必须始终学习和坚持马克思主义基本原理的道理所在}{。}}

{当前,在世界范围内和我国社会主义现代化建设中,经济、政治、思想、文化各个方面都有许多复杂的事物需要认识,许多重大问题需要回答,许多未曾认识的领域需要开拓。只有马克思主义才能引导我们深刻认识社会发展的客观规律,把握世界形势;变化的本质,提高解决建设和改革中各种实际问题的本领。同时,随着改革开放和社会主义市场经济的发展,我国社会正发生着深刻的历史变革,在社会主义主流意识形态得到坚持和发展的同时,社会生活多样、多变的特征日益凸显,各种思想观念相互交织、相互影响、相互激荡;现代社会生活多样化,人们的思想活动具有更多的独立性、选择性、多变性和差异性。面对这种情况,只有学习、掌握并坚持以马克思主义作为我们行动的指南,才能更好地坚持爱国主义、集体主义、社会主义思想的主旋律,才能有效地整合各种各样的利益诉求和价值观念,在全社会形成强大的凝聚力和共同的意志。}

{{学习马克思主义理论,}{\textbf{我们应该做到}}{:}}

{{第一}{,学习理论,武装头脑,要努力在掌握理论的科学体系上下功夫,在掌握基本原理及其精神实质上下功夫,在掌握马克思主义的立场、观点、方法并用以指导实践上下功夫。}}

{{第二}{,坚持和弘扬理论联系实际的学风,是学习马克思主义基本原理的根本方法。所谓理论联系实际,就是以马克思主义基本原理为指导,联系国际国内的大局,联系社会实际,去观察和分析问题。}}

{{第三}{,用科学的态度对待马克思主义。坚持马克思主义不动摇,这是就马克思主义的基本原理、基本观点和基本方法而言的。随着时代的发展和历史条件的变化,马克思主义创始人针对特定历史条件的一些具体论述可能不再适用,而新的实践又会提出新的问题,需要我们去认识、去解决,这就要求我们在坚持马克思主义基本原理的基础上,不断地在实践中丰富和发展马克思主义。在我国社会主义实践中,坚持和发展是统一的。}}

{{\textbf{因此}}{,我们要把马克思主义作为行动的指南,在思想上自觉地坚持以马克思主义为指导,确立马克思主义的坚定信念,树立和坚定共产主义远大理想;不断提高运用马克思主义的立场、观点和方法分析、解决问题的能力,自觉地辨别和抵制各种不良思想文化的影响;不断增强服务社会的本领,自觉投身中国特色社会主义实践。}}
