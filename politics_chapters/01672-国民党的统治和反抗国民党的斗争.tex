\textbf{国民党在全国统治的特点:}首先,国民党建立了庞大的军队。其次,国民党还建立了庞大的全国性特务系统。再次,为了控制人民,禁止革命活动,国民党还大力推行保甲制度。最后,为了控制舆论,剥夺人民的言论和出版自由,国民党还厉行文化专制主义。{}

\textbf{国民党在全国统治的性质:}维护帝国主义、封建主义、官僚资本主义的利益,巩固自身统治。{}

{1927}年{8}月{7}日,\textbf{中共中央在汉口秘密召开紧急会议}(即\textbf{{八七会议}}),\textbf{彻底清算了大革命后期陈独秀的右倾机会主义错误,确定了土地革命和武装反抗国民党的总方针,并选出了以瞿秋白为书记的中央临时政治局}。\textbf{毛泽东}在会上着重阐述了党必须依靠农民和掌握枪杆子的思想,强调党{``}\textbf{以后要非常注意军事,须知政权是由枪杆子中取得的}。{''}\textbf{八七会议开始了从大革命失败到土地革命战争兴起的{历史性转折}}。

\textbf{三大起义}{:南昌起义(8.1),广州起义,秋收起义中国革命由此发展到了一个新的阶段,即}土地革命战争{时期,或称}十年内战时期{。}
