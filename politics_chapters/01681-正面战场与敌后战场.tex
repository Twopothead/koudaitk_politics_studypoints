蒋介石集团实行的是片面抗战路线,不敢放手发动和武装民众,将希望单纯寄托在政府和正规军的抵抗上。尽管国民党军队在抗战初期对日作战比较积极比较努力,但是国民党正面战场除了台儿庄战役取得大捷外,其他战役几乎都是以退却、失败而结束。造成这种状况的客观原因,是由于在敌我力量对比上,日军占很大的优势;主观原因,则是国民党战略指导方针上的失误。主要是实行片面抗战路线及消极防御的作战方针。

{抗日战争进入相持阶段,}日本对国民党政府采取以政治诱降为主、军事打击为辅的方针{。国民党在重申坚持持久抗战的同时,其对内对外政策发生重大变化。}采取消极抗日、积极反共的政策{。}

中国共产党主张实行全面抗战的路线,即\textbf{人民战争路线}。中共在洛川召开政治局扩大会议(\textbf{洛川会议}),制定了抗日救国\textbf{十大纲领},\textbf{强调要打倒日本帝国主义,关键在于使已经发动的抗战成为全面的全民族的抗战}。

\textbf{平型关战役}歼敌{1000}余人,取得全\textbf{民族抗战以来中国军队的第一次重大胜利。}

{在抗日战争的}\textbf{初期和中期}{,}\textbf{游击战被提到了战略的地位,具有全局性的意义}{。在战略防御阶段,对阻止日军的进攻、减轻正面战场压力、使战争转入相持阶段起了关键性的作用。}\textbf{在战略相持阶段,敌后游击战争成为主要的抗日作战方式}{。游击战还为人民军队进行战略反攻准备了条件。}
