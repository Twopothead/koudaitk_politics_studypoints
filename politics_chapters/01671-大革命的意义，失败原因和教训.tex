蒋介石在上海发动``四一二''政变,汪精卫在武汉发动``七一五''政变。这标志着国共合作全面破裂,大革命最终失败。

\textbf{{大革命的意义}}:1.{
}沉重打击了帝国主义在华势力,基本推翻了北洋军阀统治。中国共产党提出的反帝反封建的口号成为广大人民的共同呼声。2.{
}教育和锻炼了各革命阶级,党领导的工农大众经受了革命的洗礼,提高了政治觉悟,为后来共产党领导的土地革命的开展奠定了群众基础。3.{
}提高了中国共产党在全国人民中的政治威望,党在马克思列宁主义指导下,制定民主革命纲领,发挥了党的政治优势和组织优势。

\textbf{{失败的原因}}{:1.
从客观方面来讲,是由于}\textbf{反革命力量的强大}{,大大超过了革命的力量;资产阶级发生严重的动摇、统一战线出现剧烈的分化;蒋介石集团、汪精卫集团先后被帝国主义势力和地主阶级、买办资产阶级拉进反革命营垒里去了。2.
从主观方面来说,是由于}\textbf{陈独秀为代表的右倾机会主义}{的错误;当时的中国共产党还处于幼年时期;共产国际的错误干预。}

\textbf{{经验教训}}{:1. 必须建立广泛的革命统一战线。2.
无产阶级领导权的中心问题是农民问题。3.
中国革命的主要斗争形式是武装斗争。4.
必须不断加强共产党思想上、政治上和组织上的建设。}
