{\textbf{意义}}

{凝聚思想共识的法治航标;推进国家治理现代化的重要举措;全面依法治国的基础工程。}

{{中国特色社会主义法律体系是在中国共产党领导下,适应中国特色社会主义建设事业的历史进程而逐步形成的。经历了个从无到有、从初步形成到基本形成再到形成、然后经过不断完善趋于更加成熟的过程。}{一个}\textbf{立足中国国情和实际、适应改革开放和社会主义现代化建设需要、集中体现党和人民意志的,以宪法为统帅,以宪法相关法、民法商法等多个法律部门的法律为主干,由法律、行政法规、地方性法规等多个层次的法律规范构成的}{\textbf{中国特色社会主义法律体}}{\textbf{系}}{已经形成}{,国家经济建设、政治建设、文化建设、社会建设以及生态文明建设的各个方面实现有法可依。}\\
}

{\textbf{内容}}

{建设完备的法律规范体系。}

{建设高效的法治实施体系。}

{建设严密的法治监督体系。}

{建设有力的法制保障体系。}

{建设完善的党内法规体系。}
