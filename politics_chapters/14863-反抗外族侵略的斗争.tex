{\textbf{{三元里人民的抗英斗争}}}{,是中国近代史上中国人民}{\textbf{{第一次}}}{大规模的反侵略武装斗争。}

{太平天国农民战争后期,太平军曾多次重创英、法侵略军和外国侵略者指挥的洋枪队``常胜军''、``常捷军''。}

{台湾人民与台湾总兵刘永福所率领的}{黑旗军}{共同抗击日本侵略。在中越边境镇南关,老将冯子材身先士卒,大败法军,取得}{镇南关大捷}{。}

{义和团运动在粉碎帝国主义列强瓜分中国的斗争中,发挥了重大的历史作用。}

{在}{1895}{年,严复就写了}{《救亡决论》}{一文,响亮地喊出了``救亡''的口号。在甲午战争后,严复翻译了}{《天演论》(}{1898}{年正式出版)。他用``物竞天择''、``适者生存''的社会进化论思想,为这种危机意识和民族意识提供了理论根据。}

{孙中山1894年11月在创立革命团体兴中会时就喊出了``}{振兴中华}{''这个时代的最强音。}
