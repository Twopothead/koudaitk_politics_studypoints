\textbf{{恩格斯}}第一次明确提出``\textbf{{{全部哲学,特别是近代哲学的重大的基本问题,是思维和存在的关系问题}}}''哲学基本问题包括\textbf{{{两个方面}}}的内容:其一,意识和物质、精神和自然界,究竟谁是\textbf{{{世界的本原}}},即物质和精神何者是\textbf{{{第一性}}}的问题;其二,思维能否认识或正确认识存在,即思维和存在有无\textbf{{{同一性}}}的问题。

根据对哲学基本问题\textbf{{{第一方面}}}的不同回答,哲学可划分为\textbf{{{唯物主义和唯心主义}}}两个对立的基本派别{。}\textbf{{{唯物主义}}}把世界的本原归结为物质,\textbf{{{主张物质第一性,意识第二性}}},意识是物质的产物,\textbf{{{唯心主义}}}把世界的本原归结为精神,\textbf{{{主张意识第一性,物质第二性}}},物质是意识的产物。

{根据对哲学基本问题}\textbf{{{第二方面}}}{的不同回答,哲学又可以划分为}\textbf{{可知论和不可知论}}{。}\textbf{{可知论}}{认为世界是可以被认识的,思维和存在具有}\textbf{{同一性}}{;}\textbf{{不可知论}}{认为世界是不能被人所认识或不能被完全认识的,}\textbf{{否认思维和存在的同一性}}{。}
