{同一真理都是真理的绝对性和相对性的统一}。

\textbf{{真理的绝对性(绝对真理)}}:一是就真理的客观性而言,任何真理都是对客观事物及其规律的正确认识,这是无条件的、绝对的。二是就人类认识的本性来说,完全可以正确认识无限发展的物质世界,这也是无条件的、绝对的。三是从真理的发展来说,无数相对真理的总和构成绝对真理。

\textbf{{真理的相对性(相对真理)}}{:真理的相对性或有三层含义:一是从}\textbf{{广度}}{上说有待于扩展;二是从}\textbf{{深度}}{上说有待于深化;三是从}\textbf{{进程}}{上说有待发展。}

{}

真理的相对性:指人们在一定条件下对事物及其发展规律的正确认识总是有限度的。相对性有两方面含义,一是真理所反映的对象是有条件的,有限的;二是真理反映客观对象的正确程度也是有条件的,有限的。任何真理都只能是主观对客观事物近似正确即相对正确的反映。

真理是由相对不断走向绝对的永无止境的发展过程。这是真理发展的一个规律。\\

\textbf{{\\
}}

\textbf{{真理的绝对性和相对性是辩证统一的,其一,二者相互依存。}}所谓相互依存是说人们对于客观事物及其本质和规律的每一个正确认识,都是在一定范围内一定程度上一定条件下的认识,因而必然是相对的和有局限性的,但是在这一范围内,一定程度上一定条件下,它又是对客观对象的正确反应,因而它又是无条件的绝对的。其二,二者相互包含,所谓相互包含,一是说真理的绝对性寓于相对性之中,二是说真理的相对性必然包含并表现着真理的绝对性,所以绝对真理和相对真理是不可分的,没有离开绝对真理的相对真理,也没有离开相对真理的绝对真理。

~

在二者的辩证关系中还要明确,{\textbf{真理永远处在由相对向绝对的转化和发展中,是从真理的相对性走向绝对性}},接近绝对性的过程,{\textbf{任何真理性的认识都是由真理的相对性向绝对性转化过程中的一个环节}},这是真理发展的规律。

既要反对绝对主义,又要反对相对主义。
