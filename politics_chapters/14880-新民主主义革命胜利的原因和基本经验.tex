{\textbf{{中共七届二中全会}}:
①提出了迅速夺取全国胜利的方针。②党的工作重心必须由乡村转移到城市。
③规定了党在全国胜利后在政治、经济、外交方面应当采取的基本政策。④他提出了``两个务必''的思想,即``{务必使同志们继续地保持谦虚、谨慎.不骄、不躁的作风,务必使同志们继续地保持艰苦奋斗的作风}''。}

{1949年9月21日,{中国人民政治协商会议第一届全体会议}在北平隆重开幕。会议通过了\textbf{{《中国人民政治协商会议共同纲领》}}。\textbf{{《共同纲领》在当时是全国人民的大宪章,起着临时宪法的作用}}。}

{新民主主义革命\textbf{{胜利原因}}:{1.}~{中国革命的发生,有着深刻的社会根源和雄厚的群众基础。2.~中国革命之所以能够走上胜利发展的道路,是由于有了中国工人阶级的先锋队------中国共产党的领导。3.~中国革命之所以能够赢得胜利,同国际无产阶级和人民群众的支持也是分不开的}。}

{\textbf{{基本经验}}:``\textbf{{统一战线,武装斗争,党的建设,是中国共产党在中国革命中战胜敌人的三个法宝}},三个主要的法宝。''}

{\textbf{{统一战线中存在着两个联盟}}:{一个是劳动者的联盟},主要是工人、农民和城市小资产阶级的联盟;{一个是劳动者与非劳动者的联盟},主要是劳动者与民族资产阶级的联盟,有时还包括与一部分大资产阶级的暂时的联盟。前者是基本的、主要的;后者是辅助的、同时又是重要的。{必须坚决依靠第一个联盟,争取建立和扩大第二个联盟}。~{巩固和扩大统一战线的关键,是坚持工人阶级及其政党的领导权}。为此,必须率领同盟者向共同的敌人作坚决的斗争并取得胜利;必须对被领导者给以物质福利,至少不损害其利益,同时对被领导者给以政治教育;必须对同工人阶级争夺领导权的资产阶级采取又联合、又斗争的政策。}

{\textbf{{中国的武装斗争实质上是工人阶级领导的农民战争}}。\textbf{{武装斗争是中国革命的特点和优点之一}}。{坚持党对军队的绝对领导是建设新型人民军队的根本原则}。{全心全意为人民服务是人民军队的唯一宗旨。它集中体现了人民军队的本质,是人民军队立于不败之地的根本所在}。}

{{中国共产党要领导革命取得胜利,必须不断加强党的思想建设、组织建设和作风建设}{。}{党内无产阶级思想和非无产阶级思想之间的矛盾成为党内思想上的主要矛盾}{。要建设一个广大群众性的、马克思主义的无产阶级政党,是一项艰巨的任务,也}{是一项伟大的工程}{。}\textbf{{加强党的建设,必须把思想建设始终放在首位}}{,克服党内的非无产阶级思想。党在领导新民主主义革命的过程中,}\textbf{{逐步形成了理论和实践相结合的作风、和人民群众紧密地联系在一起的作风以及自我批评的作风}}{,这是}\textbf{{中国共产党区别于其他任何政党的显著标志}}{。}}
