\textbf{{生产力}:}人们解决社会同自然矛盾的实际能力,是人类改造自然使其适应社会需要的物质力量。在哲学上,生产力是标志人类改造自然的实际程度和实际能力的范畴,它表示\textbf{{人和自然的关系}}。\textbf{{一是劳动资料即劳动手段}}。\textbf{{二是劳动对象}}。\textbf{{三是劳动者}}。生产力中还包含着科学技术。\textbf{{科学技术是先进生产力的集中体现和主要标志,是第一生产力}}。{生产力包括生产力水平,生产力的性质和生产力状况。}\textbf{{生产力的水平}}是生产力的\textbf{{量的规定性}}。\textbf{{生产力的性质}}是生产力的\textbf{{质的规定性}}。\textbf{{生产力状况是生产力的水平和生产力的性质的统一}}。

\textbf{{生产关系}}:人们在物质生产过程中形成的不以人的意志为转移的经济关系。\textbf{{生产关系是社会关系中最基本的关系}}。狭义的生产关系包括生产资料所有制的关系、生产中人与人的关系和产品分配关系。广义的生产关系包括\textbf{{生产}}、\textbf{{分配}}、\textbf{{交换}}和消费中的关系。在生产关系中,\textbf{{生产资料所有制的关系是最基本的}}。

\textbf{{生产力决定生产关系,生产关系反作用于生产力}}{。生产力和生产关系、经济基础和上层建筑的矛盾是}\textbf{{社会基本矛盾}}{。生产力和生产关系的矛盾是}\textbf{{更为根本的矛盾}}{。社会基本矛盾作为社会发展的}\textbf{{根本动力}}{,生产力是社会基本矛盾运动中}\textbf{{{最基本的动力因素}}}{{},是人类社会发展和进步的}\textbf{{{最终决定力量}}}{。生产力是社会发展的}\textbf{{根本内容}}{,是实现社会发展多重目标的}\textbf{{{根本条件}}}{,是社会发展的}\textbf{{{集中体现和客观标志}}}{,}\textbf{{{是衡量社会进步的根本尺度}}}{。}
