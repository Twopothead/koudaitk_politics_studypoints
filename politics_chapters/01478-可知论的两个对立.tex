{\textbf{唯物主义认识论和唯心主义认识论}}的对立:\\
1. 前提或基础即世界观不同,前者是唯物主义,后者是唯心主义。\\
2. 核心或本质不同,前者是反映论。\\
3.
认识路线(顺序)不同,前者是从物到感觉和思想,后是从思想和感觉到物。\\[2\baselineskip]{\textbf{能动反映论}}(辩证的)较\textbf{{直观反映论}}(形而上学性的)的优势:\\
1.
彻底的科学的反映论,克服了形而上学唯物主义反映论的狭隘性、机械性和被动直观性。\\
2. 把科学的实践观引入认识论,把实践作为全部认识论的基础。\\
3.
把辩证法贯彻于反映论,科学地说明了人认识发展的辩证过程,揭示了认识运动的基本规律,克服了形而上学唯物主义把认识直观化、凝固化、片面化的形而上学的缺陷。\\[2\baselineskip]
