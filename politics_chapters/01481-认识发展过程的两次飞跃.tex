{\textbf{感性认识是理性认识的基础,感性认识有待发展到理性认识}},反对割裂二者关系的教条主义和经验主义。\\[2\baselineskip]\textbf{从感性认识到理性认识的第一次飞跃}。\\
理由:认识的真正任务是要达到理性认识才能有效的指导实践。\\
条件:要有丰富而真实的感性材料,并对感性材料进行辩证思维的加工。\\[2\baselineskip]\textbf{从理性认识到实践的第二次飞跃}。\\
理由:认识的需要和要求,实践的需要和要求。\\
条件:主要是要坚持从实际出发,坚持一般理论和具体实践相结合的原则。\\[2\baselineskip]认识过程的\textbf{{反复性和无限性}}:实践、认识、再实践、再认识,无限循环,由低级阶段向高级阶段不断推移的永无止境的前进运动。\\[2\baselineskip]{思想路线是马克思主义认识论在实际工作中的具体运用},是体现在实际工作中的认识路线。党的思想路线,亦称党的认识路线:一切从实际出发,理论联系实际,实事求是,
在实践中检验真理和发展真理。\\[2\baselineskip]
