\textbf{农业:互助组,初级社,高级社}。\textbf{先合作化、后机械化}。实行\textbf{积极发展、稳步前进、逐步过渡的方针};农业互助合作的发展,要坚持\textbf{自愿和互利的原则},采取\textbf{典型示范、逐步推广的方法},发展一批,巩固一批;正确分析农村的阶级和阶层状况,制定正确的阶级政策。要始终\textbf{把是否增产作为衡量合作社是否办好的标准};要把社会改造同技术改造相结合。在实现农业合作化\textbf{{以后}},国家应努力用先进的技术和装备发展农业经济。

\textbf{手工业:}{中国共产党采取的是}\textbf{积极领导、稳步前进的方针}{。手工业合作化的组织形式,是由}\textbf{手工业生产合作小组(手工业供销小组)}{、}\textbf{手工业供销合作社到手工业生产合作社}{;步骤是从供销人手,由小到大,由低到高,逐步实行社会主义改造和生产改造。}
