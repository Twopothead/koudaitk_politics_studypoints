\textbf{{{爱国主义的基本要求}}}{:爱祖国的大好河山;爱自己的骨肉同胞(对人民群众感情的深浅程度,是检验一个人对祖国忠诚程度的}\textbf{{{试金石}}}{)。爱祖国的灿烂文化。}

\textbf{{{爱国主义的时代价值}}}{:维护祖国统一和民族团结的纽带;实现中华民族伟大复兴的动力;实现人生价值的源泉。}

{\textbf{{{新时期的爱国主义}}}{:对于大学生来说,在如何把握经济全球化趋势与爱国主义的相互关系问题上,需要看重树立以下三个观念。}}

{1.人有地域和信仰的不同,但报效祖国之心不应有差别。}

{2.科学没有国界,但科学家有祖国。}

{3.经济全球化是世界经济发展的必然趋势,但不等于全球政治、文化一体化。}

{{爱国主义与爱社会主义具有一致性;爱国主义与拥护祖国统一也是一致的}。}
