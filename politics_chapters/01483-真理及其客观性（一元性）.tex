{真理是指人类对客观事物及其规律的正确认识}。真理的根本属性是客观性,是指真理的内容是客观的,真理具有真实的客观内容;检验认识真理性的标准(实践)也是客观的。

\textbf{{实践是检验真理的唯一标准。}}

{实践标准的确定性与不确实性:}{确定性在于实践是检验真理的唯一标准。不确定性在于一定历史阶段上的具体实践具有局限性}{,它往往不能充分证明或驳倒某一认识的真理性;实践检验真理是一个过程,不是一次完成的;已被实践检验过的真理还要继续经受实践的检验。}

{}

\textbf{真理的客观性(一元性)}:一是指真理的内容是客观的;二是指真理的标准是客观的。

~真理的绝对性:真理的绝对性是指真理的内容表明了主客观统一的确定性和发展的无限性,它有两个方面的含义,一是任何真理都必然包含同客观对象相符合的客观内容,都同谬误有原则的界限,这一点是无条件的,绝对的;二是人类认识按其本性来说能够正确认识无限发展着的物质世界,认识每前进一步,都是对无限发展着的世界的接近,这一点也是无条件的,绝对的。因此承认世界的可知性,承认人能获得关于无限发展着的世界的正确认识,也就是承认了真理的绝对性。


